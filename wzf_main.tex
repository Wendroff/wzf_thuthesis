\documentclass[type=master]{thuthesis}
% 选项:
%   type=[bachelor|master|doctor|postdoctor], % 必选
%   secret,                                   % 可选
%   pifootnote,                               % 可选(建议打开)
%   openany|openright,                        % 可选,基本不用
%   arial,                                    % 可选,基本不用
%   arialtoc,                                 % 可选,基本不用
%   arialtitle                                % 可选,基本不用

% 所有其它可能用到的包都统一放到这里了,可以根据自己的实际添加或者删除。
\usepackage{thuthesis}

% 定义所有的图片文件在 figures 子目录下
\graphicspath{{../figures/}}

% 可以在这里修改配置文件中的定义。导言区可以使用中文。
% \def\myname{薛瑞尼}

\begin{document}

%%% 封面部分
\frontmatter
\thusetup{
  %******************************
  % 注意:
  %   1. 配置里面不要出现空行
  %   2. 不需要的配置信息可以删除
  %******************************
  %
  %=====
  % 秘级
  %=====
  %secretlevel={秘密},
  %secretyear={10},
  %
  %=========
  % 中文信息
  %=========
  ctitle={介质阻挡放电等离子体激发器在湍流减阻中的应用研究},
  cdegree={工学博士},
  cdepartment={航天航空学院},
  cmajor={力学},
  cauthor={王哲夫},
  csupervisor={符松教授},
  %cassosupervisor={陈文光教授}, % 副指导老师
  %ccosupervisor={某某某教授}, % 联合指导老师
  % 日期自动使用当前时间,若需指定按如下方式修改:
  % cdate={超新星纪元},
  %
  % 博士后专有部分
  %cfirstdiscipline={计算机科学与技术},
  %cseconddiscipline={系统结构},
  %postdoctordate={2009年7月——2011年7月},
  %id={编号}, % 可以留空: id={},
  %udc={UDC}, % 可以留空
  %catalognumber={分类号}, % 可以留空
  %
  %=========
  % 英文信息
  %=========
  etitle={Study of the application of Dielectric-Barrier-Discharge plasma actuators in turbulent drag reduction},
  % 这块比较复杂,需要分情况讨论:
  % 1. 学术型硕士
  %    edegree:必须为Master of Arts或Master of Science(注意大小写)
  %             “哲学、文学、历史学、法学、教育学、艺术学门类,公共管理学科
  %              填写Master of Arts,其它填写Master of Science”
  %    emajor:“获得一级学科授权的学科填写一级学科名称,其它填写二级学科名称”
  % 2. 专业型硕士
  %    edegree:“填写专业学位英文名称全称”
  %    emajor:“工程硕士填写工程领域,其它专业学位不填写此项”
  % 3. 学术型博士
  %    edegree:Doctor of Philosophy(注意大小写)
  %    emajor:“获得一级学科授权的学科填写一级学科名称,其它填写二级学科名称”
  % 4. 专业型博士
  %    edegree:“填写专业学位英文名称全称”
  %    emajor:不填写此项
  edegree={Doctor of Philosophy},
  emajor={Mechanics},
  eauthor={Wang Zhefu},
  esupervisor={Professor Fu Song},
  %eassosupervisor={Chen Wenguang},
  % 日期自动生成,若需指定按如下方式修改:
  % edate={December, 2005}
  %
  % 关键词用“英文逗号”分割
  ckeywords={DBD激发器, 湍流减阻, 转捩推迟, 流动稳定性, 流动控制},
  ekeywords={DBD actuator, turbulent drag reduction, transition delay, flow stability, flow control}
}

% 定义中英文摘要和关键字
\begin{cabstract}
  本文将大型客机减阻问题分解、抽象为三维边界层转捩推迟问题和充分发展槽道减阻问题。针对这两个问题,分别提出了应用介质阻挡放电等离子体激发器的新控制方案。

  对于三维边界层转捩推迟问题,本文先研究了顺压梯度后掠平板这一模型流动。针对这种流动,本文提出了在每个展向周期放置一个DBD激发器的谐波激励控制方案。为了得到较优的控制参数,本文推导了基于伴随方程的敏感性分析公式。通过分析敏感性因子的分布,发现在横流涡下方偏下扫的位置处添加展向正方向的体积力可以抑制扰动的发展。在这一位置放置等离子体激发器成功抑制了横流涡的增长。本文同时测试了不同的激发器流向位置和激励电压,发现都是适中的情况控制效果较好。

  在真实的后掠翼工况中,本文先测试了谐波激励控制方法。这种方法对激发器展向位置精度要求非常高,所以鲁棒性欠佳。随后又提出在每个展向周期放置两个激发器的亚谐波激励控制方案。计算发现,亚谐波控制方案对激发器位置的依赖性不强,总能起到减阻的效果。本文还测试了一个非设计工况,在不改变激发器间距的情况下依然成功降低了边界层内扰动的能量。通过一个反向控制算例,本文揭示了降低了横流强度的基本流修正模态在亚谐波激励控制中起主导作用。

  最后,对于充分发展槽道湍流,本文提出了定常激励和周期激励两种采用DBD等离子体激发器的减阻控制方案。在定常激励控制方案中,两个向相反方向吹气的激发器在槽道中产生了与槽道大约同一尺度的流向涡。这对流向涡在槽道中部产生了再层流化的现象,并因此降低了湍流摩阻。在周期激励方案中,激发器阵列被用于产生时而向左时而向右的展向周期振动。由于体积力在激励的前半个周期内会在流向涡下部产生与涡诱导速度方向相反的展向速度,这使得流向涡的生成受到阻碍。这两种控制方案均能起到大约9\%的减阻效果。

  本文的创新点主要有:
  \begin{itemize}
    \item 建立了三维边界层敏感性分析方法,推导了相关公式;
    \item 将DBD亚谐波激励控制方法应用于真实后掠翼流动;
    \item 提出了定常激励和周期激励湍流减阻控制方案,并采用条件平均技术进行了机理分析。
  \end{itemize}
\end{cabstract}

% 如果习惯关键字跟在摘要文字后面,可以用直接命令来设置,如下:
% \ckeywords{\TeX, \LaTeX, CJK, 模板, 论文}

\begin{eabstract}
   The potential use of Dielectric-Barrier-Discharge (DBD) plasma actuators for drag reduction of a passenger aircraft is studied in this paper. Two approaches are proposed to accomplish this goal: laminar-turbulent transition delay on the wing and full turbulent flow drag reduction on the fuselage.
   
   Modern airplanes always use swept wings. The boundary layer over these wings are three dimensional and usually subject to crossflow instability. Sensitivity of the boundary layer to body force induced by DBD actuators is analysed through solving the adjoint equations. It is found that force at the bottom of the crossflow vortex with opposite direction of local flow can hinder the growth of the disturbance energy. Inspired by this result, the harmonic control method, which put one actuator per fundamental wavelength, is come up with. However, this control method is not robust. Once the spanwise locations of actuators are incorrect, the transition will be promoted. The study of pressure gradient effect points out the attenuation of instability modes always links to the decrease of crossflow velocity. Thus, the subharmonic control with two actuators per fundamental wavelength is proposed to decline the crossflow velocity. This control method is more robust and it stabilizes the boundary layer with several different streamwise and spanwise actuator locations. A reverse control case reveals that the mean flow distortion mode which reduces the crossflow velocity plays an important role in this control method.
   
   Two control methods with steady actuation and periodical actuation are proposed to control full turbulence and they are tested in a turbulent channel flow. For the steady control method, two actuators mounted on the wall with different directions. They generate two large vortices whose scales are comparable to the channel height. These secondary vortices relaminarize flow at the sweep region and reduce the total drag. For the periodical actuation, DBD actuator array is employed to cause the spanwise oscillation near wall. Results of conditional average show this oscillation hinder the regeneration of streamwise vortices. Both control methods reduce nearly 9\% total drag.
\end{eabstract}

% \ekeywords{\TeX, \LaTeX, CJK, template, thesis}

% 如果使用授权说明扫描页,将可选参数中指定为扫描得到的 PDF 文件名,例如:
% \makecover[scan-auth.pdf]
\makecover

%% 目录
\tableofcontents

%% 符号对照表
\begin{denotation}[3cm]
\item[HPC] 高性能计算 (High Performance Computing)
\item[cluster] 集群
\item[Itanium] 安腾
\item[SMP] 对称多处理
\item[API] 应用程序编程接口
\item[PI] 聚酰亚胺
\item[MPI] 聚酰亚胺模型化合物,N-苯基邻苯酰亚胺
\item[PBI] 聚苯并咪唑
\item[MPBI] 聚苯并咪唑模型化合物,N-苯基苯并咪唑
\item[PY] 聚吡咙
\item[PMDA-BDA]	均苯四酸二酐与联苯四胺合成的聚吡咙薄膜
\item[$\Delta G$] 活化自由能 (Activation Free Energy)
\item[$\chi$] 传输系数 (Transmission Coefficient)
\item[$E$] 能量
\item[$m$] 质量
\item[$c$] 光速
\item[$P$] 概率
\item[$T$] 时间
\item[$v$] 速度
\item[劝学] 君子曰:学不可以已。青,取之于蓝,而青于蓝;冰,水为之,而寒于水。—— 荀况
\end{denotation}



%%% 正文部分
\mainmatter
\chapter{引言}
\label{cha:intro}

等离子体激发器由于具有响应时间短,安装方便,耗能低,器件小等众多优点,近些年得到了流动控制领域研究者们的青睐。本文主要研究了介质阻挡放电(dielectric barrier discharge,DBD)等离子体激发器在湍流减阻方面的应用。研究主要分为两个部分,分别是通过推迟转捩降低阻力和通过改变充分发展湍流的相干结构降低阻力。引言部分将会先介绍我们所采用的等离子体激发器,然后再分别综述这两种控制方法的研究现状。


\section{介质阻挡放电等离子体激发器}

介质阻挡放电等离子体激发器由两片电击和一层绝缘层构成(如图~\ref{fig:SchematicPlasma}所示)。当在两片电极上加上高电压时,两片电击之间的空气就会被电离。在电场的作用下,带电的离子会做定向运动,并通过与不带电的空气分子的碰撞作用,将动量转移到空气分子上。从宏观的角度看,等离子体激发器在开启时会产生图示方向的射流。介质阻挡放电等离子体激发器最早可以追溯到Masuda和Washizu的文章\cite{Masuda1979}。在1998年,Roth首次将其用于流动控制\cite{Roth1998}。由于本文主要是采用数值模拟的方法研究这种激发器在流动控制中的应用,所以本文将在引言的第一部分重点介绍DBD等离子体激发器的数值模拟方法与将其应用于流动控制的研究现状。
\begin{figure}
  \centering
  \includegraphics[width=\textwidth]{plasma}
  \caption{等离子体激发器示意图\cite{Whalley2012}}\label{fig:SchematicPlasma}
\end{figure}

\subsection{介质阻挡放电等离子体激发器数值模拟模型}
介质阻挡放电等离子体激发器数值模拟方法
%封面的例子请参看 \texttt{cover.tex}。主要符号表参看 \texttt{denation.tex},附录和
%个人简历分别参看 \texttt{appendix01.tex} 和 \texttt{resume.tex}。里面的命令都很直
%观,一看即会\footnote{你说还是看不懂?怎么会呢?}。
\subsection{介质阻挡放电等离子体激发器在流动控制方面的应用}
介质阻挡放电等离子体激发器在流动控制方面的应用
\section{通过推迟层流/湍流转捩减阻}
\label{sec:first}
\subsection{二维边界层失稳与转捩}
二维边界层失稳与转捩
\subsection{三维边界层失稳与转捩}
三维边界层失稳与转捩
\subsection{转捩推迟方案研究进展}
转捩推迟方案研究进展

{\kaishu 坡仙擅长行书、楷书,取法李邕、徐浩、颜真卿、杨凝式,而能自创新意。用笔丰腴}

{\fangsong 易与天地准,故能弥纶天地之道。仰以观於天文,俯以察於地理,是故知幽明之故}

% 非本科生一般用不到幼圆与隶书字体。需要的同学请查看 ctex 文档。
{\ifcsname youyuan\endcsname\youyuan\else[无 \cs{youyuan} 字体。]\fi 有天地,然后
  万物生焉。盈天地之间者,唯万物,故受之以屯;屯者盈也,屯者物之始生也。}

{\heiti 履而泰,然后安,故受之以泰;泰者通也。物不可以终通,故受之以否。}

{\ifcsname lishu\endcsname\lishu\else[无 \cs{lishu} 字体。]\fi 有事而后可大,故受
  之以临;临者大也。物大然后可观,故受之以观。可观而后有所合,故受之以噬嗑;嗑者
  合也。}

{\songti 有无妄然后可畜,故受之以大畜。物畜然后可养,故受之以颐;颐者养也。不养则不
  可动,故受之以大过。物不可以终过,故受之以坎;坎者陷也。陷必有所丽,故受之以
  离;离者丽也。}

\section{通过控制壁湍流相干结构减阻}
\subsection{湍流相干结构研究进展}
壁湍流相干结构研究进展
\subsection{湍流减阻技术研究进展}
\label{chap1:sample:table}
湍流减阻技术研究进展
\subsection{基本表格}
\label{sec:basictable}

模板中关于表格的宏包有三个:\pkg{booktabs}、\pkg{array} 和 \pkg{longtabular},命
令有一个 \cs{hlinewd}。三线表可以用 \pkg{booktabs} 提供
的 \cs{toprule}、\cs{midrule} 和 \cs{bottomrule}。它们与 \pkg{longtable} 能很好的
配合使用。如果表格比较简单的话可以直接用命令 \cs{hlinewd}\marg{width} 控制。
\begin{table}[htb]
  \centering
  \begin{minipage}[t]{0.8\linewidth} % 如果想在表格中使用脚注,minipage是个不错的办法
  \caption[模板文件]{模板文件。如果表格的标题很长,那么在表格索引中就会很不美
    观,所以要像 chapter 那样在前面用中括号写一个简短的标题。这个标题会出现在索
    引中。}
  \label{tab:template-files}
    \begin{tabularx}{\linewidth}{lX}
      \toprule[1.5pt]
      {\heiti 文件名} & {\heiti 描述} \\\midrule[1pt]
      thuthesis.ins & \LaTeX{} 安装文件,\textsc{DocStrip}\footnote{表格中的脚注} \\
      thuthesis.dtx & 所有的一切都在这里面\footnote{再来一个}。\\
      thuthesis.cls & 模板类文件。\\
      thuthesis.cfg & 模板配置文。cls 和 cfg 由前两个文件生成。\\
      thuthesis-numeric.bst    & 参考文献 BIB\TeX\ 样式文件。\\
      thuthesis-author-year.bst    & 参考文献 BIB\TeX\ 样式文件。\\
      thuthesis.sty   & 常用的包和命令写在这里,减轻主文件的负担。\\
      \bottomrule[1.5pt]
    \end{tabularx}
  \end{minipage}
\end{table}

首先来看一个最简单的表格。表 \ref{tab:template-files} 列举了本模板主要文件及其功
能。请大家注意三线表中各条线对应的命令。这个例子还展示了如何在表格中正确使用脚注。
由于 \LaTeX{} 本身不支持在表格中使用 \cs{footnote},所以我们不得不将表格放在
小页中,而且最好将表格的宽度设置为小页的宽度,这样脚注看起来才更美观。

\subsection{复杂表格}
\label{sec:complicatedtable}

我们经常会在表格下方标注数据来源,或者对表格里面的条目进行解释。前面的脚注是一种
不错的方法,如果不喜欢脚注,可以在表格后面写注释,比如表~\ref{tab:tabexamp1}。
\begin{table}[htbp]
  \centering
  \caption{复杂表格示例 1}
  \label{tab:tabexamp1}
  \begin{minipage}[t]{0.8\textwidth}
    \begin{tabularx}{\linewidth}{|l|X|X|X|X|}
      \hline
      \multirow{2}*{\diagbox[width=5em]{x}{y}} & \multicolumn{2}{c|}{First Half} & \multicolumn{2}{c|}{Second Half}\\\cline{2-5}
      & 1st Qtr &2nd Qtr&3rd Qtr&4th Qtr \\ \hline
      East$^{*}$ &   20.4&   27.4&   90&     20.4 \\
      West$^{**}$ &   30.6 &   38.6 &   34.6 &  31.6 \\ \hline
    \end{tabularx}\\[2pt]
    \footnotesize 注:数据来源《\thuthesis{} 使用手册》。\\
    *:东部\\
    **:西部
  \end{minipage}
\end{table}

此外,表~\ref{tab:tabexamp1} 同时还演示了另外两个功能:1)通过 \pkg{tabularx} 的
 \texttt{|X|} 扩展实现表格自动放大;2)通过命令 \cs{diagbox} 在表头部分
插入反斜线。

为了使我们的例子更接近实际情况,我会在必要的时候插入一些“无关”文字,以免太多图
表同时出现,导致排版效果不太理想。第一个出场的当然是我的最爱:风流潇洒、骏马绝尘、
健笔凌云的{\heiti 李太白}了。

李白,字太白,陇西成纪人。凉武昭王暠九世孙。或曰山东人,或曰蜀人。白少有逸才,志
气宏放,飘然有超世之心。初隐岷山,益州长史苏颋见而异之,曰:“是子天才英特,可比
相如。”天宝初,至长安,往见贺知章。知章见其文,叹曰:“子谪仙人也。”言于明皇,
召见金銮殿,奏颂一篇。帝赐食,亲为调羹,有诏供奉翰林。白犹与酒徒饮于市,帝坐沉香
亭子,意有所感,欲得白为乐章,召入,而白已醉。左右以水颒面,稍解,援笔成文,婉丽
精切。帝爱其才,数宴见。白常侍帝,醉,使高力士脱靴。力士素贵,耻之,摘其诗以激杨
贵妃。帝欲官白,妃辄沮止。白自知不为亲近所容,恳求还山。帝赐金放还。乃浪迹江湖,
终日沉饮。永王璘都督江陵,辟为僚佐。璘谋乱,兵败,白坐长流夜郎,会赦得还。族人阳
冰为当涂令,白往依之。代宗立,以左拾遗召,而白已卒。文宗时,诏以白歌诗、裴旻剑舞、
张旭草书为三绝云。集三十卷。今编诗二十五卷。\hfill —— 《全唐诗》诗人小传

浮动体的并排放置一般有两种情况:1)二者没有关系,为两个独立的浮动体;2)二者隶属
于同一个浮动体。对表格来说并排表格既可以像图~\ref{tab:parallel1}、
图~\ref{tab:parallel2} 使用小页环境,也可以如图~\ref{tab:subtable} 使用子表格来做。
图的例子参见第~\ref{sec:multifig} 节。

\begin{table}[htbp]
\noindent\begin{minipage}{0.5\textwidth}
\centering
\caption{第一个并排子表格}
\label{tab:parallel1}
\begin{tabular}{p{2cm}p{2cm}}
\toprule[1.5pt]
111 & 222 \\\midrule[1pt]
222 & 333 \\\bottomrule[1.5pt]
\end{tabular}
\end{minipage}%
\begin{minipage}{0.5\textwidth}
\centering
\caption{第二个并排子表格}
\label{tab:parallel2}
\begin{tabular}{p{2cm}p{2cm}}
\toprule[1.5pt]
111 & 222 \\\midrule[1pt]
222 & 333 \\\bottomrule[1.5pt]
\end{tabular}
\end{minipage}
\end{table}

然后就是忧国忧民,诗家楷模杜工部了。杜甫,字子美,其先襄阳人,曾祖依艺为巩令,因
居巩。甫天宝初应进士,不第。后献《三大礼赋》,明皇奇之,召试文章,授京兆府兵曹参
军。安禄山陷京师,肃宗即位灵武,甫自贼中遁赴行在,拜左拾遗。以论救房琯,出为华州
司功参军。关辅饥乱,寓居同州同谷县,身自负薪采梠,餔糒不给。久之,召补京兆府功曹,
道阻不赴。严武镇成都,奏为参谋、检校工部员外郎,赐绯。武与甫世旧,待遇甚厚。乃于
成都浣花里种竹植树,枕江结庐,纵酒啸歌其中。武卒,甫无所依,乃之东蜀就高適。既至
而適卒。是岁,蜀帅相攻杀,蜀大扰。甫携家避乱荆楚,扁舟下峡,未维舟而江陵亦乱。乃
溯沿湘流,游衡山,寓居耒阳。卒年五十九。元和中,归葬偃师首阳山,元稹志其墓。天宝
间,甫与李白齐名,时称李杜。然元稹之言曰:“李白壮浪纵恣,摆去拘束,诚亦差肩子美
矣。至若铺陈终始,排比声韵,大或千言,次犹数百,词气豪迈,而风调清深,属对律切,
而脱弃凡近,则李尚不能历其藩翰,况堂奥乎。”白居易亦云:“杜诗贯穿古今,  尽工尽
善,殆过于李。”元、白之论如此。盖其出处劳佚,喜乐悲愤,好贤恶恶,一见之于诗。而
又以忠君忧国、伤时念乱为本旨。读其诗可以知其世,故当时谓之“诗史”。旧集诗文共六
十卷,今编诗十九卷。

\begin{table}[htbp]
\centering
\caption{并排子表格}
\label{tab:subtable}
\subcaptionbox{第一个子表格}
{
\begin{tabular}{p{2cm}p{2cm}}
\toprule[1.5pt]
111 & 222 \\\midrule[1pt]
222 & 333 \\\bottomrule[1.5pt]
\end{tabular}
}
\hskip2cm
\subcaptionbox{第二个子表格}
{
\begin{tabular}{p{2cm}p{2cm}}
\toprule[1.5pt]
111 & 222 \\\midrule[1pt]
222 & 333 \\\bottomrule[1.5pt]
\end{tabular}
}
\end{table}

不可否认 \LaTeX{} 的表格功能没有想象中的那么强大,不过只要足够认真,足够细致,
同样可以排出来非常复杂非常漂亮的表格。请参看表~\ref{tab:tabexamp2}。
\begin{table}[htbp]
  \centering\dawu[1.3]
  \caption{复杂表格示例 2}
  \label{tab:tabexamp2}
  \begin{tabular}[c]{|m{1.5cm}|c|c|c|c|c|c|}\hline
    \multicolumn{2}{|c|}{Network Topology} & \# of nodes &
    \multicolumn{3}{c|}{\# of clients} & Server \\\hline
    GT-ITM & Waxman Transit-Stub & 600 &
    \multirow{2}{2em}{2\%}&
    \multirow{2}{2em}{10\%}&
    \multirow{2}{2em}{50\%}&
    \multirow{2}{1.2in}{Max. Connectivity}\\\cline{1-3}
    \multicolumn{2}{|c|}{Inet-2.1} & 6000 & & & &\\\hline
    \multirow{2}{1.5cm}{Xue} & Rui  & Ni &\multicolumn{4}{c|}{\multirow{2}*{\thuthesis}}\\\cline{2-3}
    & \multicolumn{2}{c|}{ABCDEF} &\multicolumn{4}{c|}{} \\\hline
\end{tabular}
\end{table}

最后就是清新飘逸、文约意赅、空谷绝响的王大侠了。王维,字摩诘,河东人。工书画,与
弟缙俱有俊才。开元九年,进士擢第,调太乐丞。坐累为济州司仓参军,历右拾遗、监察御
史、左补阙、库部郎中,拜吏部郎中。天宝末,为给事中。安禄山陷两都,维为贼所得,服
药阳喑,拘于菩提寺。禄山宴凝碧池,维潜赋诗悲悼,闻于行在。贼平,陷贼官三等定罪,
特原之,责授太子中允,迁中庶子、中书舍人。复拜给事中,转尚书右丞。维以诗名盛于开
元、天宝间,宁薛诸王驸马豪贵之门,无不拂席迎之。得宋之问辋川别墅,山水绝胜,与道
友裴迪,浮舟往来,弹琴赋诗,啸咏终日。笃于奉佛,晚年长斋禅诵。一日,忽索笔作书
数纸,别弟缙及平生亲故,舍笔而卒。赠秘书监。宝应中,代宗问缙:“朕常于诸王坐闻维
乐章,今存几何?”缙集诗六卷,文四卷,表上之。敕答云,卿伯氏位列先朝,名高希代。
抗行周雅,长揖楚辞。诗家者流,时论归美。克成编录,叹息良深。殷璠谓维诗词秀调雅,
意新理惬。在泉成珠,著壁成绘。苏轼亦云:“维诗中有画,画中有诗也。”今编诗四卷。

要想用好论文模板还是得提前学习一些 \TeX/\LaTeX{}的相关知识,具备一些基本能力,掌
握一些常见技巧,否则一旦遇到问题还真是比较麻烦。我们见过很多这样的同学,一直以来
都是使用 Word 等字处理工具,以为 \LaTeX{}模板的用法也应该类似,所以就沿袭同样的思
路来对待这种所见非所得的排版工具,结果被折腾的焦头烂额,疲惫不堪。

如果您要排版的表格长度超过一页,那么推荐使用 \pkg{longtable} 或者 \pkg{supertabular}
宏包,模板对 \pkg{longtable} 进行了相应的设置,所以用起来可能简单一些。
表~\ref{tab:performance} 就是 \pkg{longtable} 的简单示例。
\begin{longtable}[c]{c*{6}{r}}
\caption{实验数据}\label{tab:performance}\\
\toprule[1.5pt]
 测试程序 & \multicolumn{1}{c}{正常运行} & \multicolumn{1}{c}{同步} & \multicolumn{1}{c}{检查点} & \multicolumn{1}{c}{卷回恢复}
& \multicolumn{1}{c}{进程迁移} & \multicolumn{1}{c}{检查点} \\
& \multicolumn{1}{c}{时间 (s)}& \multicolumn{1}{c}{时间 (s)}&
\multicolumn{1}{c}{时间 (s)}& \multicolumn{1}{c}{时间 (s)}& \multicolumn{1}{c}{
  时间 (s)}&  文件(KB)\\\midrule[1pt]
\endfirsthead
\multicolumn{7}{c}{续表~\thetable\hskip1em 实验数据}\\
\toprule[1.5pt]
 测试程序 & \multicolumn{1}{c}{正常运行} & \multicolumn{1}{c}{同步} & \multicolumn{1}{c}{检查点} & \multicolumn{1}{c}{卷回恢复}
& \multicolumn{1}{c}{进程迁移} & \multicolumn{1}{c}{检查点} \\
& \multicolumn{1}{c}{时间 (s)}& \multicolumn{1}{c}{时间 (s)}&
\multicolumn{1}{c}{时间 (s)}& \multicolumn{1}{c}{时间 (s)}& \multicolumn{1}{c}{
  时间 (s)}&  文件(KB)\\\midrule[1pt]
\endhead
\hline
\multicolumn{7}{r}{续下页}
\endfoot
\endlastfoot
CG.A.2 & 23.05 & 0.002 & 0.116 & 0.035 & 0.589 & 32491 \\
CG.A.4 & 15.06 & 0.003 & 0.067 & 0.021 & 0.351 & 18211 \\
CG.A.8 & 13.38 & 0.004 & 0.072 & 0.023 & 0.210 & 9890 \\
CG.B.2 & 867.45 & 0.002 & 0.864 & 0.232 & 3.256 & 228562 \\
CG.B.4 & 501.61 & 0.003 & 0.438 & 0.136 & 2.075 & 123862 \\
CG.B.8 & 384.65 & 0.004 & 0.457 & 0.108 & 1.235 & 63777 \\
MG.A.2 & 112.27 & 0.002 & 0.846 & 0.237 & 3.930 & 236473 \\
MG.A.4 & 59.84 & 0.003 & 0.442 & 0.128 & 2.070 & 123875 \\
MG.A.8 & 31.38 & 0.003 & 0.476 & 0.114 & 1.041 & 60627 \\
MG.B.2 & 526.28 & 0.002 & 0.821 & 0.238 & 4.176 & 236635 \\
MG.B.4 & 280.11 & 0.003 & 0.432 & 0.130 & 1.706 & 123793 \\
MG.B.8 & 148.29 & 0.003 & 0.442 & 0.116 & 0.893 & 60600 \\
LU.A.2 & 2116.54 & 0.002 & 0.110 & 0.030 & 0.532 & 28754 \\
LU.A.4 & 1102.50 & 0.002 & 0.069 & 0.017 & 0.255 & 14915 \\
LU.A.8 & 574.47 & 0.003 & 0.067 & 0.016 & 0.192 & 8655 \\
LU.B.2 & 9712.87 & 0.002 & 0.357 & 0.104 & 1.734 & 101975 \\
LU.B.4 & 4757.80 & 0.003 & 0.190 & 0.056 & 0.808 & 53522 \\
LU.B.8 & 2444.05 & 0.004 & 0.222 & 0.057 & 0.548 & 30134 \\
EP.A.2 & 123.81 & 0.002 & 0.010 & 0.003 & 0.074 & 1834 \\
EP.A.4 & 61.92 & 0.003 & 0.011 & 0.004 & 0.073 & 1743 \\
EP.A.8 & 31.06 & 0.004 & 0.017 & 0.005 & 0.073 & 1661 \\
EP.B.2 & 495.49 & 0.001 & 0.009 & 0.003 & 0.196 & 2011 \\
EP.B.4 & 247.69 & 0.002 & 0.012 & 0.004 & 0.122 & 1663 \\
EP.B.8 & 126.74 & 0.003 & 0.017 & 0.005 & 0.083 & 1656 \\
\bottomrule[1.5pt]
\end{longtable}

\subsection{其它}
\label{sec:tableother}
如果不想让某个表格或者图片出现在索引里面,请使用命令 \cs{caption*}。
这个命令不会给表格编号,也就是出来的只有标题文字而没有“表~XX”,“图~XX”,否则
索引里面序号不连续就显得不伦不类,这也是 \LaTeX{} 里星号命令默认的规则。

有这种需求的多是本科同学的英文资料翻译部分,如果觉得附录中英文原文中的表格和图
片显示成“表”和“图”不协调的话,一个很好的办法就是用 \cs{caption*},参数
随便自己写,比如不守规矩的表~1.111 和图~1.111 能满足这种特殊需要(可以参看附录部
分)。
\begin{table}[ht]
  \begin{minipage}{0.4\linewidth}
    \centering
    \caption*{表~1.111\quad 这是一个手动编号,不出现在索引中的表格。}
    \label{tab:badtabular}
      \framebox(150,50)[c]{\thuthesis}
  \end{minipage}%
  \hfill%
  \begin{minipage}{0.4\linewidth}
    \centering
    \caption*{Figure~1.111\quad 这是一个手动编号,不出现在索引中的图。}
    \label{tab:badfigure}
    \framebox(150,50)[c]{薛瑞尼}
  \end{minipage}
\end{table}

如果的确想让它编号,但又不想让它出现在索引中的话,目前模板上不支持。

最后,虽然大家不一定会独立使用小页,但是关于小页中的脚注还是有必要提一下。请看下
面的例子。

\begin{minipage}[t]{\linewidth-2\parindent}
  柳宗元,字子厚(773-819),河东(今永济县)人\footnote{山西永济水饺。},是唐代
  杰出的文学家,哲学家,同时也是一位政治改革家。与韩愈共同倡导唐代古文运动,并称
  韩柳\footnote{唐宋八大家之首二位。}。
\end{minipage}

唐朝安史之乱后,宦官专权,藩镇割据,土地兼并日渐严重,社会生产破坏严重,民不聊生。柳宗
元对这种社会现实极为不满,他积极参加了王叔文领导的“永济革新”,并成为这一
运动的中坚人物。他们革除弊政,打击权奸,触犯了宦官和官僚贵族利益,在他们的联合反
扑下,改革失败了,柳宗元被贬为永州司马。

\section{本文研究工作与主要安排}
\label{sec:theorem}

给大家演示一下各种和证明有关的环境:

\begin{assumption}
待月西厢下,迎风户半开;隔墙花影动,疑是玉人来。
\begin{eqnarray}
  \label{eq:eqnxmp}
  c & = & a^2 - b^2\\
    & = & (a+b)(a-b)
\end{eqnarray}
\end{assumption}

千辛万苦,历尽艰难,得有今日。然相从数千里,未曾哀戚。今将渡江,方图百年欢笑,如
何反起悲伤?(引自《杜十娘怒沉百宝箱》)

\begin{definition}
子曰:「道千乘之国,敬事而信,节用而爱人,使民以时。」
\end{definition}

千古第一定义!问世间、情为何物,只教生死相许?天南地北双飞客,老翅几回寒暑。欢乐趣,离别苦,就中更有痴儿女。
君应有语,渺万里层云,千山暮雪,只影向谁去?

横汾路,寂寞当年箫鼓,荒烟依旧平楚。招魂楚些何嗟及,山鬼暗谛风雨。天也妒,未信与,莺儿燕子俱黄土。
千秋万古,为留待骚人,狂歌痛饮,来访雁丘处。

\begin{proposition}
 曾子曰:「吾日三省吾身 —— 为人谋而不忠乎?与朋友交而不信乎?传不习乎?」
\end{proposition}

多么凄美的命题啊!其日牛马嘶,新妇入青庐,奄奄黄昏后,寂寂人定初,我命绝今日,
魂去尸长留,揽裙脱丝履,举身赴清池,府吏闻此事,心知长别离,徘徊庭树下,自挂东南
枝。

\begin{remark}
天不言自高,水不言自流。
\begin{gather*}
\begin{split}
\varphi(x,z)
&=z-\gamma_{10}x-\gamma_{mn}x^mz^n\\
&=z-Mr^{-1}x-Mr^{-(m+n)}x^mz^n
\end{split}\\[6pt]
\begin{align} \zeta^0&=(\xi^0)^2,\\
\zeta^1 &=\xi^0\xi^1,\\
\zeta^2 &=(\xi^1)^2,
\end{align}
\end{gather*}
\end{remark}

天尊地卑,乾坤定矣。卑高以陈,贵贱位矣。 动静有常,刚柔断矣。方以类聚,物以群分,
吉凶生矣。在天成象,在地成形,变化见矣。鼓之以雷霆,润之以风雨,日月运行,一寒一
暑,乾道成男,坤道成女。乾知大始,坤作成物。乾以易知,坤以简能。易则易知,简则易
从。易知则有亲,易从则有功。有亲则可久,有功则可大。可久则贤人之德,可大则贤人之
业。易简,而天下矣之理矣;天下之理得,而成位乎其中矣。

\begin{axiom}
两点间直线段距离最短。
\begin{align}
x&\equiv y+1\pmod{m^2}\\
x&\equiv y+1\mod{m^2}\\
x&\equiv y+1\pod{m^2}
\end{align}
\end{axiom}

《彖曰》:大哉乾元,万物资始,乃统天。云行雨施,品物流形。大明始终,六位时成,时
乘六龙以御天。乾道变化,各正性命,保合大和,乃利贞。首出庶物,万国咸宁。

《象曰》:天行健,君子以自强不息。潜龙勿用,阳在下也。见龙再田,德施普也。终日乾
乾,反复道也。或跃在渊,进无咎也。飞龙在天,大人造也。亢龙有悔,盈不可久也。用九,
天德不可为首也。   

\begin{lemma}
《猫和老鼠》是我最爱看的动画片。
\begin{multline*}%\tag*{[a]} % 这个不出现在索引中
\int_a^b\biggl\{\int_a^b[f(x)^2g(y)^2+f(y)^2g(x)^2]
 -2f(x)g(x)f(y)g(y)\,dx\biggr\}\,dy \\
 =\int_a^b\biggl\{g(y)^2\int_a^bf^2+f(y)^2
  \int_a^b g^2-2f(y)g(y)\int_a^b fg\biggr\}\,dy
\end{multline*}
\end{lemma}

行行重行行,与君生别离。相去万余里,各在天一涯。道路阻且长,会面安可知。胡马依北
风,越鸟巢南枝。相去日已远,衣带日已缓。浮云蔽白日,游子不顾返。思君令人老,岁月
忽已晚。  弃捐勿复道,努力加餐饭。

\begin{theorem}\label{the:theorem1}
犯我强汉者,虽远必诛\hfill —— 陈汤(汉)
\end{theorem}
\begin{subequations}
\begin{align}
y & = 1 \\
y & = 0
\end{align}
\end{subequations}
道可道,非常道。名可名,非常名。无名天地之始;有名万物之母。故常无,欲以观其妙;
常有,欲以观其徼。此两者,同出而异名,同谓之玄。玄之又玄,众妙之门。上善若水。水
善利万物而不争,处众人之所恶,故几于道。曲则全,枉则直,洼则盈,敝则新,少则多,
多则惑。人法地,地法天,天法道,道法自然。知人者智,自知者明。胜人者有力,自胜
者强。知足者富。强行者有志。不失其所者久。死而不亡者寿。

\begin{proof}
燕赵古称多感慨悲歌之士。董生举进士,连不得志于有司,怀抱利器,郁郁适兹土,吾
知其必有合也。董生勉乎哉?

夫以子之不遇时,苟慕义强仁者,皆爱惜焉,矧燕、赵之士出乎其性者哉!然吾尝闻
风俗与化移易,吾恶知其今不异于古所云邪?聊以吾子之行卜之也。董生勉乎哉?

吾因子有所感矣。为我吊望诸君之墓,而观于其市,复有昔时屠狗者乎?为我谢
曰:“明天子在上,可以出而仕矣!” \hfill —— 韩愈《送董邵南序》
\end{proof}

\begin{corollary}
  四川话配音的《猫和老鼠》是世界上最好看最好听最有趣的动画片。
\begin{alignat}{3}
V_i & =v_i - q_i v_j, & \qquad X_i & = x_i - q_i x_j,
 & \qquad U_i & = u_i,
 \qquad \text{for $i\ne j$;}\label{eq:B}\\
V_j & = v_j, & \qquad X_j & = x_j,
  & \qquad U_j & u_j + \sum_{i\ne j} q_i u_i.
\end{alignat}
\end{corollary}

迢迢牵牛星,皎皎河汉女。
纤纤擢素手,札札弄机杼。
终日不成章,泣涕零如雨。
河汉清且浅,相去复几许。
盈盈一水间,脉脉不得语。

\begin{example}
  大家来看这个例子。
\begin{equation}
\label{ktc}
\left\{\begin{array}{l}
\nabla f({\mbox{\boldmath $x$}}^*)-\sum\limits_{j=1}^p\lambda_j\nabla g_j({\mbox{\boldmath $x$}}^*)=0\\[0.3cm]
\lambda_jg_j({\mbox{\boldmath $x$}}^*)=0,\quad j=1,2,\cdots,p\\[0.2cm]
\lambda_j\ge 0,\quad j=1,2,\cdots,p.
\end{array}\right.
\end{equation}
\end{example}

\begin{exercise}
  请列出 Andrew S. Tanenbaum 和 W. Richard Stevens 的所有著作。
\end{exercise}

\begin{conjecture} \textit{Poincare Conjecture} If in a closed three-dimensional
  space, any closed curves can shrink to a point continuously, this space can be
  deformed to a sphere.
\end{conjecture}

\begin{problem}
 回答还是不回答,是个问题。
\end{problem}

如何引用定理~\ref{the:theorem1} 呢?加上 \cs{label} 使用 \cs{ref} 即可。妾发
初覆额,折花门前剧。郎骑竹马来,绕床弄青梅。同居长干里,两小无嫌猜。 十四为君妇,
羞颜未尝开。低头向暗壁,千唤不一回。十五始展眉,愿同尘与灰。常存抱柱信,岂上望夫
台。 十六君远行,瞿塘滟滪堆。五月不可触,猿声天上哀。门前迟行迹,一一生绿苔。苔深
不能扫,落叶秋风早。八月蝴蝶来,双飞西园草。感此伤妾心,坐愁红颜老。

\section{参考文献}
\label{sec:bib}
当然参考文献可以直接写 \cs{bibitem},虽然费点功夫,但是好控制,各种格式可以自己随意改
写。

本模板推荐使用 BIB\TeX,分别提供数字引用(\texttt{thuthesis-numeric.bst})和作
者年份引用(\texttt{thuthesis-author-year.bst})样式,基本符合学校的参考文献格式
(如专利等引用未加详细测试)。看看这个例子,关于书的~\cite{tex, companion,
  ColdSources},还有这些~\cite{Krasnogor2004e, clzs, zjsw},关于杂志
的~\cite{ELIDRISSI94, MELLINGER96, SHELL02},硕士论文~\cite{zhubajie,
  metamori2004},博士论文~\cite{shaheshang, FistSystem01},标准文
件~\cite{IEEE-1363},会议论文~\cite{DPMG,kocher99},技术报告~\cite{NPB2},电子文
献~\cite{chuban2001,oclc2000}。中文参考文献~\cite{cnarticle}应增
加 \texttt{lang=``zh''} 字段,以便进行相应处理。另外,本模板对中文文
献~\cite{cnproceed}的支持并不是十全十美,如果有不如意的地方,请手动修
改 \texttt{bbl} 文件。

有时候不想要上标,那么可以这样~\inlinecite{shaheshang},这个非常重要。

有时候一些参考文献没有纸质出处,需要标注 URL。缺省情况下,URL 不会在连字符处断行,
这可能使得用连字符代替空格的网址分行很难看。如果需要,可以将模板类文件中
\begin{verbatim}
\RequirePackage{hyperref}
\end{verbatim}
一行改为:
\begin{verbatim}
\PassOptionsToPackage{hyphens}{url}
\RequirePackage{hyperref}
\end{verbatim}
使得连字符处可以断行。更多设置可以参考 \texttt{url} 宏包文档。

\section{公式}
\label{sec:equation}
贝叶斯公式如式~(\ref{equ:chap1:bayes}),其中 $p(y|\mathbf{x})$ 为后验;
$p(\mathbf{x})$ 为先验;分母 $p(\mathbf{x})$ 为归一化因子。
\begin{equation}
\label{equ:chap1:bayes}
p(y|\mathbf{x}) = \frac{p(\mathbf{x},y)}{p(\mathbf{x})}=
\frac{p(\mathbf{x}|y)p(y)}{p(\mathbf{x})}
\end{equation}

论文里面公式越多,\TeX{} 就越 happy。再看一个 \pkg{amsmath} 的例子:
\newcommand{\envert}[1]{\left\lvert#1\right\rvert}
\begin{equation}\label{detK2}
\det\mathbf{K}(t=1,t_1,\dots,t_n)=\sum_{I\in\mathbf{n}}(-1)^{\envert{I}}
\prod_{i\in I}t_i\prod_{j\in I}(D_j+\lambda_jt_j)\det\mathbf{A}
^{(\lambda)}(\overline{I}|\overline{I})=0.
\end{equation}

前面定理示例部分列举了很多公式环境,可以说把常见的情况都覆盖了,大家在写公式的时
候一定要好好看 \pkg{amsmath} 的文档,并参考模板中的用法:
\begin{multline*}%\tag{[b]} % 这个出现在索引中的
\int_a^b\biggl\{\int_a^b[f(x)^2g(y)^2+f(y)^2g(x)^2]
 -2f(x)g(x)f(y)g(y)\,dx\biggr\}\,dy \\
 =\int_a^b\biggl\{g(y)^2\int_a^bf^2+f(y)^2
  \int_a^b g^2-2f(y)g(y)\int_a^b fg\biggr\}\,dy
\end{multline*}

其实还可以看看这个多级规划:
\begin{equation}\label{bilevel}
\left\{\begin{array}{l}
\max\limits_{{\mbox{\footnotesize\boldmath $x$}}} F(x,y_1^*,y_2^*,\cdots,y_m^*)\\[0.2cm]
\mbox{subject to:}\\[0.1cm]
\qquad G(x)\le 0\\[0.1cm]
\qquad(y_1^*,y_2^*,\cdots,y_m^*)\mbox{ solves problems }(i=1,2,\cdots,m)\\[0.1cm]
\qquad\left\{\begin{array}{l}
    \max\limits_{{\mbox{\footnotesize\boldmath $y_i$}}}f_i(x,y_1,y_2,\cdots,y_m)\\[0.2cm]
    \mbox{subject to:}\\[0.1cm]
    \qquad g_i(x,y_1,y_2,\cdots,y_m)\le 0.
    \end{array}\right.
\end{array}\right.
\end{equation}
这些跟规划相关的公式都来自于刘宝碇老师《不确定规划》的课件。

\chapter{理论公式与数值求解方法}
\label{cha:2}
\section{等离子体模型}
\section{流动稳定性求解框架}
在研究等离子体控制扰动的问题中,本文采用研究稳定性问题的相关数值方法,来解析控制前后边界层内扰动的发展情况。稳定性的相关理论虽然并不能给出精确的转捩位置,但是能够从理论方面给出流动失稳特性,并且具有计算效率高,转捩前流动解析精度高的优点。本文中通过研究流动在控制前后稳定性方面的特性变化,来甄别控制是否有效。本文采用的研究边界层流动稳定性的步骤如下:
\begin{enumerate}
  \item 采用高精度有限元程序求解无粘流场;
  \item 以无粘流壁面上的流动参数作为边界层方程的边界条件,求解层流基本流动;
  \item 基于线性稳定性理论,判断主导转捩的模态;
  \item 采用抛物化扰动方程,求解边界层内扰动的演化;
  \item 以受扰流场作为新的基本流动进行二次失稳分析。
\end{enumerate}

与前人所做的研究不同的是,本文的研究需要将等离子体产生的体积力也考虑进来。在详细介绍求解方法之前,这里先简要介绍一下本文中对体积力的处理方法。流动所满足的控制方程(N-S方程)为:
\begin{equation}\label{EQ_NS}\left.
\begin{aligned}
    \frac{\p{\rho^*}}{\p{t^*}}
    + {\nabla^*}\cdot\left({\rho^*}{\mathbf{V}^*}\right) & =0
    \\
    {\rho^*}\left( {\frac{\p{\mathbf{V}^*}}{\p{t^*}}
    + \left( {{\mathbf{V}^*}\cdot{\nabla^*}}\right)
    {\mathbf{V}^*}} \right) & =
     - {\nabla^*}{p^*} + {\nabla^*}\left( {{\lambda ^*}\left( {{\nabla ^*} \cdot {\mathbf{V}^*}} \right)} \right) \\
    & + {\nabla ^*} \cdot \left( {{\mu ^*}\left( {{\nabla ^*}{\mathbf{V}^*} + {\nabla ^*}{\mathbf{V}^*}^T} \right)} \right) + \mathbf{f}^*
    \\
    {\rho ^*}{C_p}^*\left( {\frac{{\p{T^*}}}{{\p{t^*}}} + \left( {{{\mathbf{V}}^*} \cdot {\nabla ^*}} \right){T^*}} \right) & =
    {\nabla ^*} \cdot \left( {{\kappa ^*}{\nabla ^*}{T^*}} \right) \\
    & + \frac{{\p{p^*}}}{{\p{t^*}}} + \left( {{{\mathbf{V}}^*} \cdot {\nabla ^*}} \right){p^*} + {\Phi ^*} + \mathbf{V}^* \cdot \mathbf{f}^*
\end{aligned}~\right\}
\end{equation}

\noindent式(\ref{EQ_NS})中能量方程的耗散函数为:
\begin{equation}
    {\Phi ^*} = {\lambda ^*}{{\left( {{\nabla ^*} \cdot {{{\mathbf{V}}}^*}} \right)}^2} + \frac{{{\mu ^*}}}{2}{{\left( {{\nabla ^*}{{{\mathbf{V}}}^*} + {\nabla ^*}{{{\mathbf{V}}}^*}^T} \right)}^2}
\end{equation}
方程中星号 $*$ 表示有量纲量,${\mathbf{V}}$ 表示速度矢量,其在$x,y,z$三个方向的分量为 $u$,$v$,$w$ 。${\mathbf{f}}$ 表示体积力矢量,其分量分别为 $f_x$,$f_y$,$f_z$ 。

为封闭 N-S 方程,分别引入状态方程、Sutherland 粘性律、Stokes 假设,假定流体是量热完全气体并具有恒定的 $Pr$ 数:
\begin{equation}\left.
\begin{aligned}
p^*=\rho^*R^*T^* & \Leftrightarrow p=\frac{\rho T}{\gamma Ma^2} \\
\mu^*=\mu_s^*\frac{T^*}{T^*_s}\frac{T^*_s+S^*}{T^*+S^*} & \Leftrightarrow \mu=\mu_s\frac{T}{T_s}\frac{T_s+S}{T+S} \\
\lambda^*+2/3\mu^*=0 & \Leftrightarrow\lambda=-2/3\mu\\
\textrm{Pr}=\frac{{C_p}^*\mu^*}{\kappa^*}={const} & \Leftrightarrow\mu=\kappa\\
C_p^*=const,~ & R^*=const
\end{aligned}~\right\}
\end{equation}
Sutherland 粘性律中 $T^*_s=273K,~\mu_s^*=1.71\times10^{-5}kg/(m\cdot s),~S^*=110.4K$。

选取适当的参考长度 $l_{reff}$、参考速度 $U_{reff}$、参考密度 $\rho_{reff}$等特征量,可以对式(\ref{EQ_NS})进行无量纲化。在本文中,分别研究了后掠Hiemenz流动和后掠翼流动。在这两个流动中我们选择的特征量是不一样的,之后我们会分别介绍。为了简洁,我们将无量纲化后的N-S方程记为:
\begin{equation}
    \label{e:NS}
    \mathscr{N}(\mathbf{q})=\mathbf{F}
\end{equation}
这里, $\mathbf{q}=(\rho , u,v,w,T)^T$,即原始变量组成的5维矢量。$\mathbf{F}=(0,f_x,f_y,f_z,\mathbf{V} \cdot \mathbf{f})^T$。 上标 ``$T$" 表示转置。这里由于添加的体积力很小,我们假设其只影响扰动发展,并不影响基本流。即基本流依然满足N-S方程:
\begin{equation}
    \label{e:baseflow}
    \mathscr{N}(\mathbf{q_0})=0
\end{equation}
$\mathbf{q_0}$ 为基本流流动原始变量组成的矢量,其与$\mathbf{q}$的差即为扰动量$\mathbf{\tilde{q}}$。令式(\ref{e:NS}) - 式(\ref{e:baseflow}),即可得到扰动的控制方程:
\begin{equation}
    \label{e:disturbance1}
    \mathscr{S}(\mathbf{\tilde{q}})=\mathscr{N}(\mathbf{q_0}+\mathbf{\tilde{q}})-\mathscr{N}(\mathbf{q_0})=\mathbf{F}
\end{equation}
在之后的小节\ref{subsec:BLfun}和\ref{subsec:STBfun}中,将会分别介绍式(\ref{e:baseflow})和(\ref{e:disturbance1})所采用的求解方法。至于求解步骤1中提到的高精度有限元方法,将会在\ref{sec:DNS}节中介绍。
\subsection{边界层方程}\label{subsec:BLfun}
在边界层流动中,流向的特征尺度为常规尺度,而法向的特征尺度为边界层厚度尺度。利用这一特性,可将Navier-Stokes 方程式抛物化,得到层流边界层控制方程。本文研究的问题的基本流均满足展向均匀假设,即${\p}/{\p z^*}=0$ 。利用这些假设,式(\ref{e:baseflow})可以写为\footnote{本节讨论的均为基本流的计算方法,为了简洁,表明基本流变量的下标0在本节中都被略去。即原本的$\rho_0,u_0,v_0,w_0,T_0$在本节被记为$\rho,u,v,w,T$。有量纲量类似。}:
\begin{subequations}
\begin{align}
\frac{{\partial \left( {\rho^* u^*} \right)}}{{\partial x^* }} + \frac{{\partial \left( {\rho^* v^*} \right)}}{{\partial y^* }} &= 0 \\
\rho ^* u^* \frac{{\partial u^* }}{{\partial x^* }} + \rho ^* v^* \frac{{\partial u^* }}{{\partial y^* }} &=  - \frac{{\partial p^* }}{{\partial x^* }} + \frac{\partial }{{\partial y^* }}\left( {\mu ^* \frac{{\partial u^* }}{{\partial y^* }}} \right)\\
\rho ^* u^* \frac{{\partial w^* }}{{\partial x^* }} + \rho ^* v^* \frac{{\partial w^* }}{{\partial y^* }} &= \frac{\partial }{{\partial y^* }}\left( {\mu ^* \frac{{\partial w^* }}{{\partial y^* }}} \right)\\
\frac{{\partial p^* }}{{\partial y^* }} &= 0\\
\rho ^* u^* C_p^* \frac{{\partial T^* }}{{\partial x^* }} + \rho ^* v^* C_p^* \frac{{\partial T^* }}{{\partial y^* }} &= \frac{\partial }{{\partial y^* }}\left( {k^* \frac{{\partial T^* }}{{\partial y^* }}} \right) + u^* \frac{{\partial p^* }}{{\partial x^* }} + \mu ^* \left( {\frac{{\partial u^* }}{{\partial y^* }}} \right)^2  + \mu ^* \left( {\frac{{\partial w^* }}{{\partial y^* }}} \right)^2
\end{align}
\end{subequations}

在传统的边界层方程求解方法中,所有物理量都采用相同的参考量进行无量纲化,比如一般会采用来流的速度、密度等物理量进行无量纲化。然而,在本文研究的问题中,边界层外普遍有较大的压力梯度,这导致不同流向位置的边界层外物理量差异比较大,计算很难收敛。所以本文采用当地边界层外的物理量,即$U_e^*,T_e^*,\rho_e^*,k_e^*,\mu_e^*$,进行无量纲化,提高计算稳定性。这里边界层外的物理量是通过求解无粘流方程得到的,并作为边界层方程求解的边界条件。采用当地边界层外物理量无量纲化后的边界层方程为:
\begin{subequations}
\begin{equation}\label{e:BLE6}
  \frac{{\partial \left( {\rho u} \right)}}{{\partial x^* }} + \frac{{\partial \left( {\rho v} \right)}}{{\partial y^* }} + \frac{{\rho u}}{{\rho _e^* U_e^* }}\frac{{\partial \left( {\rho _e^* U_e^* } \right)}}{{\partial x^* }} = 0
\end{equation}
\begin{equation}\label{e:BLE7}
  \rho u\rho _e^* U_e^* U_e^* \frac{{\partial u}}{{\partial x^* }} + \rho uu\rho _e^* U_e^* \frac{{\partial U_e^* }}{{\partial x^* }} + \rho v\rho _e^* U_e^* U_e^* \frac{{\partial u}}{{\partial y^* }} = \rho _e^* U_e^* \frac{{dU_e^* }}{{dx^* }} + \mu _e^* U_e^* \frac{\partial }{{\partial y^* }}\left( {\mu \frac{{\partial u}}{{\partial y^* }}} \right)
\end{equation}
\begin{equation}\label{e:BLE8}
  \rho u\frac{{\partial w}}{{\partial x^* }} + \rho v\frac{{\partial w}}{{\partial y^* }} = \frac{{\mu _e^* }}{{\rho _e^* U_e^* }}\frac{\partial }{{\partial y^* }}\left( {\mu \frac{{\partial w}}{{\partial y^* }}} \right)
\end{equation}
\begin{multline}\label{e:BLE9}
    \rho u\rho _e^* U_e^* C_p^* \left( {T\frac{{\partial T_e^* }}{{\partial x^* }} + T_e^* \frac{{\partial T}}{{\partial x^* }}} \right) + \rho v\rho _e^* U_e^* C_p^* T_e^* \frac{{\partial T}}{{\partial y^* }} \\
    = k_e^* T_e^* \frac{\partial }{{\partial y^* }}\left( {k\frac{{\partial T}}{{\partial y^* }}} \right) - \rho _e^* U_e^* U_e^* \frac{{dU_e^* }}{{x^* }}u + \mu \mu _e^* U_e^* U_e^* \left( {\frac{{\partial u}}{{\partial y^* }}} \right)^2  + \mu \mu _e^* W_e^* W_e^* \left( {\frac{{\partial w}}{{\partial y^* }}} \right)^2
\end{multline}
\end{subequations}
注意到在上面的代换中,还用到了无粘势流中沿流线的伯努利方程:
\begin{equation}
  -\frac{\p p^*}{\p x^*}=\rho^*u^*_e\frac{du^*_e}{dx^*_e}
\end{equation}
和气体状态方程:
\begin{equation}
  \rho T=1
\end{equation}
为了消除上述边界层方程在驻点处的奇异性,引入如下相似变换:
\begin{subequations}
\begin{align}
  \xi  &= x^* \\
  \eta &= \sqrt{\frac{U_e^*}{x^*\rho_e^*\mu_e^*}}\int_{0}^{y^*}\rho^*dy^*
  =\frac{1}{L^*}\int_{0}^{y^*}T^{-1}dy^*
\end{align}
\end{subequations}
最终得到如下计算求解的方程:
\begin{subequations}\label{e:ble10}
\begin{equation}
  \xi\frac{\p u}{\p \xi}+\frac{\p \Lambda}{\p \eta}+\frac{u}{2}\left[ 1+\frac{\xi}{\mu_e^*}\frac{\p \mu_e^*}{\p \xi} + \frac{\xi}{\rho_e^*\mu_e^*}\frac{\p (\rho_e^*\mu_e^*)}{\p \xi}\right]=0
\end{equation}
\begin{equation}
  \xi u\frac{\p u}{\p \xi}+\Lambda\frac{\p u}{\p \eta} -\frac{\xi}{\mu_e^*}\frac{\p \mu_e^*}{\p \xi}(T-u^2)=\frac{\p}{\p\eta}(\frac{\mu}{T}\frac{\p u}{\p\eta})
\end{equation}
\begin{equation}
  \xi u\frac{\p w}{\p \xi}+\Lambda\frac{\p w}{\p \eta}=\frac{\p}{\p\eta}(\frac{\mu}{T}\frac{\p w}{\p\eta})
\end{equation}
\begin{equation}
  \xi u\frac{\p T}{\p \xi}+\Lambda\frac{\p T}{\p \eta} - \frac{1}{\rm Pr}\frac{\p}{\p\eta}(\frac{k}{T}\frac{\p T}{\p\eta})=(\gamma-1)\frac{\mu}{T}\left[ ({\rm Ma}_{ue}\frac{\p u}{\p \eta})^2 + ({\rm Ma}_{we}\frac{\p w}{\p \eta})^2  \right]
\end{equation}
\end{subequations}
其中:
\begin{subequations}
  \begin{align}
    L^* &= \sqrt{\frac{\mu_e^*x^*}{\rho_e^*u_e^*}} \\
    \Lambda &= \xi u\frac{\p \eta}{\p x^*}+\frac{\xi \nu}{L^*T} \\
    {\rm Ma}_{ue} &= \frac{u_e^*}{a_e^*} \\
    {\rm Ma}_{we} &= \frac{w_e^*}{a_e^*} \\
    a_e^* &= \sqrt{\gamma RT_e^*}
  \end{align}
\end{subequations}
将方程(\ref{e:ble10})在法方向采用谱方法进行离散,流向采用五阶差分格式,最后得到离散的方程简记为:
\begin{equation}\label{e:ble_dis}
  L_{dis}(\Phi)=0
\end{equation}
$\Phi=(u,w,\Lambda,T)^T$为方程(\ref{e:ble10})中实际求解的变量组成的矩阵。上式对应的Jacobian矩阵为:
\begin{equation}\label{e:jb}
  \mathbf{J}_b = \frac{\p L_{dis}(\Phi)}{\p \Phi}
\end{equation}
本文采用拟牛顿法对式(\ref{e:ble_dis})进行求解,迭代更新方法如下:
\begin{equation}\label{e:ble_iter}
  \Phi_{\rm new} = \Phi_{\rm old} - \mathbf{J}_b^{-1}L_{dis}(\Phi)
\end{equation}
式(\ref{e:ble_dis})和(\ref{e:jb})的具体形式将在附录\ref{app:ble}中给出。

为了验证程序是否正确,首先将计算结果与零压力梯度平板上的相似性解进行对比。这里采用的计算工况为:
\begin{equation}\label{}
  U_\infty  = 100{\rm m/s},T_\infty  = 300{\rm K},\nu_\infty  = 1.5 \times 10^{ - 5} {\rm m^2 /s}
\end{equation}
对比$x=1$m,即$\Rey_x=6.67\times10^6$,位置处各个物理量延法向的分布如图\ref{f:0PreassureGradienPlate}所示。其中黑色由方框标记的线为边界层方程求解出来的结果,红色由三角标记出来的先为相似性解的结果。可以看到两种算法的结果几乎完全重合了。
\begin{figure}[h]
  \centering%

  \begin{subfigure}{0.5\textwidth}
    \includegraphics[width=\textwidth]{ch2/0PreassureGradienPlate_u.jpg}
    \caption{流向速度对比}
  \end{subfigure}%
  \begin{subfigure}{0.5\textwidth}
    \includegraphics[width=\textwidth]{ch2/0PreassureGradienPlate_v.jpg}
    \caption{法向速度对比}
  \end{subfigure}%
  \bigskip

  \begin{subfigure}{0.5\textwidth}
    \includegraphics[width=\textwidth]{ch2/0PreassureGradienPlate_T.jpg}
    \caption{温度对比}
  \end{subfigure}%
  \caption{边界层方程计算结果与相似性解对比(黑线方框标记:边界层方程计算结果;红线三角标记:相似性解)}
  \label{f:0PreassureGradienPlate}
\end{figure}

本文中主要进行的是三维边界层失稳的研究,所以针对三维边界层的计算也需要验证。清华大学徐胜金老师课题组为研究三维边界层转捩在低湍流度风洞中做了后掠NLF-0415翼型的绕流实验。实验相应参数可以参考文献??????。在实验自由来流为22.3m/s的工况中,翼型上表面直至70\%弦长处均为层流。采用边界层方程计算速度分布,并取40\%和60\%弦长处的速度剖面与实验对比,结果如图\ref{f:ble_vs_exp}。其中计算结果用线表示,实验结果用点表示。这里$U_{\rm wt}$表示延风洞方向的速度分量\footnote{注意这里并不是$u$,因为在后掠翼计算中,x方向与平行于风洞的流向有45$^\circ$夹角}。蓝色表示20\%弦长处的结果,红色为40\%处。从计算的结果可以看到,我们所采用的求解方法完全满足精度需求。
\begin{figure}[h]
  \centering
  \includegraphics[width=0.7\textwidth]{ch2/compare_profiles.jpg}
  \caption{后掠翼上边界层速度剖面对比(线:计算结果;点:实验结果)}\label{f:ble_vs_exp}
\end{figure}

\subsection{扰动方程}\label{subsec:STBfun}
如之前所述,本文将流场基本变量 $\mathbf{q}=(\rho,~u,~v,~w,~T)$ 分解为基本流动 $\mathbf{q}_0$ 和扰动 $\tilde{\mathbf{q}}$ 两部分:
\begin{equation}\label{EQ_STa}
\mathbf{q}(x,y,z,t)=\mathbf{q}_0(x,y)
+\tilde{\mathbf{q}}(x,y,z,t)
\end{equation}
在小节\ref{subsec:BLfun}中已经探讨了基本流动的求解方法。在这一节中,重点讨论扰动方程(\ref{e:disturbance1})的求解方法。先假设方程(\ref{e:disturbance1})可以写成如下紧凑的形式:
\begin{multline}
 \label{e:EQ_ST}
 {\mathbf{\Gamma }}\frac{{\partial {\mathbf{\tilde q}}}}
 {{\partial t}} + {\mathbf{A}}\frac{{\partial {\mathbf{\tilde q}}}}
 {{\partial x}} + {\mathbf{B}}\frac{{\partial {\mathbf{\tilde q}}}}
 {{\partial y}} + {\mathbf{C}}\frac{{\partial {\mathbf{\tilde q}}}}
 {{\partial z}} + {\mathbf{D\tilde q}}\\ = {\mathbf{H}}_{xx} \frac{{\partial ^2 {\mathbf{\tilde q}}}}
 {{\partial x^2 }} + {\mathbf{H}}_{yz} \frac{{\partial ^2 {\mathbf{\tilde q}}}}
 {{\partial z\,\partial y}} + {\mathbf{H}}_{xy} \frac{{\partial ^2 {\mathbf{\tilde q}}}}
 {{\partial x\,\partial y}} + {\mathbf{H}}_{xz} \frac{{\partial ^2 {\mathbf{\tilde q}}}}
 {{\partial x\,\partial z}} + {\mathbf{H}}_{yy} \frac{{\partial ^2 {\mathbf{\tilde q}}}}
 {{\partial y^2 }} + {\mathbf{H}}_{zz} \frac{{\partial ^2 {\mathbf{\tilde q}}}}
 {{\partial z^2 }} + {\mathbf{N}} + {\mathbf{F}}.
\end{multline}
其中 $5\times5$ 系数矩阵 $\mathbf{\Gamma},~\mathbf{A},~\mathbf{B},~
\mathbf{C},~\mathbf{D},~\mathbf{H}_{xx},~
\mathbf{H}_{yy},~\mathbf{H}_{zz},~\mathbf{H}_{xy},~
\mathbf{H}_{xz},~\mathbf{H}_{yz}$ 是基本流动、流向曲率和 $\Rey,~\Ma,~\Prl$ 的函数,详细表达式可参见附录\ref{app:SE}。向量 $\mathbf{{N}}$ 表示非线性项,$\mathbf{F}$表示体积力产生的源项。


\subsubsection{线性稳定性理论}
由于边界层流动中,边界层厚度增长缓慢,所以可将其近似为平行剪切流。假设扰动具有行波解:
\begin{equation}\label{EQ_LST0}
    \tilde{\mathbf{q}}(x,y,z,t)=
    \hat{\mathbf{q}}(y)\exp\left(\ii (\alpha x+\beta z-\omega t)\right)+c.c.
\end{equation}
针对边界层失稳问题,其不稳定性通常是对流失稳,即边界层内的扰动并不是在原地增长,而是一边向下游传播一遍增长。针对这一类问题,通常采用空间模式求解,即给定 $\beta$ 和 $\omega$,求解$\alpha$。将式(\ref{EQ_LST0})代入扰动方程( \ref{e:EQ_ST}), 忽略非线性项整理得到
\begin{equation}\label{EQ_LST1}
\mathbf{A}_L\hat{\mathbf{q}}+
\mathbf{B}_L\frac{\p\hat{\mathbf{q}}}{\p y}
-\mathbf{H}_{yy}\frac{\p^2 \hat{\mathbf{q}}}{\p y^2}=
\alpha\left(
\mathbf{M}_L\hat{\mathbf{q}}+
\ii \mathbf{H}_{xy}\frac{\p\hat{\mathbf{q}}}{\p y}\right)
-\alpha^2\mathbf{H}_{xz}\hat{\mathbf{q}}
\end{equation}
其中
\begin{equation}\left.
\begin{aligned}
\mathbf{A}_L&= -\ii\omega\mathbf{\Gamma}
         +\ii\beta\mathbf{C}
         +\mathbf{D}
         +\beta^2\mathbf{H}_{zz}\\
\mathbf{B}_L&= \mathbf{B}-\ii\beta\mathbf{H}_{yz}\\
\mathbf{M}_L&= -\ii\mathbf{A}-\beta\mathbf{H}_{xz}
\end{aligned}~\right\}
\end{equation}
将上式中的几个微分算子记作:
\begin{subequations}\label{e:LST_short}
\begin{align}
  \mathscr{L}_0 &=\mathbf{A}_L+
  \mathbf{B}_L\frac{\p}{\p y}
  -\mathbf{H}_{yy}\frac{\p^2 }{\p y^2} \\
  \mathscr{L}_1 &=-\mathbf{M}_L-\ii \mathbf{H}_{xy}\frac{\p}{\p y}\\
  \mathscr{L}_2 &=\mathbf{H}_{xz}
\end{align}
\end{subequations}
则线性稳定性的控制方程可以写为:
\begin{equation}\label{e:LST}
  \mathscr{L}\hat{\mathbf{q}}=\mathscr{L}_0\hat{\mathbf{q}}+\alpha \mathscr{L}_1\hat{\mathbf{q}} + \alpha^2\mathscr{L}_2\hat{\mathbf{q}}=0
\end{equation}
引入一个辅助变量:
\begin{equation}\label{}
  \tilde{\mathbf{q}}_a = \alpha\tilde{\mathbf{q}}
\end{equation}
则式(\ref{e:LST})可以改写为:
\begin{equation}\label{}
  \left(
  \begin{array}{cc}
    0 & 1 \\
    \mathscr{L}_0 & \mathscr{L}_1
  \end{array}
  \right)
  \left(
  \begin{array}{c}
    \tilde{\mathbf{q}} \\
    \tilde{\mathbf{q}}_a
  \end{array}
  \right)
  =\alpha
  \left(
  \begin{array}{cc}
    1 & 0 \\
    0 & -\mathscr{L}_2
  \end{array}
  \right)
  \left(
  \begin{array}{c}
    \tilde{\mathbf{q}} \\
    \tilde{\mathbf{q}}_a
  \end{array}
  \right)
\end{equation}
很显然,式(\ref{EQ_LST1})是针对微分算子的广义特征值问题。对其进行离散求解,在法方向采用四阶精度中心差分格式:
\begin{equation}\label{EQ_CF}\left.
\begin{aligned}
    \frac{\p{\hat{\mathbf{q}}_j}}{\p y} &= \frac{{
    {\hat{\mathbf{q}}_{j - 2}}
    - 8{\hat{\mathbf{q}}_{j - 1}}
    + 8{\hat{\mathbf{q}}_{j + 1}}
    - {\hat{\mathbf{q}}_{j + 2}}}}{{12\Delta y}}\\
    \frac{{{\partial ^2}{\hat{\mathbf{q}}_j}}}{{\partial {y^2}}} &= \frac{{
    - {\hat{\mathbf{q}}_{j - 2}}
    + 16{\hat{\mathbf{q}}_{j - 1}}
    - 30{\hat{\mathbf{q}}_j}
    + 16{\hat{\mathbf{q}}_{j + 1}}
    - {\hat{\mathbf{q}}_{j + 2}}}}{{12{{\left( {\Delta y} \right)}^2}}}
\end{aligned}~\right\}
\end{equation}
便可以将这一个微分算子的广义特征值问题转化为矩阵的广义特征值问题。求解该特征值问题,得到特征向量 $\hat{\mathbf{q}}$ 即为扰动分布,特征值 $\alpha$ 虚部 $-\alpha_i$ 为扰动增长率,实部 $\alpha_r$ 为扰动流向波数。


\subsubsection{抛物化扰动方程}
线性稳定性理论有两个缺陷。首先,其采用平行流假设,导致边界层延流向的变化被忽略了。另外,线性假设忽略了非线性项,导致不同模态间的相互作用没有被考虑。抛物化扰动方程(PSE)可以克服上述 这两点缺陷,并且具有很高的求解效率。首先将物理扰动 $\mathbf{\tilde{q}}$ 和非线性项与外加源项之和 $\mathbf{{N}}+\mathbf{{F}}$ 进行 Fourier 展开:
\begin{align}
\label{e:Fourier1}
    {\mathbf{\tilde q}}\left( {x,y,z,t} \right)& = \sum\limits_{m =  - M}^M {\sum\limits_{n =  - N}^N {{\mathbf{\hat q}}_{mn} \left( {x,y} \right)\Theta _{mn} } },\\
%\end{equation}
%\begin{equation}
\label{e:Fourier2}
    {\mathbf{N}} + {\mathbf{F}} &= \sum\limits_{m =  - M}^M {\sum\limits_{n =  - N}^N {{\mathbf{S}}_{mn} \left( {x,y} \right)\Theta _{mn} } },\\
%\end{equation}
%\begin{equation}
\label{e:Fourier3}
    \Theta _{mn}  &= \exp \!\left( {i\int_{x_0 }^x {\alpha _{mn} \left( \xi  \right)d\xi }  + in\beta z - im\omega t} \right).
\end{align}
其中$\Theta _{mn}$是波数函数。代入扰动方程(\ref{e:EQ_ST}),整理得到
\begin{equation}
\label{e:unPSE}
    {\mathbf{\hat A}}\frac{{\partial {\mathbf{\hat q}}_{mn} }}{{\partial x}}
  + {\mathbf{\hat B}}\frac{{\partial {\mathbf{\hat q}}_{mn} }}{{\partial y}}
  + {\mathbf{\hat C}}\frac{{\partial^2 {\mathbf{\hat q}}_{mn} }}{{\partial x^2}}
  + {\mathbf{\hat D\hat q}}_{mn}
  - {\mathbf{H}}_{yy}\frac{{\partial ^2 {\mathbf{\hat q}}_{mn} }}{{\partial y^2 }}
  = {\mathbf{S}}_{mn},
\end{equation}
其中
\begin{equation}
\begin{aligned}
  \mathbf{\hat A} & = {\mathbf{A}} - 2i\alpha_{mn}{\mathbf{H}}_{xx} - in\beta{\mathbf{H}}_{xz}  , \\
  \mathbf{\hat B} & = {\mathbf{B}} -  i\alpha_{mn}{\mathbf{H}}_{xy} - in\beta{\mathbf{H}}_{yz}   , \\
  \mathbf{\hat C} & = {\mathbf{H}}_{xx} , \\
  \mathbf{\hat D}  &= {\mathbf{D}} - im\omega {\mathbf{\Gamma }} + i\alpha_{mn} {\mathbf{A}} + in\beta {\mathbf{C}} + {\mathbf{H}}_{xx} \left( {\alpha_{mn}^2  - i\frac{{d\alpha }}{dx}} \right) + n^2\beta ^2 {\mathbf{H}}_{zz} + n\beta\alpha_{mn}{\mathbf{H}}_{xz} . \\
\end{aligned}
\end{equation}
根据量级分析\cite{Malik1999}, ${{d\alpha }}/{dx}$ 这一项非常小可以忽略。为了使得形函数$\mathbf{\hat q}$在流向缓变,提出针对 $\alpha$的波数迭代条件:
\begin{equation}
\label{e:auxiliary}
    \int_0^\infty  {{\mathbf{\hat q}}^H {\mathbf{M}}\frac{{\partial {\mathbf{\hat q}}}}{{\partial x}}\,dy}  = 0\qquad\forall x.
\end{equation}
这里$\mathbf{M}=\mathrm{diag}(0,1,1,1,0)$, ``$H$''表示复共轭。式(\ref{e:auxiliary})又可以叫做形函数的缓变条件,这一条件使得形函数在流向的二阶偏导数可以被忽略掉,即${{\partial ^2 {\mathbf{\hat q}}_{mn} }}/{{\partial x^2 }}=0$\cite{Malik1994}。虽然二阶偏导数项被忽略掉了,但是方程(\ref{e:unPSE})依然有一些残余椭圆性\cite{LiMalik1996}。针对这一问题,将方程中的压力项修正为:
\begin{equation}
    \frac{\partial \tilde p_{mn}}{\partial x} = i\alpha_{mn}\hat p_{mn}\Theta_{mn}.
\end{equation}
采用上面所提到的诸多假设,方程(\ref{e:unPSE})可以完全被抛物化,可以流向推进求解。完整的方程为:
\begin{equation}
\label{PSE1}
    \mathscr{L}_{\rm PSE}{\mathbf{\hat q}}_{mn}  = {\mathbf{\hat A}}\frac{{\partial {\mathbf{\hat q}}_{mn} }}
    {{\partial x}} + {\mathbf{\hat B}}\frac{{\partial {\mathbf{\hat q}}_{mn} }}
    {{\partial y}} + {\mathbf{\hat D\hat q}}_{mn}  - {\mathbf{H}}_{yy} \frac{{\partial ^2 {\mathbf{\hat q}}_{mn} }}
    {{\partial y^2 }} = {\mathbf{S}}_{mn},
\end{equation}
其中$\mathscr{L}_{\rm PSE}$线性PSE算子。本文对方程(\ref{PSE1})在流向采用隐式欧拉差分,法向采用五阶中心差分进行离散求解。

为了验证程序的正确性,我们与Malik等人1994年的工作\cite{Malik1994}进行对比。该工作重点研究了后掠Hiemenz流动的失稳,计算相关参数详见他们的文献。这里计算对比$\bar{R}=500$\footnote{这个符号采用与文献\cite{Malik1994}中相同的定义}工况中主模态的能量在流向的演化,结果如图\ref{f:com_malik}示。
\begin{figure}
  \centering
  \includegraphics[width=0.7\textwidth]{ch2/comparison_Malik.jpg}
  \caption{PSE计算程序验证}\label{f:com_malik}
\end{figure}


\subsection{扰动发展的敏感性分析}
为了更好地理解流动,同时选取较优化的控制参数,本文中对三维边界层失稳进行了敏感性分析。关于流动失稳的敏感性分析最早始于2003年,是Bottaro\cite{Bottaro2003}等人针对Couette流动开展的。通过求解线性稳定性问题的伴随问题,他们找出了容易受基本流变化影响的失稳模态。之后,2008年Marquet等人\cite{Marquet2008}分析了圆柱尾迹流动对于基本流和外加体积力的敏感性,并采用这一结果进行了优化,降低了尾迹的湍流度。Alizard等人\cite{Alizard2010}2010年,对角域流动进行了分析,得到了不同失稳模态的敏感函数(敏感因子)的空间分布。2011年Brandt等人\cite{Brandt2011}对平板边界层做了相应的敏感性分析,之后学者们又对D形圆柱\cite{Meliga2012},空腔\cite{Bromwne2014},甚至湍流边界层的猝发过程进行了相应的分析\cite{Alizard2015},更加深入的了解了其流动机理。本文分别从线性稳定性理论和抛物化扰动方程出发,推导他们的伴随方程,并进而分析三维边界层失稳的敏感性。
\subsubsection{基于线性稳定性理论的敏感性分析}
记方程(\ref{e:LST})的伴随方程为:
\begin{equation}\label{e:aLST}
  \mathscr{L}^+\hat{\mathbf{p}}=\mathscr{L}_0^+\hat{\mathbf{p}}+\alpha \mathscr{L}_1^+\hat{\mathbf{p}} + \alpha^2\mathscr{L}_2^+\hat{\mathbf{p}}=0
\end{equation}
伴随方程与原方程的关系是,对于任意向量$\mathbf{a},\mathbf{b}$,都有:
\begin{equation}
    \int_{0}^{+\infty}\mathbf{a}\cdot(\mathscr{L}\mathbf{b})^Tdy=\int_{0}^{+\infty}(\mathscr{L}^+\mathbf{a})\cdot\mathbf{b}^Tdy
\end{equation}
若定义内积$<\mathbf{a},\mathbf{b}>=\int_{0}^{+\infty}\mathbf{a}\cdot\mathbf{b}^Tdy$,则有:
\begin{equation}\label{}
  <\mathbf{a},\mathscr{L}\mathbf{b}> = <\mathscr{L}^+\mathbf{a},\mathbf{b}>
\end{equation}
引入体积力后,方程变为:
\begin{equation}\label{}
  \left[\mathscr{L}_0+(\alpha + \delta\alpha) \mathscr{L}_1 + (\alpha + \delta\alpha)^2\mathscr{L}_2\right](\hat{\mathbf{q}}+\delta\hat{\mathbf{q}})=\mathbf{F}
\end{equation}
其中$\delta\alpha$和$\delta\hat{\mathbf{q}}$为因为引入体积力产生的特征值和特征向量的变化。由于本文中均采用的是微弱的体积力控制失稳,所以这两个量都是小量。将上式与伴随向量(伴随方程的解)做内积,并忽略高阶小量,得到:
\begin{equation}\label{}
  \begin{aligned}
    <\hat{\mathbf{p}},\mathbf{F}> &= <\hat{\mathbf{p}},\left[\mathscr{L}_0+(\alpha + \delta\alpha) \mathscr{L}_1 + (\alpha + \delta\alpha)^2\mathscr{L}_2\right](\hat{\mathbf{q}}+\delta\hat{\mathbf{q}})> \\
    &\approx  <\hat{\mathbf{p}},(\delta\alpha \mathscr{L}_1+2\delta\alpha\mathscr{L}_2)\hat{\mathbf{q}}>
  \end{aligned}
\end{equation}
最终得到空间模式的复特征值关于体积力的敏感性为:
\begin{equation}\label{e:LST_adjoint}
  \delta\alpha \approx \frac{<\hat{\mathbf{p}},\mathbf{F}>}{<\hat{\mathbf{p}},( \mathscr{L}_1+2\mathscr{L}_2)\hat{\mathbf{q}}>}
\end{equation}
\subsubsection{基于抛物化扰动方程的的敏感性分析}
The sensitivity is usually defined as the gradient of the input of a system with respect to the output. Here, we chose the body force as the input while the disturbance energy at outlet as the output.
%Here, the disturbance energy at the outlet is chosen as the output. Through sensitivity analysis, we can know how a manmade small changes of those inputs affects the evolution of the disturbance and the transition process. This kind of analysis is first used on  stability by Bottoaro \cite{Bottaro2003} and the the sensitivity of Orr-Sommerfeld operator's eigenvalues to modifications of the base flow in their results. Marquet \cite{Marquet2008} conducted the sensitivity analysis of the flow around a cylinder and find the most sensitive point with respect to the body force. Then they put an small cylinder at the most sensitive point to control the turbulence in the wake of the big cylinder. Their work inspires a lot of researchers to believe that the sensitivity analysis can give the important control parameters.
Previous stability investigations \cite{Marquet2008} have demonstrated that sensitivity analyses can provide the key control parameters. The present method for sensitivity analyses refers to the work by Pralits \cite{pralits2000sensitivity}. Since the mode interaction is not considered, the term ${\mathbf{S}}_{mn}$ in Eq.~(\ref{PSE1}) only includes the body force term and excludes the nonlinear term. Thus, the governing equation of each mode is decoupled from the others, and thus the index $nm$, which denotes the spanwise wave number and frequency of the harmonic modes, can be discarded. Thus, the governing equation is written as the following:
\begin{equation}
\label{LPSE1}
    \mathscr{L}{\mathbf{\hat q}}  = {\mathbf{\hat A}}\frac{{\partial {\mathbf{\hat q}} }}
    {{\partial x}} + {\mathbf{\hat B}}\frac{{\partial {\mathbf{\hat q}} }}
    {{\partial y}} + {\mathbf{\hat D\hat q}}  - {\mathbf{H}}_{yy} \frac{{\partial ^2 {\mathbf{\hat q}} }}
    {{\partial y^2 }} = {\mathbf{S}}
\end{equation}
with the output defined as:
\begin{equation}
\label{e:PSEoutput_energy}
J = E = \left[ {\frac{1}
{2}\int_0^{T_z } {\int_0^\infty  {{\mathbf{\tilde q}}^H {\mathbf{M\tilde q}}dydz} } } \right]_{x = x_1 }  %= \left[ {\frac{1}
%{2}\int_0^{T_z } {\int_0^\infty  {\left( {{\mathbf{\hat q}}\Theta } \right)^H {\mathbf{M}}\left( {{\mathbf{\hat q}}\Theta } \right)dydz} } } \right]_{x = X_1 }
= \frac{1}
{2}\int_0^{T_z } {\int_0^\infty  {\left| {\Theta _1 } \right|^2 {\mathbf{\hat q}}_1 ^H {\mathbf{M\hat q}}_1 dydz} }
\end{equation}
The subscript `1' represents the quantities at the outlet. $T_z$ is the spanwise wave length of the instability mode. Then we differentiate the output function, the governing equation (\ref{LPSE1}) and the auxiliary condition (\ref{e:auxiliary}) with respect to the control variables, namely the distributed body force, and the state variables $\alpha$ and $\mathbf{\hat q}$:
\begin{equation}
\begin{aligned}
\delta J & = \frac{1}
{2}\int_0^{T_z } {\int_0^\infty  {\left| {\Theta _1 } \right|^2 {\mathbf{\hat q}}_1 ^H {\mathbf{M}}\delta {\mathbf{\hat q}}_1 dydz} }  \\
         &+ \frac{1}{2}\int_0^{T_z } {\int_0^\infty  {\left| {\Theta _1 } \right|^2 {\mathbf{\hat q}}_1 ^H {\mathbf{M\hat q}}_1 \left( {{\rm i}\int_{x_0 }^{x_1 } {\delta \alpha (x')dx'} } \right)dydz} }  + c.c \\
\end{aligned}
\end{equation}
\begin{equation}
\label{e:dLPSE}
\mathscr{L}\delta {\mathbf{\hat q}} - \delta {\mathbf{S}} + \frac{{\partial {\mathscr{L}}}}
{{\partial \alpha }}\delta \alpha {\mathbf{\hat q}} = 0
\end{equation}
\begin{equation}
\label{e:dAuxilary}
\int_0^\infty  {\left( {\delta {\mathbf{\hat q}}^H {\mathbf{M}}\frac{{\partial {\mathbf{\hat q}}}}
{{\partial x}} + {\mathbf{\hat q}}^H {\mathbf{M}}\frac{{\partial \delta {\mathbf{\hat q}}}}
{{\partial x}}} \right)dy}  = 0
\end{equation}
Here, $c.c.$ is the complex conjugate of all the terms in the equation, $x_0$ and $x_1$  the streamwise coordinates of the inlet and the outlet, respectively. Next, we define the inner product of two arbitrary vectors $\mathbf a$ and $\mathbf b$ as the following:
\begin{equation}
\label{e:innerproduct}
<\mathbf{a},\mathbf{b}>=\int_{0}^{T_z}\int_{x_0}^{x_1}\int_{0}^\infty (\mathbf{a}^H\mathbf{b})dydxdz
\end{equation}
A complex adjoint vector ${\mathbf{\hat q}}^*$ and a complex function $r^*(x)$ are then introduced. Taking inner product of the adjoint vector with Eq.~(\ref{e:dLPSE}) and $r^*(x)$ with Eq.~(\ref{e:dAuxilary}), adding the complex conjugates of each term, we obtain the following identity:
\begin{equation}
\label{e:dadjoint1}
\begin{aligned}
     \int_{0}^{T_z} {\int_{x_0}^{x_1}{r^* \int_0^\infty  {\left( {\delta {\mathbf{\hat q}}^H {\mathbf{M}}\frac{{\partial {\mathbf{\hat q}}}}{{\partial x}} + {\mathbf{\hat q}}^H {\mathbf{M}}\frac{{\partial \delta {\mathbf{\hat q}}}}{{\partial x}}} \right)dydxdz} } }%\\
     +<{\mathbf{\hat q}}^*,{\mathscr{L}}\delta {\mathbf{\hat q}}-\delta {\mathbf{S}} + \frac{{\partial {\mathscr{L}}}}{{\partial \alpha }}\delta \alpha {\mathbf{\hat q}} > + c.c. = 0
\end{aligned}
\end{equation}
Any arbitrary vector $\mathbf{\hat q}^*$ and complex function $r^*$ can satisfy Eq.~(\ref{e:dadjoint1}). To eliminate unnecessary terms, appropriate $\mathbf{\hat q}^*$ and $r^*$ must be identified.  First, we let the adjoint vector satisfy the adjoint equation and the adjoint auxiliary condition shown in Eq.~{(\ref{e:adjointa}) and Eq.~{(\ref{e:adjoint})}. Due to the parabolic feature of the original equation, this adjoint equation is also parabolic and can be solved using a marching scheme. The only difference is that this equation should be marched from the outlet to the inlet.
%To simplify the equation (\ref{e:dadjoint1}), firstly, we let the adjoint vector satisfy the adjoint equation and the adjoint auxiliary condition shown in (\ref{e:adjointa}). Due to the parabolic feature of the original equation, this adjoint equation is also parabolic and it can be solved using a marching scheme. The only difference is that this equation should be marched from the outlet to the inlet.
\begin{equation}
\label{e:adjoint}
\mathscr{L}^* {\mathbf{\hat q}}^*  = \left( {\bar r^*  - r^* } \right){\mathbf{M}}\frac{{\partial {\mathbf{\hat q}}}}
{{\partial x}} + \frac{{\partial \bar r^* }}
{{\partial x}}{\mathbf{M\hat q}}
\end{equation}
\begin{equation}
\label{e:adjointa}
\int_0^\infty  {\left( {\mathbf{\hat q} ^{*H} \frac{{\partial {\mathscr{L}}}}
{{\partial \alpha }}\mathbf{\hat q} } \right)dy}  = \int_0^\infty  {{\rm i}\left| {\Theta _1 } \right|^2 \mathbf{\hat q} _1 ^H \mathbf{M\hat q} _1 dy}
\end{equation}
the $\mathscr{L}^*$ is the adjoint operator of the linear PSE and the bar overhead means complex conjugate. The initial value of the adjoint vector and the function $r^*$ at the outlet is shown below:
\begin{equation}
\label{e:adjointini}
%\begin{aligned}
\begin{gathered}
c = \frac{{ - \int_0^\infty  {{\rm i}\left| {\Theta _1 } \right|^2 {\mathbf{\hat q}}_1 ^H {\mathbf{M\hat q}}_1 dy} }}
{{\left. {\int_0^\infty  {\left( {{\mathbf{\hat q}}_1 ^H {\mathbf{M}}\left( {{\mathbf{\hat A}}} \right)^{ - 1} \frac{{\partial {\mathscr{L}}}}{{\partial \alpha }}{\mathbf{\hat q}}} \right)dy} } \right|_{x = x_1 } }} \hfill \\
{\mathbf{\hat q}}^* _1  =  - \bar c\left( {{\mathbf{\hat A}}^H } \right)^{ - 1} {\mathbf{M\hat q}}_1  \hfill \\
r_1^*  = c + \left| {\Theta _1 } \right|^2  \hfill \\
\end{gathered}
%\end{aligned}
\end{equation}
 If the adjoint vector and $r^*$ satisfy the Eq.~(\ref{e:adjoint}) (\ref{e:adjointa}) and (\ref{e:adjointini}), Eq.~(\ref{e:dadjoint1}) can be written as follows:
\begin{equation}
\label{e:dj}
\delta J = \frac{1}
{2} < \hat \varphi ^* ,\delta {\mathbf{S}} >  + c.c.
\end{equation}
To investigate the body-force effect on the disturbance energy growth in boundary layers, $\mathbf{F}$ and $\mathbf{S}$ are set to zero for the unexcited case. The variation of the output is thus exactly the difference between the unexcited and excited cases. The variation of the body force can be expressed as the following:
\begin{equation}
\label{e:df}
\delta {\mathbf{F}} = \Theta \delta {\mathbf{S}} + {\mathbf{S}}{\rm i}\int_{x_0 }^x {\delta \alpha (x')dx'}  = \Theta \delta {\mathbf{S}}
\end{equation}
According to Eq.~(\ref{e:df}), Eq.~(\ref{e:dj}) can be rewritten as follows:
\begin{equation}
\label{e:dj2}
\delta J = \frac{1}
{2} < \hat \varphi ^* ,\frac{{\delta {\mathbf{F}}}}
{\Theta } >  + c.c.
\end{equation}
Note that the body force term is simply a Fourier component of the total physical force because we only focus on one instability mode. To compute the variation of the kinetic energy caused by a spanwise periodical body force, the first step is to transform it into a Fourier space and then extract the corresponding component as the body-force term. Taking this transformation into account and expanding Eq.~(\ref{e:dj2}), the variation of the disturbance kinetic energy is expressed in the following Integral form:
\begin{equation}
\label{e:adjointresult}
\delta J = \int_0^{T_z } {\int_{x_0 }^{x_1 } {\int_0^\infty  {\left( {G_u \delta f_x  + G_v \delta f_y  + G_w \delta f_z } \right)dydxdz} } }
\end{equation}
\begin{equation}
\label{e:G}
\begin{gathered}
G_u  = {\rm real}(\hat u^{*H} \exp({ - {\rm i}\int_{x_0 }^x {\alpha (x')dx'}  - in\beta z}) )\\
G_v  = {\rm real}(\hat v^{*H} \exp({ - {\rm i}\int_{x_0 }^x {\alpha (x')dx'}  - in\beta z}) )\\
G_w  = {\rm real}(\hat w^{*H} \exp({ - {\rm i}\int_{x_0 }^x {\alpha (x')dx'}  - in\beta z}) )\\
\end{gathered}
\end{equation}
Here the three coefficient, $G_u$, $G_v$ and $G_w$, are sensitivity functions that indicate the disturbance sensitivity to the body force.
\section{充分发展槽道的直接数值模拟}\label{sec:DNS}


\chapter{后掠Hiemenz流动的失稳分析与控制}
后掠Hiemenz流动与后掠翼上的三维边界层流动非常相似,是非常好的模型流动。本文从这一流动出发,研究三维边界层的横流失稳。原始的二维Hiemenz流动就是一股平面射流,自上而下打到一块平板上,并向平板两边溢流开来。在无粘流的假设下,这与直角的角域流动完全等价。因此,我们可以通过构造幂指数复势解得到无粘的Hiemenz流动的流场分布。这里,将无粘流动壁面上的流速分布作为边界层外缘的速度分布。这一分布的流向速度分量是线性增加的,如式(\ref{e:HiemenzF})。式中上标`$\dagger$'表示有量纲量,$c$是一个常系数。在二维Hiemenz流动的基础上,引入展向均匀的流动,就是后掠Hiemenz流动,如图(\ref{fig:SweptHiemenz})所示。有一些研究主要着眼于其附着线的失稳研究\cite{Lin1996,Guegan2006},本文重点分析研究远离附着线区域的横流转捩问题。研究区域如图(\ref{fig:SweptHiemenz})中虚线所示。Malik等人\cite{Malik1994}对这一问题的首次失稳和二次失稳做了充分的研究,本文的控制算例也是以他们研究过的工况作为基准算例。本文中采用与文献中\cite{Malik1994}相同的方法求得这一流动的自相似解,并以此作为基本流。
\begin{equation}\label{e:HiemenzF}
  U_{\infty}=cx^{\dagger}
\end{equation}
在后掠Hiemenz流动中,引入的边界层外缘展向速度$W_{\infty}$在所有流向位置是相同的,因此将这一速度选作参考速度。Malik等人在研究这一问题时,也采用这一速度作为参考速度。$l^\dagger=(\nu/c)^{\frac{1}{2}}$ 可以用来表征边界层厚度,本文在这个算例中用这个长度作为参考长度。以$W_{\infty}$作为参考长度定义的雷诺数叫做横流雷诺数,$Re_W=W_{\infty}l^\dagger/\nu$,这个雷诺数在Malik等人的文章\cite{Malik1994}中被记做$\bar{R}$。
\section{后掠Hiemenz流动的稳定性分析}
\begin{figure}[htb]
  \centering
  % Requires \usepackage{graphicx}
  \includegraphics[width=0.7\textwidth]{ch3/plot_SweptHiemenz.eps}\\
  \caption{后掠Hiemenz流动示意图}\label{fig:SweptHiemenz}
\end{figure}
表\ref{t:testcase}列出了这一章研究的算例的具体参数。在后文中,这两个算例会被简记为Case1和Case2。其中Case1的参数与文献\cite{Malik1994}中完全相同,只是换算到了实际有量纲的情况。图\ref{f:Com_Malik1994}给出了Case1中主模态扰动延流向的发展变化。在本文的模拟计算中,流向用了600个网格点,基本上每个波长都有14个网格点。前人的文献中指出,对于PSE计算,每个波长内有3个网格点就绰绰有余了\cite{Joslin1992},所以本文中使用的网格点密度是完全满足要求的。在垂直于壁面方向,Li等人\cite{Li2015a}指出281个网格点就完全够用了,本文的计算中一共给了301个点。从图\ref{f:Com_Malik1994}所示的结果中,也可以看到,本文的计算结果与文献给出的结果完全吻合,这也再一次验证了使用计算程序的精度。这套稳定性计算程序之前还进行过其他方面的稳定性计算,读者可以查阅\cite{Xu2011a,Xu2011b,Ren2014a,Ren2014b,Ren2014c,Ren2015,Ren2016}。在本章的研究中,采用的等离子体模型为从实验中反推出来的体积力分布模型(Kriegseis` model\cite{kriegseis2013velocity})。在他们的实验中,一共测了8 , 9 , 10 kV 三个电压产生的体积力。这三个电压分别可以吹出来速度为1.7, 2.8, 3.8 m/s平行于壁面的射流。但是,在实际计算中发现,只有8kV的电压产生的体积力可以有效的控制Case1中的横流转捩,另外两个高电压产生的体积力都太强了,范围会促进转捩。所以为了研究电压的效应,在Case2中,将边界层外的展向速度提高了一倍,这样三个电压都可以产生一定的作用,并进行比较研究。
\begin{figure}[htb]
  \centering
  % Requires \usepackage{graphicx}
  \includegraphics[width=0.6\textwidth]{ch3/comparison_Malik.eps}\\
  \caption{计算得到的扰动能量与文献\cite{Malik1994}中的结果对比 ($Re_W=500$)}\label{f:Com_Malik1994}
\end{figure}
\begin{table}
  \caption{计算研究算例的参数}\label{t:testcase}
  \centering
  \begin{tabular}{p{2.3cm}<{\centering}|p{2.5cm}<{\centering}p{3.5cm}<{\centering}p{2.5cm}<{\centering}p{3.5cm}<{\centering}}%{p{3cm}p{3cm}p{3cm}p{3cm}p{3cm}}
  \hline
  % after \\: \hline or \cline{col1-col2} \cline{col3-col4} ...
        & $c{\rm (s^{-1})}$ & $l^\dagger=(\nu/c)^{\frac{1}{2}} ({\rm mm})$ & $W_\infty{\rm (m/s)}$ & $Re_W=W_\infty l^\dagger/\nu$ \\
  \hline
  Case1 & 40          & 0.6014            & 12              & 500 \\
  Case2 & 40          & 0.6014            & 24              & 1000 \\
  \hline
  \end{tabular}

\end{table}

针对这两个算例,本文首先进行了线性稳定性分析(LST)。在线性稳定性分析中,在不同的流向位置均采用小扰动假设和平行流假设,计算不同展向波数$\beta$的横流定常模态的增长率。最终得到模态的增长率随流向位置和展向波数的变化如图\ref{f:LST}。在这两个计算中,展向波数的取值范围均为$\beta\in[0.1,1]$。因为这里只关注由壁面粗糙度激发出来的定常横流模态,所以模态的频率$\omega = 0$。这里所说的流向增长率即为计算得到的复流向波数的虚部的相反数,即$-\alpha_i$。对于Case1,失稳模态首先出现在$x = 83$,失稳模态的展向波数$\beta$为0.12。最大的失稳模态增长率出现在$x =305$,展向波数$\beta$为0.33,增长率为0.0243。对于Case2,失稳模态首先出现在$x = 85$,失稳模态的展向波数$\beta$为0.05。最大的失稳模态增长率出现在$x =451$,展向波数$\beta$为0.26,增长率为0.0336。对比这两个算例,可以发现,随着横向流动的增加,失稳模态的增长率更高了,失稳的区域也更加偏向于下游。另外需要提及的是两者中性曲线,也就是增长率为0的等值线形状的变化。总的来说,随着增横向流动的增大,中性曲线的下支越来越贴近坐标的横轴线,上支的斜率越来越小。其中,上支的斜率越来越小说明高波数的模态的失稳位置更加倾向于下游。

\begin{figure}[htb]
  \centering
  % Requires \usepackage{graphicx}
  \begin{subfigure}{0.48\linewidth}
    \includegraphics[width=\linewidth]{ch3/growthrate2.eps}
    \caption{Case1}\label{modesenergycase3}
  \end{subfigure}
  \begin{subfigure}{0.48\linewidth}
    \includegraphics[width=\linewidth]{ch3/growthratecase2.eps}
    \caption{Case2}\label{modesenergycase1}
  \end{subfigure}
  %\includegraphics[width=0.6\textwidth]{ch3/growthrate2.eps}\\
  \caption{定常横流模态的流向增长率}\label{f:LST}
\end{figure}

线性稳定性计算只能够静态的得到每个模态在不同的位置的增长率,而得不到模态演化以及相互影响的过程。之后本文对这两个算例都进行了NPSE的计算。稳定性分析仅仅能够得到扰动在边界层内的增长情况,但是并不能计算得到扰动的初始值。不同的来流条件和壁面光滑程度会导致不同的扰动初值幅值,计算初始值需要对流动进行感受性分析。由于本文并不关心感受性过程,所以这里只研究一种可能的初始值情况。这里计算模拟了初始扰动主模态的展向波数为0.1、0.2、0.3、0.4和 0.5的5种情况。这些些模态分别被记为Mode1到Mode5。由于不同展向波数的模态的失稳起始位置是不一样的,所以本文的NPSE计算的起始点也是各个子算例各有不同。这5个模态分别起始于$x=86,101,134,173$和218。这也分别是LST预测的失稳起始位置。在计算起始位置只有主模态,所有高阶模态都是后续通过非线性效应激发出来。计算得到的结果如图\ref{f:findtarget}所示。可以看到,Mode1最先失稳,但是相比于其他模态,其增长率则相对较低,所以很快便被其他模态超越。Mode3,其展向波数为$\beta$=0.3在$x = 470$处首先达到饱和。定性的,首次失稳饱和之后,在饱和横流涡上发生的二次失稳会很快促发转捩。所以这里Mode3将是主导转捩的模态。之后,本章将以此模态作为控制目标模态,所有控制算例均谊在控制此模态。

\begin{figure}
  \centering
  % Requires \usepackage{graphicx}
%  \begin{subfigure}{0.48\linewidth}
%    \includegraphics[width=\linewidth]{ch3/umax4.eps}
%    \caption{Case1}\label{growthratecase3}
%  \end{subfigure}
%  \begin{subfigure}{0.48\linewidth}
%    \includegraphics[width=\linewidth]{ch3/umax4case2.eps}
%    \caption{Case2}\label{growthratecase1}
%  \end{subfigure}
  \includegraphics[width=0.6\textwidth]{ch3/umax4.eps}\\
  \caption{Case1算例中,入口扰动展向波长不同时,主模态幅值的流向演化}\label{f:findtarget}
\end{figure}
\section{后掠Hiemenz流动的敏感性分析}
\subsection{基于LST的敏感性分析}
在这一小节中,介绍一下敏感性分析的研究成果。从式(\ref{e:LST_adjoint})中我们发现伴随向量是直接起到对体积力的加权作用的,所以通过对该向量的分析,我们可以获得体积力法向敏感性分布的大致情况。三个方向伴随速度(伴随向量中对应于速度的三个分量)的分布如图\ref{f:LST_ADJOINT}:
\begin{figure}[htb]
  \centering
  \subcaptionbox{$u^*$分布云图}[0.55\textwidth]
  {\includegraphics[width=0.55\textwidth]{ch3/absau1.jpg}}
  \subcaptionbox{$u^*$不同截面剖面}[0.43\textwidth]
  {\includegraphics[width=0.43\textwidth]{ch3/absau.jpg}}
  \\\bigskip
  \subcaptionbox{$v^*$分布云图}[0.55\textwidth]
  {\includegraphics[width=0.55\textwidth]{ch3/absav1.jpg}}
  \subcaptionbox{$v^*$不同截面剖面}[0.43\textwidth]
  {\includegraphics[width=0.43\textwidth]{ch3/absav.jpg}}
  \\\bigskip
  \subcaptionbox{$w^*$分布云图}[0.55\textwidth]
  {\includegraphics[width=0.55\textwidth]{ch3/absaw1.jpg}}
  \subcaptionbox{$w^*$不同截面剖面}[0.43\textwidth]
  {\includegraphics[width=0.43\textwidth]{ch3/absaw.jpg}}
  \caption{LST伴随向量}\label{f:LST_ADJOINT}
\end{figure}
从伴随速度的分布中我们可以看到,展向和流向的伴随速度始终大于法向的伴随速度。这表明,该流动对于展向和法向的激励更加敏感,而对于法向的机理则不是那么敏感。另外,随着扰动延流向发展,其对展向的激励越来越敏感,而对流向则越来越不敏感。从物理上这也很好解释。在靠近前缘的位置,基本流的流向分量很弱,所以只需要很小的扰动就能够对其产生很大的影响。之后随着流向的推进发展,流向的基本流越来越强,对其产生扰动需要的力量也就越来越大,从而敏感性也就越来越低。

针对我们提出的等离子体控制方案,我们用所推导出的敏感性公式分析其控制效率,得到在某一固定位置,扰动模态增长率变化与激发器展向位置和安装角度的关系如图\ref{f:senVSangle_span}。
\begin{figure}[htb]
  \centering
  \subcaptionbox{敏感性随安装的展向位置和角度变化}[0.45\textwidth]
  {\includegraphics[width=0.45\textwidth]{ch3/sen_angle_span.jpg}}
  \subcaptionbox{激发器安装方位示意图}[0.45\textwidth]
  {\includegraphics[width=0.45\textwidth]{ch3/plasam_position.jpg}}
  \caption{LST敏感性分析结果}\label{f:senVSangle_span}
\end{figure}
\begin{figure}[htb]
  \centering
  \subcaptionbox{敏感性随展向安装位置的变化\label{f:senVSdz}}[0.45\textwidth]
  {\includegraphics[width=0.45\textwidth]{ch3/senVSdz.jpg}}
  \subcaptionbox{敏感性随安装角度的变化\label{f:senVSangle}}[0.45\textwidth]
  {\includegraphics[width=0.45\textwidth]{ch3/senVSangle.jpg}}
  \caption{LST敏感性分析结果}\label{f:senVSangle_span2}
\end{figure}
在这里,我们在增长率变化量后面乘了一个因子cos$(\theta)$,这是因为当偏转激发器角度的时候,激发器所能覆盖的流向位置就变短了,变成了原来的cos$(\theta)$倍。而增长率指的是单位流向长度的增长率,所有乘上了这个系数。可以看到,敏感因子随着展向的变化呈正弦规律,而最佳安装角度36$^\circ$与横流涡角度44$^\circ$差距不大。
得到不同的流向位置最大增长率减小值之后,我们对比了流向敏感性,如图\ref{f:senVSstr}。我们可以看到,在进入失稳区之后,相同大小的体积力对横流模态的影响就越来越小。然而这一结论是建立在线性、单一模态、正确相位的基础上的,实际应用时,越靠近中性点越容易激发出别的扰动模态从而影响转捩。
\begin{figure}
  \centering
  \includegraphics[width=0.5\textwidth]{ch3/senVSstr.jpg}
  \caption{敏感性随流向位置变化}\label{f:senVSstr}
\end{figure}

\subsection{基于PSE的敏感性分析}
由于基于LST的敏感性分析还是局部性质的结果,并且在推导的时候忽略了二阶项的影响,所以其结果精确度并不高。这一小节主要展示基于PSE的稳定性分析及的结果。依然是针对Case1以及我之前确定的目标模态。这里由于本文希望采用的控制方案还是在线性区进行控制,所以敏感性分析也是只针对线性区进行。因此,这里的原始方程是线性PSE方程,并不包含模态之间的非线性相互作用。图\ref{f:Guvw1}给出了流向速度和敏感因子在$x =280$位置横截面的分布。需要注意的是,在基于LSE的敏感性分析中,针对的目标变量是扰动模态的增长率,而在基于PSE的敏感性分析中,针对的目标变量是下游某一位置处的扰动能量(式(\ref{e:PSEoutput_energy}))。图\ref{f:Guvw1}展示的是输出扰动能量选取在$x =500$的结果。其中,用颜色表示的云图是流向的速度分布,线条是各个方向敏感性因子的等值线。这里用实线表示正值用虚线表示负值。从流向速度分布的云图中,可以看到横流涡正在形成,但是还没有标志着横流涡达到饱和的"上叶翻转"现象。"上叶翻转"也是强非线性产生的标志。从流向速度分布云图中可以看出,这个分析敏感性的位置还处于线性增长区,也符合本文的初衷。虽然横流涡在这里还没有完全形成,但是壁面附近已经有了高低速相间的条带。其他位置的敏感因子分布情况与图\ref{f:Guvw1}展示的类似,这里就不再赘述。值得一提的是,相比于敏感因子的大小,敏感因子的正负是指导控制方案的关键。
\begin{figure}[H]
  \centering
  % Requires \usepackage{graphicx}
  \begin{subfigure}{0.8\textwidth}
  \includegraphics[width=\textwidth]{ch3/Gu(4).eps}
  \caption{\label{f:Gu1}}
  \end{subfigure}\\
  \bigskip
  \begin{subfigure}{0.8\textwidth}
  \includegraphics[width=\textwidth]{ch3/Gv(4).eps}
  \caption{\label{f:Gv1}}
  \end{subfigure}\\
  \bigskip
  \begin{subfigure}{0.8\textwidth}
  \includegraphics[width=\textwidth]{ch3/Gw(4).eps}
  \caption{\label{f:Gw1}}
  \end{subfigure}
  \caption{Distribution of the sensitivity functions (a) $G_u$, (b) $G_v$ and (c) $G_w$ at $x =280$ with the output location at $x =500$.}\label{f:Guvw1}
\end{figure}
\begin{figure}[H]
  \centering
  % Requires \usepackage{graphicx}
  \begin{subfigure}{0.8\textwidth}
  \includegraphics[width=\textwidth]{ch3/Gu(4)_300.eps}
  \caption{\label{f:Gu1_300}}
  \end{subfigure}\\
  \bigskip
  \begin{subfigure}{0.8\textwidth}
  \includegraphics[width=\textwidth]{ch3/Gv(4)_300.eps}
  \caption{\label{f:Gv1_300}}
  \end{subfigure}\\
  \bigskip
  \begin{subfigure}{0.8\textwidth}
  \includegraphics[width=\textwidth]{ch3/Gw(4)_300.eps}
  \caption{\label{f:Gw1_300}}
  \end{subfigure}
  \caption{Distribution of the sensitivity functions (a) $G_u$, (b) $G_v$ and (c) $G_w$ at $x =280$ with the output location at $x =300$.}\label{f:Guvw1_300}
\end{figure}

如图\ref{f:Gu1}所示,$G_u$的正值分布在高速条带下方,并且斜着向上延伸,与相邻的低速条带重合。正值区与负值区相间交替出现。图\ref{f:Gv1}中,$G_v$的正值主要集中在低速条带出现的区域,而高速条带位置主要是负值。这意味着如果想要通过法向激励的方式,比如壁面垂直吹吸之类的方法控制失稳,那么就需要在低速条带下面吸气,在高速条带下面吹气。$G_v$的0值等值线并不像$G_u$和$G_w$的0值等值线那样扭曲。$G_w$的分布情况基本与$G_u$类似,只是正负值分布的区域做了交换。在D\"orr和Kloker\cite{dorr2016}提出的等离子体控制方法中,他们有两个基础的控制算例分别叫做ACF和CCF。这两个控制算例的计算结果展示在他们文章的Fig 4 中。在ACF算例,有着展向正方向分量的体积力被施加在了横流涡下方,也就是图\ref{f:Gw1}所示的$G_w$恰好是负值的位置。在他们的算例CCF中,有着与ACF中体积力相反方向的体积力被施加在了二次涡出现的位置,也就是图\ref{f:Guvw1_300}中所示高速条带的位置。这里一位置的$G_w$恰好是正的。从式(\ref{e:adjointresult})可知,负的敏感因子乘上正的体积力,或者是正的敏感因子乘上负的体积力,都可以得到负的扰动能量变化,也就是使得扰动变弱。D\"orr和Kloker\cite{dorr2016}的结果也印证了这里敏感性分析的结论。在下一小节的等离子体控制算例的结果,也会对本文推倒的敏感性分析做相关的印证。

由于并不知道扰动能量输出位置的选取对敏感性因子的计算结果有没有影响,所以本文计算了输出位置在$x=450,400,350,300$的敏感性因子分布。图\ref{f:Guvw1_300}给出了$x=280$截面上流向速度分布云图和敏感性因子的等值线,但这一次的扰动能量输出位置选取在$x=300$。相比较输出位置选取在$x=500$,流向和展向敏感因子的分布更加的贴近壁面。然而,当输出位置与所观察的截面想去较远时,如输出位置在$x=350,400,450$,则结果和图\ref{f:Guvw1}中的分布几乎完全一样。所以这里不再将这些相同的分布罗列出来。所以,通过比较输出位置在$x=500,450,400,350,300$这五个算例,可以下如下结论:当在近所关心位置上游不远处进行控制时,更加靠近壁面的控制激励效果更好。但是这一效应在远离所关心位置之后迅速衰减并消失。由于本文之后采取的控制措施都是在离关心区域较远的位置,之后展示的敏感因子均是以$x=500$为输出位置计算得到的。

\begin{figure}[htb]
  \centering
  % Requires \usepackage{graphicx}
  \includegraphics[width=\textwidth]{ch3/Gw_xz.eps}\\
  \caption{Distribution of sensitivity functions $G_w$ at $y=1$}\label{f:Gw_xz}
\end{figure}

图\ref{f:Gw_xz}给出了靠近壁面$z-x$平面上,展向敏感因子$G_w$的分布等值线和流向速度分布云图。从图中可以清楚的看到高低速条带的相间分布。其中红色代表着高速条带,蓝色代表着低速条带。在远离输出位置的区域内,可以看到$G_w$的等值线基本上与条带平行。不平行的区域只有大约不到40的无量纲长度。这意味着最佳的等离子激发器布置方案也应该是平行于高低速条带,也就是平行于横流涡轴。这样可以保证在每一个横截面内,激发器产生的体积力都处在流动最敏感的区域内。D\"orr和Klocker \cite{dorr2015stabilisation,dorr2016}提出的控制方法就总是让激发器平行于横流涡轴。

图\ref{f:Guvw_xy}给出了三个方向敏感性因子展向最大值在$y-x$平面内的分布。从图中可以看到,流向的敏感性因子最大。另外,最敏感的区域位于边界层内,但是离壁面却还有一定的距离。这是因为在壁面附近,粘性主导,速度剪切很大,体积力想改变流动非常困难,远不如远离壁面的位置改变流动容易。当然,如果出了边界层,体积力产生的扰动又不会影响失稳模态,从而敏感性也会降低。所以最敏感的区域出现在高度适中的位置。

图\ref{f:maxsen}
\begin{figure}[H]
  \centering
  % Requires \usepackage{graphicx}
  \begin{subfigure}{\textwidth}
  \includegraphics[width=0.8\textwidth]{ch3/maxGu_xy.eps}
  \caption{\label{f:Gu_xy}}
  \end{subfigure}\\
  \bigskip
  \begin{subfigure}{\textwidth}
  \includegraphics[width=0.8\textwidth]{ch3/maxGv_xy.eps}
  \caption{\label{f:Gv_xy}}
  \end{subfigure}\\
  \bigskip
  \begin{subfigure}{\textwidth}
  \includegraphics[width=0.8\textwidth]{ch3/maxGw_xy.eps}
  \caption{\label{f:Gw_xy}}
  \end{subfigure}
  \caption{Contours of maximum (a) $G_u$, (b) $G_v$ and (c) $G_w$ along the Z direction with the output location at x=500.}\label{f:Guvw_xy}
\end{figure}
\begin{figure}[htb]
  \centering
  % Requires \usepackage{graphicx}
  \includegraphics[width=0.8\textwidth]{ch3/maxG(3).eps}\\
  \caption{Streamwise distributions of the maximum value of the sensitivity functions on the y-z plane}\label{f:maxsen}
\end{figure}

Figure \ref{f:maxsen} shows the streamwise distributions of the maximum value of the sensitivity functions on each cross-section. As mentioned before, the neutral point is located at $x=134$ and it is indicated with a vertical dash line in the figure. The maximum values appear immediately upstream of the neutral point consistent with the findings by Pralits \cite{pralits2000sensitivity} for a flat-plate case. Downstream of the neutral point, all sensitive functions decrease rapidly with increasing $x$. This indicates that the actuator is more effective when placed further upstream until the neutral point. However, if the actuator is sufficiently close to the neutral point, it is likely to act as a strong disturbance that over-rides the natural disturbance and dominates transition. Nevertheless, from these sensitivity analyses, we have learnt about the features of this flow from one perspective and determined that the upstream control could be more efficient in the interval in which the natural disturbance have fully developed.
\section{采用等离子体激发器推迟后掠Hiemenz流动转捩}
However, the sensitivity analyses are based on a linear assumption, and NPSE computations are required to further investigate the actuator location effect. Here, the data for the body force distribution obtained from the experiment \cite{kriegseis2013velocity} with an actuator operating voltage of 8 kV are used. Figure \ref{f:spanwiseeffect} compares the streamwise mode-energy evolution with the actuator imposed at different spanwise locations. Here, the mode-energy is defined as following:
\begin{equation}
{\rm Energy}=\frac{1}{2}\int_{0}^{\infty}{(|\hat u|^2+|\hat v|^2+|\hat w|^2)dy}
\end{equation}
Since the mean flow correction mode, (0,0) mode, does not have a complex conjugate, its energy is defined as following:
\begin{equation}
{\rm Energy}_{00}=\frac{1}{4}\int_{0}^{\infty}{(|\hat u|^2+|\hat v|^2+|\hat w|^2)dy}
\end{equation}

Figure \ref{f:spanwiselocations} depicts the actuator locations relative to the local crossflow vortex. The spanwise locations simulated are at $z/T_z = 0.5, 0.6, 0.7, 0.8, 0.9$ and 1.0, with the corresponding Cases (a) to (f). Here, $T_z$ is the wavelength of the primary mode. In Figure \ref{f:spanwiseeffect}, the extent of the actuation region is shown within 2 blue dots; (0,0) mode represents the meanflow correction mode, (0,1) mode the primary mode, and (0,2) mode the mode with double the spanwise wave number of the primary mode. All the first number 0 indicates the frequency is zero and they are all steady modes. The characteristic of the primary mode is consistent with that of the target mode in sensitivity analyses: In Cases (d) and (e), the primary-mode energy decreases considerably with the actuator imposed at the bottom of the crossflow vortex, the negative $G_w$ region. The minimum values of the primary modes' energy are 0.0067 and 0.0077 for Cases (d) and (e), respectively. However, in Cases (a) to (d), the (0,2) mode is promoted, as also observed by D\"orr and Kloker in their DNS study \cite{dorr2016} (Fig 8 in their paper). The (0,2) mode's energy even exceeds that of the primary mode (see Figure \ref{f:d}). Fortunately, in the actuation region all-mode energy decreases in Cases (e) and (f). Therefore, the optimal spanwise actuator location is $z/Tz=0.9$. Note that results with only two spanwise locations are shown in D\"orr and Kloker's DNS\cite{dorr2016}, and in both of them the (0,2) modes were promoted.

\begin{figure}[H]
    \centering
    \begin{subfigure}{0.45\textwidth}           %
        \includegraphics[width=\linewidth]{ch3/comparemodes(a).eps}
        \caption{}\label{f:a}
    \end{subfigure}
    %\\ \bigskip
    \begin{subfigure}{0.45\textwidth}           %
        %% label for first subfigure
        \includegraphics[width=\linewidth]{ch3/comparemodes(b).eps}
        \caption{}\label{f:b}%\includegraphics[width=0.45\linewidth]{forceposition(b)}} \\
    \end{subfigure}
    \\ \bigskip
    \begin{subfigure}{0.45\textwidth}         %
        %% label for first subfigure
        \includegraphics[width=\linewidth]{ch3/comparemodes(c).eps}
        \caption{}\label{f:c}%\includegraphics[width=0.45\linewidth]{forceposition(c)}}
    \end{subfigure}
    %\\ \bigskip
    \begin{subfigure}{0.45\textwidth}          %
        %% label for first subfigure
        \includegraphics[width=\linewidth]{ch3/comparemodes(d).eps}
        \caption{}\label{f:d}
    \end{subfigure}
    \\ \bigskip
    \begin{subfigure}{0.45\textwidth}         %
        %% label for first subfigure
        \includegraphics[width=\linewidth]{ch3/comparemodes(e).eps}
        \caption{}\label{f:e}
    \end{subfigure}
    %\\ \bigskip
    \begin{subfigure}{0.45\textwidth}          %
        %% label for first subfigure
        \includegraphics[width=\linewidth]{ch3/comparemodes(f).eps}
        \caption{}\label{f:f}
    \end{subfigure}
    \caption{Comparison of the streamwise mode-energy evolution with the actuator imposed at different spanwise locations: (a) $z/T_z = 0.5$, (b) $z/T_z = 0.6$, (c) $z/T_z = 0.7$, (d) $z/T_z = 0.8$, (e) $z/T_z = 0.9$, (f) $z/T_z = 1.0$.}
    \label{f:spanwiseeffect} %% label for entire figure
\end{figure}
\begin{figure}[H]
    \centering
    \begin{subfigure}{0.45\textwidth}           %
        %% label for first subfigure
        %\includegraphics[width=0.3\linewidth]{comparemodes(a)}}
        \includegraphics[width=\linewidth]{ch3/forceposition(a).eps}
        \caption{}\label{f:a1}
    \end{subfigure}
    \begin{subfigure}{0.45\textwidth}
        %% label for first subfigure
        %\includegraphics[width=0.3\linewidth]{comparemodes(b)}}
        \includegraphics[width=\linewidth]{ch3/forceposition(b).eps}
        \caption{}\label{f:b1}
    \end{subfigure}
    \\ \bigskip
    \begin{subfigure}{0.45\textwidth}         %
        %% label for first subfigure
        %\includegraphics[width=0.3\linewidth]{comparemodes(c)}} \\
        \includegraphics[width=\linewidth]{ch3/forceposition(c).eps}
        \caption{}\label{f:c1}
    \end{subfigure}
    \begin{subfigure}{0.45\textwidth}          %
        %% label for first subfigure
        \includegraphics[width=\linewidth]{ch3/forceposition(d).eps}
        \caption{}\label{f:d1}
    \end{subfigure}
    \\ \bigskip
    \begin{subfigure}{0.45\textwidth}          %
        %% label for first subfigure
        \includegraphics[width=\linewidth]{ch3/forceposition(e).eps}
        \caption{}\label{f:e1}
    \end{subfigure}
    \begin{subfigure}{0.45\textwidth}          %
        %% label for first subfigure
        \includegraphics[width=\linewidth]{ch3/forceposition(f).eps}
        \caption{}\label{f:f1}
    \end{subfigure}
    \caption{Comparison of the streamwise mode-energy evolution with the actuator imposed at different spanwise locations: (a) $z/T_z = 0.5$, (b) $z/T_z = 0.6$, (c) $z/T_z = 0.7$, (d) $z/T_z = 0.8$, (e) $z/T_z = 0.9$, (f) $z/T_z = 1.0$.}
    \label{f:spanwiselocations} %% label for entire figure
\end{figure}
%\begin{figure}
%\ContinuedFloat
%    \centering
%    \subfloat[][]{           %
%        \label{f:d} %% label for first subfigure
%        \includegraphics[width=0.45\linewidth]{comparemodes(d)}
%        \includegraphics[width=0.45\linewidth]{forceposition(d)}} \\
    %\hspace{0.0in}
%    \subfloat[][]{           %
%        \label{f:e} %% label for first subfigure
%        \includegraphics[width=0.45\linewidth]{comparemodes(e)}
%        \includegraphics[width=0.45\linewidth]{forceposition(e)}} \\
%    \subfloat[][]{           %
%        \label{f:f} %% label for first subfigure
%        \includegraphics[width=0.45\linewidth]{comparemodes(f)}
%        \includegraphics[width=0.45\linewidth]{forceposition(f)}}
%    \caption{Comparison of crossflow profiles at different streamwise location}
%    \label{f:spanwiseeffect2} %% label for entire figure
%\end{figure}
\begin{figure}
  \centering
  % Requires \usepackage{graphicx}
  \includegraphics[width=\linewidth]{ch3/compare160427-enegy(1).eps}\\
  \caption{Comparison of control cases with actuators put on different streamwise location}\label{f:streamforce}
\end{figure}
\begin{figure}
  \centering
  % Requires \usepackage{graphicx}
  \includegraphics[width=0.6\linewidth]{ch3/compare160427-vmax.eps}\\
  \caption{Differences of maximum streamwise velocity of primary mode between uncontrolled and controlled cases in the control regions ($x_{\rm start}$: start point of control region)}\label{f:streamforce2}
\end{figure}

The optimal spanwise location of the plasma actuators is used to investigate their streamwise location effect on transition using NPSE. In sensitivity analyses, a plasma actuator location upstream as possible without contaminating the quiet boundary layer is suggested. Figure \ref{f:streamforce} compares the streamwise mode-energy evolution within different streamwise actuation regions. The actuation regions have the same extent (shown between the two large dots) but different start points, $x_{start}$, at $x=315, 358$ and 400. Here, the body force's strength is reduced to one tenth of the original, because too strong force at upstream will contaminate the boundary layer. A much lower mode energy is obtained for the case with $x_{start}=358$ compared to the one with $x_{start}=400$ because $x_{start}$ of the former case is closer to the neutral point at $x=134$, as pointed out in the sensitivity analyses. Figure \ref{f:streamforce2} further depicts the reduction of the maximum streamwise velocity of the primary mode downstream of $x_{start}$. The largest reduction is given for the case with $x_{start}=315$ in the vicinity of $x_{start}$, where the actuated body-force integration is small and therefore the sensitivity analyses based on the linear assumption are still acceptable. However, downstream of $x_{start}+18.7$, the primary mode suppression becomes weaker and weaker for the case with $x_{start}=315$ because here disturbances are introduced so far upstream that they contaminate the essentially quiet boundary layer. This explains why the case with $x_{start}=358$ provides the most effective flow-transition delay control, as shown in Figure \ref{f:streamforce}.

Figure \ref{modesenergycase3} gives the LST results of the growth rate of steady crossflow modes for Case 2, where the crossflow velocity is doubled to investigate the operating voltage effect. The first unstable mode appears at $x =85$, with $\beta$ of 0.05; the maximum mode growth rate is 0.0336 with $\beta$ of 0.26 at $x =451$. The neutral-curve slope of the upper branch is smaller than that for Case 1. Figure \ref{growthratecase3} further shows the evolution of the mode energy obtained using NPSE. The simulated wave numbers are 0.1, 0.2, 0.3 and 0.4. The mode with $\beta$ of 0.1 becomes the most unstable upstream but increases slowly compared to the others. Due to its relatively early start and large increase rate, the mode with $\beta$ of 0.2 is chosen as the target mode in the following study.

%\begin{figure}
%  \centering
%  % Requires \usepackage{graphicx}
%  \begin{subfigure}{0.48\linewidth}
%    \label{modesenergycase3}
%    \includegraphics[width=\linewidth]{ch3/growthratecase2.eps}
%  \end{subfigure}
%  \begin{subfigure}{0.48\linewidth}
%    \label{growthratecase3}
%    \includegraphics[width=\linewidth]{ch3/umax4case2.eps}
%  \end{subfigure}
%  \caption{Growth rate of steady crossflow modes (a) and evolution of mode energy (b) computed using LST and NPSE, respectively.}\label{f:baseline}
%\end{figure}
The results of the sensitivity analyses (not shown) are similar to those for Case 1. The higher freestream spanwise velocity in Case 2 allows us to adopt higher operating voltages of 9kV and 10kV. Figure \ref{f:voltage} compares the streamwise mode-energy evolution with different operating actuator voltages. The actuation region ranges from 500 to 550 in the $x$ direction with the optimum spanwise location for each case. The simulated operating voltages are 8 kV, 9 kV and 10 kV, with corresponding maximum jet velocity magnitudes of 1.7 m/s, 2.8 m/s and 3.8 m/s, respectively. Note that only 8,9,10,11,12 kV voltages were provided in the measurement \cite{kriegseis2013velocity} and the last two were too strong for our case. An increase in operating voltage leads to more efficient disturbance suppression upstream for $x = 530$. However, for the case with a 10 kV operating voltage, the disturbance energy starts to increase downstream for $x = 530$, and the energy of the mean flow distortion mode (0, 0) as well as the harmonic mode (0, 2) even exceeds the energy of the primary mode (0, 1). Eventually the harmonic mode (0, 2) dominates the flow. Forcing that is too strong may cause the formation of nocent vortices that promote transition to turbulence, as also reported by the DNS study \cite{dorr2016}. Therefore, it can be concluded that for Case 2, the moderate operating voltage of 9 kV provides the most effective flow-transition delay control.

\begin{figure}
  \centering
  % Requires \usepackage{graphicx}
  \includegraphics[width=\linewidth]{ch3/compare754-843(3).eps}\\
  \caption{Evolution of modes energy controlled by plasma actuator with different operating voltage}\label{f:voltage}
\end{figure}

\chapter{后掠翼上流动失稳分析与控制}
\section{后掠翼上流动的稳定性分析}
数值研究了后掠Hiemenz流动转捩控制之后,本文将研究目标转移到更贴近实际的工况,研究一个真实的后掠翼流动的转捩控制。清华大学(Tsinghua University)在2018年进行了NLF-0415后掠翼上翼面的转捩实验研究工作\cite{wang2018},本文以该实验的工况作为基准,研究等离子体在实际的后掠翼流动中控制转捩的效果。这一实验在后文中将会被简记为``THU实验''或``THU experiment''。这里先简述一下实验的设备和工况。该实验是在清华大学自己的低湍流度风洞中进行的,风洞试验段大小为1.2m$\times$1.2m$\times$3m。风洞内安装NLF-0412翼型,如图\ref{f:experiment}所示。在实验中采用的直角坐标系为($X_{\rm wt},Y_{\rm wt},Z_{\rm wt}$),其$X_{\rm wt}$方向与风洞的流向平行。在计算中本文用的到直角坐标系为$x,y,z$(如图\ref{f:experiment})。计算所用坐标系的$z$方向与翼型前缘线平行,这样在采用了无限展长假设之后,这一方向的物理量就是均匀的。该风洞运行时测试段的风速可以为5.0到90.0m/s,湍流度可以低到0.05\%。此次试验采用的翼型的弦长$c$=1.2m,有45$^\circ$后掠和-4$^\circ$攻角。该翼型在上翼面有很强的顺压梯度,可以有效的抑制T-S波失稳\cite{Dagenhart1999}。翼型中间到风洞入口大约1.25m。

实验中采用了单一的热线测量了来流的情况。来流风速均匀,湍流度0.08$\sim$0.1\%。采用边界层热线(TSI Model 1261A)测量翼型$X/C$ = 0.2, 0.4, 0.6位置处的边界层速度剖面。热线采用一个电脑控制的可以进行三维运动的机械臂驱动。最小的运动精度可以达到10$\mu$m。在每一个位置,沿着法向每隔0.05mm测量了沿着风洞方向的速度分量$U_{\rm wt}$。这个速度分量也就是图中所示$X_{\rm wt}$坐标的方向。沿着法向的测量一直到两倍边界层厚度的位置终止。为了防止测量热线因为接触避免被损坏,第一个测量点选取在离壁面0.15mm处。这一距离是通过镜面法测量得到的。也就是通过测量热线和其在翼面上的镜像的距离,得到其相对于翼面的距离。

为了验证本文中基本流计算的结果,这里选取了雷诺数$Re$为$1.81\times10^6$,也就是自由来流速度$U_\infty$=22.3m/s的算例进行比较。($Re={U_\infty c}/{\nu}$ 其中$U_\infty$是自由来流速度度,$c$是弦长,$\nu$是动力粘性系数。)在这个工况下,上翼面一直到70\%弦长的位置都是层流。在之后研究得我等离子体控制算例中,为了加速失稳和转捩,采用了两倍雷诺数也就是两倍来流速度的算例。

\begin{figure}
  \centering
  \includegraphics[width=\textwidth]{ch4/experiment.jpg}
  \caption{风洞俯视图( `` $\bullet$ "表示热线测量位置)}\label{f:experiment}
\end{figure}

如上一章所述,高精度重构修正有限元程序Music\cite{WangZJ2009,Zhu2016,Zh2017}被用来进行无粘流求解计算。图\ref{f:ConpareInvicidV}对比了计算求解得到的壁面上的无粘流延风洞流向速度分量和实际实验中测得的边界层外的同方向的速度分量。其中黑线是计算得到的结果,三个大红点是实验测得的结果。由于实验仅测量了三个展向位置的速度,所以这里只有三个数据点。从图中可以看到,无粘流计算得到的结果与实验吻合良好。需要提及的是,实验中并没有精确测量离翼型非常远位置处的来流速度,也就是说实际上真正的自由来流是不能准确得知的。而计算中是采取的无量纲计算,所以本位以20\%弦长位置边界层外缘的流速作为参考基准点,重新对计算结果进行了有量纲化。在实验中,虽然直到70\%弦长处都是层流,但是60\%弦长处边界层已经出现了强烈的扭曲。所以这里之对比了20\%和40\%弦长处的边界层剖面,如图\ref{f:compare_profiles}。与上图类似的,线是计算结果,点是实验测得的结果。这里并没有用所有测量数据,只是每个两个点取了一个数据。对比结果说明本文计算得到的基本流还是非常可靠地。
\begin{figure}
\centering
  % Requires \usepackage{graphicx}
  \includegraphics[width=0.7\textwidth]{ch4/compare_UexpOUT_laminarcase}
\caption{边界层外缘风洞流向速度$U_{\rm{wt}}$对比}
\label{f:ConpareInvicidV}
\end{figure}

\begin{figure}
\centering
  % Requires \usepackage{graphicx}
  \includegraphics[width=0.6\textwidth]{ch4/compare_profiles}
\caption{风洞流向速度$U_{\rm{wt}}$ 在$X/C= 20\%$和$40\%$处剖面对比}
\label{f:compare_profiles}
\end{figure}
Reibert等人\cite{Reiberit1996}也做过相同翼型相同工况的实验研究,在后文中将其命名为``Reibert's experiment''或``Reibert实验''。但是他们的实验采用的风动和模型尺寸均与本文参考的实验不一样。在Reibert实验中,模型弦长1.83m,风洞实验段尺寸1.4m$\times$1.4m$\times$5m。更重要的,他们在文献中并没有给出翼型的安装位置。这些实验设置的不同,导致了两个实验得到的上翼面压力系数的分布不太一样。图\ref{f:CpCompare}给出了THU实验和Reibert实验的上翼面压力分布。可以看到,THU实验的压力梯度要比Reibert实验的压力梯度强一些,这也说明了THU实验中流体在翼型中段的加速度更大一些。为了搞清楚压力梯度的变化对扰动的发展有什么影响,本文设置了4个算例进行对比研究。其中两个算例采用了Reibert实验得到的压力分布,他们的自由来流速度分别是22.3m/s和44.5m/s。另外连个算例采用THU实验得到的压力分布,自由来流速度也分别是22.3m/s和44.5m/s。这四个算例中边界层外缘流向速度分布与边界层位移厚度展示在图\ref{fig:CompOutFlow}中。可以看到,在采用THU算例的工况中,流体在靠近前缘位置加速比较慢,但是在翼型中段,其速度快速提升,并在30\%弦长处超过了采用Reibert实验工况中的边界层外缘流向速度。不管在哪一个雷诺数条件下,采用THU压力分布的算例中的边界层厚度都要比采用Reibert的对应算例要厚。在70\%弦长处,压力达到最低值,在之后均是逆压梯度,边界层快速增长。
\begin{figure}[htb]
\centering
  % Requires \usepackage{graphicx}
  \includegraphics[width=0.7\textwidth]{ch4/compareCp_Reibert}
  \caption{THU(清华)实验中压力系数(红)与Reibert博士论文 \cite{Reiberit1996}中给出的压力系数(蓝)对比}\label{f:CpCompare}
\end{figure}
\begin{figure}[htb]
\centering
\subcaptionbox{不同算例中边界层外缘流向速度对比\label{fig:CompOutFlow:a}}[0.48\linewidth] %% label for first subfigure
    {\includegraphics[width=0.48\linewidth]{ch4/compare-outflow3}}
%\hspace{0.0in}
\subcaptionbox{不同算例中边界层位移厚度对比\label{fig:CompOutFlow:b} }[0.48\linewidth]%% label for second subfigure
    {\includegraphics[width=0.48\linewidth]{ch4/DisplacementThickness-4(2)}}
\caption{不同算例边界层对比}
\label{fig:CompOutFlow} %% label for entire figure
\end{figure}
\begin{figure}[htb]
\centering
\subcaptionbox{$U_\infty=22.3m/s$\label{fig:CompCrossProfiles:a} }[0.48\linewidth]%% label for first subfigure
{\includegraphics[width=0.48\linewidth]{ch4/compare223Wt(scaledUe)}}
%\hspace{0.0in}
\subcaptionbox{$U_\infty=44.5m/s$\label{fig:CompCrossProfiles:b}} [0.48\linewidth]%% label for second subfigure
{\includegraphics[width=0.48\linewidth]{ch4/compare445Wt(scaledUe)}}
\caption{不同算例不同位置横流速度剖面对比}
\label{fig:CompCrossProfiles} %% label for entire figure
\end{figure}

图\ref{fig:CompCrossProfiles}展示了20\%, 40\%, 和 60\%处的横流速度剖面。如之前提到的,横流速度必须在壁面和边界层外消失。这里,随着边界层的厚度从20\%到60\%逐渐增加,横流速度消失位置的高度也在逐渐增加。可以看到,所有采用THU压力分布的算例中的横流速度的最大值都要大于采用Reibert压力分布的对应算例对应位置的横流速度最大值。从图\ref{f:CpCompare}中可以看到在这三个位置THU实验中的压力梯度均大于Reibert实验中的压力梯度,这也正是横流速度峰值不同的主要原因。众所周知,横流的产生正是因为边界层内由主流速度转向产生的离心力不足以平衡压力梯度产生的压差力。所以在压力梯度越强诱导出来的横流就越强。

本文先用传统的$e^N$方法来研究不同算例中的流动稳定性。这里$N$的定义为:
\begin{equation}\label{e:eNdef}
  N=\int_{x_0}^x( -\alpha_i)\,dx,
\end{equation}
这里,$\alpha_i$是采用局部线性稳定性理论(LST)计算得到的流向复波数的虚部。其相反数$-\alpha_i$也就是LST预测出来的模态空间增长率。其物理意义就是下游单位长度距离处失稳模态幅值与当地失稳模态幅值的比值。$x_0$是模态首次失稳的流向位置。所以,这里$N$的物理意义就是,当地扰动模态的幅值相对于模态首次失稳位置的幅值的比的对数。采用这一定义的一个隐藏假设是所有模态在中性点位置通过感受性过程所获得的初始模态幅值是相同的。$e^N$虽然不是十分精确,但是可以定性反应失稳过程。图\ref{fig:CompN:a}给出了不同算例$N$值的包线,也就是不同位置$N$的最大值。其中蓝线和红线分别代表来流速度为44.5与22.5m/s的算例,方块和圆点分别代表采用THU压力分布和采用Reibert压力分布的算例。从结果中可以看到,所有44.5m/s自由来流速度的算例中的$N$值均比对应的22.5m/s自由来流的算例高。这说明了来流速度雷诺数越高,失稳模态增长越快,流动越不稳定。另外,还可以看到,在自由来流速度一样的前提下,采用THU压力分布的算例中的边界层要比采用Reibert压力梯度的更不稳定。可见,横流速度的微弱变化(见图\ref{fig:CompCrossProfiles})都会导致$N$值的剧烈变化。另外,必须要提及的是,如之前所说的,$N$是扰动幅值的对数,用数学式子表达也就是$A=A_0e^{N}$($A_0$是中性点的模态初始值)。所以这里$N$的变化转换到实际的模态幅值上则会是更强烈的变化。图\ref{fig:CompN:b}对比了不同算例中不同位置$N$值最大的定常横流失稳模态的展向波长。这也就是LST所预测的最不稳定模态的展向波长。可以看到,压力梯度的变化并没有对最不稳定模态的展向波长产生较大影响。基于以上分析,这里可以做一个简单的总结,更大的压力梯度会导致更强的横流速度,并进而导致更强的横流失稳模态的增长率。也就是压力梯度越大有横流的边界层越容易失稳。然而,这一变化对最不稳定横流模态的波长并没有显著影响,也就是说主导转捩的横流涡的展向并不会有太大变化。在THU实验中观测到的失稳波长与Reibert文献中的相同,也印证了这里的分析。
\begin{figure}
\centering
\subcaptionbox{$N$值包络线           %
\label{fig:CompN:a}} [0.48\linewidth]%% label for first subfigure
{\includegraphics[width=0.48\linewidth]{ch4/compare-Nmax3}}
%\hspace{0.0in}
\subcaptionbox{不同流向位置$N$值最大模态展向波长        %
\label{fig:CompN:b}} [0.48\linewidth] %% label for second subfigure
{\includegraphics[width=0.48\linewidth]{ch4/compare-lamda}}
\caption{不同算例$e^{N}$方法结果对比}
\label{fig:CompN} %% label for entire figure
\end{figure}

本文选取了上面对比的四个算例中最不稳定的一个进行控制,即自由来流44.5\,m/s同时采用THU压力分布的算例。这样能够更明显的体现出控制的效果。在之后的计算中,自由来流速度44.5\,m/s被选作参考速度。图\ref{fig:Nfactor445}给出了$N$这个基准算例中不同展向波长的横流驻涡模态的$N$值延流向变化的情况。通常,横流模态的$N=6$的时候会触发专列。如果采用这一标准,4mm展向波长的模态会在24\%弦长位置触发转捩。然而,用$N$判断转捩的方法并不十分稳定可靠。有文献指出\cite{saric2011},在前缘抛光,来流湍流度非常低的条件下,临界点$N$值甚至可以高于14。在这个基准算例中,在70\%弦长之前(压力最低点之前)没有任何模态的$N$达到14。在翼型前半段,展向波长为4、5、6mm的模态的$N$值依次领跑。这意味着这几个展向波长的模态都有可能主导转捩。这里需要重点指出的是,3mm展向波长的模态在靠近前缘的位置增长飞快,但是在20\%弦长处达到了峰值之后便开始减弱。在50\%弦长处其幅值甚至小于其初始幅值。波长更小的模态,如2mm展向波长的模态在计算域内几乎不增长。这一短波长高波数模态不增长的特性与之前分析的后掠Hiemenz流动完全不同。在后掠Hiemenz流动中,短波长高波数的模态仅仅是失稳的晚一些,并不会出现完全不失稳的情况。这一特性将在之后的控制中起到非常关键的作用。

$e^{N}$方法是线性稳定性理论下的一种半经验方法,也就是说它还是要依赖小扰动可线性化假设和边界层增长近似为零的平行流假设。但实际上,当扰动的幅值增长到一定程度也就是大约扰动速度达到十分之一自由来流速度的时候,非线性会起作用,失稳进入非线性阶段。本文采用非线性抛物华扰动方程(NPSE)求解非线性阶段的扰动发展变化。在线性稳定性分析得到的几个最有可能主导转捩的模态被用来初始化NPSE计算。这几个模态被放置在计算入口处,并且幅值都设置为$5\times10^{-5}$。这里NPSE中失稳模态的幅值定义为:
\begin{equation}
\mathrm{Amp}=\exp\!\left(\int_{x_0}^x -\alpha _i\,d\xi\right)\max\!\left(\sqrt{\left| \hat{u} \right|^2+\left| \hat{v} \right|^2+\left| \hat{w} \right|^2}\right)_y.
\end{equation}
NPSE计算得到的几个算例中的主模态幅值沿着流向演化的曲线展示在图\ref{f:NPSE}中。这里的做法和后掠Hiemenz流动中选取控制目标模态的方法相同。图\ref{f:NPSE}中的曲线并不是一个算例中不同模态的演化曲线,而是4个主模态不同的算例的结果。在这几个算例中,除了主模态是计算入口就给入的,其他模态都是通过非线性依靠主模态激发出来的高阶谐波模态。从图中可以看到,3mm展向波长的模态的幅值基本上比其他模态的幅值低了一个量级。这和线性稳定性理论预测出来的相似,可见这一模态就并没有能够成功发展进入非线性阶段。5mm展向波长的模态是最先进入饱和的。因此在这一章之后的研究中,均以这一模态作为目标模态。
\begin{figure}
\centering
  % Requires \usepackage{graphicx}
  \includegraphics[width=\textwidth]{ch4/Nvalue(1)}
  \caption{自由来流$U_\infty= 44.5$\,m/s算例中不同展向波长的横流驻涡模态的$N$值延流向变化}%
  \label{fig:Nfactor445}
\end{figure}
\begin{figure}
\centering
  % Requires \usepackage{graphicx}
  \includegraphics[width=0.48\textwidth]{ch4/CompareCsesVmax_Amp0=1e-4(ScaledWithLocalUout0_5)(1)} \includegraphics[width=0.48\textwidth]{ch4/CompareCsesVmax_Amp0=1e-4(ScaledWithLocalUout0_5)}
  \caption{NPSE计算得到的不同展向波长模态幅值延流向变化}\label{f:NPSE}%
\end{figure}


\section{采用等离子体激发器推迟后掠翼上流动转捩}

\subsection{谐波激励:每个展向周期放置一个激发器}\label{subs:control1}
如之前在介绍等离子体模型中提到了,Maden模型中有7个模型参数需要确定,他们分别是$a_0$, $a_1$, $a_2$, $b_0$, $b_1$, $b_2$和$c_{\rm force}$。D\"orr和Kloker\cite{dorr2015stabilisation}指出体积力分布最好分散在边界层内,但是不要超过边界层外缘。图\ref{f:BLvelocityprofile}展示了$X/C=0.15,0.2,0.25$位置处的主流和横流速度。可以看到,这里的边界层厚度大约为1.2mm。在这个厚度以外,主流速度保持为一个常数,横流速度快速下降并回归到零。基于这一基本流的速度剖面分布,本文调整体积力模型中的7个参数,设计了如图\ref{f:forceshape}所示的体积力分布。这一分布对应的模型参数列在了表\ref{t:constantsPmodel}中。其中参数$c_{\rm force}$是用来调整体积力强度的,并不会影响体积力的实际分布。这里一共给出了三个不同的该参数取值,以便在之后的研究中分析体积力强度效应。图\ref{f:forceshape}中的体积力分布对应$c_{\rm force}=30$。可以看到,本文设计的体积力基本完全分布在1.2mm高度以下,也就是完全在边界层内。这一体积力分布的展向长度小于2.5mm,即小于是基础展向波长5mm的一半。$c_{\rm force}=30,, 50, 70$时所对应的有量纲最大体积力密度分别为2986, 4976, 6967\,N/m$^3$,在$y-z$横截面上的总得积分分别为1.467$\times 10^{-3}$, 2.446$\times 10^{-3}$, 3.424$\times 10^{-3}$N/m。需要提及的是,在Kriegseis的实验中\cite{kriegseis2013velocity},最大体积力密度为 7000\,N/m$^3$。可见这里所需的由等离子体激发器产生的体积力强度在实际中完全能够产生。
\begin{table}[htb]
\caption{DBD模型参数}\label{t:constantsPmodel}
%\begin{ruledtabular}
%\begin{tabular*}{\textwidth}{@{\extracolsep{\fill}}ccccccc}
    \begin{center}
    \begin{tabular}{p{1.5cm}p{1.5cm}p{1.5cm}p{1.5cm}p{1.5cm}p{1.5cm}p{2cm}}
    %  \hline
      % after \\: \hline or \cline{col1-col2} \cline{col3-col4} ...
      %\br
      \toprule[1.5pt]
      $a_0$ & $a_1$ & $a_2$ & $b_0$ & $b_1$ & $b_2$ & $c_{\rm force}$ \\\midrule[1pt]
      %\mr
      2.0 & 0.08 & 0.001 & 7.76 & 2.1 & 1.8 & 30,50,70 \\
      %\br
      \bottomrule[1.5pt]
    %  \hline
    \end{tabular}
    \end{center}
%\end{ruledtabular}
\end{table}
\begin{figure}[htb]
\centering
  % Requires \usepackage{graphicx}
  \includegraphics[width=0.48\textwidth]{ch4/Ut(ScaledWithUinf)} \includegraphics[width=0.48\textwidth]{ch4/Wt(ScaledWithUinf)}
  \caption{不同流向位置主流(左)和横流(右)速度剖面}%
  \label{f:BLvelocityprofile}
\end{figure}

\begin{figure}[htb]
\centering
  % Requires \usepackage{graphicx}
  \includegraphics[width=0.8\textwidth]{ch4/abs(bodyforce)}
  \caption{单一DBD激发器产生的体积力分布}%
  \label{f:forceshape}
\end{figure}

\begin{figure}[htb]
\centering
  % Requires \usepackage{graphicx}
  \includegraphics[width=0.6\textwidth]{ch4/bodyforceXZ(y=0_1mm)}
  \caption{DBD激发器阵列产生的体积力在$X$--$Z$平面上的分布($y= 0.1$\,mm) 激发器间距为一个展向波长}%
  \label{f:force_XZ_1perwavelength}
\end{figure}

\begin{figure}[htb]
\centering
  % Requires \usepackage{graphicx}
  \includegraphics[width=0.6\textwidth]{ch4/Vmax_compare(scaledUe)-improved}
  \caption{最佳控制算例与最差控制算例主模态能量延流向演化与无控制算例结果比较}%
  \label{f:bestworst}
\end{figure}
\begin{figure}[htb]
\centering
  % Requires \usepackage{graphicx}
  \subcaptionbox{$z_0/Tz=0.4$\label{f:thebest}}[\textwidth]
  {\includegraphics[width=\textwidth]{ch4/force-position-wt(scaledUinf)_z0=04}
}
  \\\bigskip
  \subcaptionbox{$z_0/Tz=0.9$\label{f:theworst}}[\textwidth]
  {
  \includegraphics[width=\textwidth]{ch4/force_position_wt(scaledUinf)_z0=09}
  }
  \caption{体积力分布(颜色云图)与扰动分布(黑色等值线)的相对位置}
  \label{f:pla_postion}
\end{figure}

在后掠翼算例中,本文首先尝试了与之前后掠Hiemenz流动中相同的谐波激励控制方法,也就是让激发器与横流涡相平行,然后每个展现波长内放置一个激发器。无量纲的体积力$f$ 在$y=0.1$\,mm高度$X$-$Z$平面内的分布如图\ref{f:force_XZ_1perwavelength}所示。这里没有再展示敏感性分析的结果,这是因为其敏感性因子的分布与之前后掠Hiemenz流动中的基本完全相同。这里控制的中心位置在25\%弦长处,控制开始于23.7\%弦长处,结束于26.2\%弦长处。这里体积力强度参数$c_{\rm force}= 30$。本文计算了激发器在十个不同展向位置的结果。其控制效果与之前后掠Hiemenz流动中的类似,这里就不再一一列出。结论也是在横流涡下方偏下扫位置处控制效果较好。这里重点对比了这十个算例中效果最好的和效果最差的以得到其控制的内在机理。图\ref{f:bestworst}展示了最好算例和最差中主模态延流向的演化。其中黑线是无控制的结果,绿线和红线分别是最差算例和最好算例的结果。$T_z$是展向波长,$z_0$是激发器中点的展向位置坐标。可以看到在激发器位于$z_0/T_z=0.4$处,在控制区域横流主模态能量大幅下降,并且在控制之后的区域始终低于无控制工况中的横流主模态能量。而在另一个算例中,激发器位于$z_0/T_z=0.9$,也就是激发器刚好平移了半个波长。这时主模态在控制区域反而被促进了,这也意味之转捩将会被提前。

由于横流涡是斜的,所以横流涡的位置会随着观察的流向位置而变化,所以但看$z_0$是没有意义的,更重要的是看激发器与横流涡的相对位置。在之前后掠Hiemenz流动的控制分析中,本文给出了施加体积力和横流涡的相对位置,这里给出体积力和横流方向扰动速度分布的相对位置,以说明控制的机理。图\ref{f:pla_postion}中颜色云图表示体积力横流方向分量在横截面上的分布,曲线表示在横流方向的扰动速度分量。其中实线表示值为正,虚线表示值为负。可以看到,当把扰动和体积力都投影到横流方向的时候,如果他们的符号相反,也就是力的方向与扰动的速度方向相反(如图\ref{f:thebest}),则主模态的能量就会被降低。相反的,如果外加力与扰动速度在横流方向的投影得到的符号相同,则主模态反而会被促进,如图\ref{f:theworst}。这一结果非常直观,也说明了这种谐波控制的控制方法实际就是通过用外加体积力去抵消因为失稳产生的扰动。当然,由于横流失稳产生的扰动在展向的分布是时正时负,所以激发器的展向位置非常关键。所以这一特性也给实际应用带来了风险。因为并不是控制总能起到效果,有时反而会适得其反。再加上实际中横流涡的位置并不是很容易预测,至少笔者并不知道有效的横流涡展向位置的预测方法。所以,这种谐波激励的方法在笔者看来并不实用,缺少足够的鲁棒性。还需要寻找更鲁棒的方法。

\subsection{Subharmonic control: two actuators per wavelength}\label{subs:control2}
Since it is known that the magnitude of the crossflow velocity greatly influences  crossflow instability, another idea is to use plasma actuators to attenuate the crossflow velocity. To avoid exciting the primary mode, two actuators are positioned per wavelength. The body force distribution in the $X$---$Z$ plane is shown in figure~\ref{f:force2perwavelength}. The number of plasma actuators is doubled compared with the previous scheme. The control region starts at 18.7\% chord length and ends at 21.2\% chord length. The electrodes are still parallel to the isophasal curves of the primary instability mode. $c_\mathrm{force}=50$ in the first case; the cases with  $c_\mathrm{force}=30$ and 70 will be given afterwards.

Figure \ref{f:basecase} shows the evolution of the disturbance energy. The red curves stand for the controlled case and the black curves for the uncontrolled case. The right figure uses a normal coordinate, whereas the left uses a logarithmic coordinate to show the harmonics more clearly. Again, the control region is denoted by two blue vertical lines in both figures, and the center of the region is at 20\% chord length. Since the distance between two neighboring actuators is half the fundamental wavelength, the harmonic mode $(0,2)$, whose wavelength is also half  the fundamental wavelength, is excited directly. It can be seen that there exists a small peak right at the end of the control region. When the mode leaves  the control region, its energy decreases rapidly, and at 30\% chord length its energy becomes two orders of magnitude lower than the peak value. The reason for this energy decline is because this mode with 2.5\,mm wavelength is predicted to be stable by the $e^{N}$ method, meaning that it will die out soon without plasma stimulation. The behaviors of other harmonics,  from $(0,3)$ to $(0,5)$ modes, and all the higher-order harmonics (not shown) are all similar to that of the  $(0,2)$ mode. However, in the middle section of the wing, from 30\% to 40\% chord length, all the modes are weaker than their counterparts from the case without control.
\begin{figure}
\centering
  % Requires \usepackage{graphicx}
  \includegraphics[width=0.6\textwidth]{ch4/bodyforce_Forshowy=0_1mm(twoactuators)}
  \caption{Distribution of  body force in the $X$--$Z$ plane (two actuators per wavelength).}%
  \label{f:force2perwavelength}
\end{figure}
\begin{figure}
\centering
  % Requires \usepackage{graphicx}
  \subcaptionbox{\label{f:basecase_a}}[0.48\textwidth]{

    \includegraphics[width=0.48\textwidth]{ch4/compare_modes_energy(scaledUinf)1-improved}}
  \subcaptionbox{\label{f:basecase_b}}[0.48\textwidth]{

    \includegraphics[width=0.48\textwidth]{ch4/compare_modes_energy(scaledUinf)2-improved}}
  \caption{Evolution of mode energy with and without control.}%
  \label{f:basecase}
\end{figure}

\begin{figure}
\centering
\subcaptionbox{\label{fig:ContU0216WOC}}[0.48\linewidth]{           %
%% label for first subfigure
\includegraphics[width=0.48\linewidth]{ch4/XC=0216(scaledUinf)WOC}}
%\hspace{0.0in}
\subcaptionbox{\label{fig:ContU0216WC}}[0.48\linewidth]{
%% label for second subfigure
\includegraphics[width=0.48\linewidth]{ch4/XC=0216(scaledUinf)WC}}
\caption{Contours of streamwise velocity at $X/C=0.216$ (a) without  and (b) with control.}
\label{fig:ContU0216} %% label for entire figure
\end{figure}

\begin{figure}
\centering
\subcaptionbox{\label{fig:ContU0350WOC}}[0.48\linewidth]{           %
%% label for first subfigure
\includegraphics[width=0.48\linewidth]{ch4/XC=035(scaledUinf)WOC}}
%\hspace{0.0in}
\subcaptionbox{\label{fig:ContU0350WC}}[0.48\linewidth]{
%% label for second subfigure
\includegraphics[width=0.48\linewidth]{ch4/XC=035(scaledUinf)WC}}
\caption{Contours of streamwise velocity at $X/C=0.35$ (a) without  and (b) with control.}
\label{fig:ContU0350} %% label for entire figure
\end{figure}
%\clearpage %REMEMBER TO DELATE THIS AFTER YOU ADD ALL WORDS IN THIS PAPER!!!!!!!!!!!!!!!!!!!!!!!!!!!!!!!!!!!
Figures \ref{fig:ContU0216} and \ref{fig:ContU0350} show  contours of the streamwise velocity at $X/C=0.216$ and 0.35, respectively. At $X/C=0.216$, close to the end of the control region $X/C=0.212$, the boundary layer looks quiet and clean without control. The instability modes are very weak there. When the plasma is induced, small waves are generated, seen in figure~\ref{fig:ContU0216WC}. These small waves are mainly caused by the  (0,2) mode and have  wavelength  2.5\,mm. At 35\% chord length, in the case without control, a strong crossflow vortex appears and  convects low-momentum fluid away from the wall. A rollover structure that indicates the beginning of the saturation stage also appears. However, for the controlled case, there appear only small ripples, and no strong convection emerges. From these figures, it can be concluded that even though the plasma actuators do not affect the primary mode directly, their effects do ultimately hinder the evolution of crossflow vortices.
\begin{figure}
\centering
  % Requires \usepackage{graphicx}
\includegraphics[width=0.24\textwidth]{ch4/compare_Wt_XC=025(scaledUinf)2}
\includegraphics[width=0.24\textwidth]{ch4/compare_Wt_XC=030(scaledUinf)2}
\includegraphics[width=0.24\textwidth]{ch4/compare_Wt_XC=035(scaledUinf)2}
\includegraphics[width=0.24\textwidth]{ch4/compare_Wt_XC=040(scaledUinf)2}
\caption{Crossflow velocity profiles.}%
\label{f:CFprofiles}
\end{figure}

Figure \ref{f:CFprofiles} shows the crossflow velocity profiles at different streamwise locations. The blue curves stand for the crossflow velocity profiles of the baseflow. The black curves represent the uncontrolled case, and they deviate from the baseflow profile owing to the mean flow distortion mode, namely, the $(0,0)$ mode. The red curves stand for the controlled case. It can be seen that at $X/C=0.25$, the black curve coincides with the blue one, because all the disturbance modes, including the mean flow distortion mode $(0,0)$, are weak there. Meanwhile, since the direction of the plasma-induced body force is  opposite to that of the crossflow velocity, the profile in the controlled case is lower than in the other two cases. The situation is similar at $X/C=0.3$. The controlled crossflow profile grows marginally, but it is still lower than that in the other two cases. From 25\% to 30\% chord length, all the instability modes in the controlled case grow slower than their counterparts in the uncontrolled case, and some of them even shrink [see figure~\ref{f:basecase_b}]. This is mainly attributed to the decrease in the crossflow velocity profile. At 35\% and 40\% chord length,  the effect of nonlinearity  promotes  distortion of  the mean flow and a decrease in the crossflow for the uncontrolled case. However, since the development of these instability modes is hindered in the controlled case, the nonlinearity is not significant and thus the distortion of the controlled baseflow is not as intense as that in the case without control. Then, the controlled crossflow become stronger than in the case without control, as can be seen in the last picture in figure~\ref{f:CFprofiles}.
\begin{figure}
\centering
  % Requires \usepackage{graphicx}
\includegraphics[width=0.48\textwidth]{ch4/(scaledUinf)Amp0=1e-4_c_force=VARYING_X0C=020_dzdTz=00_Nplasma=2-improved}
\includegraphics[width=0.48\textwidth]{ch4/(scaledUinf)Amp0=1e-4_c_force=VARYING_X0C=020_dzdTz=00_Nplasma=2___zoomin-improved}
\caption{Evolution of mode energy for different body force strengths.}%
\label{f:forcestrength}
\end{figure}

The effect of force strength  is studied by varying the coefficient $c_{\rm force}$, and the results are shown in figure~\ref{f:forcestrength}. The right figure zooms in the vicinity of the control region in the left figure. The orange, red, and green curves stand for the cases with $c_{\rm force}=30$, 50, and 70, respectively. In all the controlled cases, the energies of the primary modes and the $(0,2)$ modes are all lower than those in the case without control, and a stronger body force results in weaker instability. A stronger body force also leads to higher peak value of the energy of the harmonic $(0,2)$ mode near the end of the control region. It can be seen that the peak values are $2 \times 10^{-4}$, $8 \times 10^{-4}$, and  $2 \times 10^{-3}$ for the cases with $c_{\rm force}=30$, 50, and 70, respectively.

Figure \ref{f:streamwiselocationeffect} compares the results for cases with actuators positioned at different streamwise locations. The green, red, and orange curves stand for the cases with DBD centers located at 15\%, 20\%, and 25\% chord length. The control regions are not plotted on the figure, but they can still  be recognized by the small peaks on the dashed curves denoting the $(0,2)$ instability modes, because, like all the other cases shown previously, in the vicinity of the control region, the $(0,2)$ modes are all excited. The peak of the $(0,2)$ mode in the case with excitation at 15\% chord length is lower than that at 20\% chord length, which in turn is lower than that at 25\% chord length. The reason is that the amplitude of the uncontrolled $(0,2)$ mode is larger  downstream, and thus, when it is excited by the same force, the originally strong mode reaches an even higher level. From 30\% to 40\% chord length, all the mode energies for the case controlled at 15\% chord length are lower than their counterparts in the other two cases. The primary mode controlled by the DBD actuators at 25\% chord length does not deviate from the primary mode without control until 33\% chord length. Fortunately, its energy decreases after that. No matter where the actuators are placed, all the mode energies are lower than in the original case without control.
\begin{figure}
\centering
  % Requires \usepackage{graphicx}
\includegraphics[width=0.48\textwidth]{ch4/(scaledUinf)Amp0=1e-4_c_force=50_X0C=VARYING_dzdTz=00_Nplasma=2-improved}
\includegraphics[width=0.48\textwidth]{ch4/(scaledUinf)Amp0=1e-4_c_force=50_X0C=VARYING_dzdTz=00_Nplasma=2____zoomin-improved}
\caption{Evolution of mode energies with plasma actuators at different streamwise locations.}%
\label{f:streamwiselocationeffect}
\end{figure}
\begin{figure}
\centering
  % Requires \usepackage{graphicx}
\includegraphics[width=0.48\textwidth]{ch4/(scaledUinf)Amp0=1e-4_c_force=50_X0C=020_dzdTz=VARYING_Nplasma=2-improved}
\includegraphics[width=0.48\textwidth]{ch4/(scaledUinf)Amp0=1e-4_c_force=50_X0C=020_dzdTz=VARYING_Nplasma=2___zoomin-improved}
\caption{Evolution of mode energies with plasma actuators at different spanwise locations.}%
\label{f:spanwiselocationeffect}
\end{figure}
\begin{figure}
\centering
  % Requires \usepackage{graphicx}
\includegraphics[width=0.24\textwidth]{ch4/XC=025_Modified_baseflow(scaledUinf)-z0Tz=025}
\includegraphics[width=0.24\textwidth]{ch4/XC=030_Modified_baseflow(scaledUinf)-z0Tz=025}
\includegraphics[width=0.24\textwidth]{ch4/XC=035_Modified_baseflow(scaledUinf)-z0Tz=025}
\includegraphics[width=0.24\textwidth]{ch4/XC=040_Modified_baseflow(scaledUinf)-z0Tz=025}
\caption{Comparison of modified mean flow profiles with plasma actuators  at different spanwise locations.}%
\label{f:Wt_SpVar}
\end{figure}

It has already been mentioned in Section~\ref{subs:control1} that the controlled results with one plasma actuator per wavelength are remarkably sensitive to the spanwise location of the actuators. The sensitivity of the control method with two actuators per wavelength to  spanwise location is examined here, as shown in figure~\ref{f:spanwiselocationeffect}. Since the wavelength of the array of plasma actuators is half  the fundamental wavelength $T_z$, the actuators are moved one-quarter of the fundamental wavelength reversed for phase in the spanwise direction. The red curves stand for the original case, and the green curves represent the new case with the  spanwise location of the actuators shifted. Again, the energy of the $(0,2)$ mode increases in the control region and decreases elsewhere. The energies of  the $(0,1)$ modes in both cases remain the same in the control region. However, they begin to deviate from each other just slightly downstream of the control region. From 25\% to 45\% chord length, there is only a small difference between these two curves. In addition, both of them are below the black curve, the one without control, indicating that the spanwise locations of the actuators are not crucial.

To explain the slight difference between the primary modes from the cases with and without a shift in the spanwise location of the actuators, the mean flow profiles in the crossflow direction are compared in figure~\ref{f:Wt_SpVar}. The blue  and  red curves denote the mean crossflow velocity profiles in the cases with and without a spanwise shift of the  actuators, respectively. At 25\% and 30\% chord length, the red curves are perfectly superposed on the blue curves. It should be recalled that at these two streamwise locations, the energies of the primary modes have already deviated from each other. Hence, it can be concluded that the difference between the $(0,1)$ modes in the two cases is not caused by the mean crossflow velocity profile.
\begin{figure}
\centering
  % Requires \usepackage{graphicx}
\includegraphics[width=0.6\textwidth]{ch4/force_XZ(scaledUinf)-reversed}
\caption{Body force distribution of inverse plasma actuators in the $X$--$Z$ plane.}%
\label{f:force_reversed}
\end{figure}

To date, it is clear that  manipulation of the $(0,0)$ mode \cite{Saric1998} or the $(0,2)$ mode can result in a decrease in the energy of the primary mode. In the DBD plasma actuators control scheme presented here, the $(0,0)$ and $(0,2)$ modes are both altered directly and the primary mode is  affected only downstream of the control region. It is not clear which mode, $(0,0)$ or $(0,2)$, contributes more to the decline in the energy of the primary mode. To answer this question, a reversed control case is examined. All the DBD actuators are turned 180$^\circ$, with  the body force in the opposite direction. In the computation, the force appears as a source term. When analyzing each mode, the force term is decomposed into Fourier series with respect to the spanwise coordinate. These Fourier components affect the corresponding instability modes. For instance, the zeroth-order Fourier component affects the $(0,0)$ mode directly and the second affects the $(0,2)$ mode. For this reverse control, the sign of the force term and that of its Fourier component are switched. Owing to the properties of trigonometric functions, the sign switch of the second Fourier component is equivalent to a phase shift. This phase shift effect has been investigated above by comparing  results with actuators at different spanwise locations, and has been shown to be trivial. Thus, the biggest difference in this reverse control is that the sign of the force term corresponding to the $(0,0)$ mode is switched. If this reverse control still works and reduces the energy of the primary mode, then the $(0,2)$ mode rather than the $(0,0)$ mode will play a more important role in our control scheme. Otherwise, the conclusion will be that the $(0,0)$ mode is more important.

The body force distribution in the $X$--$Z$ plane is shown in figure~\ref{f:force_reversed}. The evolution of the mode energies in reverse control cases is shown in figure~\ref{f:model_energy_revers}. As mentioned above, reversal of the force direction  will also lead to a phase shift of the Fourier component corresponding to the $(0,2)$ mode. This phase shift can be achieved by moving the actuators in the spanwise direction. To eliminate this small ambiguity, actuators located at $Z_0/T_z=0.0$ and $Z_0/T_z=0.25$ are both simulated. It can be seen that in both cases the energies of the primary modes are higher than that in the controlled case. Also, the effect of actuator spanwise location  is not significant, and this agrees well with the conclusion reached above (see Fig.~ \ref{f:Wt_SpVar}). This result indicates that the $(0,0)$ mode is more important than the $(0,2)$ mode and  is the main cause of the decline in the energy of the primary mode.
\begin{figure}
\centering
  % Requires \usepackage{graphicx}
\includegraphics[width=0.48\textwidth]{ch4/(scaledUinf)Amp0=1e-4_c_force=50_X0C=020_dzdTz=VARYING_Nplasma=2_revers-improved}
\includegraphics[width=0.48\textwidth]{ch4/(scaledUinf)Amp0=1e-4_c_force=50_X0C=020_dzdTz=VARYING_Nplasma=2_revers___zoomin-improved}
\caption{Evolution of mode energy in reversed control cases (with actuators at two different locations).}%
\label{f:model_energy_revers}
\end{figure}

\begin{figure}
\centering
  % Requires \usepackage{graphicx}
\includegraphics[width=0.32\textwidth]{ch4/XC=025_Modified_baseflow_reversed(scaledUinf)}
\includegraphics[width=0.32\textwidth]{ch4/XC=030_Modified_baseflow_reversed(scaledUinf)}
\includegraphics[width=0.32\textwidth]{ch4/XC=035_Modified_baseflow_reversed(scaledUinf)}
\caption{Mean crossflow profile in the reversed control case.}%
\label{f:inverse_meanflow}
\end{figure}

Figure \ref{f:inverse_meanflow} shows the mean crossflow velocity profiles at different streamwise locations in the reverse control case. It is obvious that the crossflow is enhanced at 25\% chord length, just downstream of the control region. Thereafter, the crossflow falls back and finally returns to the same level as that in the case without control. From this result, it can be concluded that the crossflow velocity has a significant effect on  crossflow instability. When the crossflow is weakened by the actuator, the instability is attenuated. Otherwise, the instability is intensified.
\begin{figure}
\centering
  % Requires \usepackage{graphicx}
\includegraphics[width=0.6\textwidth]{ch4/(scaledUinf)Amp0=1e-4_c_force=70_X0C=VARYING_dzdTz=0000_Nplasma=3-improved}
\caption{Evolution of mode energy in cases targeted at the 7.5\,mm wavelength mode.}%
\label{f:7.5mm}
\end{figure}
\subsection{Off-designed case}
All the simulations shown above assume that the mode with 5\,mm spanwise wavelength dominates the transition, and the 2.5\,mm mode, happens to be the $(0,2)$ mode with respect to the 5\,mm fundamental wavelength. Here, another situation is considered in which the 7.5\,mm mode becomes dominant but the distance between two neighboring actuators is still 2.5\,mm. Thus, the control mode is the $(0,3)$ mode. Plasma actuators are positioned at three different streamwise locations, and the evolution of the mode energy is shown in figure~\ref{f:7.5mm}. The green, red, and orange curves stand for the cases controlled at 15\%, 20\%, and 25\% chord length, respectively. The small peak in the energy of the $(0,3)$ mode, that is, the control mode, becomes stronger and stronger when the actuators are moved downstream. Fortunately, all the modes in all the controlled cases are weaker than those in the uncontrolled case downstream of 30\% chord length. This result proves that the presented control method performs well even in an un-designed case.

\chapter{等离子体激发器控制充分发展槽道湍流}
本章研究应用DBD激发器控制壁湍流的方法。基准算例选取为壁面摩擦尺度雷诺数Re$_\tau=180$(Re$_m=5600$)的湍流槽道。计算程序采用的是王志坚课题组的hpMusic\cite{WangZJ2009,Zhu2016,Zh2017},时间步长dt=0.00015,用三阶的显式Eular进行时间推进。空间采用4阶精度计算,单元内采用高斯点,每个单元内有$5\times5\times5$个点,三个方向总的自由度数分别为$235\times155\times200$。在后处理的时候,对单元界面上空间位置相同的点进行了平均。平均处理之后用于显示的网格点数与Kim(1987)\cite{Kim1987}文献中的网格相近。计算域大小为$4\pi \times 2\times 2\pi$。$\gamma$=1.4, 气体常数为1.0,普朗特常数Pr = 0.72,粘性系数 $\mu=3.57\times10^{-4}$, Ma=0.1。计算时在流场内添加全场均匀的体积力,使得槽道内的质量平均流速在稳定在1.0。计算的初始条件为抛物线形速度刨面,流速峰值$u_{\rm max} = 1.327$。为了加快转捩,在这个流场上叠加上10个不同流向展向波数的扰动波。初始的密度$\rho=1.0$,压力$p= 31.75$。计算得到的近壁涡结构和条带结构如图\ref{f:nearwallvortex}和\ref{f:nearwallstreak}。其中用于展示涡结构的Q值等值面取了与文献中相同的值(Q=0.5)。涡结构上的颜色采用高度进行渲染。从图中可以清楚的看到近壁区的流向条带,以及这些条带经过发展,抬高之后形成的发卡涡。图\ref{f:nearwallstreak}中的流向脉动速度云图取自$y^+=5$。可以看到本文计算得到的近壁结构基本与文献中的类似。
\begin{figure}[htb]
  \centering
  \subcaptionbox{本文计算结果}[0.45\textwidth] %标题的长度,超过则会换行,如下一个小图。
    {\includegraphics[width=0.45\textwidth]{ch5/VorStru_Nocontrol.jpg}}%
  \subcaptionbox{Wei和Pollard\cite{Wei2011}计算结果}[0.45\textwidth]
    {\includegraphics[width=0.45\textwidth]{ch5/VorStru_Nocontrol_ref.jpg}}%
  \caption{计算得到涡结构对比}\label{f:nearwallvortex}
\end{figure}
\begin{figure}[htb]
  \centering
  \subcaptionbox{本文计算结果}[0.5\textwidth] %标题的长度,超过则会换行,如下一个小图。
    {\includegraphics[width=0.5\textwidth]{ch5/streaks_nocontrol.jpg}}%
  \subcaptionbox{Wei和Pollard\cite{Wei2011}计算结果}[0.4\textwidth]
    {\includegraphics[width=0.4\textwidth]{ch5/streaks_paper.jpg}}%
  \caption{计算得到条带结构对比}\label{f:nearwallstreak}
\end{figure}

计算得到的对数律分布如图\ref{f:loglow}。其为采用190万瞬时结果进行时间平均和空间平均(流向和展向)之后的结果。其中线性区对比的理论公式为:
\begin{equation}\label{e:linear_region}
  U^+=y^+
\end{equation}
对数区对比的理论公式为:
\begin{equation}\label{e:loglaw}
  U^+=2.5{\rm ln}y^++5.5
\end{equation}
可以看到,在$y^+<5$的区域,平均流速分布基本上完全符合线性关系,在$20<y^+<110$的区间,平均速度与对数律完美吻合。另外在图中同时画出了上壁面和下壁面的平均流速分布,y在这里均表示测量点到壁面的距离。上下壁面的速度分布分别用红线和黑线表示。可以看到,图中红线完全将黑线覆盖。
\begin{figure}[htb]
  \centering
  \includegraphics[width=0.7\textwidth]{ch5/loglow.jpg}
  \caption{计算结果与对数律对比}\label{f:loglow}
\end{figure}

计算得到的二阶统计量与文献对比如图\ref{f:2orderaver}。这四张图依次是流向、法向、展向脉动速度均方根,以及雷诺切应力$<uv>^+$。这些量均采用壁面摩擦尺度进行无量纲化。其中红线均来自本文的计算结果,绿线为文献\cite{Kim1987}中给出的结果。可以看到两条线基本上完全重合,这表明本文的计算精度是满足要求的。
\begin{figure}[htb]
  \centering
  \subcaptionbox{$u_{rms}^+$}[0.45\textwidth] %标题的长度,超过则会换行,如下一个小图。
    {\includegraphics[width=0.45\textwidth]{ch5/urms_NoControl.jpeg}}%
  \subcaptionbox{$v_{rms}^+$}[0.45\textwidth]
    {\includegraphics[width=0.45\textwidth]{ch5/vrms_NoControl.jpeg}}%
  \\\bigskip
  \subcaptionbox{$w_{rms}^+$}[0.45\textwidth] %标题的长度,超过则会换行,如下一个小图。
    {\includegraphics[width=0.45\textwidth]{ch5/wrms_NoControl.jpeg}}%
  \subcaptionbox{$<uv>^+$}[0.45\textwidth]
    {\includegraphics[width=0.45\textwidth]{ch5/uv_NoControl.jpeg}}%
  \caption{计算得到的二阶统计量与文献中的结果对比}\label{f:2orderaver}
\end{figure}
\begin{figure}[htb]
  \centering
  \includegraphics[width=\textwidth]{ch5/condit_aver.jpeg}
  \caption{$y^+=20$条件平均}\label{f:base_condition_average}
\end{figure}

前人的研究\cite{Hamilton1995}指出,壁湍流的产生来自近壁拟序结构的自维持机制。也就是流向涡通过上抛和下扫产生高速和低速条带,然后高速和低速条带又由于失稳机制再次破碎产生新的流向涡。虽然从瞬时涡结构中也能观察到近壁的流向涡,但是其分布过于杂乱无章,不便于分析。这里采用一种条件平均技术对结果进行后处理。首先对每一个瞬时结果在$y^+=20$的平面上进行涡探测。之所以选择$y^+=20$是因为流向涡量均方极大值位于这个高度\cite{Jeong1997},所以也被普遍认为是流向涡涡心所处在的高度。探测时采用$Q=1$作为涡的识别标准。探测到涡之后,选取涡内$Q$的最大值作为涡心。之后将所有探测得到的涡的涡心的位置置为$z=0$,然后对涡心附近的流场进行平均,得到图\ref{f:base_condition_average}。这里只平均了涡心处流向涡量为正的涡。负涡量的涡与之对称,这里不再展示。图中最中心的粗黑线是$Q$的等值线,最外面一圈是$Q=1$也就是本文用来判断涡的准则。图上平均之后$Q$的最大值为3.87。图中的细线是横截面内的流线,可以看到$Q$判据在这里非常准确的预测了这个流向涡。图中的颜色云图是流向速度脉动。这里颜色的显示为-0.1到+0.1。可以看到图中左侧的高速条带和右侧的低速条带关于涡并不对称。低速条带中心大致位于$y^+=20,z^+=-15$位置处,高速条带中心大致位于$y^+=10,z^+=+15$位置处。脉动速度在正负值上的分布也不是对称的,最小值为-0.13,最大值为0.10。在之后的控制中,本文也会用类似的条件平均方法,分析其控制机理。

\section{定常激励控制方案}
这一小节主要介绍采用定常激励的DBD激发器降低湍流槽道阻力的控制方案和相关计算结果。首先,这里简要介绍一下控制方案的灵感来源。在2016年,Canton等人\cite{Canton2016}提出了一种用大涡形状分布的体积力进行减阻的方案,图\ref{f:vor_f}为其体积力分布示意图。其中箭头表示力的方向。这种形状的体积力在槽道内的分布可以完全用解析式表达:
\begin{subequations}\label{e:vvf_equations}
\begin{align}
  F_y \left( {y,z} \right) &= A\beta \cos \left( {\beta z} \right)\left( {1 + \cos \left( {\pi y/h} \right)} \right)\\
  F_z \left( {y,z} \right) &= A\pi /h\sin \left( {\beta z} \right)\sin \left( {\pi y/h} \right)
\end{align}
\end{subequations}
其中$F_y,F_z$分别表示体积力法向和展向的分量,$A$是添加体积力的强度系数,$\beta$是展向波数,$h$是半槽宽,在本文的计算中就是1。
在他们的工作中,最初研究的并不是采用体积力,而是采用通过数值的手段在进行数值模拟的过程中强制在槽道内固定一个大涡。他们的结果显示,无论是体积力产生的还是强制数值方法产生的,只要槽道内有大涡,并且参数满足一定条件,就能够减阻。
\begin{figure}[htb]
  \centering
  \includegraphics[width=0.8\textwidth]{ch5/vortex.JPG}
  \caption{Canton等人\cite{Canton2016}采用的减阻控制的体积力}\label{f:vor_f}
\end{figure}
\begin{figure}[htb]
  \centering
  \includegraphics[width=0.8\textwidth]{ch5/parameter_study.JPG}
  \caption{Canton等人\cite{Canton2016}给出的涡强度与减阻率的关系}\label{f:vort_strength_vs_DR}
\end{figure}

他们还对这种减阻控制方式进行了参数研究。他们的结果表明,最终所产生的二次涡的强度要在一定的范围内才能有效减阻。除此之外,涡的展向波长也是决定减阻率的关键参数。本文也复现了他们文章中提到的一个减阻算例,其参数设定为$\beta=1,A=5\times10^4$。在后文中将这一算例命名为“Vortex Force”。除了这个算例,作者还计算了其十倍强度的大涡状体积力控制算例。但是涡变强后阻力反而增加了14\%,这里就不再展示。

虽然这种大涡状的体积力可以减阻,但是并不现实,因为实际中并不能产生这种形状的体积力。所以本文接下来重点研究采用DBD是否可以产生相同效果。这里依然选用Maden等人2013年提出来的模型,与之前研究后掠翼转捩推迟中用到的模型相同。之所以不用更实际一点的模型主要是因为槽道本身就是一个理论模型算例,其长度速度等都是无量纲的,一般也不会用专门的实际工况去与之对应。所以这里还是选用了可以调整参数的近似体积力分布模型,做理论方面的研究。本文中采用的DBD生成的体积力分布如图\ref{f:steady_force}所示。其中红色表述展向正方向的体积力,也就是向右侧的,蓝色表示展向负方向的体积力,也就是向左侧的。这里仅展示分布的形状,具体的强度在不同的计算工况中不一样。模型用到的参数列在表\ref{t:parameters_steadycontrol}中。
\begin{figure}[htb]
  \centering
  \includegraphics[width=0.8\textwidth]{ch5/steady_force.jpg}
  \caption{定常激励采用的DBD体积力示意图}\label{f:steady_force}
\end{figure}

可以看到,图\ref{f:steady_force}中仅仅用到了两个激发器。这是因为笔者希望用于控制的结构尽可能的简单。只在下壁面布置是因为将来希望将这种控制方法推广到湍流边界层中,所以这里只关注下壁面的阻力。之后提到的阻力也都只是流体作用在下壁面的阻力。

\begin{table}
  \centering
  \caption{DBD激发器体积力模型参数选择}\label{t:parameters_steadycontrol}
  \begin{tabularx}{\linewidth}{XXXXXXc}
    \toprule[1.5pt]
    $a_0$ & $a_1$ &$a_2$ &$b_0$ &$b_1$ &$b_2$ &$c_{\rm force}$  \\\midrule[1pt]
    2     & 0.1   & 0.05 & 9    & 0.07 & 0.002& 10$^4$或$2.5\times10^3$ \\
    \bottomrule[1.5pt]
  \end{tabularx}
\end{table}
\begin{table}
  \centering
  \caption{不同算例产生的涡强度}\label{t:vortexStrength}
  \begin{tabularx}{\textwidth}{*2{>{\centering\arraybackslash}X}}
    \toprule[1.5pt]
    {\heiti 算例名称} & {\heiti max$(V)/U_b$} \\\midrule[1pt]
        Plasma(Strong) & 0.19 \\
        Plasma(Weak) & 0.05 \\
        Vortex Force & 0.06 \\
    \bottomrule[1.5pt]
  \end{tabularx}
\end{table}
由于最开始并不知道体积力诱导的涡强度与真实涡强度的关系,所以做了很多参数试验。试验发现过大的体积力反而会起到增阻的的作用,和前人研究的用大涡状的体积力进行控制的规律相同。这里仅展示一个增阻的算例和一个减阻的算例,在后文中把他们分别记做算例Plasma(Strong)和算例Plasam(Weak)(分别对应表\ref{t:parameters_steadycontrol}中$c_{\rm force}=10^4$和$2.5\times10^3$)。他们的产生的涡强度如表\ref{t:vortexStrength}。这里袭承了Canton\cite{Canton2016}等人的做法,采用max$(V)/U_b$表征产生的涡的强度。在这里也可以用其他衡量涡强度的方法,比如半个展向周期内横截面上的涡量积分。然而实际稍加推导,就可以发现其实差别不大。例如若采用半个横截面上的流向涡量积分作为涡强度的判据,稍微应用一下高斯-奥斯特罗格拉德斯基公式,就可以将其转换为环绕这半个横截面的环线积分。由于壁面上速度为零,所以这个环线积分就变成了槽道中心线和周期边界上法向速度$V$的积分。实际计算发现不同算例这两个位置法向速度$V$的分布变化不大,所以采用其极值max$(V)$作为涡强度大小的量度和采用半个展向周期内横截面上的涡量积分作为涡强度大小的量度没有太大区别。另外也是为了和文献中的结果进行对比,这里依然采用max$(V)/U_b$来表征涡的强度。

\begin{figure}[htb]
  \centering
  \subcaptionbox{平均法向速度\label{f:nobackV}}[0.58\textwidth]%标题的长度,超过则会换行,如下一个小图。
    {\includegraphics[width=0.58\textwidth]{ch5/V_nobackflow.jpg}}%
  %\\\bigskip
  \subcaptionbox{平均展向速度与流线\label{f:nobackW}}[0.58\textwidth]
    {\includegraphics[width=0.58\textwidth]{ch5/W_nobackflow.jpg}}%
  \caption{定常DBD在无背景流动的槽道中诱导出来的流场}\label{f:nobackflow}
\end{figure}
在没有背景流动的槽道中诱导出来的流场如图\ref{f:nobackflow}。其中图\ref{f:nobackV}是法向平均速度云图,图\ref{f:nobackW}是展向平均速度云图。图\ref{f:nobackW}中还画出了这个横截面内的流线。从法向速度的分布云图中,可以看到体积力诱发出来的向上的流动和向下的流动并不对称。其中向下的流动分布的范围并不广,只集中在槽道中间很小的一部分,而向上的流动则铺开在很大的范围内。另外,向下的最大流速比向上的最大流速大很多。正的最大法向速度大约0.01而负的是-0.03。可见这样布置的体积力分布会产生一个很强的将流体从上往下吸的力。展向高速流动主要集中在体积力作用区。从流线上可以发现,DBD产生的二次涡的涡心更加靠近壁面。

各个算例下壁面阻力对比如图\ref{f:darg_steady}所示。其中绿线是无控制工况的。所有控制激励都是在无量纲时间到了620的时候打开的。之后红、蓝、灰分别是Plasma(Weak)、Plasma(Strong)和Vortex Force三个算例的结果。可以看到,在没有控制的时候下壁面总的摩阻为0.32,采用大涡形体积力将阻力降低到0.27左右,采用较弱的等离子激励将阻力降低到0.295左右。然而当体积力较强时,阻力反而升到了0.36。另外这三个工况阻力的波动幅值都没有明显的变化。这一现象将会结合流场内的涡结构图加以说明。
\begin{figure}
  \centering
  \includegraphics[width=0.5\textwidth]{ch5/FX_steady_forcing.jpeg}
  \caption{各个算例下壁面阻力对比}\label{f:darg_steady}
\end{figure}

图\ref{f:aver_velocity_steady1}、\ref{f:aver_velocity_steady2}和\ref{f:aver_velocity_steady3}对比了三个定常激励算例的部分一二阶统计量。这里从前到后依次是平均流向速度$U$、平均法向速度$V$、平均展向速度$W$、雷诺切应力$<uv>$和雷诺主应力$<uu>$及$<vv>$。对于流向平均速度$U$,三个算例使用的是相同的云图颜色划分。如果没有控制,平均流向速度在展向应该是均匀的。所以在控制算例中平均流向速度在展向分布的不均匀性也正说明了这种大涡产生的高低速流体之间对流的强弱。Plasma(Strong)算例是涡强度最大的算例,可以看到其流向平均速度分布与另外两个最大的不同是,代表最大速度的红色区在槽道中间被分开了。这也正是说明由于大涡强烈的将近壁的低速流体卷起,甚至将一部分低速流体带到了槽道中心。而另外两个算例,槽道中心的高速流动区域仅仅是被向另一侧挤压了些。另外在Plasma(Weak)算例中,槽道中部,也就是$z=\pi$位置处的高速流体连接没有被切断,反而是周期边界处的高速流体连接被阻隔了。这是因为等离子体仅仅在下壁面安装,而下壁面的上抛现象正好发生在周期边界位置处。
\begin{figure}[htb]
  \centering
  \subcaptionbox{Vortex Force: $U$}[0.49\textwidth]
    {\includegraphics[width=0.49\textwidth]{ch5/VortexF/U.png}}
  \subcaptionbox{Vortex Force: $V$}[0.49\textwidth]
    {\includegraphics[width=0.49\textwidth]{ch5/VortexF/V.png}}
  \\\bigskip
  \subcaptionbox{Plasma(Strong): $U$}[0.49\textwidth]
    {\includegraphics[width=0.49\textwidth]{ch5/PlasmaStronge/U.png}}
  \subcaptionbox{Plasma(Strong): $V$}[0.49\textwidth]
    {\includegraphics[width=0.49\textwidth]{ch5/PlasmaStronge/V.png}}
  \\\bigskip
  \subcaptionbox{Plasma(Weak): $U$}[0.49\textwidth]
    {\includegraphics[width=0.49\textwidth]{ch5/PlasmaWeak/U.png}}
  \subcaptionbox{Plasma(Weak): $V$}[0.49\textwidth]
    {\includegraphics[width=0.49\textwidth]{ch5/PlasmaWeak/V.png}}
  \caption{平均速度对比}\label{f:aver_velocity_steady1}
\end{figure}

在图\ref{f:aver_velocity_steady1}右侧一列展示的平均法向速度里,不同的算例采用的是不同的显示值域范围。这是因为平均法向速度完全是外加的体积力诱导出来的,其强度反映了生成的二次涡的强度,受体积力强度的影响很大。为了能够让每一个算例都清楚的显示,这里都分别选择了它们各自合适的值域范围。可以看到,三个算例中在槽道中心$z=\pi$处都有一块高速向下的流动区域。不同的是,在算例Vortex Force 和Plasma(Strong)中,这个区域呈蝴蝶形分布,而Plasm(Weak)中是椭圆形的。
\begin{figure}[htb]
  \centering
  \subcaptionbox{Vortex Force: $W$}[0.49\textwidth]
    {\includegraphics[width=0.49\textwidth]{ch5/VortexF/W.png}}
  \subcaptionbox{Vortex Force: $<uv>$}[0.49\textwidth]
    {\includegraphics[width=0.49\textwidth]{ch5/VortexF/uv.png}}
  \\\bigskip
  \subcaptionbox{Plasma(Strong): $W$}[0.49\textwidth]
    {\includegraphics[width=0.49\textwidth]{ch5/PlasmaStronge/W.png}}
  \subcaptionbox{Plasma(Strong): $<uv>$}[0.49\textwidth]
    {\includegraphics[width=0.49\textwidth]{ch5/PlasmaStronge/uv.png}}
  \\\bigskip
  \subcaptionbox{Plasma(Weak): $W$}[0.49\textwidth]
    {\includegraphics[width=0.49\textwidth]{ch5/PlasmaWeak/W.png}}
  \subcaptionbox{Plasma(Weak): $<uv>$}[0.49\textwidth]
    {\includegraphics[width=0.49\textwidth]{ch5/PlasmaWeak/uv.png}}
  \caption{平均展向速度与雷诺切应力}\label{f:aver_velocity_steady2}
\end{figure}
\begin{figure}[htb]
  \centering
  \subcaptionbox{Vortex Force: $<uu>$}[0.49\textwidth]
    {\includegraphics[width=0.49\textwidth]{ch5/VortexF/uu.png}}
  \subcaptionbox{Vortex Force: $<vv>$}[0.49\textwidth]
    {\includegraphics[width=0.49\textwidth]{ch5/VortexF/vv.png}}
  \\\bigskip
  \subcaptionbox{Plasma(Strong): $<uu>$}[0.49\textwidth]
    {\includegraphics[width=0.49\textwidth]{ch5/PlasmaStronge/uu.png}}
  \subcaptionbox{Plasma(Strong): $<vv>$}[0.49\textwidth]
    {\includegraphics[width=0.49\textwidth]{ch5/PlasmaStronge/vv.png}}
  \\\bigskip
  \subcaptionbox{Plasma(Weak): $<uu>$}[0.49\textwidth]
    {\includegraphics[width=0.49\textwidth]{ch5/PlasmaWeak/uu.png}}
  \subcaptionbox{Plasma(Weak): $<vv>$}[0.49\textwidth]
    {\includegraphics[width=0.49\textwidth]{ch5/PlasmaWeak/vv.png}}
  \caption{流向与法向雷诺正应力}\label{f:aver_velocity_steady3}
\end{figure}

图\ref{f:aver_velocity_steady2}左侧是平均展向速度,右侧是雷诺切应力。可以看到,由于DBD激发器仅仅加在下壁面,所以平均展向速度在下壁面附近有两处绝对值较大的集中分布区,而在上壁面附近则没有。这与Vortex Force算例中上下严格对称的分布情况形成了鲜明的对比。另外,从图中展示的流线,也可以看到,Plasma(Weak)中的涡显得方方正正的,而另外两个算例中的涡就有一点斜。这也很好地解释为什么这两个算例中的平均法向速度分布在槽道中部呈蝴蝶形。雷诺应力是湍流阻力的关键。可以看到三个算例中,贴近下壁面的高雷诺切应力区都被驱赶到周期边界附近。另外,在Plasma(Weak)算例中,上壁面附近的高雷诺切应力区并没有像其他两个算例一样被卷起带到槽道中间。特别要注意的是,在Plasma(Strong)算例中,在周期边界两侧,也就是$z=1$和6位置附近,有两道狭长的高雷诺切应力区。并且这里的雷诺切应力是正的,而其上下都是负的。这两道狭长的区域在雷诺正应力$<uu>$中也有体现。这可以说明在那个位置的平均流场流向发生了很大的剪切,所以产生了非常强的雷诺应力。

\begin{figure}[htb]
  \centering
  \includegraphics[width=\textwidth]{ch5/PlasmaWeak/8367500.png}
  \caption{Plasma(Weak)涡结构}\label{f:PlasmaWeak_vortexstructure}
\end{figure}
之后将重点分析Plasma(Weak)算例的结果。该算例流场的涡结构如图\ref{f:PlasmaWeak_vortexstructure}。这张图中,在空间分布的白色结构是$Q=1$的等值线。底面的颜色云图为壁面摩阻,左侧面的颜色云图为瞬时流向速度,后端的颜色云图为流向涡量。可以看到,在流体被DBD吸下来区域,壁面由于受到流体的冲击,摩阻比较大。在流体被卷上去的区域,摩阻相对较小。但是涡结构和涡量还是主要集中在流体被卷上去的区域。在流体被吸下来的区域流动反而比较干净,没什么涡结构,也没什么涡量。另外,在上抛区的涡结构中,依然可以看到流向涡抬升并形成发卡涡的过程。

\begin{figure}
  \centering
  \subcaptionbox{湍动能}[0.49\textwidth]
    {\includegraphics[width=0.49\textwidth]{ch5/PlasmaWeak/k.png}}
  \subcaptionbox{湍动能生成项\label{f:Productionb}}[0.49\textwidth]
    {\includegraphics[width=0.49\textwidth]{ch5/PlasmaWeak/Production.png}}
  \caption{Plasma(Weak)湍动能和湍动能生成项}\label{f:production}
\end{figure}
图\ref{f:production}给出了Plasma(Weak)算例的湍动能与湍动能生成项的分布。这里先重点看图\ref{f:Productionb}给出的湍动能生成项分布。图中白色粗线包围的区域内湍动能生成项为负值。这表明,在DBD将流体吸到壁面附近这一过程中,流体不仅被加速,还产生了在层流化现象。这一现象使得槽道中部贴近壁面处湍动能极低。这同时非常好的印证了图\ref{f:PlasmaWeak_vortexstructure}中流体下扫区涡结构较少这一观察。这里,可以将上壁面近似当成无控制的壁面进行对比观察。可以看到上壁面附近就有一层高湍动能区,几乎平平的铺在整个上壁面上。而下壁面的高湍动能区则集中在流体上扫区,并且极大值位置也略有提升。所以可见,该算例主要是通过二次涡产生的层流化效应降低湍流脉动,其次在上抛区抬高高湍动能区的位置,从而实现减阻。

\section{周期激励控制方案}
\begin{figure}[htb]
  \centering
  \includegraphics[width=0.5\textwidth]{ch5/POLIMI/Drag.jpeg}
  \caption{米兰理工(POLIMI)体积力激励方案阻力变化}\label{f:POLIMI_DARG}
\end{figure}
除了产生流向大涡可以降低湍流槽道的阻力外,迫使近壁流体做展向的周期振动也是一种非常有效的减阻方法。一般采用的激励方式有体积力和壁面振动。米兰理工(之后简称为POLIMI)在这一方面有很详细系统的研究\cite{Gatti2016,Gatti2013,Quadrio2009}。他们不仅研究了展向均匀振动的效果,还研究了这种振动形式沿着展向或流向进行传播的控制效果。其中,他们研究的展向行波形式的激励可以用解析式表达为:
\begin{equation}\label{e:f_POLIMI}
  F_z  = F_z \left( {y,t} \right) = A_f e^{ - y/D} \cos \left( k_zz + \omega t \right)
\end{equation}
其中$\omega$是圆频率,其激励周期为$T=2\pi/\omega$。POLIMI研究了诸多不同参数的算例,后来发现当式(\ref{e:f_POLIMI})中的参数取$k_z=0,A_f=2,D=0.04,T^+=52$的时候减阻率最高。本文先复现了POLIMI给出的最佳的减阻算例,并计算了激励周期为最佳控制周期两倍,其余参数一样的增阻算例进行比较($T^+=104$)。这两个工况得到的阻力结果如图\ref{f:POLIMI_DARG}所示。注意到这里控制的最佳周期与流向涡的生存寿命$T^+\approx$50\cite{Jimenez1999}到60\cite{del2006}大致相当,但是比整个壁湍流自维持机制的周期$T^+\approx400$\cite{Jimenez2005}要小很多。图中显示,$T^+=52$算例中的壁面摩阻几乎没有脉动,并且阻力降低了近乎一半。这基本上表明产生的周期振荡成功抑制了湍流脉动。然而两倍周期,也就是 $T^+=104$ 的结果就不是很乐观。摩阻出现了同频率的大幅震荡。这表明用于激励的体积力和流动产生了共振。本文之后便不再讨 $T^+=104$ 的增阻算例,论重点分析 $T^+=52$ 的减阻算例,并将其简记为算例POLIMI。

\begin{figure}[htb]
  \centering
  \subcaptionbox{$U^+$\label{f:POLIMI_phaseaverU}}[0.495\textwidth]
    {\includegraphics[width=0.495\textwidth]{ch5/POLIMI/phase_average_U+.jpeg}}
  \subcaptionbox{$W$\label{f:POLIMI_phaseaverW}}[0.495\textwidth]
    {\includegraphics[width=0.495\textwidth]{ch5/POLIMI/phase_average_W.jpeg}}
  \caption{$T^+=52$相平均}\label{f:POLIMI_phaseaver}
\end{figure}
图\ref{f:POLIMI_phaseaver}给出了POLIMI算例中流向速度和展向速度进行相平均的结果。由于DNS计算量过大,为了不浪费数据,每一个相位均指的是这一相位区间内所有瞬时的平均结果。例如图例中给出的$0.25\pi$实际上是平均了$[0.25\pi,0.5\pi]$区间内的所有结果。图\ref{f:POLIMI_phaseaverU}是相位平均后的流向速度,采用壁面摩擦尺度进行无量纲化。可以看到,所有相位的结果完全重合。图中的黑线是无控制时的结果。控制后的平均流向速度在粘性底层有所降低。另外,控制后的流向速度剖面没有明显的符合对数律的区域。图\ref{f:POLIMI_phaseaverW}展示的是展向速度相平均的结果。这里需要注意的是在$0.75\pi$和$1.75\pi$相位处,展向速度开始反向,其紧贴壁面的流动方向和更高处一些流体的流动方向不同。

\begin{figure}[htb]
  \centering
  \includegraphics[width=0.8\textwidth]{ch5/POLIMI/5973000.png}
  \caption{POLIMI涡结构图}\label{f:POLIM_vortex_structure}
\end{figure}
图\ref{f:POLIM_vortex_structure}展示的是流动的涡结构。这里还是使用$Q=0.5$作为涡判据,下壁面用摩阻分布进行颜色渲染,左侧壁面用瞬时流向速度进行渲染,后壁面用流向涡量进行渲染。通过观察流动发展的过程,发现壁面摩阻的分布不仅随着随着涡结构一起向前流动,同时还会受体积力影响在展向做左右的周期摆动。这两种运动的叠加导致壁面摩阻分布会以蛇形的方式向前走。然而,远离壁面的涡结构却并不怎么受到近壁体积力的影响,只是向前运动,并没有左右摆动。后端面显示的是流向涡量。由于体积力的引入,近壁会出现时正时负的高涡量层。除此之外,下半个槽道出现的涡量团比上半个要少很多。这也从侧面反映了激励成功的抑制了湍流脉动。

\begin{figure}[htb]
  \centering
  \subcaptionbox{正向激励时体积力分布}[0.8\textwidth]
  {\includegraphics[width=0.8\textwidth]{ch5/6DBD1_new.jpg}}
  \\\bigskip
  \subcaptionbox{反向激励时体积力分布}[0.83\textwidth]
  {\includegraphics[width=0.83\textwidth]{ch5/6DBD2_new.jpg}}
  \caption{6DBD算例激励示意图}\label{f:6dbd_force}
\end{figure}
\begin{figure}[htb]
  \centering
  \subcaptionbox{正向激励时体积力分布\label{f:16dbd_forcea}}[\textwidth]
  {\includegraphics[width=\textwidth]{ch5/16dbd1_new.jpg}}
  \\\bigskip
  \subcaptionbox{反向激励时体积力分布\label{f:16dbd_forceb}}[\textwidth]
  {\includegraphics[width=\textwidth]{ch5/16dbd2_new.jpg}}
  \caption{16DBD算例激励示意图}\label{f:16dbd_force}
\end{figure}
%\begin{figure}
%  \centering
%  \includegraphics[width=0.5\textwidth]{ch5/FX_unsteady_forcing.jpeg}
%  \caption{周期激励各个算例控制效果}\label{f:period_actuation}
%\end{figure}
POLIMI算例的控制效果非常好,但是实际中无法产生这样的体积力,所以本文之后主要研究如何用DBD等离子体激发器产生相同的效果。这里还是采用Maden的\cite{Maden2013}DBD模型。本文最初提出的激发器布置方案如图\ref{f:6dbd_force}所示。在前半个周期采用3个DBD激发器向右吹,在后半个周期采用3个激发器向左吹。由于一共用到了6个激发器,这里将其简记为6DBD算例。这里激励的周期还是采用$T^+=52$,添加进去的体积力在全场积分的值与POLIMI算例相同,力在垂直壁面方向的分布也近似相同。然而,这个算例中阻力增大了将近三倍(如图\ref{f:period_actuation})。由于周期、力强度等参数都与POLIMI算例中相仿,最有可能导致阻力增加的原因就是展向的不均匀性。笔者为了对比6DBD形式的体积力和POLIMI形式的体积力分别会带来多大的扰动,做了一个层流计算。还是相同的算例,只是背景流动采用的是层流槽道的速度剖面。对这两个层流槽道添加两种不同形式的激励后,POLIMI的体积力在很长一段时间都不会导致流动从层流转变为湍流,而6DBD形式的体积力则会使得流动很快进入湍流状态。这表明在采用DBD控制的时候,会引入额外的扰动,并进而引起阻力的增加。究其原因,产生这些扰动的罪魁祸首就是展向不均匀性。为了克服这一现象,本文又提出了采用16个DBD的控制方案。其前后半个周期的体积力分布分别如图\ref{f:16dbd_forcea}和\ref{f:16dbd_forceb}所示。这里先不去考虑实际中是否可以进行这样的布置,仅仅探讨加密布置能不能够削弱因展向不均匀性引入的额外扰动,并进而达到减阻的目的。图\ref{f:period_actuation}的控制结果显示这么做是有效果的,下壁面阻力从0.32降低到了0.29。

\begin{figure}[htb]
\begin{minipage}{0.48\textwidth}
  \centering
  \includegraphics[width=\textwidth]{ch5/U.jpeg}
  \caption{平均流向速度对比}\label{f:period_U}
\end{minipage}\hfill
\begin{minipage}{0.48\textwidth}
  \centering
  \includegraphics[width=\textwidth]{ch5/FX_unsteady_forcing.jpeg}
  \caption{周期激励各个算例控制效果}\label{f:period_actuation}
\end{minipage}
\end{figure}
%\begin{figure}[htb]
%  \centering
%  \includegraphics[width=0.6\textwidth]{ch5/U.jpeg}
%  \caption{周期激励控制算例平均流向速度对比}\label{f:period_U}
%\end{figure}
\begin{figure}[htb]
  \centering
  \subcaptionbox{$<uu>$}[0.495\textwidth] %标题的长度,超过则会换行,如下一个小图。
    {\includegraphics[width=0.495\textwidth]{ch5/16DBD/uu.jpeg}}%
  \subcaptionbox{$<vv>$}[0.495\textwidth]
    {\includegraphics[width=0.495\textwidth]{ch5/16DBD/vv.jpeg}}%
  \\\bigskip
  \subcaptionbox{$<ww>$}[0.495\textwidth] %标题的长度,超过则会换行,如下一个小图。
    {\includegraphics[width=0.495\textwidth]{ch5/16DBD/ww.jpeg}}%
  \subcaptionbox{$<uv>$}[0.495\textwidth]
    {\includegraphics[width=0.495\textwidth]{ch5/16DBD/uv.jpeg}}%
  \caption{二阶统计量对比}\label{f:period_2order}
\end{figure}
图\ref{f:period_U}和\ref{f:period_2order}分别给出了平均流向速度和二阶统计量的对比。可以看到,在POLIMI算例中,两个方向的雷诺正应力$<uu>$和$<vv>$在下半个槽道均大幅减小。尤其是$<vv>$,本来应该在下壁面出现的峰值都被抹平,仅仅是出现了一个平台。这表明下壁面附近的湍流被极大地抑制住了。由于展向速度的周期脉动也被统计到了雷诺应力里,所以在控制算例中的展向雷诺正应力分量要远高于无控制算例的,其中POLIMI算例中的展向雷诺正应力分量峰值达到了0.27,没有在图中画出。另外,雷诺切应力$<uv>$在POLIMI算例中也被极大地削弱了。从二阶量统计量的结果对比来看,16DBD算例像是一个弱化版的POLIMI算例。雷诺应力减小和增减的方向相同,只是变化的程度不及POLIMI算例。由于控制只加在了下壁面,导致下壁面阻力降低而上壁面阻力不变,流体更容易从槽道的下半部分通过,所以槽道下半部分的平均流速高于上半部分的(图\ref{f:period_U})

\begin{figure}[htb]
  \centering
  \subcaptionbox{POLIMI}[0.45\textwidth]
    {\includegraphics[width=0.45\textwidth]{ch5/POLIMI/POLIMI_num_vort.jpeg}}
  \subcaptionbox{16DBD}[0.45\textwidth]
    {\includegraphics[width=0.45\textwidth]{ch5/16DBD/16DBD_num_vort.jpeg}}
  \caption{不同相位探测到的涡个数}\label{f:vortex_num}
\end{figure}
\begin{figure}[htb]
  \centering
  \includegraphics[width=0.8\textwidth]{ch5/16DBD/phaseQu.png}
  \caption{16DBD算例中$Q$最大值,$u$最大最小值与无控制时候的比值}\label{f:Qu}
\end{figure}
\begin{figure}[htb]
  \centering
  \includegraphics[height=5cm]{ch5/POLIMI/condit_0.jpeg}
  \includegraphics[height=5cm]{ch5/POLIMI/phase_0.jpeg}\\
  \includegraphics[height=5cm]{ch5/POLIMI/condit_1.jpeg}
  \includegraphics[height=5cm]{ch5/POLIMI/phase_1.jpeg}\\
  \includegraphics[height=5cm]{ch5/POLIMI/condit_2.jpeg}
  \includegraphics[height=5cm]{ch5/POLIMI/phase_2.jpeg}\\
  \includegraphics[height=5cm]{ch5/POLIMI/condit_3.jpeg}
  \includegraphics[height=5cm]{ch5/POLIMI/phase_3.jpeg}
  \caption{POLIMI控制方案在第一个四分之一周期内的条件平均结果(左:条件平均流向涡附件流场;右:相平均展向速度)}\label{f:polimi_conditionally1}
\end{figure}
\begin{figure}[htb]
  \centering
  \includegraphics[height=5cm]{ch5/POLIMI/condit_4.jpeg}
  \includegraphics[height=5cm]{ch5/POLIMI/phase_4.jpeg}\\
  \includegraphics[height=5cm]{ch5/POLIMI/condit_5.jpeg}
  \includegraphics[height=5cm]{ch5/POLIMI/phase_5.jpeg}\\
  \includegraphics[height=5cm]{ch5/POLIMI/condit_6.jpeg}
  \includegraphics[height=5cm]{ch5/POLIMI/phase_6.jpeg}\\
  \includegraphics[height=5cm]{ch5/POLIMI/condit_7.jpeg}
  \includegraphics[height=5cm]{ch5/POLIMI/phase_7.jpeg}
  \caption{POLIMI控制方案在第二个四分之一周期内的条件平均结果(左:条件平均流向涡附件流场;右:相平均展向速度)}\label{f:polimi_conditionally2}
\end{figure}
\begin{figure}[htb]
  \centering
  \includegraphics[height=5cm]{ch5/POLIMI/condit_8.jpeg}
  \includegraphics[height=5cm]{ch5/POLIMI/phase_8.jpeg}\\
  \includegraphics[height=5cm]{ch5/POLIMI/condit_9.jpeg}
  \includegraphics[height=5cm]{ch5/POLIMI/phase_9.jpeg}\\
  \includegraphics[height=5cm]{ch5/POLIMI/condit_10.jpeg}
  \includegraphics[height=5cm]{ch5/POLIMI/phase_10.jpeg}\\
  \includegraphics[height=5cm]{ch5/POLIMI/condit_11.jpeg}
  \includegraphics[height=5cm]{ch5/POLIMI/phase_11.jpeg}
  \caption{POLIMI控制方案在第三个四分之一周期内的条件平均结果(左:条件平均流向涡附件流场;右:相平均展向速度)}\label{f:polimi_conditionally3}
\end{figure}
\begin{figure}[htb]
  \centering
  \includegraphics[height=5cm]{ch5/POLIMI/condit_12.jpeg}
  \includegraphics[height=5cm]{ch5/POLIMI/phase_12.jpeg}\\
  \includegraphics[height=5cm]{ch5/POLIMI/condit_13.jpeg}
  \includegraphics[height=5cm]{ch5/POLIMI/phase_13.jpeg}\\
  \includegraphics[height=5cm]{ch5/POLIMI/condit_14.jpeg}
  \includegraphics[height=5cm]{ch5/POLIMI/phase_14.jpeg}\\
  \includegraphics[height=5cm]{ch5/POLIMI/condit_15.jpeg}
  \includegraphics[height=5cm]{ch5/POLIMI/phase_15.jpeg}
  \caption{POLIMI控制方案在第四个四分之一周期内的条件平均结果(左:条件平均流向涡附件流场;右:相平均展向速度)}\label{f:polimi_conditionally4}
\end{figure}
\begin{figure}[htb]
  \centering
  \includegraphics[height=5cm]{ch5/16DBD/condit_0.jpeg}
  \includegraphics[height=5cm]{ch5/16DBD/phase_0.jpeg}\\
  \includegraphics[height=5cm]{ch5/16DBD/condit_1.jpeg}
  \includegraphics[height=5cm]{ch5/16DBD/phase_1.jpeg}\\
  \includegraphics[height=5cm]{ch5/16DBD/condit_2.jpeg}
  \includegraphics[height=5cm]{ch5/16DBD/phase_2.jpeg}\\
  \includegraphics[height=5cm]{ch5/16DBD/condit_3.jpeg}
  \includegraphics[height=5cm]{ch5/16DBD/phase_3.jpeg}
  \caption{16DBD控制方案在第一个四分之一周期内的条件平均结果(左:条件平均流向涡附件流场;右:相平均展向速度)}\label{f:16DBD_conditional_1}
\end{figure}
\begin{figure}[htb]
  \centering
  \includegraphics[height=5cm]{ch5/16DBD/condit_4.jpeg}
  \includegraphics[height=5cm]{ch5/16DBD/phase_4.jpeg}\\
  \includegraphics[height=5cm]{ch5/16DBD/condit_5.jpeg}
  \includegraphics[height=5cm]{ch5/16DBD/phase_5.jpeg}\\
  \includegraphics[height=5cm]{ch5/16DBD/condit_6.jpeg}
  \includegraphics[height=5cm]{ch5/16DBD/phase_6.jpeg}\\
  \includegraphics[height=5cm]{ch5/16DBD/condit_7.jpeg}
  \includegraphics[height=5cm]{ch5/16DBD/phase_7.jpeg}
  \caption{16DBD控制方案在第二个四分之一周期内的条件平均结果(左:条件平均流向涡附件流场;右:相平均展向速度)}\label{f:16DBD_conditional_2}
\end{figure}
\begin{figure}[htb]
  \centering
  \includegraphics[height=5cm]{ch5/16DBD/condit_8.jpeg}
  \includegraphics[height=5cm]{ch5/16DBD/phase_8.jpeg}\\
  \includegraphics[height=5cm]{ch5/16DBD/condit_9.jpeg}
  \includegraphics[height=5cm]{ch5/16DBD/phase_9.jpeg}\\
  \includegraphics[height=5cm]{ch5/16DBD/condit_10.jpeg}
  \includegraphics[height=5cm]{ch5/16DBD/phase_10.jpeg}\\
  \includegraphics[height=5cm]{ch5/16DBD/condit_11.jpeg}
  \includegraphics[height=5cm]{ch5/16DBD/phase_11.jpeg}
  \caption{16DBD控制方案在第三个四分之一周期内的条件平均结果(左:条件平均流向涡附件流场;右:相平均展向速度)}\label{f:16DBD_conditional_3}
\end{figure}
\begin{figure}[htb]
  \centering
  \includegraphics[height=5cm]{ch5/16DBD/condit_12.jpeg}
  \includegraphics[height=5cm]{ch5/16DBD/phase_12.jpeg}\\
  \includegraphics[height=5cm]{ch5/16DBD/condit_13.jpeg}
  \includegraphics[height=5cm]{ch5/16DBD/phase_13.jpeg}\\
  \includegraphics[height=5cm]{ch5/16DBD/condit_14.jpeg}
  \includegraphics[height=5cm]{ch5/16DBD/phase_14.jpeg}\\
  \includegraphics[height=5cm]{ch5/16DBD/condit_15.jpeg}
  \includegraphics[height=5cm]{ch5/16DBD/phase_15.jpeg}
  \caption{16DBD控制方案在第四个四分之一周期内的条件平均结果(左:条件平均流向涡附件流场;右:相平均展向速度)}\label{f:16DBD_conditional_4}
\end{figure}
为了分析POLIMI算例和16DBD算例的减阻机理,本文分别对其进行了区分相位的条件平均(或者叫相条件平均)。在不同相位统计平均的流向涡数量与无控制工况下统计到的流向涡数量比值如图\ref{f:vortex_num}所示。可以看到,在POLIMI算例中,探测到的流向涡数量大约只有无控制时的5\%。这一统计结果表明POLIMI形式的展向激励的确可以减弱湍流在近壁区的生成。而在16DBD算例中,大约探测到了原来74\%的流向涡。很明显,等离子体激发器也限制了流向涡的生成,但是其效果相比于展向均匀激励要弱很多。这一结果也印证了之前图\ref{f:period_2order}中的观察与分析。

图\ref{f:polimi_conditionally1}到\ref{f:polimi_conditionally4}展示了POLIMI算例在16个不同的相位进行条件平均的结果。每张图的右侧是对应相位的平均展向速度。可以看到在第一个四分之一周期,近壁面的展向速度都是向右的。这里所探测到并拿来进行平均的涡都是涡量为正的,也就是涡量方向指向纸面里的。探测涡的平面还是取在$y^+=20$的位置。可以发现,体积力诱导产生的展向速度主要都在$y^+=20$一下,也就是在无控制的涡的涡核下方。很显然,涡量指向纸面里的涡在涡核下方的流动应该是向左的。也就是说,这时的体积力完全在阻碍这个方向的流向涡生成。图中依然是用粗的黑线表示$Q$的等值线,最外围的等值线为$Q=1$。在$2/8\pi$和$3/8\pi$两个相位,探测到的涡完全不是一个完整的,涡核在$y^+=20$附近的流向涡。仅仅是涡核位于更高位置处的涡的一部分。在$2/8\pi$相位中,流向脉动速度的最大最小值,也就是高低速条带的极值分别为0.05和-0.048,仅为无控制时的大约一半。

从$3/8\pi$相位开始,在所探测到的涡(涡核在$y^+=20$)上方,逐渐开始形成一个高速条带区。显然我们所探测到的流向涡是无法在其上方形成一个高速条带的。一个比较合理的解释是这这时探测到的涡仅仅是高度更高的涡诱导出来的一个二次涡。这一观点可以在$4/8\pi$相位的时候得到更充分的验证。从$Q$图上看,在$y^+=35$位置还有一处$Q$的极大值,并且流线上也显示那里有一个涡量为负的涡。这个涡将上面的高速流体卷下来,从而形成了这个新的高速条带。之后,随着近壁的展向速度方向逐渐增大,$y^+=20$处的涡逐渐摆脱更高处涡的控制,渐渐将高速条带拽到自己右侧。在$8/8\pi$相位的时候,涡下部的展向流动方向与涡诱导出来的流动方向相同,于是又形成了高低速条带与流向涡分别位于右左中的格局。之后流向涡越来越强,逐渐将高速条带向自己的下方拖拽,将低速条带抬起。在$13/8\pi$相位的时候,低速条带中心的高度已经达到了接近$y^+=30$。最后随着展向速度的反向,流向涡开始减弱并逐渐被破坏,并进入下一个周期。

图\ref{f:16DBD_conditional_1}到\ref{f:16DBD_conditional_4}展示了16DBD控制算例在一个周期内的条件平均的结果。首先需要说明的是在这个算例中,体积力诱导出来的展向速度要比POLIMI算例弱很多。POLIMI中展向速度可以到0.7,而这个算例中只有0.2。实际的控制效果也表明随着展向速度的减弱,流向涡-条带的近壁结构并没有被完全破坏。在所有相位的条件平均中,均还是可以观察到这一结构。在第一个四分之一周期,体积力诱导出来的展向速度在涡核下方,与涡诱导出来的相反。这一阶段是体积力对流向涡抑制效果最强的阶段。这一阶段的典型代表——$3/8\pi$相位的结果中,$Q$的最大值为2.79,脉动速度最大最小值分别为0.075和-0.105,均比无控制算例中对应的3.87、0.10、-0.13要弱。之后展向速度逐渐反向。在反向过程中,还出现了底层流体与高层流体展向速度不同的情况($6/8\pi$)。这时在流向涡下发还产生了一个偏平的流向涡。在第三个四分之一周期内,体积力诱导的展向速度已经与流向涡诱导的流向速度方向相同了,在$11/8\pi$相位展向速度达到最大值,在此之后,低速条带的范围迅速扩大。到了$14/8\pi$和$15/8\pi$相位,展向速度又出现了底层与高层流动方向不一致的情况,贴近壁面的二次涡再一次生成。

由于在16DBD算例中,流向涡-条带结构并没有像POLIMI算例中一样被大幅破坏掉,所以单从云图中并看不出所以然来。为了定量对比,图\ref{f:Qu}给出了不同相位$Q$最大值、$u$最大最小值与无控制时候的比值。从图中首先可以看到的是,这三个量在整个周期内都比1小,也意味着流向涡-条带结构相比于无控制工况始终是被抑制的。另外,高低速条带的变化趋势始终相反。也就是高速条带强的时候,低速调点弱,而低速条带强的时候高速条带若。相比涡的强度,条带的强度有一定的滞后效应。可以看到,在$12/8\pi$相位的时候$Q$已经达到了最大值,而高低速条带的极值分别在$13/8\pi$和$15/8\pi$相位处达到。总的来说,采用展向周期激励的DBD控制方案依然是通过抑制湍流的自维持机制,减弱湍流的产生,来达到减阻的目的。

\section{本章小结}
本章针对$Re_\tau=180$的充分发展槽道湍流,提出了定常激励和周期激励两种采用DBD等离子体激发器的控制方案。这两种控制方案均能起到大约9\%的减阻效果。

在定常激励控制方案中,两个向相反方向吹气的激发器在槽道中产生了与槽道大约同一尺度的流向涡。这个流向涡在槽道中部将流体快速的吸到壁面,然后又在计算的周期边界处,将流体向上抛回高速区。在向下吸的加速区,流动发生了再层流化,使得壁面附近整体的湍动能降低,从而降低的湍流摩阻。在这种控制方案中,产生的流向涡的强度至关重要,太强的流向涡反而会使得阻力增加。

在周期激励方案中,激发器阵列被用于产生时而向左时而向右的展向周期振动。从分相位的条件平均结果来看,由于体积力在激励的前半个周期内会在流向涡下部产生与涡诱导速度方向相反的展向速度,这使得流向涡的生成受到阻碍。采用展向均匀的体积力激励会比采用有展向分布的DBD产生的体积力效果好。在这一控制方案中,激发器的密度是一个关键参数。密集布置的激发器可以降低展向不均匀性,从而取得较好的控制效果,而过稀的激发器布置反而会增加阻力。
\chapter{结论}




%%% 其它部分
\backmatter

%% 本科生要这几个索引,研究生不要。选择性留下。
% 插图索引
%\listoffigures
% 表格索引
%\listoftables
% 公式索引
%\listofequations


%% 参考文献
% 注意:至少需要引用一篇参考文献,否则下面两行可能引起编译错误。
% 如果不需要参考文献,请将下面两行删除或注释掉。
% 数字式引用
\bibliographystyle{thuthesis-numeric}
% 作者-年份式引用
% \bibliographystyle{thuthesis-author-year}
\bibliography{ref/refs}


%% 致谢
% 如果使用声明扫描页,将可选参数指定为扫描后的 PDF 文件名,例如:
% \begin{acknowledgement}[scan-statement.pdf]
\begin{acknowledgement}
  衷心感谢导师符松教授和王亮老师对本人的精心指导。他们的言传身教将使我终生受益。

  同时衷心感谢徐国亮师兄,任杰师兄在流动稳定性方面的悉心指导,帮助我解决了不少问题。感谢朱辉师兄教我使用hpMusic计算程序。感谢徐胜金教授和他的学生王庆洋提供实验数据。

  感谢我的父母和杨希同学对我生活上的照顾。
  
  感谢 \LaTeX 和 \thuthesis\cite{thuthesis},帮我节省了不少时间。
\end{acknowledgement}


%% 附录
\begin{appendix}
\chapter{扰动方程推导}
前面两个附录主要是给本科生做例子。其它附录的内容可以放到这里,当然如果你愿意,可
以把这部分也放到独立的文件中,然后将其 \cs{input} 到主文件中。

\end{appendix}

%% 个人简历
\begin{resume}

  \resumeitem{个人简历}

  1991 年 10 月 5 日出生于陕西省西安市长安县(现 长安区)。

  2009 年 9 月考入清华大学航天航空学院工程力学系钱学森力学班,2013 年 7 月本科毕业并获得工学学士学位。

  2013 年 9 月免试进入清华大学大学航天航空学院攻读力学博士学位至今。

  \researchitem{发表的学术论文} % 发表的和录用的合在一起

  % 1. 已经刊载的学术论文(本人是第一作者,或者导师为第一作者本人是第二作者)
  \begin{publications}
    \item Zhefu Wang, Liang Wang, and Song Fu. "Control of stationary crossflow modes in swept Hiemenz flows with dielectric barrier discharge plasma actuators", Physics of Fluids, 2017, 29(9): 094105. (SCI 收录,WOS:000412105100038)
    \item Zhefu Wang, Liang Wang, and Song Fu. "Sensitivity analysis of crossflow boundary layer and transition delay using plasma actuator", 8th AIAA Flow Control Conference, AIAA AVIATION Forum, (AIAA 2016-3933). (EI 收录,ISBN-13: 9781624104329)
    \item Zhefu Wang, Liang Wang, and Song Fu. "Control of crossflow instability using plasma actuators", 7th Asia-Pacific International Symposium on Aerospace Technology, 25-27 November 2015, Cairns, Australia. (会议论文)
    \item Zhefu Wang and Song Fu. "Control of crossflow instability using plasma actuators", XXIV ICTAM, 21-26 August 2016, Montreal, Canada.(会议论文)
    \item Zhefu Wang and Song Fu. "Transition delay using DBD plasma actuators", European Drag Reduction and Flow Control Meeting, 3-6 April 2017, Rome, Italy. (会议论文)
  \end{publications}

  % 2. 尚未刊载,但已经接到正式录用函的学术论文(本人为第一作者,或者
  %    导师为第一作者本人是第二作者)。
  %\begin{publications}[before=\publicationskip,after=\publicationskip]
%    \item Yang Y, Ren T L, Zhu Y P, et al. PMUTs for handwriting recognition. In
%      press. (已被 Integrated Ferroelectrics 录用. SCI 源刊.)
%  \end{publications}

  % 3. 其他学术论文。可列出除上述两种情况以外的其他学术论文,但必须是
  %    已经刊载或者收到正式录用函的论文。
  %\begin{publications}
%    \item Wu X M, Yang Y, Cai J, et al. Measurements of ferroelectric MEMS
%      microphones. Integrated Ferroelectrics, 2005, 69:417-429. (SCI 收录, 检索号
%      :896KM)
%    \item 贾泽, 杨轶, 陈兢, 等. 用于压电和电容微麦克风的体硅腐蚀相关研究. 压电与声
%      光, 2006, 28(1):117-119. (EI 收录, 检索号:06129773469)
%    \item 伍晓明, 杨轶, 张宁欣, 等. 基于MEMS技术的集成铁电硅微麦克风. 中国集成电路,
%      2003, 53:59-61.
%  \end{publications}

  %\researchitem{研究成果} % 有就写,没有就删除
%  \begin{achievements}
%    \item 任天令, 杨轶, 朱一平, 等. 硅基铁电微声学传感器畴极化区域控制和电极连接的
%      方法: 中国, CN1602118A. (中国专利公开号)
%    \item Ren T L, Yang Y, Zhu Y P, et al. Piezoelectric micro acoustic sensor
%      based on ferroelectric materials: USA, No.11/215, 102. (美国发明专利申请号)
%  \end{achievements}

\end{resume}


%% 本科生进行格式审查是需要下面这个表格,答辩可能不需要。选择性留下。
% 综合论文训练记录表
\includepdf[pages=-]{scan-record.pdf}
\end{document}
