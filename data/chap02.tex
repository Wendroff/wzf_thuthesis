\chapter{理论公式与数值求解方法}
\label{cha:2}
\section{等离子体模型}
在本文的研究中,主要使用了两种等离子体模型。分别是Kriegseis等人\cite{kriegseis2013velocity}用实验数据反推出来的体积力分布模型(之后简称为Kriegseis模型)和在此基础上用解析式近似拟合的Maden模型\cite{Maden2013}。其中Kriegseis模型更加精确,但是其仅仅适用于他们实验中测试的那一种几何外形的激发器。而这种尺寸的激发器无法胜任本文之后提到的后掠翼减阻的工作。具体而言,他们实验测量的激发器太大的,远远超过了横流涡的尺寸。所以在后掠翼推迟转捩的研究中本文使用了可以通过改变参数来调整体积力分布的Maden模型。在最后的槽道湍流减阻中,由于这个工况并无实际物理尺寸对应,所以也是使用了较为理想化的Maden模型。

这里先介绍Kriegseis模型。在引言中提到过,从实验的PIV数据反推体积力存在一个问题,就是因为无法测量空间的压力分布,导致未知量个数多于方程的个数。Albrecht等人\cite{albrecht2011method}提出用涡量方程跳过压力项的方法。他们推导得到的涡量方程为:
\begin{equation}\label{e:force1}
\frac{1}{\rho} \left( \frac{\partial f_x}{\partial y}-\frac{\partial f_y}{\partial x} \right)=  {u\frac{{\partial \omega_z }}
{{\partial x}} + v\frac{{\partial \omega_z }}
{{\partial y}} - \frac{\mu }
{\rho }\left( {\frac{{\partial ^2 \omega_z }}
{{\partial x^2 }} + \frac{{\partial ^2 \omega_z }}
{{\partial y^2 }}} \right)}
\end{equation}
其中$f_x,f_y$分别为平行壁面与垂直壁面体积力分量。z方向涡量$\omega_z$的定义为:
\begin{equation}\label{e:vorticity}
  \omega_z=\frac{\partial v}{\partial x}-\frac{\partial u}{\partial y}
\end{equation}
但是这时未知量个数($f_x$和$f_y$)还是多于方程个数。Albrecht等人\cite{albrecht2011method}认为$f_x$比$f_y$要大一个量级,所以便将后者忽略掉。之后在对上式延法向积分,得到:
\begin{equation}\label{e:force3}
f_x(x,y) =  - \rho \int_{+\infty} ^y {\left[ {U\frac{{\partial \omega_z }}
{{\partial x}} + V\frac{{\partial \omega_z }}
{{\partial y}} - \frac{\mu }
{\rho }\left( {\frac{{\partial ^2 \omega_z }}
{{\partial x^2 }} + \frac{{\partial ^2 \omega_z }}
{{\partial y^2 }}} \right)} \right]dy}
\end{equation}
这里交换了积分上下限是因为积分需要有一个已知的初始值,而在无穷远处,体积力显然为0。所以积分从无穷远开始。Kriegseis等人\cite{kriegseis2013velocity}使用的实验装置示意图及其测得的体积力分布如图\ref{f:kmodel}所示。
\begin{figure}
  \centering
  \includegraphics[height=5cm]{ch1/kmodel1.JPG}
  \includegraphics[height=5cm]{ch1/kmodel2.JPG}
  \caption{Kriegseis等人\cite{kriegseis2013velocity}使用的实验装置示意图(上)及其测得的体积力分布(下)}\label{f:kmodel}
\end{figure}

之后,Maden等人\cite{Maden2013}将模型进一步简化,提出了式(\ref{f:MadenModel})对之前测得的空间分布体积力进行了近似:
\begin{subequations}\label{f:MadenModel}
\begin{align}
    X(x)&=[a_1(x-x_{\rm PA})+a_2(x-x_{\rm PA})^2]\exp[-a_0(x-x_{\rm PA})] &x>x_{\rm PA}\\
    Y(y)&=(b_1y+b_2y^2)\exp(-b_0y^{2/5}) &y>0\\
    f_x(x,y)&=c_{\rm force}X(x)Y(y)&
\end{align}
\end{subequations}
其中的系数$a_0$, $a_1$, $a_2$, $b_0$, $b_1$,  $b_2$ 和 $c_{\rm force}$可以根据需要进行调整。$x_{\rm PA}$是激发器介质上电极片边缘位置坐标。D\"orr和Kloker\cite{dorr2016,dorr2015stabilisation}在他们的等离子体转捩推迟研究中就使用的是这个Maden模型
\section{流动稳定性求解框架}
在研究等离子体推迟转捩的问题中,本文采用研究稳定性问题的相关数值方法,来解析边界层内扰动的发展情况。稳定性的相关理论虽然并不能给出精确的转捩位置,但是能够从理论方面给出流动失稳特性,并且具有计算效率高,转捩前流动解析精度高的优点。本文中通过研究流动在控制前后稳定性方面的特性变化,来甄别控制是否有效。本文采用的研究边界层流动稳定性的步骤如下:
\begin{enumerate}
  \item 采用高精度有限元程序求解无粘流场;
  \item 以无粘流壁面上的流动参数作为边界层方程的边界条件,求解层流基本流动;
  \item 基于线性稳定性理论,判断主导转捩的模态;
  \item 采用抛物化扰动方程,求解边界层内扰动的演化。
\end{enumerate}

与前人所做的研究不同的是,本文的研究需要将等离子体产生的体积力也考虑进来。在详细介绍求解方法之前,这里先简要介绍一下本文中对体积力的处理方法。流动所满足的控制方程(N-S方程)为:
\begin{equation}\label{EQ_NS}\left.
\begin{aligned}
    \frac{\p{\rho^\dagger}}{\p{t^\dagger}}
    + {\nabla^\dagger}\cdot\left({\rho^\dagger}{\mathbf{V}^\dagger}\right) & =0
    \\
    {\rho^\dagger}\left( {\frac{\p{\mathbf{V}^\dagger}}{\p{t^\dagger}}
    + \left( {{\mathbf{V}^\dagger}\cdot{\nabla^\dagger}}\right)
    {\mathbf{V}^\dagger}} \right) & =
     - {\nabla^\dagger}{p^\dagger} + {\nabla^\dagger}\left( {{\lambda ^\dagger}\left( {{\nabla ^\dagger} \cdot {\mathbf{V}^\dagger}} \right)} \right) \\
    & + {\nabla ^\dagger} \cdot \left( {{\mu ^\dagger}\left( {{\nabla ^\dagger}{\mathbf{V}^\dagger} + {\nabla ^\dagger}{\mathbf{V}^\dagger}^T} \right)} \right) + \mathbf{f}^\dagger
    \\
    {\rho ^\dagger}{C_p}^\dagger\left( {\frac{{\p{T^\dagger}}}{{\p{t^\dagger}}} + \left( {{{\mathbf{V}}^\dagger} \cdot {\nabla ^\dagger}} \right){T^\dagger}} \right) & =
    {\nabla ^\dagger} \cdot \left( {{\kappa ^\dagger}{\nabla ^\dagger}{T^\dagger}} \right) \\
    & + \frac{{\p{p^\dagger}}}{{\p{t^\dagger}}} + \left( {{{\mathbf{V}}^\dagger} \cdot {\nabla ^\dagger}} \right){p^\dagger} + {\Phi ^\dagger} + \mathbf{V}^\dagger \cdot \mathbf{f}^\dagger
\end{aligned}~\right\}
\end{equation}

\noindent式(\ref{EQ_NS})中能量方程的耗散函数为:
\begin{equation}
    {\Phi ^\dagger} = {\lambda ^\dagger}{{\left( {{\nabla ^\dagger} \cdot {{{\mathbf{V}}}^\dagger}} \right)}^2} + \frac{{{\mu ^\dagger}}}{2}{{\left( {{\nabla ^\dagger}{{{\mathbf{V}}}^\dagger} + {\nabla ^\dagger}{{{\mathbf{V}}}^\dagger}^T} \right)}^2}
\end{equation}
方程中 $\dagger$ 表示有量纲量,${\mathbf{V}}$ 表示速度矢量,其在$x,y,z$三个方向的分量为 $u$,$v$,$w$ 。${\mathbf{f}}$ 表示体积力矢量,其分量分别为 $f_x$,$f_y$,$f_z$ 。

为封闭 N-S 方程,分别引入状态方程、Sutherland 粘性律、Stokes 假设,假定流体是量热完全气体并具有恒定的 $Pr$ 数:
\begin{equation}\left.
\begin{aligned}
p^\dagger=\rho^\dagger R^\dagger T^\dagger & \Leftrightarrow p=\frac{\rho T}{\gamma Ma^2} \\
\mu^\dagger=\mu_s^\dagger\frac{T^\dagger}{T^\dagger_s}\frac{T^\dagger_s+S^\dagger}{T^\dagger+S^\dagger} & \Leftrightarrow \mu=\mu_s\frac{T}{T_s}\frac{T_s+S}{T+S} \\
\lambda^\dagger+2/3\mu^\dagger=0 & \Leftrightarrow\lambda=-2/3\mu\\
\textrm{Pr}=\frac{{C_p}^\dagger\mu^\dagger}{\kappa^\dagger}={const} & \Leftrightarrow\mu=\kappa\\
C_p^\dagger=const,~ & R^\dagger=const
\end{aligned}~\right\}
\end{equation}
Sutherland 粘性律中 $T^\dagger_s=273K,~\mu_s^\dagger=1.71\times10^{-5}kg/(m\cdot s),~S^\dagger=110.4K$。

选取适当的参考长度 $l_{reff}$、参考速度 $U_{reff}$、参考密度 $\rho_{reff}$等特征量,可以对式(\ref{EQ_NS})进行无量纲化。本文分别研究了后掠Hiemenz流动和后掠翼流动。在这两个流动中我们选择的特征量是不一样的,之后我们会分别介绍。为了简洁,我们将无量纲化后的N-S方程记为:
\begin{equation}
    \label{e:NS}
    \mathscr{N}(\mathbf{q})=\mathbf{F}
\end{equation}
这里, $\mathbf{q}=(\rho , u,v,w,T)^T$,即原始变量组成的5维矢量。$\mathbf{F}=(0,f_x,f_y,f_z,\mathbf{V} \cdot \mathbf{f})^T$。 上标 ``$T$" 表示转置。这里由于添加的体积力很小,我们假设其只影响扰动发展,并不影响基本流。即基本流依然满足N-S方程:
\begin{equation}
    \label{e:baseflow}
    \mathscr{N}(\mathbf{q_0})=0
\end{equation}
$\mathbf{q_0}$ 为基本流流动原始变量组成的矢量,其与$\mathbf{q}$的差即为扰动量$\mathbf{\tilde{q}}$。令式(\ref{e:NS}) - 式(\ref{e:baseflow}),即可得到扰动的控制方程:
\begin{equation}
    \label{e:disturbance1}
    \mathscr{S}(\mathbf{\tilde{q}})=\mathscr{N}(\mathbf{q_0}+\mathbf{\tilde{q}})-\mathscr{N}(\mathbf{q_0})=\mathbf{F}
\end{equation}
在之后的小节\ref{subsec:BLfun}和\ref{subsec:STBfun}中,将会分别介绍式(\ref{e:baseflow})和(\ref{e:disturbance1})所采用的求解方法。至于求解步骤1中提到的高精度有限元方法,将会在\ref{sec:DNS}节中介绍。
\subsection{边界层方程}\label{subsec:BLfun}
在边界层流动中,流向的特征尺度为常规尺度,而法向的特征尺度为边界层厚度尺度。利用这一特性,可将Navier-Stokes 方程式抛物化,得到层流边界层控制方程。本文研究的问题的基本流均满足展向均匀假设,即${\p}/{\p z^\dagger}=0$ 。利用这些假设,式(\ref{e:baseflow})可以写为\footnote{本节讨论的均为基本流的计算方法,为了简洁,表示基本流变量的下标0在本节中都被略去。即原本的$\rho_0,u_0,v_0,w_0,T_0$在本节被记为$\rho,u,v,w,T$。有量纲量类似。}:
\begin{subequations}
\begin{align}
\frac{{\partial \left( {\rho^\dagger u^\dagger} \right)}}{{\partial x^\dagger }} + \frac{{\partial \left( {\rho^\dagger v^\dagger} \right)}}{{\partial y^\dagger }} &= 0 \\
\rho ^\dagger u^\dagger \frac{{\partial u^\dagger }}{{\partial x^\dagger }} + \rho ^\dagger v^\dagger \frac{{\partial u^\dagger }}{{\partial y^\dagger }} &=  - \frac{{\partial p^\dagger }}{{\partial x^\dagger }} + \frac{\partial }{{\partial y^\dagger }}\left( {\mu ^\dagger \frac{{\partial u^\dagger }}{{\partial y^\dagger }}} \right)\\
\rho ^\dagger u^\dagger \frac{{\partial w^\dagger }}{{\partial x^\dagger }} + \rho ^\dagger v^\dagger \frac{{\partial w^\dagger }}{{\partial y^\dagger }} &= \frac{\partial }{{\partial y^\dagger }}\left( {\mu ^\dagger \frac{{\partial w^\dagger }}{{\partial y^\dagger }}} \right)\\
\frac{{\partial p^\dagger }}{{\partial y^\dagger }} &= 0\\
\rho ^\dagger u^\dagger C_p^\dagger \frac{{\partial T^\dagger }}{{\partial x^\dagger }} + \rho ^\dagger v^\dagger C_p^\dagger \frac{{\partial T^\dagger }}{{\partial y^\dagger }} &= \frac{\partial }{{\partial y^\dagger }}\left( {k^\dagger \frac{{\partial T^\dagger }}{{\partial y^\dagger }}} \right) + u^\dagger \frac{{\partial p^\dagger }}{{\partial x^\dagger }} + \mu ^\dagger \left( {\frac{{\partial u^\dagger }}{{\partial y^\dagger }}} \right)^2  + \mu ^\dagger \left( {\frac{{\partial w^\dagger }}{{\partial y^\dagger }}} \right)^2
\end{align}
\end{subequations}

在传统的边界层方程求解方法中,所有物理量都采用相同的参考量进行无量纲化,比如一般会采用来流的速度、密度等物理量进行无量纲化。然而,在本文研究的问题中,边界层外普遍有较大的压力梯度,这导致不同流向位置的边界层外物理量差异比较大,计算很难收敛。所以本文采用当地边界层外的物理量,即$U_e^\dagger,T_e^\dagger,\rho_e^\dagger,k_e^\dagger,\mu_e^\dagger$,进行无量纲化,提高计算稳定性。这里边界层外的物理量是通过求解无粘流方程得到的,并作为边界层方程求解的边界条件。采用当地边界层外物理量无量纲化后的边界层方程为:
\begin{subequations}
\begin{equation}\label{e:BLE6}
  \frac{{\partial \left( {\rho u} \right)}}{{\partial x^\dagger }} + \frac{{\partial \left( {\rho v} \right)}}{{\partial y^\dagger }} + \frac{{\rho u}}{{\rho _e^\dagger U_e^\dagger }}\frac{{\partial \left( {\rho _e^\dagger U_e^\dagger } \right)}}{{\partial x^\dagger }} = 0
\end{equation}
\begin{equation}\label{e:BLE7}
  \rho u\rho _e^\dagger U_e^\dagger U_e^\dagger \frac{{\partial u}}{{\partial x^\dagger }} + \rho uu\rho _e^\dagger U_e^\dagger \frac{{\partial U_e^\dagger }}{{\partial x^\dagger }} + \rho v\rho _e^\dagger U_e^\dagger U_e^\dagger \frac{{\partial u}}{{\partial y^\dagger }} = \rho _e^\dagger U_e^\dagger \frac{{dU_e^\dagger }}{{dx^\dagger }} + \mu _e^\dagger U_e^\dagger \frac{\partial }{{\partial y^\dagger }}\left( {\mu \frac{{\partial u}}{{\partial y^\dagger }}} \right)
\end{equation}
\begin{equation}\label{e:BLE8}
  \rho u\frac{{\partial w}}{{\partial x^\dagger }} + \rho v\frac{{\partial w}}{{\partial y^\dagger }} = \frac{{\mu _e^\dagger }}{{\rho _e^\dagger U_e^\dagger }}\frac{\partial }{{\partial y^\dagger }}\left( {\mu \frac{{\partial w}}{{\partial y^\dagger }}} \right)
\end{equation}
\begin{multline}\label{e:BLE9}
    \rho u\rho _e^\dagger U_e^\dagger C_p^\dagger \left( {T\frac{{\partial T_e^\dagger }}{{\partial x^\dagger }} + T_e^\dagger \frac{{\partial T}}{{\partial x^\dagger }}} \right) + \rho v\rho _e^\dagger U_e^\dagger C_p^\dagger T_e^\dagger \frac{{\partial T}}{{\partial y^\dagger }} \\
    = k_e^\dagger T_e^\dagger \frac{\partial }{{\partial y^\dagger }}\left( {k\frac{{\partial T}}{{\partial y^\dagger }}} \right) - \rho _e^\dagger U_e^\dagger U_e^\dagger \frac{{dU_e^\dagger }}{{x^\dagger }}u + \mu \mu _e^\dagger U_e^\dagger U_e^\dagger \left( {\frac{{\partial u}}{{\partial y^\dagger }}} \right)^2  + \mu \mu _e^\dagger W_e^\dagger W_e^\dagger \left( {\frac{{\partial w}}{{\partial y^\dagger }}} \right)^2
\end{multline}
\end{subequations}
注意到在上面的代换中,还用到了无粘势流中沿流线的伯努利方程:
\begin{equation}
  -\frac{\p p^\dagger}{\p x^\dagger}=\rho^\dagger u^\dagger_e\frac{du^\dagger_e}{dx^\dagger_e}
\end{equation}
和气体状态方程:
\begin{equation}
  \rho T=1
\end{equation}
为了消除上述边界层方程在驻点处的奇异性,引入如下相似变换:
\begin{subequations}
\begin{align}
  \xi  &= x^\dagger \\
  \eta &= \sqrt{\frac{U_e^\dagger}{x^\dagger\rho_e^\dagger\mu_e^\dagger}}\int_{0}^{y^\dagger}\rho^\dagger dy^\dagger
  =\frac{1}{L^\dagger}\int_{0}^{y^\dagger}T^{-1}dy^\dagger
\end{align}
\end{subequations}
最终得到如下计算求解的方程:
\begin{subequations}\label{e:ble10}
\begin{equation}
  \xi\frac{\p u}{\p \xi}+\frac{\p \Lambda}{\p \eta}+\frac{u}{2}\left[ 1+\frac{\xi}{\mu_e^\dagger}\frac{\p \mu_e^\dagger}{\p \xi} + \frac{\xi}{\rho_e^\dagger\mu_e^\dagger}\frac{\p (\rho_e^\dagger\mu_e^\dagger)}{\p \xi}\right]=0
\end{equation}
\begin{equation}
  \xi u\frac{\p u}{\p \xi}+\Lambda\frac{\p u}{\p \eta} -\frac{\xi}{\mu_e^\dagger}\frac{\p \mu_e^\dagger}{\p \xi}(T-u^2)=\frac{\p}{\p\eta}(\frac{\mu}{T}\frac{\p u}{\p\eta})
\end{equation}
\begin{equation}
  \xi u\frac{\p w}{\p \xi}+\Lambda\frac{\p w}{\p \eta}=\frac{\p}{\p\eta}(\frac{\mu}{T}\frac{\p w}{\p\eta})
\end{equation}
\begin{equation}
  \xi u\frac{\p T}{\p \xi}+\Lambda\frac{\p T}{\p \eta} - \frac{1}{\rm Pr}\frac{\p}{\p\eta}(\frac{k}{T}\frac{\p T}{\p\eta})=(\gamma-1)\frac{\mu}{T}\left[ ({\rm Ma}_{ue}\frac{\p u}{\p \eta})^2 + ({\rm Ma}_{we}\frac{\p w}{\p \eta})^2  \right]
\end{equation}
\end{subequations}
其中:
\begin{subequations}
  \begin{align}
    L^\dagger &= \sqrt{\frac{\mu_e^\dagger x^\dagger}{\rho_e^\dagger u_e^\dagger}} \\
    \Lambda &= \xi u\frac{\p \eta}{\p x^\dagger}+\frac{\xi \nu}{L^\dagger T} \\
    {\rm Ma}_{ue} &= \frac{u_e^\dagger}{a_e^\dagger} \\
    {\rm Ma}_{we} &= \frac{w_e^\dagger}{a_e^\dagger} \\
    a_e^\dagger &= \sqrt{\gamma RT_e^\dagger}
  \end{align}
\end{subequations}
将方程(\ref{e:ble10})在法方向采用谱方法进行离散,流向采用五阶差分格式,最后得到离散的方程简记为:
\begin{equation}\label{e:ble_dis}
  L_{dis}(\Phi)=0
\end{equation}
$\Phi=(u,w,\Lambda,T)^T$为方程(\ref{e:ble10})中实际求解的变量组成的矩阵。上式对应的Jacobian矩阵为:
\begin{equation}\label{e:jb}
  \mathbf{J}_b = \frac{\p L_{dis}(\Phi)}{\p \Phi}
\end{equation}
本文采用拟牛顿法对式(\ref{e:ble_dis})进行求解,迭代更新方法如下:
\begin{equation}\label{e:ble_iter}
  \Phi_{\rm new} = \Phi_{\rm old} - \mathbf{J}_b^{-1}L_{dis}(\Phi)
\end{equation}
式(\ref{e:ble_dis})和(\ref{e:jb})的具体形式将在附录\ref{app:ble}中给出。

为了验证程序是否正确,首先将计算结果与零压力梯度平板上的相似性解进行对比。这里采用的计算工况为:
\begin{equation}
  U_\infty  = 100{\rm m/s},T_\infty  = 300{\rm K},\nu_\infty  = 1.5 \times 10^{ - 5} {\rm m^2 /s}
\end{equation}
对比$x=1$m,即$\Rey_x=6.67\times10^6$位置处各个物理量延法向的分布如图\ref{f:0PreassureGradienPlate}所示。其中黑色由方框标记的线为边界层方程求解出来的结果,红色由三角标记出来的先为相似性解的结果。可以看到两种算法的结果几乎完全重合了。
\begin{figure}[htb]
  \centering%

  \begin{subfigure}{0.5\textwidth}
    \includegraphics[width=\textwidth]{ch2/0PreassureGradienPlate_u.jpg}
    \caption{流向速度对比}
  \end{subfigure}%
  \begin{subfigure}{0.5\textwidth}
    \includegraphics[width=\textwidth]{ch2/0PreassureGradienPlate_v.jpg}
    \caption{法向速度对比}
  \end{subfigure}%
  \bigskip

  \begin{subfigure}{0.5\textwidth}
    \includegraphics[width=\textwidth]{ch2/0PreassureGradienPlate_T.jpg}
    \caption{温度对比}
  \end{subfigure}%
  \caption{边界层方程计算结果与相似性解对比(黑线方框标记:边界层方程计算结果;红线三角标记:相似性解)}
  \label{f:0PreassureGradienPlate}
\end{figure}
\begin{figure}[htb]
  \centering
  \includegraphics[width=0.6\textwidth]{ch2/compare_profiles.jpg}
  \caption{后掠翼上边界层速度剖面对比(线:计算结果;点:实验结果)}\label{f:ble_vs_exp}
\end{figure}

本文中主要进行的是三维边界层失稳的研究,所以针对三维边界层的计算也需要验证。清华大学徐胜金老师课题组为研究三维边界层转捩在低湍流度风洞中做了后掠NLF-0415翼型的绕流实验。实验相应参数可以参考文献\cite{wang2018}。在实验自由来流为22.3m/s的工况中,翼型上表面直至70\%弦长处均为层流。采用边界层方程计算速度分布,并取20\%和40\%弦长处的速度剖面与实验对比,结果如图\ref{f:ble_vs_exp}。其中计算结果用线表示,实验结果用点表示。这里$U_{\rm wt}$表示延风洞方向的速度分量\footnote{注意这里并不是$u$,因为在后掠翼计算中,x方向与平行于风洞的流向有45$^\circ$夹角}。蓝色表示20\%弦长处的结果,红色为40\%处。从计算的结果可以看到,我们所采用的求解方法完全满足精度需求。

\subsection{扰动方程}\label{subsec:STBfun}
如之前所述,本文将流场基本变量 $\mathbf{q}=(\rho,~u,~v,~w,~T)$ 分解为基本流动 $\mathbf{q}_0$ 和扰动 $\tilde{\mathbf{q}}$ 两部分:
\begin{equation}\label{EQ_STa}
\mathbf{q}(x,y,z,t)=\mathbf{q}_0(x,y)
+\tilde{\mathbf{q}}(x,y,z,t)
\end{equation}
在小节\ref{subsec:BLfun}中已经探讨了基本流动的求解方法。在这一节中,重点讨论扰动方程(\ref{e:disturbance1})的求解方法。先假设方程(\ref{e:disturbance1})可以写成如下紧凑的形式:
\begin{multline}
 \label{e:EQ_ST}
 {\mathbf{\Gamma }}\frac{{\partial {\mathbf{\tilde q}}}}
 {{\partial t}} + {\mathbf{A}}\frac{{\partial {\mathbf{\tilde q}}}}
 {{\partial x}} + {\mathbf{B}}\frac{{\partial {\mathbf{\tilde q}}}}
 {{\partial y}} + {\mathbf{C}}\frac{{\partial {\mathbf{\tilde q}}}}
 {{\partial z}} + {\mathbf{D\tilde q}}\\ = {\mathbf{H}}_{xx} \frac{{\partial ^2 {\mathbf{\tilde q}}}}
 {{\partial x^2 }} + {\mathbf{H}}_{yz} \frac{{\partial ^2 {\mathbf{\tilde q}}}}
 {{\partial z\,\partial y}} + {\mathbf{H}}_{xy} \frac{{\partial ^2 {\mathbf{\tilde q}}}}
 {{\partial x\,\partial y}} + {\mathbf{H}}_{xz} \frac{{\partial ^2 {\mathbf{\tilde q}}}}
 {{\partial x\,\partial z}} + {\mathbf{H}}_{yy} \frac{{\partial ^2 {\mathbf{\tilde q}}}}
 {{\partial y^2 }} + {\mathbf{H}}_{zz} \frac{{\partial ^2 {\mathbf{\tilde q}}}}
 {{\partial z^2 }} + {\mathbf{N}} + {\mathbf{F}}.
\end{multline}
其中 $5\times5$ 系数矩阵 $\mathbf{\Gamma},~\mathbf{A},~\mathbf{B},~
\mathbf{C},~\mathbf{D},~\mathbf{H}_{xx},~
\mathbf{H}_{yy},~\mathbf{H}_{zz},~\mathbf{H}_{xy},~
\mathbf{H}_{xz},~\mathbf{H}_{yz}$ 是基本流动、流向曲率和 $\Rey,~\Ma,~\Prl$ 的函数,详细表达式可参见附录\ref{app:SE}。向量 $\mathbf{{N}}$ 表示非线性项,$\mathbf{F}$表示体积力产生的源项。


\subsubsection{线性稳定性理论}
由于边界层流动中,边界层厚度增长缓慢,所以可将其近似为平行剪切流。假设扰动具有行波解:
\begin{equation}\label{EQ_LST0}
    \tilde{\mathbf{q}}(x,y,z,t)=
    \hat{\mathbf{q}}(y)e^{\ii (\alpha x+\beta z-\omega t)}+c.c.
\end{equation}
针对边界层失稳问题,其不稳定性通常是对流失稳,即边界层内的扰动并不是在原地增长,而是一边向下游传播一遍增长。针对这一类问题,通常采用空间模式求解,即给定 $\beta$ 和 $\omega$,求解$\alpha$。将式(\ref{EQ_LST0})代入扰动方程( \ref{e:EQ_ST}), 忽略非线性项整理得到
\begin{equation}\label{EQ_LST1}
\mathbf{A}_L\hat{\mathbf{q}}+
\mathbf{B}_L\frac{\p\hat{\mathbf{q}}}{\p y}
-\mathbf{H}_{yy}\frac{\p^2 \hat{\mathbf{q}}}{\p y^2}=
\alpha\left(
\mathbf{M}_L\hat{\mathbf{q}}+
\ii \mathbf{H}_{xy}\frac{\p\hat{\mathbf{q}}}{\p y}\right)
-\alpha^2\mathbf{H}_{xz}\hat{\mathbf{q}}
\end{equation}
其中
\begin{equation}\left.
\begin{aligned}
\mathbf{A}_L&= -\ii\omega\mathbf{\Gamma}
         +\ii\beta\mathbf{C}
         +\mathbf{D}
         +\beta^2\mathbf{H}_{zz}\\
\mathbf{B}_L&= \mathbf{B}-\ii\beta\mathbf{H}_{yz}\\
\mathbf{M}_L&= -\ii\mathbf{A}-\beta\mathbf{H}_{xz}
\end{aligned}~\right\}
\end{equation}
将上式中的几个微分算子记作:
\begin{subequations}\label{e:LST_short}
\begin{align}
  \mathscr{L}_0 &=\mathbf{A}_L+
  \mathbf{B}_L\frac{\p}{\p y}
  -\mathbf{H}_{yy}\frac{\p^2 }{\p y^2} \\
  \mathscr{L}_1 &=-\mathbf{M}_L-\ii \mathbf{H}_{xy}\frac{\p}{\p y}\\
  \mathscr{L}_2 &=\mathbf{H}_{xz}
\end{align}
\end{subequations}
则线性稳定性的控制方程可以写为:
\begin{equation}\label{e:LST}
  \mathscr{L}\hat{\mathbf{q}}=\mathscr{L}_0\hat{\mathbf{q}}+\alpha \mathscr{L}_1\hat{\mathbf{q}} + \alpha^2\mathscr{L}_2\hat{\mathbf{q}}=0
\end{equation}
引入一个辅助变量:
\begin{equation}
  \tilde{\mathbf{q}}_a = \alpha\tilde{\mathbf{q}}
\end{equation}
则式(\ref{e:LST})可以改写为:
\begin{equation}
  \left(
  \begin{array}{cc}
    0 & 1 \\
    \mathscr{L}_0 & \mathscr{L}_1
  \end{array}
  \right)
  \left(
  \begin{array}{c}
    \tilde{\mathbf{q}} \\
    \tilde{\mathbf{q}}_a
  \end{array}
  \right)
  =\alpha
  \left(
  \begin{array}{cc}
    1 & 0 \\
    0 & -\mathscr{L}_2
  \end{array}
  \right)
  \left(
  \begin{array}{c}
    \tilde{\mathbf{q}} \\
    \tilde{\mathbf{q}}_a
  \end{array}
  \right)
\end{equation}
很显然,式(\ref{EQ_LST1})是针对微分算子的广义特征值问题。对其进行离散求解,在法方向采用四阶精度中心差分格式:
\begin{equation}\label{EQ_CF}\left.
\begin{aligned}
    \frac{\p{\hat{\mathbf{q}}_j}}{\p y} &= \frac{{
    {\hat{\mathbf{q}}_{j - 2}}
    - 8{\hat{\mathbf{q}}_{j - 1}}
    + 8{\hat{\mathbf{q}}_{j + 1}}
    - {\hat{\mathbf{q}}_{j + 2}}}}{{12\Delta y}}\\
    \frac{{{\partial ^2}{\hat{\mathbf{q}}_j}}}{{\partial {y^2}}} &= \frac{{
    - {\hat{\mathbf{q}}_{j - 2}}
    + 16{\hat{\mathbf{q}}_{j - 1}}
    - 30{\hat{\mathbf{q}}_j}
    + 16{\hat{\mathbf{q}}_{j + 1}}
    - {\hat{\mathbf{q}}_{j + 2}}}}{{12{{\left( {\Delta y} \right)}^2}}}
\end{aligned}~\right\}
\end{equation}
便可以将这一个微分算子的广义特征值问题转化为矩阵的广义特征值问题。求解该特征值问题,得到特征向量 $\hat{\mathbf{q}}$ 即为扰动分布,特征值 $\alpha$ 虚部相反数 $-\alpha_i$ 为扰动增长率,实部 $\alpha_r$ 为扰动流向波数。


\subsubsection{抛物化扰动方程}
线性稳定性理论有两个缺陷。首先,其采用平行流假设,导致边界层延流向的变化被忽略了。另外,线性假设忽略了非线性项,导致不同模态间的相互作用没有被考虑。抛物化扰动方程(PSE)可以克服上述 这两点缺陷,并且具有很高的求解效率。首先将物理扰动 $\mathbf{\tilde{q}}$ 和非线性项与外加源项之和 $\mathbf{{N}}+\mathbf{{F}}$ 进行 Fourier 展开:
\begin{align}
\label{e:Fourier1}
    {\mathbf{\tilde q}}\left( {x,y,z,t} \right)& = \sum\limits_{m =  - M}^M {\sum\limits_{n =  - N}^N {{\mathbf{\hat q}}_{mn} \left( {x,y} \right)\Theta _{mn} } },\\
%\end{equation}
%\begin{equation}
\label{e:Fourier2}
    {\mathbf{N}} + {\mathbf{F}} &= \sum\limits_{m =  - M}^M {\sum\limits_{n =  - N}^N {{\mathbf{S}}_{mn} \left( {x,y} \right)\Theta _{mn} } },\\
%\end{equation}
%\begin{equation}
\label{e:Fourier3}
    \Theta _{mn}  &= \exp \!\left( {i\int_{x_0 }^x {\alpha _{mn} \left( \xi  \right)d\xi }  + in\beta z - im\omega t} \right).
\end{align}
其中$\Theta _{mn}$是波数函数。代入扰动方程(\ref{e:EQ_ST}),整理得到
\begin{equation}
\label{e:unPSE}
    {\mathbf{\hat A}}\frac{{\partial {\mathbf{\hat q}}_{mn} }}{{\partial x}}
  + {\mathbf{\hat B}}\frac{{\partial {\mathbf{\hat q}}_{mn} }}{{\partial y}}
  + {\mathbf{\hat C}}\frac{{\partial^2 {\mathbf{\hat q}}_{mn} }}{{\partial x^2}}
  + {\mathbf{\hat D\hat q}}_{mn}
  - {\mathbf{H}}_{yy}\frac{{\partial ^2 {\mathbf{\hat q}}_{mn} }}{{\partial y^2 }}
  = {\mathbf{S}}_{mn},
\end{equation}
其中
\begin{equation}
\begin{aligned}
  \mathbf{\hat A} & = {\mathbf{A}} - 2i\alpha_{mn}{\mathbf{H}}_{xx} - in\beta{\mathbf{H}}_{xz}  , \\
  \mathbf{\hat B} & = {\mathbf{B}} -  i\alpha_{mn}{\mathbf{H}}_{xy} - in\beta{\mathbf{H}}_{yz}   , \\
  \mathbf{\hat C} & = {\mathbf{H}}_{xx} , \\
  \mathbf{\hat D}  &= {\mathbf{D}} - im\omega {\mathbf{\Gamma }} + i\alpha_{mn} {\mathbf{A}} + in\beta {\mathbf{C}} + {\mathbf{H}}_{xx} \left( {\alpha_{mn}^2  - i\frac{{d\alpha }}{dx}} \right) + n^2\beta ^2 {\mathbf{H}}_{zz} + n\beta\alpha_{mn}{\mathbf{H}}_{xz} . \\
\end{aligned}
\end{equation}
根据量级分析\cite{Malik1999}, ${{d\alpha }}/{dx}$ 这一项非常小可以忽略。为了使得形函数$\mathbf{\hat q}$在流向缓变,提出针对 $\alpha$的波数迭代条件:
\begin{equation}
\label{e:auxiliary}
    \int_0^\infty  {{\mathbf{\hat q}}^H {\mathbf{M}}\frac{{\partial {\mathbf{\hat q}}}}{{\partial x}}\,dy}  = 0\qquad\forall x.
\end{equation}
这里$\mathbf{M}=\mathrm{diag}(0,1,1,1,0)$, ``$H$''表示复共轭。式(\ref{e:auxiliary})又可以叫做形函数的缓变条件,这一条件使得形函数在流向的二阶偏导数可以被忽略掉,即${{\partial ^2 {\mathbf{\hat q}}_{mn} }}/{{\partial x^2 }}=0$\cite{Malik1994}。虽然二阶偏导数项被忽略掉了,但是方程(\ref{e:unPSE})依然有一些残余椭圆性\cite{LiMalik1996}。针对这一问题,将方程中的压力项修正为:
\begin{equation}
    \frac{\partial \tilde p_{mn}}{\partial x} = i\alpha_{mn}\hat p_{mn}\Theta_{mn}.
\end{equation}
采用上面所提到的诸多假设,方程(\ref{e:unPSE})可以完全被抛物化,可以流向推进求解。完整的方程为:
\begin{equation}
\label{PSE1}
    \mathscr{L}_{\rm PSE}{\mathbf{\hat q}}_{mn}  = {\mathbf{\hat A}}\frac{{\partial {\mathbf{\hat q}}_{mn} }}
    {{\partial x}} + {\mathbf{\hat B}}\frac{{\partial {\mathbf{\hat q}}_{mn} }}
    {{\partial y}} + {\mathbf{\hat D\hat q}}_{mn}  - {\mathbf{H}}_{yy} \frac{{\partial ^2 {\mathbf{\hat q}}_{mn} }}
    {{\partial y^2 }} = {\mathbf{S}}_{mn},
\end{equation}
其中$\mathscr{L}_{\rm PSE}$线性PSE算子。本文对方程(\ref{PSE1})在流向采用隐式欧拉差分,法向采用五阶中心差分进行离散求解。

为了验证程序的正确性,我们与Malik等人1994年的工作\cite{Malik1994}进行对比。该工作重点研究了后掠Hiemenz流动的失稳,计算相关参数详见他们的文献。这里计算对比$\bar{R}=500$\footnote{这个符号采用与文献\cite{Malik1994}中相同的定义}工况中主模态的能量在流向的演化,结果如图\ref{f:com_malik}。可见该计算求解程序是满足要求的。
\begin{figure}
  \centering
  \includegraphics[width=0.7\textwidth]{ch2/comparison_Malik.jpg}
  \caption{PSE计算程序验证}\label{f:com_malik}
\end{figure}


\subsection{扰动发展的敏感性分析}
为了更好地理解流动,选取较优化的控制参数,本文中对三维边界层失稳进行了敏感性分析。关于流动失稳的敏感性分析最早始于2003年,是Bottaro\cite{Bottaro2003}等人针对Couette流动开展的。通过求解线性稳定性问题的伴随问题,他们找出了容易受基本流变化影响的失稳模态。之后,2008年Marquet等人\cite{Marquet2008}分析了圆柱尾迹流动对于基本流和外加体积力的敏感性,并采用这一结果进行了优化,降低了尾迹的湍流度。Alizard等人\cite{Alizard2010}2010年,对角域流动进行了分析,得到了不同失稳模态的敏感函数(敏感因子)的空间分布。2011年Brandt等人\cite{Brandt2011}对平板边界层做了相应的敏感性分析,之后学者们又对D形圆柱\cite{Meliga2012},空腔\cite{Bromwne2014},甚至湍流边界层的猝发过程进行了相应的分析\cite{Alizard2015},更加深入的了解了其流动机理。本文分别从线性稳定性理论和抛物化扰动方程出发,推导他们的伴随方程,并进而分析三维边界层失稳的敏感性。
\subsubsection{基于线性稳定性理论的敏感性分析}
记方程(\ref{e:LST})的伴随方程为:
\begin{equation}\label{e:aLST}
  \mathscr{L}^*\hat{\mathbf{p}}=\mathscr{L}_0^*\hat{\mathbf{p}}+\alpha \mathscr{L}_1^*\hat{\mathbf{p}} + \alpha^2\mathscr{L}_2^*\hat{\mathbf{p}}=0
\end{equation}
伴随方程与原方程的关系是,对于任意向量$\mathbf{a},\mathbf{b}$,都有:
\begin{equation}
    \int_{0}^{+\infty}\mathbf{a}\cdot(\mathscr{L}\mathbf{b})^Tdy=\int_{0}^{+\infty}(\mathscr{L}^*\mathbf{a})\cdot\mathbf{b}^Tdy
\end{equation}
若定义内积$<\mathbf{a},\mathbf{b}>=\int_{0}^{+\infty}\mathbf{a}\cdot\mathbf{b}^Tdy$,则有:
\begin{equation}
  <\mathbf{a},\mathscr{L}\mathbf{b}> = <\mathscr{L}^*\mathbf{a},\mathbf{b}>
\end{equation}
引入体积力后,方程变为:
\begin{equation}\label{}
  \left[\mathscr{L}_0+(\alpha + \delta\alpha) \mathscr{L}_1 + (\alpha + \delta\alpha)^2\mathscr{L}_2\right](\hat{\mathbf{q}}+\delta\hat{\mathbf{q}})=\mathbf{F}
\end{equation}
其中$\delta\alpha$和$\delta\hat{\mathbf{q}}$为因为引入体积力产生的特征值和特征向量的变化。由于本文中均采用的是微弱的体积力控制失稳,所以这两个量都是小量。将上式与伴随向量(伴随方程的解)做内积,并忽略高阶小量,得到:
\begin{equation}\label{}
  \begin{aligned}
    <\hat{\mathbf{p}},\mathbf{F}> &= <\hat{\mathbf{p}},\left[\mathscr{L}_0+(\alpha + \delta\alpha) \mathscr{L}_1 + (\alpha + \delta\alpha)^2\mathscr{L}_2\right](\hat{\mathbf{q}}+\delta\hat{\mathbf{q}})> \\
    &\approx  <\hat{\mathbf{p}},(\delta\alpha \mathscr{L}_1+2\delta\alpha\mathscr{L}_2)\hat{\mathbf{q}}>
  \end{aligned}
\end{equation}
最终得到空间模式的复特征值关于体积力的敏感性为:
\begin{equation}\label{e:LST_adjoint}
  \delta\alpha \approx \frac{<\hat{\mathbf{p}},\mathbf{F}>}{<\hat{\mathbf{p}},( \mathscr{L}_1+2\mathscr{L}_2)\hat{\mathbf{q}}>}
\end{equation}
\subsubsection{基于抛物化扰动方程的的敏感性分析}
这一小节推导PSE的伴随方程和相应的敏感性分析公式。这里的推导与Pralits\cite{pralits2000sensitivity}文章中的类似。据作者所知,本文中对三维边界层的敏感性分析还尚属国内外首次。由于本文研究的控制方法主要还是集中在扰动的线性增长区,所以这里的PSE还只是线性的PSE,并没有考虑非线性效应。所以式(\ref{PSE1})中的源项${\mathbf{S}}_{mn}$只包含与流畅解耦的体积力项,并不包含非线性项。另外,忽略了非线性效应也就相当于忽略了模态互相干扰,也使得每次计算中只需要考虑一个模态。 这样,本身是用来区分模态的下标$nm$就可以略去不写。于是PSE方程可以写为\footnote{为了简洁,这一节将所有“PSE”下标略去}:
\begin{equation}
\label{LPSE1}
    \mathscr{L}{\mathbf{\hat q}}  = {\mathbf{\hat A}}\frac{{\partial {\mathbf{\hat q}} }}
    {{\partial x}} + {\mathbf{\hat B}}\frac{{\partial {\mathbf{\hat q}} }}
    {{\partial y}} + {\mathbf{\hat D\hat q}}  - {\mathbf{H}}_{yy} \frac{{\partial ^2 {\mathbf{\hat q}} }}
    {{\partial y^2 }} = {\mathbf{S}}
\end{equation}
在研究敏感性问题时,需要有一个输出目标函数。在基于LST的敏感性分析中,扰动模态的增长率是一个天然且绝佳的候选项。在PSE的敏感性分析中,这里将下游某一位置的扰动能量作为输出函数。实际上敏感性分析可以理解为这一输出函数对输入变量,本文中也就是体积力的导数。输出函数的具体定义见式(\ref{e:PSEoutput_energy})
\begin{equation}
\label{e:PSEoutput_energy}
J = E = \left[ {\frac{1}
{2}\int_0^{T_z } {\int_0^\infty  {{\mathbf{\tilde q}}^H {\mathbf{M\tilde q}}dydz} } } \right]_{x = x_1 }  %= \left[ {\frac{1}
%{2}\int_0^{T_z } {\int_0^\infty  {\left( {{\mathbf{\hat q}}\Theta } \right)^H {\mathbf{M}}\left( {{\mathbf{\hat q}}\Theta } \right)dydz} } } \right]_{x = X_1 }
= \frac{1}
{2}\int_0^{T_z } {\int_0^\infty  {\left| {\Theta _1 } \right|^2 {\mathbf{\hat q}}_1 ^H {\mathbf{M\hat q}}_1 dydz} }
\end{equation}
上式中,下标`1'代表着计算出口处,也就是提取输出函数位置的物理量。$T_z$是失稳模态的展向波长。之后,对输出函数、控制方程(\ref{LPSE1})和波数迭代条件(\ref{e:auxiliary})做变分,得到下式:
\begin{equation}\label{e:bianfenJ}
\begin{aligned}
\delta J & = \frac{1}
{2}\int_0^{T_z } {\int_0^\infty  {\left| {\Theta _1 } \right|^2 {\mathbf{\hat q}}_1 ^H {\mathbf{M}}\delta {\mathbf{\hat q}}_1 dydz} }  \\
         &+ \frac{1}{2}\int_0^{T_z } {\int_0^\infty  {\left| {\Theta _1 } \right|^2 {\mathbf{\hat q}}_1 ^H {\mathbf{M\hat q}}_1 \left( {{\rm i}\int_{x_0 }^{x_1 } {\delta \alpha (x')dx'} } \right)dydz} }  + c.c \\
\end{aligned}
\end{equation}
\begin{equation}
\label{e:dLPSE}
\mathscr{L}\delta {\mathbf{\hat q}} - \delta {\mathbf{S}} + \frac{{\partial {\mathscr{L}}}}
{{\partial \alpha }}\delta \alpha {\mathbf{\hat q}} = 0
\end{equation}
\begin{equation}
\label{e:dAuxilary}
\int_0^\infty  {\left( {\delta {\mathbf{\hat q}}^H {\mathbf{M}}\frac{{\partial {\mathbf{\hat q}}}}
{{\partial x}} + {\mathbf{\hat q}}^H {\mathbf{M}}\frac{{\partial \delta {\mathbf{\hat q}}}}
{{\partial x}}} \right)dy}  = 0
\end{equation}
这里$c.c.$代表方程中所有项的复共轭,$x_0$和$x_1$分别是入口和出口处的流向位置。在进行下一步推导之前,需要先定义内积。在PSE的框架下,本文将内积定义为两个场向量点积之后在全计算域内的积分,即$\mathbf a$与$\mathbf b$的积分记作:
\begin{equation}
\label{e:innerproduct}
<\mathbf{a},\mathbf{b}>=\int_{0}^{T_z}\int_{x_0}^{x_1}\int_{0}^\infty (\mathbf{a}^H \cdot \mathbf{b})dydxdz
\end{equation}
其实接下来需要做的就是想办法消去式(\ref{e:bianfenJ})中的中间变量。这里,引入一个所谓的伴随向量${\mathbf{\hat q}}^*$和复变量$r^*(x)$。令这两个量分别与方程(\ref{e:dLPSE})和(\ref{e:dAuxilary})做内积,并加上他们的复共轭,可以得到恒等式:
\begin{equation}
\label{e:dadjoint1}
\begin{aligned}
     \int_{0}^{T_z} {\int_{x_0}^{x_1}{r^* \int_0^\infty  {\left( {\delta {\mathbf{\hat q}}^H {\mathbf{M}}\frac{{\partial {\mathbf{\hat q}}}}{{\partial x}} + {\mathbf{\hat q}}^H {\mathbf{M}}\frac{{\partial \delta {\mathbf{\hat q}}}}{{\partial x}}} \right)dydxdz} } }%\\
     +<{\mathbf{\hat q}}^*,{\mathscr{L}}\delta {\mathbf{\hat q}}-\delta {\mathbf{S}} + \frac{{\partial {\mathscr{L}}}}{{\partial \alpha }}\delta \alpha {\mathbf{\hat q}} > + c.c. = 0
\end{aligned}
\end{equation}
很显然,对于任意的$\mathbf{\hat q}^*$和$r^*$都能够使上面的等式成立。为了消除中间变量,需要选取好的$\mathbf{\hat q}^*$和$r^*$。这里,本文令这两个变量分别满足伴随方程(\ref{e:adjointa})和伴随波数迭代条件(\ref{e:adjoint})。很容易证明,伴随方程也是抛物型的,计算的时候可以从下游向上游回溯推进求解。
\begin{equation}
\label{e:adjoint}
\mathscr{L}^* {\mathbf{\hat q}}^*  = \left( {\bar r^*  - r^* } \right){\mathbf{M}}\frac{{\partial {\mathbf{\hat q}}}}
{{\partial x}} + \frac{{\partial \bar r^* }}
{{\partial x}}{\mathbf{M\hat q}}
\end{equation}
\begin{equation}
\label{e:adjointa}
\int_0^\infty  {\left( {\mathbf{\hat q} ^{*H} \frac{{\partial {\mathscr{L}}}}
{{\partial \alpha }}\mathbf{\hat q} } \right)dy}  = \int_0^\infty  {{\rm i}\left| {\Theta _1 } \right|^2 \mathbf{\hat q} _1 ^H \mathbf{M\hat q} _1 dy}
\end{equation}
$\mathscr{L}^*$是PSE伴随方程算子,字母上的横线表示相应变量的复共轭。伴随向量和$r^*$计算的初始值,也就是出口处的值为:
\begin{equation}
\label{e:adjointini}
%\begin{aligned}
\begin{gathered}
c = \frac{{ - \int_0^\infty  {{\rm i}\left| {\Theta _1 } \right|^2 {\mathbf{\hat q}}_1 ^H {\mathbf{M\hat q}}_1 dy} }}
{{\left. {\int_0^\infty  {\left( {{\mathbf{\hat q}}_1 ^H {\mathbf{M}}\left( {{\mathbf{\hat A}}} \right)^{ - 1} \frac{{\partial {\mathscr{L}}}}{{\partial \alpha }}{\mathbf{\hat q}}} \right)dy} } \right|_{x = x_1 } }} \hfill \\
{\mathbf{\hat q}}^* _1  =  - \bar c\left( {{\mathbf{\hat A}}^H } \right)^{ - 1} {\mathbf{M\hat q}}_1  \hfill \\
r_1^*  = c + \left| {\Theta _1 } \right|^2  \hfill \\
\end{gathered}
%\end{aligned}
\end{equation}
综上,如果伴随向量和$r^*$满足上面给出的式(\ref{e:adjoint})(\ref{e:adjointa})和(\ref{e:adjointini}),那么式(\ref{e:dadjoint1})可以简化为:
\begin{equation}
\label{e:dj}
\delta J = \frac{1}
{2} < \hat \varphi ^* ,\delta {\mathbf{S}} >  + c.c.
\end{equation}
也就是说,不再需要求解PSE方程,直接通过计算伴随向量和外加体积力的内积,便可以直接得到体积力对下游扰动能量的影响。当然,式(\ref{e:dj})中的$\delta {\mathbf{S}}$还是形函数控制方程中的源项的变化,需要将其转换到实际的体积力中。本文做稳定分析重点在于对比添加控制和不添加控制下的扰动能量,所以令初始的体积力为0,这样的话,体积力的变化可以写为:
\begin{equation}
\label{e:df}
\delta {\mathbf{F}} = \Theta \delta {\mathbf{S}} + {\mathbf{S}}{\rm i}\int_{x_0 }^x {\delta \alpha (x')dx'}  = \Theta \delta {\mathbf{S}}
\end{equation}
代入式(\ref{e:dj}),得到:
\begin{equation}
\label{e:dj2}
\delta J = \frac{1}
{2} < \hat \varphi ^* ,\frac{{\delta {\mathbf{F}}}}
{\Theta } >  + c.c.
\end{equation}
将上式展开并写成分量的形式,最终得到:
\begin{equation}
\label{e:adjointresult}
\delta J = \int_0^{T_z } {\int_{x_0 }^{x_1 } {\int_0^\infty  {\left( {G_u \delta f_x  + G_v \delta f_y  + G_w \delta f_z } \right)dydxdz} } }
\end{equation}
\begin{equation}
\label{e:G}
\begin{gathered}
G_u  = {\rm real}(\hat u^{*H} \exp({ - {\rm i}\int_{x_0 }^x {\alpha (x')dx'}  - in\beta z}) )\\
G_v  = {\rm real}(\hat v^{*H} \exp({ - {\rm i}\int_{x_0 }^x {\alpha (x')dx'}  - in\beta z}) )\\
G_w  = {\rm real}(\hat w^{*H} \exp({ - {\rm i}\int_{x_0 }^x {\alpha (x')dx'}  - in\beta z}) )\\
\end{gathered}
\end{equation}
本文将$G_u$, $G_v$ and $G_w$命名为流向、法向、展向敏感性因子,并在之后展示的结果中给出他们的具体分布,以探究体积力对失稳过程的影响。
\section{充分发展槽道的直接数值模拟}\label{sec:DNS}
本文采用王志坚教授课题组开发的hpMusic高精度多物理场求解程序进行充分发展槽道的直接数值模拟。该程序采用高精度通量重构方法(Flux Reconstruction,FR 或 Correction Porcedure via Reconstruction,CPR)。这一方法最早是由Huynh\cite{Huynh2007}提出来的,之后Wang和Gao\cite{WangGao2009}将其扩展到包含各种不同形状的混合非结构网格上。这一算法的详细介绍读者可以参考相关文献\cite{Vincent2011,Jameson2012,GaoWang2013,Zhu2016,Zh2017}。这一方法属于间断有限元方法的一种,有些类似间断伽辽金方法。由于CPR方法有微分形式的表达,不需要进行显示的数值积分。另外,CPR方法也提供了一种通用的框架,其他高精度的有限元方法也可以在该框架下表达。

这里以双曲对流方程为例,简要的介绍一下FR/CPR方法。将双曲对流方程简记为:
\begin{equation}\label{e:hyperbolic}
  \frac{\p \mathbf{U}}{\p t} + \nabla \cdot \mathbf{F}(\mathbf{U}) = 0
\end{equation}
其中$\mathbf{U}$为包含守恒变量的向量,$\mathbf{F}$为通量张量\footnote{注意:在这一小节$\mathbf{F}$均表示通量(Flux)}。采用非交错网格单元离散计算域,并在单元内引入任意的权函数$W$,在单元$V_i$上的加权残差公式可以写为:
\begin{equation}\label{e:wresidual}
  \int\limits_{V_i}\left[ \frac{\p \mathbf{U}}{\p t} + \nabla \cdot \mathbf{F}(\mathbf{U})  \right]W{\rm d}\Omega=0
\end{equation}
在FR/CPR方法中,单元内的守恒变量分布都被假设为多项式形式。这些多项式的具体形式都采用单元内求解点(solution points,SPs)上的值来表达。对通量散度项进行分部积分,将法通量用黎曼通量$\mathbf{F}_{\rm com}^n$替换掉,得到:
\begin{equation}\label{e:CPR1}
  \int\limits_{V_i}\frac{\p \mathbf{U}_i}{\p t}W{\rm d}\Omega
  + \int\limits_{V_i}W \nabla \cdot \mathbf{F}(\mathbf{U}_i) {\rm d}\Omega
  + \int\limits_{\p V_i}W [\mathbf{F}_{\rm com}^n - \mathbf{F}^n(\mathbf{U}_i)] {\rm d}S
  =0
\end{equation}
其中,黎曼通量定义为:
\begin{equation}
  \mathbf{F}_{\rm com}^n = \mathbf{F}_{\rm com}^n(\mathbf{U}_i,\mathbf{U}_{i+},\mathbf{n})
\end{equation}
这里$\mathbf{U}_{i+}$是求解单元界面外的物理量。$\mathbf{n}$表示界面的法向量。界面的法通量为:
\begin{equation}
  \mathbf{F}^n(\mathbf{U}_i)=\mathbf{F}(\mathbf{U}_i) \cdot \mathbf{n}
\end{equation}
显然,如果方程(\ref{e:CPR1})中的界面积分能够被转化成单元内积分,那么权函数就可以被消去。于是引入提升算子$\delta_i$,其在不同单元内的定义如下:
\begin{equation}\label{e:CPR2}
  \int\limits_{V_i}W\delta_i{\rm d}\Omega = \int\limits_{\p V_i}W [\mathbf{F}^n] {\rm d}S
\end{equation}
其中$[\mathbf{F}^n]=\mathbf{F}_{\rm com}^n - \mathbf{F}^n(\mathbf{U}_i)$是法通量差。由方程(\ref{e:CPR1})和(\ref{e:CPR2})可以得到:
\begin{equation}\label{e:CPR3}
  \int\limits_{V_i}\left[ \frac{\p \mathbf{U}_i}{\p t} + \nabla \cdot \mathbf{F}(\mathbf{U_i})  + \delta_i\right]W{\rm d}\Omega=0
\end{equation}
由于N-S方程的通量是守恒变量的非线性函数,所以通量的散度不在多项式空间内,需要进行投影:
\begin{equation}
  \int\limits_{V_i}\Pi [\nabla \cdot \mathbf{F}(\mathbf{U}_i)]W{\rm d}\Omega
  = \int\limits_{V_i}\nabla \cdot \mathbf{F}(\mathbf{U}_i)W{\rm d}\Omega
\end{equation}
由于权函数$W$是任意选取的,所以式(\ref{e:CPR3})可以拿掉积分号,得到微分形式:
\begin{equation}
  \frac{\p \mathbf{U}_i}{\p t} + \Pi[\nabla \cdot \mathbf{F}(\mathbf{U}_i)]  + \delta_i = 0
\end{equation}
之前提到过,结果的多项式的具体形式是由求解点上的物理量上的值决定的。求解点位置的选择并不会影响线性问题的精度,但是会影响非线性问题的精度。在本文中,求解点均选为高斯点。最终求解方程的离散表达为:
\begin{equation}
  \frac{\p \mathbf{U}_{i,j}}{\p t} + \Pi_j[\nabla \cdot \mathbf{F}(\mathbf{U}_{i,j})]  + \delta_{i,j} = 0
\end{equation}
其中j为求解点编号。对于映射算子$\Pi$这里采用拉格朗日插值法,即假设通量也在相同的多项式空间内。提升算子$\delta_i$的计算依赖于界面上求解通量的通量点的位置选择。其表达式为:
\begin{equation}
  \delta_{i,j}=\frac{1}{\left| V_i\right|}\sum\limits_{f,m}\alpha_{j,f,m}[\mathbf{F}^n_{f,m}]S_f
\end{equation}
其中f为界面单元编号,m为界面上通量点编号。$\alpha$为线性组合系数,只取决于相关几何信息和权函数的选择。

对于方程中的粘性项,本文所用程序采用Bassi–Rebay 2格式求解\cite{Bassi1997}。这一部分不是CPR框架的重点,在此不再赘述。具体细节参见原文献或朱辉的博士论文\cite{zhuhui}。在计算中,体积力产生的源项直接放在残差中考虑。