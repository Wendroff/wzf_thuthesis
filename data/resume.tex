\begin{resume}

  \resumeitem{个人简历}

  1991 年 10 月 5 日出生于陕西省西安市长安县(现 长安区)。

  2009 年 9 月考入清华大学航天航空学院工程力学系钱学森力学班,2013 年 7 月本科毕业并获得工学学士学位。

  2013 年 9 月免试进入清华大学大学航天航空学院攻读力学博士学位至今。

  \researchitem{发表的学术论文\footnote{审稿中文章:Zhefu Wang, Liang Wang, Qingyang Wang, Shengjin Xu and Song Fu. "Control of crossflow instability over a swept wing using dielectric-barrier-discharge plasma actuators", International Journal of Heat and Fluid Flow}} % 发表的和录用的合在一起

  % 1. 已经刊载的学术论文(本人是第一作者,或者导师为第一作者本人是第二作者)
  \begin{publications}
    \item Zhefu Wang, Liang Wang, and Song Fu. "Control of stationary crossflow modes in swept Hiemenz flows with dielectric barrier discharge plasma actuators", Physics of Fluids, 2017, 29(9): 094105. (SCI 收录,WOS:000412105100038)
    \item Zhefu Wang, Liang Wang, and Song Fu. "Sensitivity analysis of crossflow boundary layer and transition delay using plasma actuator", 8th AIAA Flow Control Conference, AIAA AVIATION Forum, (AIAA 2016-3933). (EI 收录,ISBN-13: 9781624104329)
    \item Zhefu Wang, Liang Wang, and Song Fu. "Control of crossflow instability using plasma actuators", 7th Asia-Pacific International Symposium on Aerospace Technology, 25-27 November 2015, Cairns, Australia. (会议论文)
    \item Zhefu Wang and Song Fu. "Control of crossflow instability using plasma actuators", XXIV ICTAM, 21-26 August 2016, Montreal, Canada.(会议论文)
    \item Zhefu Wang and Song Fu. "Transition delay using DBD plasma actuators", European Drag Reduction and Flow Control Meeting, 3-6 April 2017, Rome, Italy. (会议论文)
  \end{publications}

  % 2. 尚未刊载,但已经接到正式录用函的学术论文(本人为第一作者,或者
  %    导师为第一作者本人是第二作者)。
  %\begin{publications}[before=\publicationskip,after=\publicationskip]
%    \item Yang Y, Ren T L, Zhu Y P, et al. PMUTs for handwriting recognition. In
%      press. (已被 Integrated Ferroelectrics 录用. SCI 源刊.)
%  \end{publications}

  % 3. 其他学术论文。可列出除上述两种情况以外的其他学术论文,但必须是
  %    已经刊载或者收到正式录用函的论文。
  %\begin{publications}
%    \item Wu X M, Yang Y, Cai J, et al. Measurements of ferroelectric MEMS
%      microphones. Integrated Ferroelectrics, 2005, 69:417-429. (SCI 收录, 检索号
%      :896KM)
%    \item 贾泽, 杨轶, 陈兢, 等. 用于压电和电容微麦克风的体硅腐蚀相关研究. 压电与声
%      光, 2006, 28(1):117-119. (EI 收录, 检索号:06129773469)
%    \item 伍晓明, 杨轶, 张宁欣, 等. 基于MEMS技术的集成铁电硅微麦克风. 中国集成电路,
%      2003, 53:59-61.
%  \end{publications}

  %\researchitem{研究成果} % 有就写,没有就删除
%  \begin{achievements}
%    \item 任天令, 杨轶, 朱一平, 等. 硅基铁电微声学传感器畴极化区域控制和电极连接的
%      方法: 中国, CN1602118A. (中国专利公开号)
%    \item Ren T L, Yang Y, Zhu Y P, et al. Piezoelectric micro acoustic sensor
%      based on ferroelectric materials: USA, No.11/215, 102. (美国发明专利申请号)
%  \end{achievements}

\end{resume}
