\chapter{后掠翼上流动失稳分析与控制}
\section{后掠翼上流动的稳定性分析}
数值研究了后掠Hiemenz流动转捩控制之后,本文将研究目标转移到更贴近实际的工况,研究一个真实的后掠翼流动的转捩控制。清华大学(Tsinghua University)在2018年进行了NLF-0415后掠翼上翼面的转捩实验研究工作\cite{wang2018},本文以该实验的工况作为基准,研究等离子体在实际的后掠翼流动中控制转捩的效果。这一实验在后文中将会被简记为``THU实验''或``THU experiment''。这里先简述一下实验的设备和工况。该实验是在清华大学自己的低湍流度风洞中进行的,风洞试验段大小为1.2m$\times$1.2m$\times$3m。风洞内安装NLF-0412翼型,如图\ref{f:experiment}所示。在实验中采用的直角坐标系为($X_{\rm wt},Y_{\rm wt},Z_{\rm wt}$),其$X_{\rm wt}$方向与风洞的流向平行。在计算中本文用的到直角坐标系为$x,y,z$(如图\ref{f:experiment})。计算所用坐标系的$z$方向与翼型前缘线平行,这样在采用了无限展长假设之后,这一方向的物理量就是均匀的。该风洞运行时测试段的风速可以为5.0到90.0m/s,湍流度可以低到0.05\%。此次试验采用的翼型的弦长$c$=1.2m,有45$^\circ$后掠和-4$^\circ$攻角。该翼型在上翼面有很强的顺压梯度,可以有效的抑制T-S波失稳\cite{Dagenhart1999}。翼型中间到风洞入口大约1.25m。

实验中采用了单一的热线测量了来流的情况。来流风速均匀,湍流度0.08$\sim$0.1\%。采用边界层热线(TSI Model 1261A)测量翼型$X/C$ = 0.2, 0.4, 0.6位置处的边界层速度剖面。热线采用一个电脑控制的可以进行三维运动的机械臂驱动。最小的运动精度可以达到10$\mu$m。在每一个位置,沿着法向每隔0.05mm测量了沿着风洞方向的速度分量$U_{\rm wt}$。这个速度分量也就是图中所示$X_{\rm wt}$坐标的方向。沿着法向的测量一直到两倍边界层厚度的位置终止。为了防止测量热线因为接触避免被损坏,第一个测量点选取在离壁面0.15mm处。这一距离是通过镜面法测量得到的。也就是通过测量热线和其在翼面上的镜像的距离,得到其相对于翼面的距离。

为了验证本文中基本流计算的结果,这里选取了雷诺数$Re$为$1.81\times10^6$,也就是自由来流速度$U_\infty$=22.3m/s的算例进行比较。($Re={U_\infty c}/{\nu}$ 其中$U_\infty$是自由来流速度度,$c$是弦长,$\nu$是动力粘性系数。)在这个工况下,上翼面一直到70\%弦长的位置都是层流。在之后研究得我等离子体控制算例中,为了加速失稳和转捩,采用了两倍雷诺数也就是两倍来流速度的算例。

\begin{figure}
  \centering
  \includegraphics[width=\textwidth]{ch4/experiment.jpg}
  \caption{风洞俯视图( `` $\bullet$ "表示热线测量位置)}\label{f:experiment}
\end{figure}

如上一章所述,高精度重构修正有限元程序Music\cite{WangZJ2009,Zhu2016,Zh2017}被用来进行无粘流求解计算。图\ref{f:ConpareInvicidV}对比了计算求解得到的壁面上的无粘流延风洞流向速度分量和实际实验中测得的边界层外的同方向的速度分量。其中黑线是计算得到的结果,三个大红点是实验测得的结果。由于实验仅测量了三个展向位置的速度,所以这里只有三个数据点。从图中可以看到,无粘流计算得到的结果与实验吻合良好。需要提及的是,实验中并没有精确测量离翼型非常远位置处的来流速度,也就是说实际上真正的自由来流是不能准确得知的。而计算中是采取的无量纲计算,所以本位以20\%弦长位置边界层外缘的流速作为参考基准点,重新对计算结果进行了有量纲化。在实验中,虽然直到70\%弦长处都是层流,但是60\%弦长处边界层已经出现了强烈的扭曲。所以这里之对比了20\%和40\%弦长处的边界层剖面,如图\ref{f:compare_profiles}。与上图类似的,线是计算结果,点是实验测得的结果。这里并没有用所有测量数据,只是每个两个点取了一个数据。对比结果说明本文计算得到的基本流还是非常可靠地。
\begin{figure}
\centering
  % Requires \usepackage{graphicx}
  \includegraphics[width=0.7\textwidth]{ch4/compare_UexpOUT_laminarcase}
\caption{边界层外缘风洞流向速度$U_{\rm{wt}}$对比}
\label{f:ConpareInvicidV}
\end{figure}

\begin{figure}
\centering
  % Requires \usepackage{graphicx}
  \includegraphics[width=0.6\textwidth]{ch4/compare_profiles}
\caption{风洞流向速度$U_{\rm{wt}}$ 在$X/C= 20\%$和$40\%$处剖面对比}
\label{f:compare_profiles}
\end{figure}
Reibert等人\cite{Reiberit1996}也做过相同翼型相同工况的实验研究,在后文中将其命名为``Reibert's experiment''或``Reibert实验''。但是他们的实验采用的风动和模型尺寸均与本文参考的实验不一样。在Reibert实验中,模型弦长1.83m,风洞实验段尺寸1.4m$\times$1.4m$\times$5m。更重要的,他们在文献中并没有给出翼型的安装位置。这些实验设置的不同,导致了两个实验得到的上翼面压力系数的分布不太一样。图\ref{f:CpCompare}给出了THU实验和Reibert实验的上翼面压力分布。可以看到,THU实验的压力梯度要比Reibert实验的压力梯度强一些,这也说明了THU实验中流体在翼型中段的加速度更大一些。为了搞清楚压力梯度的变化对扰动的发展有什么影响,本文设置了4个算例进行对比研究。其中两个算例采用了Reibert实验得到的压力分布,他们的自由来流速度分别是22.3m/s和44.5m/s。另外连个算例采用THU实验得到的压力分布,自由来流速度也分别是22.3m/s和44.5m/s。这四个算例中边界层外缘流向速度分布与边界层位移厚度展示在图\ref{fig:CompOutFlow}中。可以看到,在采用THU算例的工况中,流体在靠近前缘位置加速比较慢,但是在翼型中段,其速度快速提升,并在30\%弦长处超过了采用Reibert实验工况中的边界层外缘流向速度。不管在哪一个雷诺数条件下,采用THU压力分布的算例中的边界层厚度都要比采用Reibert的对应算例要厚。在70\%弦长处,压力达到最低值,在之后均是逆压梯度,边界层快速增长。
\begin{figure}[htb]
\centering
  % Requires \usepackage{graphicx}
  \includegraphics[width=0.7\textwidth]{ch4/compareCp_Reibert}
  \caption{THU(清华)实验中压力系数(红)与Reibert博士论文 \cite{Reiberit1996}中给出的压力系数(蓝)对比}\label{f:CpCompare}
\end{figure}
\begin{figure}[htb]
\centering
\subcaptionbox{不同算例中边界层外缘流向速度对比\label{fig:CompOutFlow:a}}[0.48\linewidth] %% label for first subfigure
    {\includegraphics[width=0.48\linewidth]{ch4/compare-outflow3}}
%\hspace{0.0in}
\subcaptionbox{不同算例中边界层位移厚度对比\label{fig:CompOutFlow:b} }[0.48\linewidth]%% label for second subfigure
    {\includegraphics[width=0.48\linewidth]{ch4/DisplacementThickness-4(2)}}
\caption{不同算例边界层对比}
\label{fig:CompOutFlow} %% label for entire figure
\end{figure}
\begin{figure}[htb]
\centering
\subcaptionbox{$U_\infty=22.3m/s$\label{fig:CompCrossProfiles:a} }[0.48\linewidth]%% label for first subfigure
{\includegraphics[width=0.48\linewidth]{ch4/compare223Wt(scaledUe)}}
%\hspace{0.0in}
\subcaptionbox{$U_\infty=44.5m/s$\label{fig:CompCrossProfiles:b}} [0.48\linewidth]%% label for second subfigure
{\includegraphics[width=0.48\linewidth]{ch4/compare445Wt(scaledUe)}}
\caption{不同算例不同位置横流速度剖面对比}
\label{fig:CompCrossProfiles} %% label for entire figure
\end{figure}

图\ref{fig:CompCrossProfiles}展示了20\%, 40\%, 和 60\%处的横流速度剖面。如之前提到的,横流速度必须在壁面和边界层外消失。这里,随着边界层的厚度从20\%到60\%逐渐增加,横流速度消失位置的高度也在逐渐增加。可以看到,所有采用THU压力分布的算例中的横流速度的最大值都要大于采用Reibert压力分布的对应算例对应位置的横流速度最大值。从图\ref{f:CpCompare}中可以看到在这三个位置THU实验中的压力梯度均大于Reibert实验中的压力梯度,这也正是横流速度峰值不同的主要原因。众所周知,横流的产生正是因为边界层内由主流速度转向产生的离心力不足以平衡压力梯度产生的压差力。所以在压力梯度越强诱导出来的横流就越强。

本文先用传统的$e^N$方法来研究不同算例中的流动稳定性。这里$N$的定义为:
\begin{equation}\label{e:eNdef}
  N=\int_{x_0}^x( -\alpha_i)\,dx,
\end{equation}
这里,$\alpha_i$是采用局部线性稳定性理论(LST)计算得到的流向复波数的虚部。其相反数$-\alpha_i$也就是LST预测出来的模态空间增长率。其物理意义就是下游单位长度距离处失稳模态幅值与当地失稳模态幅值的比值。$x_0$是模态首次失稳的流向位置。所以,这里$N$的物理意义就是,当地扰动模态的幅值相对于模态首次失稳位置的幅值的比的对数。采用这一定义的一个隐藏假设是所有模态在中性点位置通过感受性过程所获得的初始模态幅值是相同的。$e^N$虽然不是十分精确,但是可以定性反应失稳过程。图\ref{fig:CompN:a}给出了不同算例$N$值的包线,也就是不同位置$N$的最大值。其中蓝线和红线分别代表来流速度为44.5与22.5m/s的算例,方块和圆点分别代表采用THU压力分布和采用Reibert压力分布的算例。从结果中可以看到,所有44.5m/s自由来流速度的算例中的$N$值均比对应的22.5m/s自由来流的算例高。这说明了来流速度雷诺数越高,失稳模态增长越快,流动越不稳定。另外,还可以看到,在自由来流速度一样的前提下,采用THU压力分布的算例中的边界层要比采用Reibert压力梯度的更不稳定。可见,横流速度的微弱变化(见图\ref{fig:CompCrossProfiles})都会导致$N$值的剧烈变化。另外,必须要提及的是,如之前所说的,$N$是扰动幅值的对数,用数学式子表达也就是$A=A_0e^{N}$($A_0$是中性点的模态初始值)。所以这里$N$的变化转换到实际的模态幅值上则会是更强烈的变化。图\ref{fig:CompN:b}对比了不同算例中不同位置$N$值最大的定常横流失稳模态的展向波长。这也就是LST所预测的最不稳定模态的展向波长。可以看到,压力梯度的变化并没有对最不稳定模态的展向波长产生较大影响。基于以上分析,这里可以做一个简单的总结,更大的压力梯度会导致更强的横流速度,并进而导致更强的横流失稳模态的增长率。也就是压力梯度越大有横流的边界层越容易失稳。然而,这一变化对最不稳定横流模态的波长并没有显著影响,也就是说主导转捩的横流涡的展向并不会有太大变化。在THU实验中观测到的失稳波长与Reibert文献中的相同,也印证了这里的分析。
\begin{figure}
\centering
\subcaptionbox{$N$值包络线           %
\label{fig:CompN:a}} [0.48\linewidth]%% label for first subfigure
{\includegraphics[width=0.48\linewidth]{ch4/compare-Nmax3}}
%\hspace{0.0in}
\subcaptionbox{不同流向位置$N$值最大模态展向波长        %
\label{fig:CompN:b}} [0.48\linewidth] %% label for second subfigure
{\includegraphics[width=0.48\linewidth]{ch4/compare-lamda}}
\caption{不同算例$e^{N}$方法结果对比}
\label{fig:CompN} %% label for entire figure
\end{figure}

本文选取了上面对比的四个算例中最不稳定的一个进行控制,即自由来流44.5\,m/s同时采用THU压力分布的算例。这样能够更明显的体现出控制的效果。在之后的计算中,自由来流速度44.5\,m/s被选作参考速度。图\ref{fig:Nfactor445}给出了$N$这个基准算例中不同展向波长的横流驻涡模态的$N$值延流向变化的情况。通常,横流模态的$N=6$的时候会触发专列。如果采用这一标准,4mm展向波长的模态会在24\%弦长位置触发转捩。然而,用$N$判断转捩的方法并不十分稳定可靠。有文献指出\cite{saric2011},在前缘抛光,来流湍流度非常低的条件下,临界点$N$值甚至可以高于14。在这个基准算例中,在70\%弦长之前(压力最低点之前)没有任何模态的$N$达到14。在翼型前半段,展向波长为4、5、6mm的模态的$N$值依次领跑。这意味着这几个展向波长的模态都有可能主导转捩。这里需要重点指出的是,3mm展向波长的模态在靠近前缘的位置增长飞快,但是在20\%弦长处达到了峰值之后便开始减弱。在50\%弦长处其幅值甚至小于其初始幅值。波长更小的模态,如2mm展向波长的模态在计算域内几乎不增长。这一短波长高波数模态不增长的特性与之前分析的后掠Hiemenz流动完全不同。在后掠Hiemenz流动中,短波长高波数的模态仅仅是失稳的晚一些,并不会出现完全不失稳的情况。这一特性将在之后的控制中起到非常关键的作用。

$e^{N}$方法是线性稳定性理论下的一种半经验方法,也就是说它还是要依赖小扰动可线性化假设和边界层增长近似为零的平行流假设。但实际上,当扰动的幅值增长到一定程度也就是大约扰动速度达到十分之一自由来流速度的时候,非线性会起作用,失稳进入非线性阶段。本文采用非线性抛物华扰动方程(NPSE)求解非线性阶段的扰动发展变化。在线性稳定性分析得到的几个最有可能主导转捩的模态被用来初始化NPSE计算。这几个模态被放置在计算入口处,并且幅值都设置为$5\times10^{-5}$。这里NPSE中失稳模态的幅值定义为:
\begin{equation}
\mathrm{Amp}=\exp\!\left(\int_{x_0}^x -\alpha _i\,d\xi\right)\max\!\left(\sqrt{\left| \hat{u} \right|^2+\left| \hat{v} \right|^2+\left| \hat{w} \right|^2}\right)_y.
\end{equation}
NPSE计算得到的几个算例中的主模态幅值沿着流向演化的曲线展示在图\ref{f:NPSE}中。这里的做法和后掠Hiemenz流动中选取控制目标模态的方法相同。图\ref{f:NPSE}中的曲线并不是一个算例中不同模态的演化曲线,而是4个主模态不同的算例的结果。在这几个算例中,除了主模态是计算入口就给入的,其他模态都是通过非线性依靠主模态激发出来的高阶谐波模态。从图中可以看到,3mm展向波长的模态的幅值基本上比其他模态的幅值低了一个量级。这和线性稳定性理论预测出来的相似,可见这一模态就并没有能够成功发展进入非线性阶段。5mm展向波长的模态是最先进入饱和的。因此在这一章之后的研究中,均以这一模态作为目标模态。
\begin{figure}
\centering
  % Requires \usepackage{graphicx}
  \includegraphics[width=\textwidth]{ch4/Nvalue(1)}
  \caption{自由来流$U_\infty= 44.5$\,m/s算例中不同展向波长的横流驻涡模态的$N$值延流向变化}%
  \label{fig:Nfactor445}
\end{figure}
\begin{figure}
\centering
  % Requires \usepackage{graphicx}
  \includegraphics[width=0.48\textwidth]{ch4/CompareCsesVmax_Amp0=1e-4(ScaledWithLocalUout0_5)(1)} \includegraphics[width=0.48\textwidth]{ch4/CompareCsesVmax_Amp0=1e-4(ScaledWithLocalUout0_5)}
  \caption{NPSE计算得到的不同展向波长模态幅值延流向变化}\label{f:NPSE}%
\end{figure}


\section{采用等离子体激发器推迟后掠翼上流动转捩}

\subsection{Harmonic control: one actuator per wavelength}\label{subs:control1}
As mentioned before, there are seven constants ($a_0$, $a_1$, $a_2$, $b_0$, $b_1$, $b_2$, and $c_{\rm force}$) that need to be determined in the plasma model. D\"orr and Kloker pointed out that the body force should spread over the boundary layer but  not extend beyond its edge \cite{dorr2015stabilisation}. Figure \ref{f:BLvelocityprofile} shows the primary  and  crossflow velocity profiles at streamwise locations $X/C=0.15$, 0.2, and 0.25. It can be seen that the boundary-layer edge is at roughly  1.2\,mm height. Beyond this height, the primary velocities became constant and the crossflow velocities vanish rapidly. Based on these baseflow velocity profiles, the designed body force distribution is shown in figure~\ref{f:forceshape} and the corresponding constants are listed in Table~\ref{t:constantsPmodel}. The coefficient $c_{\rm force}$ controls the force strength. The force strength effect is thus studied by choosing 3 values of $c_{\rm force}$. Figure \ref{f:forceshape}  shows the case with $c_{\rm force}=30$. For the other two cases, the distributions have the same shape, but with the values  all increased proportionally. The body force is distributed under 1.2\,mm and the spread length in the $z$ direction is less than 2.5\,mm, nearly  half of the target mode wavelength. The maximum force densities are 2986, 4976, and 6967\,N/m$^3$ corresponding to $c_{\rm force}$ of 30, 50, and 70, respectively. The total forces, obtained by integrating over the whole y-z cross-section containing a DBD, are 1.467$\times 10^{-3}$, 2.446$\times 10^{-3}$, and 3.424$\times 10^{-3}$N/m, respectively. It is worth mentioning that the maximum force density was as high as 7000\,N/m$^3$ in Kriegseis's experiment~\cite{kriegseis2013velocity}.

Actuators with the previously described body force distribution are utilized to attenuate  crossflow instability. To hinder  crossflow vortices directly, one actuator is positioned per wavelength. The distribution of the dimensionless force $f$ in the $X$--$Z$ plane at $y=0.1$\,mm is shown in Fig.~\ref{f:force_XZ_1perwavelength}. All the electrodes are parallel to the isophasal lines of the primary crossflow instability mode. The spanwise distance of each pair of neighboring actuators is just the wavelength of the primary mode. The control region starts at 23.7\% chord length and ends at 26.2\% chord length. In this case,  $c_{\rm force}= 30$. Ten different spanwise locations of the actuators are examined to find the optimal one. The flow does not always become stabler in all cases, with some arrangements can promote the transition. Figure \ref{f:bestworst} shows the evolution of the fundamental modes in the most stable and  unstable cases. The black curve stands for the case without control, and the green and red curves for the most stable and  unstable cases, respectively. Here, $T_z$ is the fundamental spanwise wavelength and $z_0$ is the spanwise coordinate of the central point of the middle actuator. The control region is indicated by the two vertical blue lines. When the actuators locate at $z_0/T_z=0.4$, the primary mode is weakened in the control region and its amplitude is lower than that in the case without control downstream. However, when the actuators locate at $z_0/T_z=0.9$, just half a wavelength away from the formal most stable case, the primary instability mode is promoted. The amplitudes increase in the control region and become much higher than that in the no-control case.

Figures \ref{f:pla_postion} depict the actuator locations relative to the instability disturbance. The colors denote the body force, and the iso-lines show the  disturbance velocities without control. All the forces and  velocities are projected onto the direction perpendicular to the crossflow vortex. It can be seen that when the force and the local disturbance have the same sign, both negtive shown in figure~\ref{f:theworst}, instability is promoted. In contrast, when the force and the disturbance velocities are in opposite directions (see figure~\ref{f:thebest}), the disturbance will be damped and thus the instability will be attenuated. This result implies that the spanwise position is critical, and an unfavorable position can even lead to a stronger disturbance that may bring the transition further upstream. For application of this control method  to a real aircraft, it would be essential to locate the positions of all the crossflow vortices,  which would be a tremendous challenge. Hence, in our view, this method is not suitable for practical application.
\begin{table}
\caption{Constants in the plasma model}\label{t:constantsPmodel}
%\begin{ruledtabular}
%\begin{tabular*}{\textwidth}{@{\extracolsep{\fill}}ccccccc}
    \begin{center}
    \begin{tabular}{p{1.5cm}p{1.5cm}p{1.5cm}p{1.5cm}p{1.5cm}p{1.5cm}p{2cm}}
    %  \hline
      % after \\: \hline or \cline{col1-col2} \cline{col3-col4} ...
      %\br
      \toprule[1.5pt]
      $a_0$ & $a_1$ & $a_2$ & $b_0$ & $b_1$ & $b_2$ & $c_{\rm force}$ \\\midrule[1pt]
      %\mr
      2.0 & 0.08 & 0.001 & 7.76 & 2.1 & 1.8 & 30,50,70 \\
      %\br
      \bottomrule[1.5pt]
    %  \hline
    \end{tabular}
    \end{center}
%\end{ruledtabular}
\end{table}
%\begin{table}
%\caption{\label{tab:table1} Transitions selected for thermometry}
%\begin{ruledtabular}
%\begin{tabular}{lcccccc}
%& Transition& & \multicolumn{2}{c}{}\\\cline{2-2}
%Line& $\nu \prime\prime $& & \textit{J}$\prime\prime $& Frequency, cm$^{-1}$& \textit{FJ}, cm$^{-1}$& \textit{G}$\nu $, cm$^{-1}$\\\hline
%a& 0& P$_{12}$& 2.5& 44069.416& 73.58& 948.66\\
%b& 1& R$_{2}$& 2.5& 42229.348& 73.41& 2824.76\\
%c& 2& R$_{21}$& 805& 40562.179& 71.37& 4672.68\\
%d& 0& R$_{2}$& 23.5& 42516.527& 1045.85& 948.76\\
%\end{tabular}
%\end{ruledtabular}
%\end{table}
\begin{figure}
\centering
  % Requires \usepackage{graphicx}
  \includegraphics[width=0.48\textwidth]{ch4/Ut(ScaledWithUinf)} \includegraphics[width=0.48\textwidth]{ch4/Wt(ScaledWithUinf)}
  \caption{Primary velocity profiles (left) and  crossflow (secondary) velocity profiles (right).}%
  \label{f:BLvelocityprofile}
\end{figure}

\begin{figure}
\centering
  % Requires \usepackage{graphicx}
  \includegraphics[width=0.8\textwidth]{ch4/abs(bodyforce)}
  \caption{Distribution of  body force induced by a single plasma actuator}%
  \label{f:forceshape}
\end{figure}

\begin{figure}
\centering
  % Requires \usepackage{graphicx}
  \includegraphics[width=0.6\textwidth]{ch4/bodyforceXZ(y=0_1mm)}
  \caption{Distribution of  body force in the $X$--$Z$ plane ($y= 0.1$\,mm) with one actuator per wavelength.}%
  \label{f:force_XZ_1perwavelength}
\end{figure}

\begin{figure}
\centering
  % Requires \usepackage{graphicx}
  \includegraphics[width=0.6\textwidth]{ch4/Vmax_compare(scaledUe)-improved}
  \caption{Amplitudes of  fundamental modes with actuators placed at different spanwise locations.}%
  \label{f:bestworst}
\end{figure}

\begin{figure}
\centering
  % Requires \usepackage{graphicx}
  \subcaptionbox{$z_0/Tz=0.4$\label{f:thebest}}[\textwidth]
  {\includegraphics[width=\textwidth]{ch4/force-position-wt(scaledUinf)_z0=04}
}
  %\caption{Relative positions of  body force and  crossflow-wise disturbance velocity in the case .}%

%\end{figure}

%\begin{figure}
%\centering
  % Requires \usepackage{graphicx}
  \subcaptionbox{$z_0/Tz=0.9$\label{f:theworst}}[\textwidth]
  {
  \includegraphics[width=\textwidth]{ch4/force_position_wt(scaledUinf)_z0=09}
  }
  \caption{Relative positions of body force and crossflow-wise disturbance velocity. The colors denote the body force, and the iso-lines show the disturbance velocities without control.}
  \label{f:pla_postion}

\end{figure}


\subsection{Subharmonic control: two actuators per wavelength}\label{subs:control2}
Since it is known that the magnitude of the crossflow velocity greatly influences  crossflow instability, another idea is to use plasma actuators to attenuate the crossflow velocity. To avoid exciting the primary mode, two actuators are positioned per wavelength. The body force distribution in the $X$---$Z$ plane is shown in figure~\ref{f:force2perwavelength}. The number of plasma actuators is doubled compared with the previous scheme. The control region starts at 18.7\% chord length and ends at 21.2\% chord length. The electrodes are still parallel to the isophasal curves of the primary instability mode. $c_\mathrm{force}=50$ in the first case; the cases with  $c_\mathrm{force}=30$ and 70 will be given afterwards.

Figure \ref{f:basecase} shows the evolution of the disturbance energy. The red curves stand for the controlled case and the black curves for the uncontrolled case. The right figure uses a normal coordinate, whereas the left uses a logarithmic coordinate to show the harmonics more clearly. Again, the control region is denoted by two blue vertical lines in both figures, and the center of the region is at 20\% chord length. Since the distance between two neighboring actuators is half the fundamental wavelength, the harmonic mode $(0,2)$, whose wavelength is also half  the fundamental wavelength, is excited directly. It can be seen that there exists a small peak right at the end of the control region. When the mode leaves  the control region, its energy decreases rapidly, and at 30\% chord length its energy becomes two orders of magnitude lower than the peak value. The reason for this energy decline is because this mode with 2.5\,mm wavelength is predicted to be stable by the $e^{N}$ method, meaning that it will die out soon without plasma stimulation. The behaviors of other harmonics,  from $(0,3)$ to $(0,5)$ modes, and all the higher-order harmonics (not shown) are all similar to that of the  $(0,2)$ mode. However, in the middle section of the wing, from 30\% to 40\% chord length, all the modes are weaker than their counterparts from the case without control.
\begin{figure}
\centering
  % Requires \usepackage{graphicx}
  \includegraphics[width=0.6\textwidth]{ch4/bodyforce_Forshowy=0_1mm(twoactuators)}
  \caption{Distribution of  body force in the $X$--$Z$ plane (two actuators per wavelength).}%
  \label{f:force2perwavelength}
\end{figure}
\begin{figure}
\centering
  % Requires \usepackage{graphicx}
  \subcaptionbox{\label{f:basecase_a}}[0.48\textwidth]{

    \includegraphics[width=0.48\textwidth]{ch4/compare_modes_energy(scaledUinf)1-improved}}
  \subcaptionbox{\label{f:basecase_b}}[0.48\textwidth]{

    \includegraphics[width=0.48\textwidth]{ch4/compare_modes_energy(scaledUinf)2-improved}}
  \caption{Evolution of mode energy with and without control.}%
  \label{f:basecase}
\end{figure}

\begin{figure}
\centering
\subcaptionbox{\label{fig:ContU0216WOC}}[0.48\linewidth]{           %
%% label for first subfigure
\includegraphics[width=0.48\linewidth]{ch4/XC=0216(scaledUinf)WOC}}
%\hspace{0.0in}
\subcaptionbox{\label{fig:ContU0216WC}}[0.48\linewidth]{
%% label for second subfigure
\includegraphics[width=0.48\linewidth]{ch4/XC=0216(scaledUinf)WC}}
\caption{Contours of streamwise velocity at $X/C=0.216$ (a) without  and (b) with control.}
\label{fig:ContU0216} %% label for entire figure
\end{figure}

\begin{figure}
\centering
\subcaptionbox{\label{fig:ContU0350WOC}}[0.48\linewidth]{           %
%% label for first subfigure
\includegraphics[width=0.48\linewidth]{ch4/XC=035(scaledUinf)WOC}}
%\hspace{0.0in}
\subcaptionbox{\label{fig:ContU0350WC}}[0.48\linewidth]{
%% label for second subfigure
\includegraphics[width=0.48\linewidth]{ch4/XC=035(scaledUinf)WC}}
\caption{Contours of streamwise velocity at $X/C=0.35$ (a) without  and (b) with control.}
\label{fig:ContU0350} %% label for entire figure
\end{figure}
%\clearpage %REMEMBER TO DELATE THIS AFTER YOU ADD ALL WORDS IN THIS PAPER!!!!!!!!!!!!!!!!!!!!!!!!!!!!!!!!!!!
Figures \ref{fig:ContU0216} and \ref{fig:ContU0350} show  contours of the streamwise velocity at $X/C=0.216$ and 0.35, respectively. At $X/C=0.216$, close to the end of the control region $X/C=0.212$, the boundary layer looks quiet and clean without control. The instability modes are very weak there. When the plasma is induced, small waves are generated, seen in figure~\ref{fig:ContU0216WC}. These small waves are mainly caused by the  (0,2) mode and have  wavelength  2.5\,mm. At 35\% chord length, in the case without control, a strong crossflow vortex appears and  convects low-momentum fluid away from the wall. A rollover structure that indicates the beginning of the saturation stage also appears. However, for the controlled case, there appear only small ripples, and no strong convection emerges. From these figures, it can be concluded that even though the plasma actuators do not affect the primary mode directly, their effects do ultimately hinder the evolution of crossflow vortices.
\begin{figure}
\centering
  % Requires \usepackage{graphicx}
\includegraphics[width=0.24\textwidth]{ch4/compare_Wt_XC=025(scaledUinf)2}
\includegraphics[width=0.24\textwidth]{ch4/compare_Wt_XC=030(scaledUinf)2}
\includegraphics[width=0.24\textwidth]{ch4/compare_Wt_XC=035(scaledUinf)2}
\includegraphics[width=0.24\textwidth]{ch4/compare_Wt_XC=040(scaledUinf)2}
\caption{Crossflow velocity profiles.}%
\label{f:CFprofiles}
\end{figure}

Figure \ref{f:CFprofiles} shows the crossflow velocity profiles at different streamwise locations. The blue curves stand for the crossflow velocity profiles of the baseflow. The black curves represent the uncontrolled case, and they deviate from the baseflow profile owing to the mean flow distortion mode, namely, the $(0,0)$ mode. The red curves stand for the controlled case. It can be seen that at $X/C=0.25$, the black curve coincides with the blue one, because all the disturbance modes, including the mean flow distortion mode $(0,0)$, are weak there. Meanwhile, since the direction of the plasma-induced body force is  opposite to that of the crossflow velocity, the profile in the controlled case is lower than in the other two cases. The situation is similar at $X/C=0.3$. The controlled crossflow profile grows marginally, but it is still lower than that in the other two cases. From 25\% to 30\% chord length, all the instability modes in the controlled case grow slower than their counterparts in the uncontrolled case, and some of them even shrink [see figure~\ref{f:basecase_b}]. This is mainly attributed to the decrease in the crossflow velocity profile. At 35\% and 40\% chord length,  the effect of nonlinearity  promotes  distortion of  the mean flow and a decrease in the crossflow for the uncontrolled case. However, since the development of these instability modes is hindered in the controlled case, the nonlinearity is not significant and thus the distortion of the controlled baseflow is not as intense as that in the case without control. Then, the controlled crossflow become stronger than in the case without control, as can be seen in the last picture in figure~\ref{f:CFprofiles}.
\begin{figure}
\centering
  % Requires \usepackage{graphicx}
\includegraphics[width=0.48\textwidth]{ch4/(scaledUinf)Amp0=1e-4_c_force=VARYING_X0C=020_dzdTz=00_Nplasma=2-improved}
\includegraphics[width=0.48\textwidth]{ch4/(scaledUinf)Amp0=1e-4_c_force=VARYING_X0C=020_dzdTz=00_Nplasma=2___zoomin-improved}
\caption{Evolution of mode energy for different body force strengths.}%
\label{f:forcestrength}
\end{figure}

The effect of force strength  is studied by varying the coefficient $c_{\rm force}$, and the results are shown in figure~\ref{f:forcestrength}. The right figure zooms in the vicinity of the control region in the left figure. The orange, red, and green curves stand for the cases with $c_{\rm force}=30$, 50, and 70, respectively. In all the controlled cases, the energies of the primary modes and the $(0,2)$ modes are all lower than those in the case without control, and a stronger body force results in weaker instability. A stronger body force also leads to higher peak value of the energy of the harmonic $(0,2)$ mode near the end of the control region. It can be seen that the peak values are $2 \times 10^{-4}$, $8 \times 10^{-4}$, and  $2 \times 10^{-3}$ for the cases with $c_{\rm force}=30$, 50, and 70, respectively.

Figure \ref{f:streamwiselocationeffect} compares the results for cases with actuators positioned at different streamwise locations. The green, red, and orange curves stand for the cases with DBD centers located at 15\%, 20\%, and 25\% chord length. The control regions are not plotted on the figure, but they can still  be recognized by the small peaks on the dashed curves denoting the $(0,2)$ instability modes, because, like all the other cases shown previously, in the vicinity of the control region, the $(0,2)$ modes are all excited. The peak of the $(0,2)$ mode in the case with excitation at 15\% chord length is lower than that at 20\% chord length, which in turn is lower than that at 25\% chord length. The reason is that the amplitude of the uncontrolled $(0,2)$ mode is larger  downstream, and thus, when it is excited by the same force, the originally strong mode reaches an even higher level. From 30\% to 40\% chord length, all the mode energies for the case controlled at 15\% chord length are lower than their counterparts in the other two cases. The primary mode controlled by the DBD actuators at 25\% chord length does not deviate from the primary mode without control until 33\% chord length. Fortunately, its energy decreases after that. No matter where the actuators are placed, all the mode energies are lower than in the original case without control.
\begin{figure}
\centering
  % Requires \usepackage{graphicx}
\includegraphics[width=0.48\textwidth]{ch4/(scaledUinf)Amp0=1e-4_c_force=50_X0C=VARYING_dzdTz=00_Nplasma=2-improved}
\includegraphics[width=0.48\textwidth]{ch4/(scaledUinf)Amp0=1e-4_c_force=50_X0C=VARYING_dzdTz=00_Nplasma=2____zoomin-improved}
\caption{Evolution of mode energies with plasma actuators at different streamwise locations.}%
\label{f:streamwiselocationeffect}
\end{figure}
\begin{figure}
\centering
  % Requires \usepackage{graphicx}
\includegraphics[width=0.48\textwidth]{ch4/(scaledUinf)Amp0=1e-4_c_force=50_X0C=020_dzdTz=VARYING_Nplasma=2-improved}
\includegraphics[width=0.48\textwidth]{ch4/(scaledUinf)Amp0=1e-4_c_force=50_X0C=020_dzdTz=VARYING_Nplasma=2___zoomin-improved}
\caption{Evolution of mode energies with plasma actuators at different spanwise locations.}%
\label{f:spanwiselocationeffect}
\end{figure}
\begin{figure}
\centering
  % Requires \usepackage{graphicx}
\includegraphics[width=0.24\textwidth]{ch4/XC=025_Modified_baseflow(scaledUinf)-z0Tz=025}
\includegraphics[width=0.24\textwidth]{ch4/XC=030_Modified_baseflow(scaledUinf)-z0Tz=025}
\includegraphics[width=0.24\textwidth]{ch4/XC=035_Modified_baseflow(scaledUinf)-z0Tz=025}
\includegraphics[width=0.24\textwidth]{ch4/XC=040_Modified_baseflow(scaledUinf)-z0Tz=025}
\caption{Comparison of modified mean flow profiles with plasma actuators  at different spanwise locations.}%
\label{f:Wt_SpVar}
\end{figure}

It has already been mentioned in Section~\ref{subs:control1} that the controlled results with one plasma actuator per wavelength are remarkably sensitive to the spanwise location of the actuators. The sensitivity of the control method with two actuators per wavelength to  spanwise location is examined here, as shown in figure~\ref{f:spanwiselocationeffect}. Since the wavelength of the array of plasma actuators is half  the fundamental wavelength $T_z$, the actuators are moved one-quarter of the fundamental wavelength reversed for phase in the spanwise direction. The red curves stand for the original case, and the green curves represent the new case with the  spanwise location of the actuators shifted. Again, the energy of the $(0,2)$ mode increases in the control region and decreases elsewhere. The energies of  the $(0,1)$ modes in both cases remain the same in the control region. However, they begin to deviate from each other just slightly downstream of the control region. From 25\% to 45\% chord length, there is only a small difference between these two curves. In addition, both of them are below the black curve, the one without control, indicating that the spanwise locations of the actuators are not crucial.

To explain the slight difference between the primary modes from the cases with and without a shift in the spanwise location of the actuators, the mean flow profiles in the crossflow direction are compared in figure~\ref{f:Wt_SpVar}. The blue  and  red curves denote the mean crossflow velocity profiles in the cases with and without a spanwise shift of the  actuators, respectively. At 25\% and 30\% chord length, the red curves are perfectly superposed on the blue curves. It should be recalled that at these two streamwise locations, the energies of the primary modes have already deviated from each other. Hence, it can be concluded that the difference between the $(0,1)$ modes in the two cases is not caused by the mean crossflow velocity profile.
\begin{figure}
\centering
  % Requires \usepackage{graphicx}
\includegraphics[width=0.6\textwidth]{ch4/force_XZ(scaledUinf)-reversed}
\caption{Body force distribution of inverse plasma actuators in the $X$--$Z$ plane.}%
\label{f:force_reversed}
\end{figure}

To date, it is clear that  manipulation of the $(0,0)$ mode \cite{Saric1998} or the $(0,2)$ mode can result in a decrease in the energy of the primary mode. In the DBD plasma actuators control scheme presented here, the $(0,0)$ and $(0,2)$ modes are both altered directly and the primary mode is  affected only downstream of the control region. It is not clear which mode, $(0,0)$ or $(0,2)$, contributes more to the decline in the energy of the primary mode. To answer this question, a reversed control case is examined. All the DBD actuators are turned 180$^\circ$, with  the body force in the opposite direction. In the computation, the force appears as a source term. When analyzing each mode, the force term is decomposed into Fourier series with respect to the spanwise coordinate. These Fourier components affect the corresponding instability modes. For instance, the zeroth-order Fourier component affects the $(0,0)$ mode directly and the second affects the $(0,2)$ mode. For this reverse control, the sign of the force term and that of its Fourier component are switched. Owing to the properties of trigonometric functions, the sign switch of the second Fourier component is equivalent to a phase shift. This phase shift effect has been investigated above by comparing  results with actuators at different spanwise locations, and has been shown to be trivial. Thus, the biggest difference in this reverse control is that the sign of the force term corresponding to the $(0,0)$ mode is switched. If this reverse control still works and reduces the energy of the primary mode, then the $(0,2)$ mode rather than the $(0,0)$ mode will play a more important role in our control scheme. Otherwise, the conclusion will be that the $(0,0)$ mode is more important.

The body force distribution in the $X$--$Z$ plane is shown in figure~\ref{f:force_reversed}. The evolution of the mode energies in reverse control cases is shown in figure~\ref{f:model_energy_revers}. As mentioned above, reversal of the force direction  will also lead to a phase shift of the Fourier component corresponding to the $(0,2)$ mode. This phase shift can be achieved by moving the actuators in the spanwise direction. To eliminate this small ambiguity, actuators located at $Z_0/T_z=0.0$ and $Z_0/T_z=0.25$ are both simulated. It can be seen that in both cases the energies of the primary modes are higher than that in the controlled case. Also, the effect of actuator spanwise location  is not significant, and this agrees well with the conclusion reached above (see Fig.~ \ref{f:Wt_SpVar}). This result indicates that the $(0,0)$ mode is more important than the $(0,2)$ mode and  is the main cause of the decline in the energy of the primary mode.
\begin{figure}
\centering
  % Requires \usepackage{graphicx}
\includegraphics[width=0.48\textwidth]{ch4/(scaledUinf)Amp0=1e-4_c_force=50_X0C=020_dzdTz=VARYING_Nplasma=2_revers-improved}
\includegraphics[width=0.48\textwidth]{ch4/(scaledUinf)Amp0=1e-4_c_force=50_X0C=020_dzdTz=VARYING_Nplasma=2_revers___zoomin-improved}
\caption{Evolution of mode energy in reversed control cases (with actuators at two different locations).}%
\label{f:model_energy_revers}
\end{figure}

\begin{figure}
\centering
  % Requires \usepackage{graphicx}
\includegraphics[width=0.32\textwidth]{ch4/XC=025_Modified_baseflow_reversed(scaledUinf)}
\includegraphics[width=0.32\textwidth]{ch4/XC=030_Modified_baseflow_reversed(scaledUinf)}
\includegraphics[width=0.32\textwidth]{ch4/XC=035_Modified_baseflow_reversed(scaledUinf)}
\caption{Mean crossflow profile in the reversed control case.}%
\label{f:inverse_meanflow}
\end{figure}

Figure \ref{f:inverse_meanflow} shows the mean crossflow velocity profiles at different streamwise locations in the reverse control case. It is obvious that the crossflow is enhanced at 25\% chord length, just downstream of the control region. Thereafter, the crossflow falls back and finally returns to the same level as that in the case without control. From this result, it can be concluded that the crossflow velocity has a significant effect on  crossflow instability. When the crossflow is weakened by the actuator, the instability is attenuated. Otherwise, the instability is intensified.
\begin{figure}
\centering
  % Requires \usepackage{graphicx}
\includegraphics[width=0.6\textwidth]{ch4/(scaledUinf)Amp0=1e-4_c_force=70_X0C=VARYING_dzdTz=0000_Nplasma=3-improved}
\caption{Evolution of mode energy in cases targeted at the 7.5\,mm wavelength mode.}%
\label{f:7.5mm}
\end{figure}
\subsection{Off-designed case}
All the simulations shown above assume that the mode with 5\,mm spanwise wavelength dominates the transition, and the 2.5\,mm mode, happens to be the $(0,2)$ mode with respect to the 5\,mm fundamental wavelength. Here, another situation is considered in which the 7.5\,mm mode becomes dominant but the distance between two neighboring actuators is still 2.5\,mm. Thus, the control mode is the $(0,3)$ mode. Plasma actuators are positioned at three different streamwise locations, and the evolution of the mode energy is shown in figure~\ref{f:7.5mm}. The green, red, and orange curves stand for the cases controlled at 15\%, 20\%, and 25\% chord length, respectively. The small peak in the energy of the $(0,3)$ mode, that is, the control mode, becomes stronger and stronger when the actuators are moved downstream. Fortunately, all the modes in all the controlled cases are weaker than those in the uncontrolled case downstream of 30\% chord length. This result proves that the presented control method performs well even in an un-designed case.
