\chapter{后掠翼上流动失稳分析与控制}
\section{后掠翼上流动的稳定性分析}
数值研究了后掠Hiemenz流动转捩控制之后,本文将研究目标转移到更贴近实际的工况,研究一个真实的后掠翼流动的转捩控制。清华大学(Tsinghua University)在2018年进行了NLF-0415后掠翼上翼面的转捩实验研究工作\cite{wang2018},本文以该实验的工况作为基准,研究等离子体在实际的后掠翼流动中控制转捩的效果。这一实验在后文中将会被简记为``THU实验''或``THU experiment''。这里先简述一下实验的设备和工况。该实验是在清华大学自己的低湍流度风洞中进行的,风洞试验段大小为1.2m$\times$1.2m$\times$3m。风洞内安装NLF-0412翼型,如图\ref{f:experiment}所示。在实验中采用的直角坐标系为($X_{\rm wt},Y_{\rm wt},Z_{\rm wt}$),其$X_{\rm wt}$方向与风洞的流向平行。在计算中本文用的到直角坐标系为$x,y,z$(如图\ref{f:experiment})。计算所用坐标系的$z$方向与翼型前缘线平行,这样在采用了无限展长假设之后,这一方向的物理量就是均匀的。该风洞运行时测试段的风速可以为5.0到90.0m/s,湍流度可以低到0.05\%。此次试验采用的翼型的弦长$c$=1.2m,有45$^\circ$后掠和-4$^\circ$攻角。该翼型在上翼面有很强的顺压梯度,可以有效的抑制T-S波失稳\cite{Dagenhart1999}。翼型中间到风洞入口大约1.25m。

实验中采用了单一的热线测量了来流的情况。来流风速均匀,湍流度0.08$\sim$0.1\%。采用边界层热线(TSI Model 1261A)测量翼型$X/C$ = 0.2, 0.4, 0.6位置处的边界层速度剖面。热线被一个可以进行三维运动的机械臂驱动,该机械臂采用电脑控制的。最小的运动精度可以达到10$\mu$m。在每一个位置,沿着法向每隔0.05mm测量了沿着风洞方向的速度分量$U_{\rm wt}$。这个速度分量也就是图中所示$X_{\rm wt}$坐标的方向。沿着法向的测量一直到两倍边界层厚度的位置终止。为了防止测量热线因为接触壁面被损坏,第一个测量点选取在离壁面0.15mm处。这一距离是通过镜面法测量得到的。也就是通过测量热线和其在翼面上的镜像的距离,得到其相对于翼面的距离。

为了验证本文中基本流计算的结果,这里选取了雷诺数$Re$为$1.81\times10^6$,也就是自由来流速度$U_\infty$=22.3m/s的算例进行比较。($Re={U_\infty c}/{\nu}$ 其中$U_\infty$是自由来流速度,$c$是弦长,$\nu$是动力粘性系数。)在这个工况下,上翼面一直到70\%弦长的位置都是层流。在之后研究的等离子体控制算例中,为了加速失稳和转捩,采用了两倍雷诺数也就是两倍来流速度的算例。

\begin{figure}
  \centering
  \includegraphics[width=\textwidth]{ch4/experiment.jpg}
  \caption{风洞俯视图( `` $\bullet$ "表示热线测量位置)}\label{f:experiment}
\end{figure}

如上一章所述,高精度通量修正有限元程序Music\cite{WangZJ2009,Zhu2016,Zh2017}被用来进行无粘流求解计算。图\ref{f:ConpareInvicidV}对比了计算求解得到的壁面上的无粘流延风洞流向速度分量和实际实验中测得的边界层外的同方向的速度分量。其中黑线是计算得到的结果,三个大红点是实验测得的结果。由于实验仅测量了三个展向位置的速度,所以这里只有三个数据点。从图中可以看到,无粘流计算得到的结果与实验吻合良好。需要提及的是,实验中并没有精确测量离翼型非常远位置处的来流速度,也就是说实际上真正的自由来流是不能准确得知的。而计算中是采取的无量纲计算,所以本文以20\%弦长位置边界层外缘的流速作为参考基准点,重新对计算结果进行了有量纲化。在实验中,虽然直到70\%弦长处都是层流,但是60\%弦长处边界层已经出现了强烈的扭曲。所以这里只对比了20\%和40\%弦长处的边界层剖面,如图\ref{f:compare_profiles}。与上图类似的,线是计算结果,点是实验测得的结果。这里并没有用所有测量数据,只是每个两个点取了一个数据。对比结果说明本文计算得到的基本流还是非常可靠地。
\begin{figure}
\centering
  % Requires \usepackage{graphicx}
  \includegraphics[width=0.7\textwidth]{ch4/compare_UexpOUT_laminarcase}
\caption{边界层外缘风洞流向速度$U_{\rm{wt}}$对比}
\label{f:ConpareInvicidV}
\end{figure}

\begin{figure}
\centering
  % Requires \usepackage{graphicx}
  \includegraphics[width=0.6\textwidth]{ch4/compare_profiles}
\caption{风洞流向速度$U_{\rm{wt}}$ 在$X/C= 20\%$和$40\%$处剖面对比}
\label{f:compare_profiles}
\end{figure}
Reibert等人\cite{Reiberit1996}也做过相同翼型相同工况的实验研究,在后文中将其命名为``Reibert's experiment''或``Reibert实验''。但是他们的实验采用的风洞和模型尺寸均与本文参考的实验不一样。在Reibert实验中,模型弦长1.83m,风洞实验段尺寸1.4m$\times$1.4m$\times$5m。更重要的,他们在文献中并没有给出翼型的安装位置。这些实验设置的不同,导致了两个实验得到的上翼面压力系数的分布不太一样。图\ref{f:CpCompare}给出了THU实验和Reibert实验的上翼面压力分布。可以看到,THU实验的压力梯度要比Reibert实验的压力梯度强一些,这也说明了THU实验中流体在翼型中段的加速度更大一些。为了搞清楚压力梯度的变化对扰动的发展有什么影响,本文设置了4个算例进行对比研究。其中两个算例采用了Reibert实验得到的压力分布,他们的自由来流速度分别是22.3m/s和44.5m/s。另外两个算例采用THU实验得到的压力分布,自由来流速度也分别是22.3m/s和44.5m/s。这四个算例中边界层外缘流向速度分布与边界层位移厚度展示在图\ref{fig:CompOutFlow}中。可以看到,在采用THU算例的工况中,流体在靠近前缘位置加速比较慢,但是在翼型中段,其速度快速提升,并在30\%弦长处超过了采用Reibert实验工况中的边界层外缘流向速度。不管在哪一个雷诺数条件下,采用THU压力分布的算例中的边界层厚度都要比采用Reibert压力分布的要厚。在70\%弦长处,压力达到最低值,在之后均是逆压梯度,边界层快速增长。
\begin{figure}[htb]
\centering
  % Requires \usepackage{graphicx}
  \includegraphics[width=0.7\textwidth]{ch4/compareCp_Reibert}
  \caption{THU(清华)实验中压力系数(红)与Reibert博士论文 \cite{Reiberit1996}中给出的压力系数(蓝)对比}\label{f:CpCompare}
\end{figure}
\begin{figure}[htb]
\centering
\subcaptionbox{不同算例中边界层外缘流向速度对比\label{fig:CompOutFlow:a}}[0.48\linewidth] %% label for first subfigure
    {\includegraphics[width=0.48\linewidth]{ch4/compare-outflow3}}
%\hspace{0.0in}
\subcaptionbox{不同算例中边界层位移厚度对比\label{fig:CompOutFlow:b} }[0.48\linewidth]%% label for second subfigure
    {\includegraphics[width=0.48\linewidth]{ch4/DisplacementThickness-4(2)}}
\caption{不同算例边界层对比}
\label{fig:CompOutFlow} %% label for entire figure
\end{figure}
\begin{figure}[htb]
\centering
\subcaptionbox{$U_\infty=22.3m/s$\label{fig:CompCrossProfiles:a} }[0.48\linewidth]%% label for first subfigure
{\includegraphics[width=0.48\linewidth]{ch4/compare223Wt(scaledUe)}}
%\hspace{0.0in}
\subcaptionbox{$U_\infty=44.5m/s$\label{fig:CompCrossProfiles:b}} [0.48\linewidth]%% label for second subfigure
{\includegraphics[width=0.48\linewidth]{ch4/compare445Wt(scaledUe)}}
\caption{不同算例不同位置横流速度剖面对比}
\label{fig:CompCrossProfiles} %% label for entire figure
\end{figure}

图\ref{fig:CompCrossProfiles}展示了20\%, 40\%, 和 60\%处的横流速度剖面。如之前提到的,横流速度必须在壁面和边界层外消失。这里,随着边界层的厚度从20\%到60\%逐渐增加,横流速度消失位置的高度也在逐渐增加。可以看到,所有采用THU压力分布的算例中的横流速度的最大值都要大于采用Reibert压力分布的对应算例对应位置的横流速度最大值。从图\ref{f:CpCompare}中可以看到在这三个位置THU实验中的压力梯度均大于Reibert实验中的压力梯度,这也正是横流速度峰值不同的主要原因。众所周知,横流的产生正是因为边界层内由主流速度转向产生的离心力不足以平衡压力梯度产生的压差力。所以压力梯度越强诱导出来的横流就越强。

本文先用传统的$e^N$方法来研究不同算例中的流动稳定性。这里$N$的定义为:
\begin{equation}\label{e:eNdef}
  N=\int_{x_0}^x( -\alpha_i)\,dx,
\end{equation}
这里,$\alpha_i$是采用局部线性稳定性理论(LST)计算得到的流向复波数的虚部。其相反数$-\alpha_i$也就是LST预测出来的模态空间增长率。其物理意义就是下游单位长度距离处失稳模态幅值与当地失稳模态幅值之比值。$x_0$是模态首次失稳的流向位置。所以,这里$N$的物理意义就是,当地扰动模态的幅值相对于模态首次失稳位置的幅值的比的对数。采用这一定义的一个隐藏假设是所有模态在中性点位置通过感受性过程所获得的初始模态幅值是相同的。$e^N$虽然不是十分精确,但是可以定性反映失稳过程。图\ref{fig:CompN:a}给出了不同算例$N$值的包线,也就是不同位置$N$的最大值。其中蓝线和红线分别代表来流速度为44.5与22.5m/s的算例,方块和圆点分别代表采用THU压力分布和采用Reibert压力分布的算例。从结果中可以看到,所有44.5m/s自由来流速度的算例中的$N$值均比对应的22.5m/s自由来流的算例高。这说明了来流速度雷诺数越高,失稳模态增长越快,流动越不稳定。另外,还可以看到,在自由来流速度一样的前提下,采用THU压力分布的算例中的边界层要比采用Reibert压力梯度的更不稳定。可见,横流速度的微弱变化(见图\ref{fig:CompCrossProfiles})都会导致$N$值的剧烈变化。另外,必须要提及的是,如之前所说的,$N$是扰动幅值的对数,用数学式子表达也就是$A=A_0e^{N}$($A_0$是中性点的模态初始值)。所以模态幅值的变化更强烈。图\ref{fig:CompN:b}对比了不同算例中不同位置$N$值最大的定常横流失稳模态的展向波长。这也就是LST所预测的最不稳定模态的展向波长。可以看到,压力梯度的变化并没有对最不稳定模态的展向波长产生较大影响。基于以上分析,这里可以做一个简单的总结,更大的压力梯度会产生更强的横流速度,并进而导致更强的横流失稳模态增长率。也就是说压力梯度越大三维边界层越容易失稳。然而,这一变化对最不稳定横流模态的波长并没有显著影响,也就是说主导转捩的横流涡的展向波长并不会有太大变化。在THU实验中观测到的失稳波长与Reibert文献中的相同,也印证了这里的分析。
\begin{figure}
\centering
\subcaptionbox{$N$值包络线           %
\label{fig:CompN:a}} [0.48\linewidth]%% label for first subfigure
{\includegraphics[width=0.48\linewidth]{ch4/compare-Nmax3}}
%\hspace{0.0in}
\subcaptionbox{不同流向位置$N$值最大模态展向波长        %
\label{fig:CompN:b}} [0.48\linewidth] %% label for second subfigure
{\includegraphics[width=0.48\linewidth]{ch4/compare-lamda}}
\caption{不同算例$e^{N}$方法结果对比}
\label{fig:CompN} %% label for entire figure
\end{figure}

本文选取了上面对比的四个算例中最不稳定的一个进行控制,即自由来流为44.5\,m/s,同时采用THU压力分布的算例。这样能够更明显的体现出控制的效果。在之后的计算中,自由来流速度44.5\,m/s被选作参考速度。图\ref{fig:Nfactor445}给出了这个基准算例中不同展向波长的横流驻涡模态的$N$值延流向变化的情况。通常,横流模态的$N=6$的时候会触发转捩。如果采用这一标准,4mm展向波长的模态会在24\%弦长位置触发转捩。然而,用$N$判断转捩的方法并不十分稳定可靠。有文献指出\cite{saric2011},在前缘抛光,来流湍流度非常低的条件下,临界点$N$值甚至可以高于14。在这个基准算例中,在70\%弦长之前(压力最低点之前)没有任何模态的$N$达到14。在翼型前半段,展向波长为4、5、6mm的模态的$N$值依次领跑。这意味着这几个展向波长的模态都有可能主导转捩。这里需要重点指出的是,3mm展向波长的模态在靠近前缘的位置增长飞快,但是在20\%弦长处达到了峰值之后便开始减弱。在50\%弦长处其幅值甚至小于其初始幅值。波长更小的模态,如2mm展向波长的模态在计算域内几乎不增长。这一短波长高波数模态不增长的特性与之前分析的后掠Hiemenz流动完全不同。在后掠Hiemenz流动中,短波长高波数的模态仅仅是失稳的晚一些,并不会出现完全不失稳的情况。这一特性将在之后的控制中起到非常关键的作用。

$e^{N}$方法是线性稳定性理论下的一种半经验方法,也就是说它还是要依赖小扰动可线性化假设和边界层增长近似为零的平行流假设。但实际上,当扰动的幅值增长到一定程度也就是大约扰动速度达到十分之一自由来流速度的时候,非线性会起作用,失稳进入非线性阶段。之后采用NPSE求解非线性阶段的扰动发展变化。在线性稳定性分析得到的几个最有可能主导转捩的模态被用来初始化NPSE计算。这几个模态被放置在计算入口处,并且幅值都设置为$5\times10^{-5}$。这里NPSE中失稳模态的幅值定义为:
\begin{equation}
\mathrm{Amp}=\exp\!\left(\int_{x_0}^x -\alpha _i\,d\xi\right)\max\!\left(\sqrt{\left| \hat{u} \right|^2+\left| \hat{v} \right|^2+\left| \hat{w} \right|^2}\right)_y.
\end{equation}
NPSE计算得到的几个算例中的主模态幅值沿着流向演化的曲线展示在图\ref{f:NPSE}中。这里的做法和后掠Hiemenz流动中选取控制目标模态的方法相同。图\ref{f:NPSE}中的曲线并不是一个算例中不同模态的演化曲线,而是4个主模态不同的算例的结果。在这几个算例中,除了主模态是计算入口就给入的,其他模态都是通过非线性依靠主模态激发出来的高阶谐波模态。从图中可以看到,3mm展向波长的模态的幅值基本上比其他模态的幅值低了一个量级。这和线性稳定性理论预测出来的相似,可见这一模态就并没有能够成功发展进入非线性阶段。5mm展向波长的模态是最先进入饱和的。因此在这一章之后的研究中,均以这一模态作为目标模态。这一模态在翼型上产生的横流涡的形状如图\ref{f:crossflowonwing}所示。图\ref{f:crossflowonwing}中展示了$X/C=0.1,0.2,0.3,0.4$四个截面上的流向速度云图。侧端面展示的是横流速度延流向的变化。可以看到,在$X/C=0.4$位置处,失稳模态饱和,横流涡已经完全形成,并且在云图上出现了低速流体向下翻转的情况。
\begin{figure}[htb]
\centering
  % Requires \usepackage{graphicx}
  \includegraphics[width=\textwidth]{ch4/Nvalue(1)}
  \caption{自由来流$U_\infty= 44.5$\,m/s算例中不同展向波长的横流驻涡模态的$N$值延流向变化}%
  \label{fig:Nfactor445}
\end{figure}
\begin{figure}[htb]
\centering
  % Requires \usepackage{graphicx}
  \includegraphics[width=0.48\textwidth]{ch4/CompareCsesVmax_Amp0=1e-4(ScaledWithLocalUout0_5)(1)} \includegraphics[width=0.48\textwidth]{ch4/CompareCsesVmax_Amp0=1e-4(ScaledWithLocalUout0_5)}
  \caption{NPSE计算得到的不同展向波长模态幅值延流向变化}\label{f:NPSE}%
\end{figure}
\begin{figure}[htb]
  \centering
  \includegraphics[width=\textwidth]{ch4/crossflow_vortices.jpeg}
  \caption{横流涡在翼型上的发展}\label{f:crossflowonwing}
\end{figure}


\section{采用等离子体激发器推迟后掠翼上流动转捩}

\subsection{谐波激励:每个展向周期放置一个激发器}\label{subs:control1}
如之前在介绍等离子体模型中提到的,Maden模型中有7个模型参数需要确定,他们分别是$a_0$, $a_1$, $a_2$, $b_0$, $b_1$, $b_2$和$c_{\rm force}$。D\"orr和Kloker\cite{dorr2015stabilisation}指出体积力分布最好分散在边界层内,但是不要超过边界层外缘。图\ref{f:BLvelocityprofile}展示了$X/C=0.15,0.2,0.25$位置处的主流和横流速度。可以看到,这里的边界层厚度大约为1.2mm。在这个厚度以外,主流速度保持为一个常数,横流速度快速下降并回归到零。基于这一基本流的速度剖面分布,本文调整体积力模型中的7个参数,设计了如图\ref{f:forceshape}所示的体积力分布。这一分布对应的模型参数列在了表\ref{t:constantsPmodel}中。其中参数$c_{\rm force}$是用来调整体积力强度的,并不会影响体积力的实际分布。这里一共给出了三个不同的该参数取值,以便在之后的研究中分析体积力强度效应。图\ref{f:forceshape}中的体积力分布对应$c_{\rm force}=30$。可以看到,本文设计的体积力基本完全分布在1.2mm高度以下,也就是完全在边界层内。这一体积力分布的展向长度小于2.5mm,即小于是基础展向波长5mm的一半。$c_{\rm force}=30,, 50, 70$时所对应的有量纲最大体积力密度分别为2986, 4976, 6967\,N/m$^3$,在$y-z$横截面上的总得积分分别为1.467$\times 10^{-3}$, 2.446$\times 10^{-3}$, 3.424$\times 10^{-3}$N/m。需要提及的是,在Kriegseis的实验中\cite{kriegseis2013velocity},最大体积力密度为 7000\,N/m$^3$。可见这里所需的由等离子体激发器产生的体积力强度在实际中完全能够产生。
\begin{table}[htb]
\caption{DBD模型参数}\label{t:constantsPmodel}
%\begin{ruledtabular}
%\begin{tabular*}{\textwidth}{@{\extracolsep{\fill}}ccccccc}
    \begin{center}
    \begin{tabular}{p{1.5cm}p{1.5cm}p{1.5cm}p{1.5cm}p{1.5cm}p{1.5cm}p{2cm}}
    %  \hline
      % after \\: \hline or \cline{col1-col2} \cline{col3-col4} ...
      %\br
      \toprule[1.5pt]
      $a_0$ & $a_1$ & $a_2$ & $b_0$ & $b_1$ & $b_2$ & $c_{\rm force}$ \\\midrule[1pt]
      %\mr
      2.0 & 0.08 & 0.001 & 7.76 & 2.1 & 1.8 & 30,50,70 \\
      %\br
      \bottomrule[1.5pt]
    %  \hline
    \end{tabular}
    \end{center}
%\end{ruledtabular}
\end{table}
\begin{figure}[htb]
\centering
  % Requires \usepackage{graphicx}
  \includegraphics[width=0.48\textwidth]{ch4/Ut(ScaledWithUinf)} \includegraphics[width=0.48\textwidth]{ch4/Wt(ScaledWithUinf)}
  \caption{不同流向位置主流(左)和横流(右)速度剖面}%
  \label{f:BLvelocityprofile}
\end{figure}

\begin{figure}[htb]
\centering
  % Requires \usepackage{graphicx}
  \includegraphics[width=0.8\textwidth]{ch4/abs(bodyforce)}
  \caption{单一DBD激发器产生的体积力分布}%
  \label{f:forceshape}
\end{figure}

\begin{figure}[htb]
\centering
  % Requires \usepackage{graphicx}
  \includegraphics[width=0.6\textwidth]{ch4/bodyforceXZ(y=0_1mm)}
  \caption{DBD激发器阵列产生的体积力在$X$--$Z$平面上的分布($y= 0.1$\,mm) 激发器间距为一个展向波长}%
  \label{f:force_XZ_1perwavelength}
\end{figure}

\begin{figure}[htb]
\centering
  % Requires \usepackage{graphicx}
  \includegraphics[width=0.6\textwidth]{ch4/Vmax_compare(scaledUe)-improved}
  \caption{最佳控制算例与最差控制算例主模态能量延流向演化与无控制算例结果比较}%
  \label{f:bestworst}
\end{figure}
\begin{figure}[htb]
\centering
  % Requires \usepackage{graphicx}
  \subcaptionbox{$z_0/Tz=0.4$\label{f:thebest}}[\textwidth]
  {\includegraphics[width=\textwidth]{ch4/force-position-wt(scaledUinf)_z0=04}
}
  \\\bigskip
  \subcaptionbox{$z_0/Tz=0.9$\label{f:theworst}}[\textwidth]
  {
  \includegraphics[width=\textwidth]{ch4/force_position_wt(scaledUinf)_z0=09}
  }
  \caption{体积力分布(颜色云图)与扰动分布(黑色等值线)的相对位置}
  \label{f:pla_postion}
\end{figure}

在后掠翼算例中,本文首先尝试了与之前后掠Hiemenz流动中相同的谐波激励控制方法,也就是让激发器与横流涡相平行,然后每个展向波长内放置一个激发器。无量纲的体积力$f$ 在$y=0.1$\,mm高度$X$-$Z$平面内的分布如图\ref{f:force_XZ_1perwavelength}所示。这里没有再展示敏感性分析的结果,这是因为其敏感性因子的分布与之前后掠Hiemenz流动中的基本完全相同。这里控制的中心位置在25\%弦长处,控制开始于23.7\%弦长处,结束于26.2\%弦长处。这里体积力强度参数$c_{\rm force}= 30$。本文计算了激发器在十个不同展向位置的结果。其控制效果与之前后掠Hiemenz流动中的类似,这里就不再一一列出。结论也是在横流涡下方偏下扫位置处控制效果较好。这里重点对比了这十个算例中效果最好的和效果最差的算例,以得到其控制的内在机理。图\ref{f:bestworst}展示了最好和最差算例中主模态延流向的演化。其中黑线是无控制的结果,绿线和红线分别是最差算例和最好算例的结果。$T_z$是展向波长,$z_0$是激发器中点的展向位置坐标。可以看到在激发器位于$z_0/T_z=0.4$处,在控制区域横流主模态能量大幅下降,并且在控制之后的区域始终低于无控制工况中的横流主模态能量。而在另一个算例中,激发器位于$z_0/T_z=0.9$,也就是激发器刚好平移了半个波长。这时主模态在控制区域反而被促进了,这也意味着转捩将会被提前。

由于横流涡是斜的,所以横流涡的位置会随着观察的流向位置而变化,所以但看$z_0$是没有意义的,更重要的是看激发器与横流涡的相对位置。在之前后掠Hiemenz流动的控制分析中,本文给出了施加体积力和横流涡的相对位置,这里给出体积力和横流方向扰动速度分布的相对位置,以说明控制的机理。图\ref{f:pla_postion}中颜色云图表示体积力横流方向分量在横截面上的分布,曲线表示在横流方向的扰动速度分量。其中实线表示值为正,虚线表示值为负。可以看到,当把扰动和体积力都投影到横流方向的时候,如果他们的符号相反,也就是力的方向与扰动的速度方向相反(如图\ref{f:thebest}),则主模态的能量就会被降低。相反的,如果外加力与扰动速度在横流方向的投影符号相同,则主模态反而会被促进,如图\ref{f:theworst}。这一结果非常直观,也说明了这种谐波控制的控制方法实际就是通过用外加体积力去抵消因为失稳产生的扰动。当然,由于横流失稳产生的扰动在展向的分布是时正时负,所以激发器的展向位置非常关键。所以这一特性也给实际应用带来了风险。因为并不是控制总能起到效果,有时会适得其反。再加上实际中横流涡的位置并不是很容易预测,至少笔者并不知道有效的横流涡展向位置的预测方法。所以,这种谐波激励的方法在笔者看来并不实用,缺少足够的鲁棒性。还需要寻找更鲁棒的方法。

\subsection{亚谐波激励:每个展向周期放置两个激发器}\label{subs:control2}
在上一小节中,通过研究压力梯度对横流失稳的影响,发现压力梯度增大导致横流速度提高,并进一步导致失稳加剧。这不禁令人猜想,是不是通过其他方式减弱横流也可以减缓横流失稳的发展过程。在这一小节中,本文提出一种采用DBD等离子体激发器削弱横流来抑制横流失稳的方法。依旧在翼型前缘附近安装等离子体激发器阵列,令其产生的体积力与横流方向相反。这里在每个展向周期内放置两个等离子体激发器。这种布置方式会直接激发出半波长的(0,2)模态,也就是2.5mm展向波长的模态。读者可以回忆图\ref{fig:Nfactor445}中线性稳定性理论给出的结果,波长小于3mm的模态都几乎不增长。这也意味着不用担心这个被额外激发出来的模态会取代之前的主模态重新主导转捩。这里需要提一下的是,之所以没有在后掠Hiemenz流动中应用这一控制方法,是因为后掠Hiemenz流动并没有高波数短波长模态不增长这一关键特性。读者可以回忆图\ref{f:LST},在后掠Hiemenz流动中,高波数短波长的模态只是失稳的位置比较靠下游,但终究还是要失稳的。笔者也曾试图采用这种亚谐波激励的方式控制后掠Hiemenz流动,但往往激发出来的高波数短波长的模态会取代原主模态主导转捩。体积力分布的俯视图如图\ref{f:force2perwavelength}。可以看到相比之前的控制方案,激发器的布置密了一倍。控制区域开始于18.7\%弦长处,结束于21.2\%弦长处,中心位置处于20\%弦长处。激发器平行于主模态的等相位线。这里先令$c_\mathrm{force}=50$,展示被控制算例的普遍特性,之后再对比展示$c_\mathrm{force}=30$和70的结果。
\begin{figure}[htb]
\centering
  % Requires \usepackage{graphicx}
  \includegraphics[width=0.6\textwidth]{ch4/bodyforce_Forshowy=0_1mm(twoactuators)}
  \caption{体积力在$X$-$Z$平面上的分布(每个展向波长放置两个激发器)}%
  \label{f:force2perwavelength}
\end{figure}
\begin{figure}[htb]
\centering
  % Requires \usepackage{graphicx}
  \subcaptionbox{\label{f:basecase_a}}[0.48\textwidth]{

    \includegraphics[width=0.48\textwidth]{ch4/compare_modes_energy(scaledUinf)1-improved}}
  \subcaptionbox{\label{f:basecase_b}}[0.48\textwidth]{

    \includegraphics[width=0.48\textwidth]{ch4/compare_modes_energy(scaledUinf)2-improved}}
  \caption{有控制和无控制时各阶模态能量延流向演化过程}%
  \label{f:basecase}
\end{figure}

图\ref{f:basecase}给出了有控制和无控制时各阶失稳模态能量随流向的变化。其中红线代表控制算例的结果,黑线代表无控制算例的结果。图\ref{f:basecase_a}采用的是正常坐标表示,图\ref{f:basecase_b}对扰动的能量取了对数,这样可以更清楚的看到能量较小的高阶模态。控制区域用两条竖线表示出来。如之前所分析的,由于激发器的间距是基础展向波长的一半,所以半波长模态(0,2)被直接激发出来。可以看到,这一模态在控制区域一直增长,一直到控制区域结束位置,形成了一个小峰值。但是,除了控制区域之后,这一模态又迅速回落。在30\%弦长处,其能量比控制区域结束位置的能量峰值低了将近两个数量级。这主要是因为这个展向波长2.5mm的模态在这一区域本身就是衰减的。所以,即便这一模态能够从等离子体激励中获得一点能量,但这些能量也会迅速地耗散掉。除了(0,2)模态,更高阶的模态((0,3)到(0,5))也被等离子体激发了出来,但是同样的,他们也都是稳定的模态,在翼型前缘是衰减的。所以他们的演化情况与(0,2)模态类似,也是出了控制区域后一路衰减,一直到30\%弦长左右处达到最低点。之后因为主模态幅值已经近饱和,其通过非线性效应将一部分能量转移到高阶模态上。依靠这些来自主模态的能量,高阶模态在30\%弦长之后又重新开始增长。但是,直到50\%弦长处,控制算例的各阶模态的能量都低于他们在无控制算例中的对应模态。这表明等离子体激发器成功的抑制了扰动。

\begin{figure}[htb]
\centering
\subcaptionbox{无控制\label{fig:ContU0216WOC}}[0.48\linewidth]{           %
%% label for first subfigure
\includegraphics[width=0.48\linewidth]{ch4/XC=0216(scaledUinf)WOC}}
%\hspace{0.0in}
\subcaptionbox{有控制\label{fig:ContU0216WC}}[0.48\linewidth]{
%% label for second subfigure
\includegraphics[width=0.48\linewidth]{ch4/XC=0216(scaledUinf)WC}}
\caption{$X/C=0.216$位置处流向速度云图}
\label{fig:ContU0216} %% label for entire figure
\end{figure}
\begin{figure}[htb]
\centering
\subcaptionbox{无控制\label{fig:ContU0350WOC}}[0.48\linewidth]{           %
%% label for first subfigure
\includegraphics[width=0.48\linewidth]{ch4/XC=035(scaledUinf)WOC}}
%\hspace{0.0in}
\subcaptionbox{有控制\label{fig:ContU0350WC}}[0.48\linewidth]{
%% label for second subfigure
\includegraphics[width=0.48\linewidth]{ch4/XC=035(scaledUinf)WC}}
\caption{$X/C=0.35$位置处流向速度云图}
\label{fig:ContU0350} %% label for entire figure
\end{figure}
%\clearpage %REMEMBER TO DELATE THIS AFTER YOU ADD ALL WORDS IN THIS PAPER!!!!!!!!!!!!!!!!!!!!!!!!!!!!!!!!!!!
图\ref{fig:ContU0216}和\ref{fig:ContU0350}给出了$X/C=0.216$和0.35位置处的流向速度分布云图。其中$X/C=0.216$非常靠近控制区域的结束位置,也就是$X/C=0.212$。这是图\ref{f:basecase}中(0,2)模态能量出现小峰值的位置。可以看到,这一位置无控制的边界层十分干净,几乎没有什么扰动。加了控制之后,边界层反倒起了一些小的涟漪状的波动。所以可见这种亚谐波的控制方式并不像之前展示的谐波激励,直接在控制区域就起到作用,抑制扰动的能量。相反,在亚谐波激励区域,扰动的的总能量反而因为激发了半波长模态而提高了。亚谐波激励的控制效果在下游才体现出来。如图\ref{fig:ContU0350},在$X/C=0.35$位置处,无控制工况内横流涡已经形成,近壁区的低速流体被卷起,甚至要翻转下来。然而,在控制工况里,这个位置还没有形成明显的横流涡,流向速度云图中也只是起了一点波动,并没有强烈的高低速流体对流现象。从这一位置的流向速度云图结果对比中,可以看到在控制区域的下游,横流涡的生成过程被减缓了。
\begin{figure}
\centering
  % Requires \usepackage{graphicx}
\includegraphics[width=0.24\textwidth]{ch4/compare_Wt_XC=025(scaledUinf)2}
\includegraphics[width=0.24\textwidth]{ch4/compare_Wt_XC=030(scaledUinf)2}
\includegraphics[width=0.24\textwidth]{ch4/compare_Wt_XC=035(scaledUinf)2}
\includegraphics[width=0.24\textwidth]{ch4/compare_Wt_XC=040(scaledUinf)2}
\caption{横流速度剖面}%
\label{f:CFprofiles}
\end{figure}

图\ref{f:CFprofiles}给出了不同算例中不同流向位置横流速度剖面的对比。其中蓝线表示基本流中的横流速度剖面。黑线表示无控制工况中横流速度剖面。这一剖面是基本流的叠加上基本流修正模态,也就是(0,0)模态得到的。可以看到,最开始黑线基本上和蓝线重合,这也意味着基本流修正模态在开始的时候非常微弱。之后,随着横流模态的发展,平均流场被横流涡所扭曲,基本流修正模态的幅值开始提高,黑线也逐渐开始和蓝线分离。可以看到,黑线与蓝线在$X/C=0.35$位置处已经有很大的分离了,而这一位置也就是无控制工况中主模态近乎饱和的位置。图中的红线表示采用了等离子体激励算例中的横流速度剖面分布。可以看到,等离子体对平均流场的影响是显著的。在$X/C=0.25$位置,其横流剖面就已经非常明显的低于无控制算例和基本流中的横流剖面。回忆之前研究压力梯度效应的结果,微弱的横流速度变化都能对之后失稳的发展起到强烈的影响。因此,可以认为正是等离子体降低了横流速度,从而使得流动更加稳定,并阻碍了横流模态的发展。另外,从图\ref{f:CFprofiles}中也可以看到,四个流向位置处,控制工况下的横流强度并没有明显的变化。这主要是因为在控制工况中,横流模态的增长较慢。而基本流修正模态的能量主要来自其他模态对其的非线性作用。而这种非线性作用正比于其他模态本身的强度。所以当横流模态的发展受到抑制,其模态幅值在翼型中段较低,无法有效的通过非线性项作用到基本流修正模态上,使得在控制工况中基本流修正模态并没有明显的变化。
\begin{figure}
\centering
  % Requires \usepackage{graphicx}
\includegraphics[width=0.48\textwidth]{ch4/(scaledUinf)Amp0=1e-4_c_force=VARYING_X0C=020_dzdTz=00_Nplasma=2-improved}
\includegraphics[width=0.48\textwidth]{ch4/(scaledUinf)Amp0=1e-4_c_force=VARYING_X0C=020_dzdTz=00_Nplasma=2___zoomin-improved}
\caption{不同体积力强度控制下的扰动模态能量演化}%
\label{f:forcestrength}
\end{figure}

通过改变等离子体模型中的系数$c_{\rm force}$,本文研究了不同体积力强度对控制效果的影响,计算结果如图\ref{f:forcestrength}所示。其中右边的图是左边图中在控制区域附近的放大。橙色、红色、绿色的线分别代表了$c_{\rm force}=30$、50、70的计算结果。可以看到,在所有控制工况中,主模态和(0,2)模态的能量都有所下降。随着体积力的增强,在控制区域激发出来的(0,2)模态的峰值会逐渐增高,但是在翼型中段各阶失稳模态的能量增长率会逐渐降低。这表明用于控制的体积力越强,控制效果越好。

图\ref{f:streamwiselocationeffect}对比了激发器放置在不同流向位置的算例中的失稳模态能量演化情况。其中绿色、红色和橙色的曲线分别代表DBD激发器位于15\%、20\%和25\%弦长位置处的算例。从图中可以看到,在控制区域,(0,2)模态被激发并形成了一个小的峰值。右侧的图中依然是对激励区域的放大。可以看到,在15\%位置处等离子体激发出来的(0,2)模态能量的峰值最矮,其次是20\%,而25\%位置的最高。这是因为在上游扰动能量本身就低,所以激发出来的模态也略弱。从30\%到40\%弦长位置处,在15\%弦长位置处进行控制的算例的所有模态的能量都比其他算例中对应模态的能量低。在25\%弦长处进行控制的算例中,主模态刚开始并没有受到等离子体激励的影响,一直到33\%弦长位置其能量才开始大幅下降。这三个控制算例说明激发器的位置即便在流向一定范围内有所变化,其主模态的能量依然能够被削弱。

\begin{figure}
\centering
  % Requires \usepackage{graphicx}
\includegraphics[width=0.48\textwidth]{ch4/(scaledUinf)Amp0=1e-4_c_force=50_X0C=VARYING_dzdTz=00_Nplasma=2-improved}
\includegraphics[width=0.48\textwidth]{ch4/(scaledUinf)Amp0=1e-4_c_force=50_X0C=VARYING_dzdTz=00_Nplasma=2____zoomin-improved}
\caption{激发器在不同流向位置时,扰动模态能量演化对比}%
\label{f:streamwiselocationeffect}
\end{figure}
\begin{figure}
\centering
  % Requires \usepackage{graphicx}
\includegraphics[width=0.48\textwidth]{ch4/(scaledUinf)Amp0=1e-4_c_force=50_X0C=020_dzdTz=VARYING_Nplasma=2-improved}
\includegraphics[width=0.48\textwidth]{ch4/(scaledUinf)Amp0=1e-4_c_force=50_X0C=020_dzdTz=VARYING_Nplasma=2___zoomin-improved}
\caption{激发器在不同展向位置时,扰动模态能量演化对比}%
\label{f:spanwiselocationeffect}
\end{figure}
\begin{figure}
\centering
  % Requires \usepackage{graphicx}
\includegraphics[width=0.24\textwidth]{ch4/XC=025_Modified_baseflow(scaledUinf)-z0Tz=025}
\includegraphics[width=0.24\textwidth]{ch4/XC=030_Modified_baseflow(scaledUinf)-z0Tz=025}
\includegraphics[width=0.24\textwidth]{ch4/XC=035_Modified_baseflow(scaledUinf)-z0Tz=025}
\includegraphics[width=0.24\textwidth]{ch4/XC=040_Modified_baseflow(scaledUinf)-z0Tz=025}
\caption{激发器在不同展向位置时,横流速度剖面对比}%
\label{f:Wt_SpVar}
\end{figure}

在小节\ref{subs:control1}中提到过,激发器的展向位置对每个展向周期放置一个激发器的谐波激励控制至关重要。这里研究了激发器展向位置对每个展向周期放置两个激发器的亚谐波激励控制的影响。两个激发器展向位置相差半个相位的控制结果在图\ref{f:spanwiselocationeffect}中进行了对比。由于激发器的展向周期实际是失稳模态的一半,这样让激发器产生反相位需要将其在展向平移四分之一个主模态展向周期。红色的线代表原来的算例,绿色的线代表激发器展向位置发生平移之后的新算例。与之前的算例类似,(0,2)模态在控制区域被激发,但是出了控制区域后又迅速降低。这两个控制工况中的主模态能量在控制区域几乎完全相同,但是一出了控制区域,其能量则开始有了一定的分化。从25\%到45\%弦长处,展向平移之后的控制算例中的主模态一直略高于没有展向位置平移的。然而这一差距并不大,并且无论激发器的位置在展向平移与否,其翼型中段的扰动能量总是低于无控制算例中的扰动能量。

在图\ref{f:spanwiselocationeffect}中已经看到不同展向位置的DBD激发器激发的(0,2)模态的能量几乎没有区别。除了(0,2)模态,剩下的对主模态影响最大的就是基本流修正模态,即(0,0)模态。图\ref{f:Wt_SpVar}对比了这两个算例中的横流平均流速度剖面。可以看到,在25\%和30\%弦长位置,这两个算例中的横流速度都被等离子体降低了,并且降低的幅度完全一样。在随后两个位置他们才出现了轻微的差别。之前提到过,主模态幅值在达到一定程度之后会对基本流修正模态产生一定的抑制作用,这轻微的差异主要是来源于其主模态能量的差异(图\ref{f:spanwiselocationeffect})。至此,展向位置的位置引起的主模态能量差异可以被归结为被激发(0,2)模态的相位差,而不是(0,2)或(0,0)模态的幅值。

\begin{figure}[htb]
  \centering
  \includegraphics[width=\textwidth]{ch4/reversecontrolmechanisim.JPG}
  \caption{反向控制揭示控制机理示意图}\label{f:reversecontrolmechanisim}
\end{figure}
\begin{figure}[htb]
\centering
  % Requires \usepackage{graphicx}
\includegraphics[width=0.6\textwidth]{ch4/force_XZ(scaledUinf)-reversed}
\caption{反向控制算例中体积力在$X$-$Z$平面上的分布}%
\label{f:force_reversed}
\end{figure}

截止目前,人们已经清楚改变(0,0)模态\cite{dorr2016}或者是激发(0,2)模态\cite{Saric1998}可以有效的降低主模态的能量。在这里用到的DBD激发器亚谐波激励控制方案中,(0,0)和(0,2)模态都被等离子体直接改变了,主模态的能量变化是在控制区域下游才出现的。所以到底是(0,0)还是(0,2)模态引起了主模态的降低还并不是十分清楚。为了解决这一问题,笔者设计了一个反向控制工况,用来说明这个问题。在这个反向控制工况内,等离子体激发器被旋转了180$^\circ$。这样体积力就指向了相反的方向。在计算中,体积力是以源项的形式出现。在分析不同模态的演化过程的时候,这一源项先被傅里叶分解成若干不同展向波长的波,然后这些波再作用于不同的模态上。示意图\ref{f:reversecontrolmechanisim}中为了说明问题,假设激发器产生的源项是正弦形式的S$_{\rm total}$。其中S$_{\rm total}$可以被分解为延展向均匀的分量S$_{(0,0)}$和波动形式的分量S$_{(0,2)}$(为了说明问题假设其波动形式分量恰为两倍波长谐波形式)。显然,当激发器的周期只有主模态展向周期的一半的时候,体积力源项对应主模态展向波数的傅里叶分量为零,也就是体积力并不直接作用在主模态上。这从之前的计算结果中也反映了出来。亚谐波激励形式的体积力会直接作用在(0,0),(0,2),(0,4)等模态上。其中(0,0)和(0,2)对主模态影响最大。这里将体积力反向,对于展向均匀的(0,0)模态上的体积力分量S$_{(0,0)}$,其刚好被反号了,从图\ref{f:reversecontrolmechanisim}中的蓝线变成了红线。而对于(0,2),相对应的体积力源项S$_ {(0,2)}$是正弦波形式的,反号也就相当于进行了相位平移,也就是展向位置平移。之前已经证实过展向位置对主模态的演化影响甚微(图\ref{f:spanwiselocationeffect})。所以综上,反向控制中仅仅对(0,0)模态产生了刚好相反的效果,其余模态几乎不受影响。如果控制依然有效,说明(0,0)模态并不是控制有效的主要因素。反之,则说明是。

反向控制算例的体积力在$X$--$Z$平面的分布如图\ref{f:force_reversed}所示。不同模态能量在流向的演化如图\ref{f:model_energy_revers}所示。可以看到,采用了反向控制之后,主模态能量大幅提升。如之前分析的,这表示(0,0)模态在本文提出的亚谐波激励的控制方案中是起主导作用的。这里,为了更好的摆脱(0,2)模态影响,同时测试了相同展向位置的反向控制和展向平移四分之一波长的反向控制。之前提到了,由于体积力作用在(0,2)模态上的分量也是正弦波形式,所以反向相当于移动了半个相位。这里在将半个相位移回来,就完全与之前相同。另外,也可以看到,在这增加扰动能量的两个算例里,激发器展向位置可以轻微的影响模态能量,但是依然无法扭转扰动能量增加这一情况。

\begin{figure}[htb]
\centering
  % Requires \usepackage{graphicx}
\includegraphics[width=0.48\textwidth]{ch4/(scaledUinf)Amp0=1e-4_c_force=50_X0C=020_dzdTz=VARYING_Nplasma=2_revers-improved}
\includegraphics[width=0.48\textwidth]{ch4/(scaledUinf)Amp0=1e-4_c_force=50_X0C=020_dzdTz=VARYING_Nplasma=2_revers___zoomin-improved}
\caption{反向控制算例中模态能量演化}%
\label{f:model_energy_revers}
\end{figure}

\begin{figure}[htb]
\centering
  % Requires \usepackage{graphicx}
\includegraphics[width=0.32\textwidth]{ch4/XC=025_Modified_baseflow_reversed(scaledUinf)}
\includegraphics[width=0.32\textwidth]{ch4/XC=030_Modified_baseflow_reversed(scaledUinf)}
\includegraphics[width=0.32\textwidth]{ch4/XC=035_Modified_baseflow_reversed(scaledUinf)}
\caption{反向控制算例中横流速度剖面}%
\label{f:inverse_meanflow}
\end{figure}

图\ref{f:inverse_meanflow}展示了反向控制算例中不同位置横流速度剖面。可以看到,在25\%弦长处,也就是激励位置下游不远处,横流就已经增大了。这是因为激励器转了向,其体积力方向就与横流方向相同了。之后随着主模态的增长,横流速度剖面受到非线性效应的影响,其幅值大幅下降并最终达到了与无控制工况相当的强度。从这一结果可以看出,横流模态的增长受边界层内横流速度的影响非常大。当横流速度被减弱时,失稳模态的增长就会变得缓慢,当横流速度增强时,失稳就会加剧。
\begin{figure}[htb]
\centering
  % Requires \usepackage{graphicx}
\includegraphics[width=0.6\textwidth]{ch4/(scaledUinf)Amp0=1e-4_c_force=70_X0C=VARYING_dzdTz=0000_Nplasma=3-improved}
\caption{以7.5mm为目标模态的非设计工况控制效果}%
\label{f:7.5mm}
\end{figure}

之前的工况都是以展向波长为5mm的横流驻涡模态作为控制的目标模态来进行控制的。然而,真实的转捩过程受来流条件的影响很大。理论预测的最不稳定模态并不一定会成为主导转捩。这里,本文考虑了一种可能的非设计工况,假设7.5mm展向波长的模态取代了5mm展向波长的模态成为主导模态。由于事先并不知道这一变化,所以用于控制的等离子体激发器间距还是2.5mm。这个时候,控制模态就从(0,2)模态转变为(0,3)模态。在这一非设计工况中,本文测试了三个不同的流向激励位置。各阶模态的能量演化如图\ref{f:7.5mm}所示。其中绿、红和橙色曲线分别代表DBD激发器位于15\%、20\%和25\%弦长处。可以看到,与之前类似,(0,3)模态在控制区域被激发出来。并且激励位置越向下游其激发出来的控制模态的幅值越高。但是这些被激发出来的控制模态在出了激励区域之后其幅值又迅速降低。 在30\%弦长之后,所有控制算例的所有模态的能量都要低于无控制算例中的对应模态。这表明这一控制方案即便在非设计工况中也能起到效果。

\section{本章小结}
在这一章中,本文对后掠翼上三维边界层失稳进行了分析,重点比较了不同压力梯度对三维边界层失稳造成的影响。研究发现,压力梯度增大会提高边界层内的横流速度,并降低边界层的稳定性。但是压力梯度的变化对最不稳定横流模态展向波长的影响并不显著。

在推迟转捩方面,本文先测试了之前在后掠Hiemenz流动中使用过的谐波激励控制方法,也就是在每个展向波长内放置一个DBD激发器。然而这种方法对激发器展向位置精度要求非常高,在错位的展向位置激发器反而会促进转捩。受之前压力梯度效应研究的启发,提出采用等离子体激发器抑制横流的方法推迟转捩。由于控制中在每个展向周期放置两个激发器,所以这一控制方法又被叫做亚谐波激励。本文测试了将激发器放置在不同展向、流向位置,发现其都能起到推迟转捩的效果。甚至在一个非设计工况,都成功的降低了边界层内扰动的能量。通过一个反向控制算例,本文揭示了降低横流强度的基本流修正模态在控制中起主导作用。所以,亚谐波激励控制实际上靠的是DBD产生的体积力削弱了横流,其激发出的亚谐模态仅仅是附带效果。并且,在这种压力梯度变化的翼型上,这一亚谐模态(本例中的展向波长2.5mm模态)是稳定模态。这也保证了新激励出的模态并不会成为新的主导模态。这一亚谐激励控制在后掠Hiemenz流动中的应用效果就不如在后掠翼流动中(没有在文中展示),因为其高阶谐波在下游都会增长,一旦被激发出来就有可能成为新的主导模态。
