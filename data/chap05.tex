\chapter{等离子体激发器控制充分发展槽道湍流}
本文进行控制的基本算例选取为壁面摩擦尺度雷诺数Re$_\tau=180$(Re$_m=5600$)的湍流槽道。计算程序采用的是王志坚课题组的hpMusic\cite{WangZJ2009,Zhu2016,Zh2017},时间步长dt=0.00015,用三阶的显式Eular进行时间推进;空间采用4阶精度计算,单元内采用高斯点,每个单元内有$5\times5\times5$个点,三个方向总的自由度数分别为$235\times155\times200$。在后处理的时候,对单元界面上空间位置相同的点进行了平均。平均处理之后用于显示的网格点数与Kim(1987)\cite{Kim1987}文献中的网格相近。计算域大小为$4\pi \times 2\times 2\pi$。$\gamma$=1.4, 气体常数为1.0,普朗特常数Pr = 0.72,粘性系数 $\mu=3.57\times10^{-4}$, Ma=0.1。计算时在流场内添加全场均匀的体积力,使得槽道内的质量平均流速在稳定在1.0;计算的初始条件为抛物线形速度刨面,流速峰值$u_{\rm max} = 1.327$。为了加快转捩,在这个流场上叠加上10个不同流向展向波数的扰动波。初始的密度$\rho=1.0$;压力$p= 31.74603174603175$;这两个量在计算的时候基本上不会变化。计算得到的近壁涡结构和条带结构如图\ref{f:nearwallvortex}和\ref{f:nearwallstreak}。其中用于展示涡结构的Q值等值面取了与文献中相同的值(Q=0.5)。涡结构上的颜色采用高度进行渲染。从图中可以清楚的看到近壁区的流向条带,以及这些条带经过发展,抬高之后形成的发卡涡。图\ref{f:nearwallstreak}中的流向脉动速度云图取自$y^+=5$。可以看到本文计算得到的近壁结构基本与文献中的类似。
\begin{figure}[htb]
  \centering
  \subcaptionbox{本文计算结果}[0.45\textwidth] %标题的长度,超过则会换行,如下一个小图。
    {\includegraphics[width=0.45\textwidth]{ch5/VorStru_Nocontrol.jpg}}%
  \subcaptionbox{Liang Wei 和 Andrew Pollard (2011)\cite{Wei2011}}[0.45\textwidth]
    {\includegraphics[width=0.45\textwidth]{ch5/VorStru_Nocontrol_ref.jpg}}%
  \caption{计算得到涡结构对比}\label{f:nearwallvortex}
\end{figure}
\begin{figure}[htb]
  \centering
  \subcaptionbox{本文计算结果}[0.5\textwidth] %标题的长度,超过则会换行,如下一个小图。
    {\includegraphics[width=0.5\textwidth]{ch5/streaks_nocontrol.jpg}}%
  \subcaptionbox{Liang Wei 和 Andrew Pollard (2011)\cite{Wei2011}}[0.4\textwidth]
    {\includegraphics[width=0.4\textwidth]{ch5/streaks_paper.jpg}}%
  \caption{计算得到条带结构对比}\label{f:nearwallstreak}
\end{figure}

计算得到的对数律分布如图\ref{f:loglow}。其为采用190万瞬时结果进行时间平均和空间平均(流向和展向)之后的结果。其中线性区对比的理论公式为:
\begin{equation}\label{e:linear_region}
  U^+=y^+
\end{equation}
对数区对比的理论公式为:
\begin{equation}\label{e:loglaw}
  U^+=2.5{\rm ln}y^++5.5
\end{equation}
可以看到,在$y^+<5$的区域,平均流速分布基本上完全符合线性关系,在$20<y^+<110$的区间,平均速度与对数律完美吻合。另外在图中同时画出了上壁面和下壁面的平均流速分布,y在这里均表示测量点到壁面的距离。上下壁面的速度分布分别用红线和黑线表示。可以看到,图中红线完全将黑线覆盖。
\begin{figure}[htb]
  \centering
  \includegraphics[width=0.7\textwidth]{ch5/loglow.jpg}
  \caption{计算结果与对数律对比}\label{f:loglow}
\end{figure}

计算得到的二阶统计量与文献对比如图\ref{f:2orderaver}。这四张图依次是流向、法向、展向脉动速度均方根,以及雷诺切应力$<uv>^+$。这些量均采用壁面摩擦尺度进行无量纲化。其中红线均来自本文的计算结果,绿线为文献\cite{Kim1987}中给出的结果。可以看到两条线基本上完全重合,这表明本文的计算精度是满足要求的。
\begin{figure}[htb]
  \centering
  \subcaptionbox{$u_{rms}^+$}[0.45\textwidth] %标题的长度,超过则会换行,如下一个小图。
    {\includegraphics[width=0.45\textwidth]{ch5/urms_NoControl.jpeg}}%
  \subcaptionbox{$v_{rms}^+$}[0.45\textwidth]
    {\includegraphics[width=0.45\textwidth]{ch5/vrms_NoControl.jpeg}}%
  \\\bigskip
  \subcaptionbox{$w_{rms}^+$}[0.45\textwidth] %标题的长度,超过则会换行,如下一个小图。
    {\includegraphics[width=0.45\textwidth]{ch5/wrms_NoControl.jpeg}}%
  \subcaptionbox{$<uv>^+$}[0.45\textwidth]
    {\includegraphics[width=0.45\textwidth]{ch5/uv_NoControl.jpeg}}%
  \caption{计算得到的二阶统计量与文献中的结果对比}\label{f:2orderaver}
\end{figure}
\begin{figure}[htb]
  \centering
  \includegraphics[width=\textwidth]{ch5/condit_aver.jpeg}
  \caption{$y^+=20$条件平均}\label{f:base_condition_average}
\end{figure}

前人的研究\cite{Hamilton1995}指出,壁湍流的产生来自近壁拟序结构的自维持机制。也就是流向涡通过上抛和下扫产生高速和低速条带,然后高速和低速条带又由于失稳机制再次破碎产生新的流向涡。虽然从瞬时涡结构中也能观察到近壁的流向涡,但是其分布过于杂乱无章,不便于分析。这里采用一种条件平均技术对结果进行后处理。首先对每一个瞬时结果在$y^+=20$的平面上进行涡探测。之所以选择$y^+=20$是因为流向涡量均方极大值位于这个高度\cite{Jeong1997},所以也被普遍认为是流向涡涡心所处在的高度。探测时采用$Q=1$作为涡的识别标准。探测到涡之后,选取涡内$Q$的最大值作为涡心。之后将所有探测得到的涡的涡心的位置置为$z=0$,然后对涡心附近的流场进行平均,得到图\ref{f:base_condition_average}。



\section{定常激励控制方案}
文献中给出了一种大涡形状的体积力分布用于减阻
\begin{figure}
  \centering
  \includegraphics[width=0.8\textwidth]{ch5/vortex.JPG}
  \caption{J Canton et al. (2016 FTC)采用的减阻控制的体积力}\label{f:vor_f}
\end{figure}
他们对这种减阻控制方式进行了参数研究:
\begin{figure}
  \centering
  \includegraphics[width=0.8\textwidth]{ch5/parameter_study.JPG}
  \caption{涡强度与减阻率的关系}\label{f:vort_strength_vs_DR}
\end{figure}
然而,实际中并不能产生这种形状的体积力,所以本文考虑采用DBD产生相同的效果。本文中采用的体积力分布情况如下:
\begin{figure}
  \centering
  \includegraphics[width=0.8\textwidth]{ch5/steady_force.jpg}
  \caption{定常激励采用的DBD体积力示意图}\label{f:steady_force}
\end{figure}
在没有背景流动的槽道中诱导出来的流动如下:
\begin{figure}[h]
  \centering
  \subcaptionbox{V}[0.8\textwidth] %标题的长度,超过则会换行,如下一个小图。
    {\includegraphics[width=0.8\textwidth]{ch5/V_nobackflow.jpg}}%
  \\\bigskip
  \subcaptionbox{W}[0.8\textwidth]
    {\includegraphics[width=0.8\textwidth]{ch5/W_nobackflow.jpg}}%
  \caption{定常DBD在无背景流动的槽道中诱导出来的流场}
\end{figure}
由于最开始并不知道体积力诱导的涡强度与真实涡强度的关系,所以做了很多参数的实验。最终发现确实过大的体积力反而会起到增阻的的作用。这里仅展示一个增阻的算例和一个减阻的算例,他们的产生的涡强度如下:
\begin{table}
  \centering
  \caption{不同算例产生的涡强度}
  \begin{tabularx}{\textwidth}{*2{>{\centering\arraybackslash}X}}
    \toprule[1.5pt]
    {\heiti 算例名称} & {\heiti max$(V)/U_b$} \\\midrule[1pt]
        Plasma(Strong) & 0.19 \\
        Plasma(Weak) & 0.05 \\
        Vortex Force & 0.06 \\
    \bottomrule[1.5pt]
  \end{tabularx}
\end{table}
各个算例下壁面阻力对比:
\begin{figure}
  \centering
  \includegraphics[width=0.5\textwidth]{ch5/FX_steady_forcing.jpeg}
  \caption{各个算例下壁面阻力对比}
\end{figure}
流向平均与时间平均对比:
\begin{figure}[h]
  \centering
  \subcaptionbox{Vortex Force: U}[0.45\textwidth]
    {\includegraphics[width=0.45\textwidth]{ch5/VortexF/U.png}}
  \subcaptionbox{Vortex Force: V}[0.45\textwidth]
    {\includegraphics[width=0.45\textwidth]{ch5/VortexF/V.png}}
  \\\bigskip
  \subcaptionbox{Plasma(Strong): U}[0.45\textwidth]
    {\includegraphics[width=0.45\textwidth]{ch5/PlasmaStronge/U.png}}
  \subcaptionbox{Plasma(Strong): V}[0.45\textwidth]
    {\includegraphics[width=0.45\textwidth]{ch5/PlasmaStronge/V.png}}
  \\\bigskip
  \subcaptionbox{Plasma(Weak): U}[0.45\textwidth]
    {\includegraphics[width=0.45\textwidth]{ch5/PlasmaWeak/U.png}}
  \subcaptionbox{Plasma(Weak): V}[0.45\textwidth]
    {\includegraphics[width=0.45\textwidth]{ch5/PlasmaWeak/V.png}}
  \caption{流向与法向平均速度}
\end{figure}
\begin{figure}[h]
  \centering
  \subcaptionbox{Vortex Force: W}[0.45\textwidth]
    {\includegraphics[width=0.45\textwidth]{ch5/VortexF/W.png}}
  \subcaptionbox{Vortex Force: <uv>}[0.45\textwidth]
    {\includegraphics[width=0.45\textwidth]{ch5/VortexF/uv.png}}
  \\\bigskip
  \subcaptionbox{Plasma(Strong): W}[0.45\textwidth]
    {\includegraphics[width=0.45\textwidth]{ch5/PlasmaStronge/W.png}}
  \subcaptionbox{Plasma(Strong): <uv>}[0.45\textwidth]
    {\includegraphics[width=0.45\textwidth]{ch5/PlasmaStronge/uv.png}}
  \\\bigskip
  \subcaptionbox{Plasma(Weak): W}[0.45\textwidth]
    {\includegraphics[width=0.45\textwidth]{ch5/PlasmaWeak/W.png}}
  \subcaptionbox{Plasma(Weak): <uv>}[0.45\textwidth]
    {\includegraphics[width=0.45\textwidth]{ch5/PlasmaWeak/uv.png}}
  \caption{展向平均速度与雷诺切应力}
\end{figure}
\begin{figure}[h]
  \centering
  \subcaptionbox{Vortex Force: $<uu>$}[0.45\textwidth]
    {\includegraphics[width=0.45\textwidth]{ch5/VortexF/uu.png}}
  \subcaptionbox{Vortex Force: $<vv>$}[0.45\textwidth]
    {\includegraphics[width=0.45\textwidth]{ch5/VortexF/vv.png}}
  \\\bigskip
  \subcaptionbox{Plasma(Strong): $<uu>$}[0.45\textwidth]
    {\includegraphics[width=0.45\textwidth]{ch5/PlasmaStronge/uu.png}}
  \subcaptionbox{Plasma(Strong): $<vv>$}[0.45\textwidth]
    {\includegraphics[width=0.45\textwidth]{ch5/PlasmaStronge/vv.png}}
  \\\bigskip
  \subcaptionbox{Plasma(Weak): $<uu>$}[0.45\textwidth]
    {\includegraphics[width=0.45\textwidth]{ch5/PlasmaWeak/uu.png}}
  \subcaptionbox{Plasma(Weak): $<vv>$}[0.45\textwidth]
    {\includegraphics[width=0.45\textwidth]{ch5/PlasmaWeak/vv.png}}
  \caption{流向与法向雷诺主应力}
\end{figure}
Plasma(Weak)算例中的涡结构:
\begin{figure}
  \centering
  \includegraphics[width=\textwidth]{ch5/PlasmaWeak/8367500.png}
  \caption{Plasma(Weak)涡结构}
\end{figure}
Plasma(Weak)湍动能和湍动能生成项:
\begin{figure}
  \centering
  \subcaptionbox{湍动能}[0.45\textwidth]
    {\includegraphics[width=0.45\textwidth]{ch5/PlasmaWeak/k.jpeg}}
  \subcaptionbox{湍动能生成项}[0.45\textwidth]
    {\includegraphics[width=0.45\textwidth]{ch5/PlasmaWeak/Production.jpeg}}
  \caption{Plasma(Weak)湍动能和湍动能生成项}
\end{figure}

\section{周期激励控制方案}
米兰理工体积力形式:
\begin{equation}\label{e:f_POLIMI}
  F_z  = F_z \left( {y,t} \right) = A_f e^{ - y/D} \cos \left( {2\pi \frac{t}{T}} \right);A_f  = 2;D = 0.04
\end{equation}
控制效果:
\begin{figure}
  \centering
  \includegraphics[width=0.5\textwidth]{ch5/POLIMI/Drag.jpeg}
  \caption{米兰理工体积力激励方案阻力变化}\label{f:POLIMI_DARG}
\end{figure}
$T^+=52$相平均:
\begin{figure}
  \centering
  \subcaptionbox{$U^+$}[0.45\textwidth]
    {\includegraphics[width=0.45\textwidth]{ch5/POLIMI/phase_average_U+.jpeg}}
  \subcaptionbox{$W$}[0.45\textwidth]
    {\includegraphics[width=0.45\textwidth]{ch5/POLIMI/phase_average_W.jpeg}}
  \caption{$T^+=52$相平均}
\end{figure}
涡结构图:
\begin{figure}
  \centering
  \includegraphics[width=0.8\textwidth]{ch5/POLIMI/5973000.png}
  \caption{POLIMI涡结构图}
\end{figure}
各阶统计量对比:
\begin{figure}[h]
  \centering
  \subcaptionbox{$<uu>$}[0.45\textwidth] %标题的长度,超过则会换行,如下一个小图。
    {\includegraphics[width=0.45\textwidth]{ch5/POLIMI/uu.png}}%
  \subcaptionbox{$<vv>$}[0.45\textwidth]
    {\includegraphics[width=0.45\textwidth]{ch5/POLIMI/vv.png}}%
  \\\bigskip
  \subcaptionbox{$<ww>$}[0.45\textwidth] %标题的长度,超过则会换行,如下一个小图。
    {\includegraphics[width=0.45\textwidth]{ch5/POLIMI/ww.png}}%
  \subcaptionbox{$<uv>$}[0.45\textwidth]
    {\includegraphics[width=0.45\textwidth]{ch5/POLIMI/uv.png}}%
  \caption{POLIMI二阶统计量对比}
\end{figure}
\begin{figure}
  \centering
  \subcaptionbox{POLIMI}[0.45\textwidth]
    {\includegraphics[width=0.45\textwidth]{ch5/POLIMI/POLIMI_num_vort.jpeg}}
  \subcaptionbox{16DBD}[0.45\textwidth]
    {\includegraphics[width=0.45\textwidth]{ch5/16DBD/16DBD_num_vort.jpeg}}
  \caption{不同相位探测到的涡个数}
\end{figure}
\begin{figure}
  \centering
  \subcaptionbox{条件平均}[0.8\textwidth]
    {\includegraphics[height=7cm]{ch5/POLIMI/condit_0.jpeg}}
  \subcaptionbox{展向速度}[0.18\textwidth]
    {\includegraphics[height=7cm]{ch5/POLIMI/phase_0.jpeg}}
  \caption{POLIMI体积力分布$0/16\pi$相平均}
\end{figure}
\begin{figure}
  \centering
  \subcaptionbox{条件平均}[0.8\textwidth]
    {\includegraphics[height=7cm]{ch5/POLIMI/condit_1.jpeg}}
  \subcaptionbox{展向速度}[0.18\textwidth]
    {\includegraphics[height=7cm]{ch5/POLIMI/phase_1.jpeg}}
  \caption{POLIMI体积力分布$1/16\pi$相平均}
\end{figure}
\begin{figure}
  \centering
  \subcaptionbox{条件平均}[0.8\textwidth]
    {\includegraphics[height=7cm]{ch5/POLIMI/condit_2.jpeg}}
  \subcaptionbox{展向速度}[0.18\textwidth]
    {\includegraphics[height=7cm]{ch5/POLIMI/phase_2.jpeg}}
  \caption{POLIMI体积力分布$2/16\pi$相平均}
\end{figure}
\begin{figure}
  \centering
  \subcaptionbox{条件平均}[0.8\textwidth]
    {\includegraphics[height=7cm]{ch5/POLIMI/condit_3.jpeg}}
  \subcaptionbox{展向速度}[0.18\textwidth]
    {\includegraphics[height=7cm]{ch5/POLIMI/phase_3.jpeg}}
  \caption{POLIMI体积力分布$3/16\pi$相平均}
\end{figure}
\begin{figure}
  \centering
  \subcaptionbox{条件平均}[0.8\textwidth]
    {\includegraphics[height=7cm]{ch5/POLIMI/condit_4.jpeg}}
  \subcaptionbox{展向速度}[0.18\textwidth]
    {\includegraphics[height=7cm]{ch5/POLIMI/phase_4.jpeg}}
  \caption{POLIMI体积力分布$4/16\pi$相平均}
\end{figure}
\begin{figure}
  \centering
  \subcaptionbox{条件平均}[0.8\textwidth]
    {\includegraphics[height=7cm]{ch5/POLIMI/condit_5.jpeg}}
  \subcaptionbox{展向速度}[0.18\textwidth]
    {\includegraphics[height=7cm]{ch5/POLIMI/phase_5.jpeg}}
  \caption{POLIMI体积力分布$5/16\pi$相平均}
\end{figure}
\begin{figure}
  \centering
  \subcaptionbox{条件平均}[0.8\textwidth]
    {\includegraphics[height=7cm]{ch5/POLIMI/condit_6.jpeg}}
  \subcaptionbox{展向速度}[0.18\textwidth]
    {\includegraphics[height=7cm]{ch5/POLIMI/phase_6.jpeg}}
  \caption{POLIMI体积力分布$6/16\pi$相平均}
\end{figure}
\begin{figure}
  \centering
  \subcaptionbox{条件平均}[0.8\textwidth]
    {\includegraphics[height=7cm]{ch5/POLIMI/condit_7.jpeg}}
  \subcaptionbox{展向速度}[0.18\textwidth]
    {\includegraphics[height=7cm]{ch5/POLIMI/phase_7.jpeg}}
  \caption{POLIMI体积力分布$7/16\pi$相平均}
\end{figure}
\begin{figure}
  \centering
  \subcaptionbox{条件平均}[0.8\textwidth]
    {\includegraphics[height=7cm]{ch5/POLIMI/condit_8.jpeg}}
  \subcaptionbox{展向速度}[0.18\textwidth]
    {\includegraphics[height=7cm]{ch5/POLIMI/phase_8.jpeg}}
  \caption{POLIMI体积力分布$8/16\pi$相平均}
\end{figure}
\begin{figure}
  \centering
  \subcaptionbox{条件平均}[0.8\textwidth]
    {\includegraphics[height=7cm]{ch5/POLIMI/condit_9.jpeg}}
  \subcaptionbox{展向速度}[0.18\textwidth]
    {\includegraphics[height=7cm]{ch5/POLIMI/phase_9.jpeg}}
  \caption{POLIMI体积力分布$9/16\pi$相平均}
\end{figure}
\begin{figure}
  \centering
  \subcaptionbox{条件平均}[0.8\textwidth]
    {\includegraphics[height=7cm]{ch5/POLIMI/condit_10.jpeg}}
  \subcaptionbox{展向速度}[0.18\textwidth]
    {\includegraphics[height=7cm]{ch5/POLIMI/phase_10.jpeg}}
  \caption{POLIMI体积力分布$10/16\pi$相平均}
\end{figure}
\begin{figure}
  \centering
  \subcaptionbox{条件平均}[0.8\textwidth]
    {\includegraphics[height=7cm]{ch5/POLIMI/condit_11.jpeg}}
  \subcaptionbox{展向速度}[0.18\textwidth]
    {\includegraphics[height=7cm]{ch5/POLIMI/phase_11.jpeg}}
  \caption{POLIMI体积力分布$11/16\pi$相平均}
\end{figure}
\begin{figure}
  \centering
  \subcaptionbox{条件平均}[0.8\textwidth]
    {\includegraphics[height=7cm]{ch5/POLIMI/condit_12.jpeg}}
  \subcaptionbox{展向速度}[0.18\textwidth]
    {\includegraphics[height=7cm]{ch5/POLIMI/phase_12.jpeg}}
  \caption{POLIMI体积力分布$12/16\pi$相平均}
\end{figure}
\begin{figure}
  \centering
  \subcaptionbox{条件平均}[0.8\textwidth]
    {\includegraphics[height=7cm]{ch5/POLIMI/condit_13.jpeg}}
  \subcaptionbox{展向速度}[0.18\textwidth]
    {\includegraphics[height=7cm]{ch5/POLIMI/phase_13.jpeg}}
  \caption{POLIMI体积力分布$13/16\pi$相平均}
\end{figure}
\begin{figure}
  \centering
  \subcaptionbox{条件平均}[0.8\textwidth]
    {\includegraphics[height=7cm]{ch5/POLIMI/condit_14.jpeg}}
  \subcaptionbox{展向速度}[0.18\textwidth]
    {\includegraphics[height=7cm]{ch5/POLIMI/phase_14.jpeg}}
  \caption{POLIMI体积力分布$14/16\pi$相平均}
\end{figure}
\begin{figure}
  \centering
  \subcaptionbox{条件平均}[0.8\textwidth]
    {\includegraphics[height=7cm]{ch5/POLIMI/condit_15.jpeg}}
  \subcaptionbox{展向速度}[0.18\textwidth]
    {\includegraphics[height=7cm]{ch5/POLIMI/phase_15.jpeg}}
  \caption{POLIMI体积力分布$15/16\pi$相平均}
\end{figure}
\begin{figure}
  \centering
  \subcaptionbox{条件平均}[0.8\textwidth]
    {\includegraphics[height=7cm]{ch5/16DBD/condit_0.jpeg}}
  \subcaptionbox{展向速度}[0.18\textwidth]
    {\includegraphics[height=7cm]{ch5/16DBD/phase_0.jpeg}}
  \caption{16DBD体积力分布$0/16\pi$相平均}
\end{figure}
\begin{figure}
  \centering
  \subcaptionbox{条件平均}[0.8\textwidth]
    {\includegraphics[height=7cm]{ch5/16DBD/condit_1.jpeg}}
  \subcaptionbox{展向速度}[0.18\textwidth]
    {\includegraphics[height=7cm]{ch5/16DBD/phase_1.jpeg}}
  \caption{16DBD体积力分布$1/16\pi$相平均}
\end{figure}
\begin{figure}
  \centering
  \subcaptionbox{条件平均}[0.8\textwidth]
    {\includegraphics[height=7cm]{ch5/16DBD/condit_2.jpeg}}
  \subcaptionbox{展向速度}[0.18\textwidth]
    {\includegraphics[height=7cm]{ch5/16DBD/phase_2.jpeg}}
  \caption{16DBD体积力分布$2/16\pi$相平均}
\end{figure}
\begin{figure}
  \centering
  \subcaptionbox{条件平均}[0.8\textwidth]
    {\includegraphics[height=7cm]{ch5/16DBD/condit_3.jpeg}}
  \subcaptionbox{展向速度}[0.18\textwidth]
    {\includegraphics[height=7cm]{ch5/16DBD/phase_3.jpeg}}
  \caption{16DBD体积力分布$3/16\pi$相平均}
\end{figure}
\begin{figure}
  \centering
  \subcaptionbox{条件平均}[0.8\textwidth]
    {\includegraphics[height=7cm]{ch5/16DBD/condit_4.jpeg}}
  \subcaptionbox{展向速度}[0.18\textwidth]
    {\includegraphics[height=7cm]{ch5/16DBD/phase_4.jpeg}}
  \caption{16DBD体积力分布$4/16\pi$相平均}
\end{figure}
\begin{figure}
  \centering
  \subcaptionbox{条件平均}[0.8\textwidth]
    {\includegraphics[height=7cm]{ch5/16DBD/condit_5.jpeg}}
  \subcaptionbox{展向速度}[0.18\textwidth]
    {\includegraphics[height=7cm]{ch5/16DBD/phase_5.jpeg}}
  \caption{16DBD体积力分布$5/16\pi$相平均}
\end{figure}
\begin{figure}
  \centering
  \subcaptionbox{条件平均}[0.8\textwidth]
    {\includegraphics[height=7cm]{ch5/16DBD/condit_6.jpeg}}
  \subcaptionbox{展向速度}[0.18\textwidth]
    {\includegraphics[height=7cm]{ch5/16DBD/phase_6.jpeg}}
  \caption{16DBD体积力分布$6/16\pi$相平均}
\end{figure}
\begin{figure}
  \centering
  \subcaptionbox{条件平均}[0.8\textwidth]
    {\includegraphics[height=7cm]{ch5/16DBD/condit_7.jpeg}}
  \subcaptionbox{展向速度}[0.18\textwidth]
    {\includegraphics[height=7cm]{ch5/16DBD/phase_7.jpeg}}
  \caption{16DBD体积力分布$7/16\pi$相平均}
\end{figure}
\begin{figure}
  \centering
  \subcaptionbox{条件平均}[0.8\textwidth]
    {\includegraphics[height=7cm]{ch5/16DBD/condit_8.jpeg}}
  \subcaptionbox{展向速度}[0.18\textwidth]
    {\includegraphics[height=7cm]{ch5/16DBD/phase_8.jpeg}}
  \caption{16DBD体积力分布$8/16\pi$相平均}
\end{figure}
\begin{figure}
  \centering
  \subcaptionbox{条件平均}[0.8\textwidth]
    {\includegraphics[height=7cm]{ch5/16DBD/condit_9.jpeg}}
  \subcaptionbox{展向速度}[0.18\textwidth]
    {\includegraphics[height=7cm]{ch5/16DBD/phase_9.jpeg}}
  \caption{16DBD体积力分布$9/16\pi$相平均}
\end{figure}
\begin{figure}
  \centering
  \subcaptionbox{条件平均}[0.8\textwidth]
    {\includegraphics[height=7cm]{ch5/16DBD/condit_10.jpeg}}
  \subcaptionbox{展向速度}[0.18\textwidth]
    {\includegraphics[height=7cm]{ch5/16DBD/phase_10.jpeg}}
  \caption{16DBD体积力分布$10/16\pi$相平均}
\end{figure}
\begin{figure}
  \centering
  \subcaptionbox{条件平均}[0.8\textwidth]
    {\includegraphics[height=7cm]{ch5/16DBD/condit_11.jpeg}}
  \subcaptionbox{展向速度}[0.18\textwidth]
    {\includegraphics[height=7cm]{ch5/16DBD/phase_11.jpeg}}
  \caption{16DBD体积力分布$11/16\pi$相平均}
\end{figure}
\begin{figure}
  \centering
  \subcaptionbox{条件平均}[0.8\textwidth]
    {\includegraphics[height=7cm]{ch5/16DBD/condit_12.jpeg}}
  \subcaptionbox{展向速度}[0.18\textwidth]
    {\includegraphics[height=7cm]{ch5/16DBD/phase_12.jpeg}}
  \caption{16DBD体积力分布$12/16\pi$相平均}
\end{figure}
\begin{figure}
  \centering
  \subcaptionbox{条件平均}[0.8\textwidth]
    {\includegraphics[height=7cm]{ch5/16DBD/condit_13.jpeg}}
  \subcaptionbox{展向速度}[0.18\textwidth]
    {\includegraphics[height=7cm]{ch5/16DBD/phase_13.jpeg}}
  \caption{16DBD体积力分布$13/16\pi$相平均}
\end{figure}
\begin{figure}
  \centering
  \subcaptionbox{条件平均}[0.8\textwidth]
    {\includegraphics[height=7cm]{ch5/16DBD/condit_14.jpeg}}
  \subcaptionbox{展向速度}[0.18\textwidth]
    {\includegraphics[height=7cm]{ch5/16DBD/phase_14.jpeg}}
  \caption{16DBD体积力分布$14/16\pi$相平均}
\end{figure}
\begin{figure}
  \centering
  \subcaptionbox{条件平均}[0.8\textwidth]
    {\includegraphics[height=7cm]{ch5/16DBD/condit_15.jpeg}}
  \subcaptionbox{展向速度}[0.18\textwidth]
    {\includegraphics[height=7cm]{ch5/16DBD/phase_15.jpeg}}
  \caption{16DBD体积力分布$15/16\pi$相平均}
\end{figure} 