\chapter{引言}
\label{cha:intro}

为了配合我国大型客机发展的战略需求,帮助新一代自主研制客机在节能减排绿色环保的道路上顺利前进,研究边界层减阻相关技术至关重要。对于大型民用客机,摩擦阻力几乎占到了总阻力的一半\cite{Schrauf2005}。而这些摩擦阻力又主要由机身和机翼贡献。就机翼而言,其流向长度不大,将层流向湍流转捩的过程推迟的层流技术,非常有应用前景\cite{Joslin1998}。但是对于机身,其流向尺度过大,机身中后段边界层已经无可避免的发展为湍流,这个时候层流技术毫无用武之地,必须使用通过改变充分发展湍流中拟序结构的方法 降低阻力。针对这两种完全不同的情况,本文将其抽象为两个分别与之对应的科学问题进行研究。关于机翼减阻的问题,本文将其简化为三维边界层层流-湍流转捩推迟问题。关于机身上减阻的问题,本文将其抽象成充分发展槽道壁湍流减阻控制问题。
\begin{figure}[htb]
  \centering
  \includegraphics[width=0.6\textwidth]{ch1/Schrauf2005.jpg}
  \caption{大型客机阻力分解\cite{Schrauf2005}}\label{f:Schrauf2005}
\end{figure}

为了完成减阻的目标,本文以介质阻挡放电(DBD)等离子体激发器作为控制装置。这种激发器由于有响应时间短,安装方便,耗能低,器件小等众多优点,近些年得到了流动控制领域研究者们的青睐。在引言的第一节,本文先简要的介绍一下这种流动控制装置的特点、数值模拟方法与相关应用。在随后的两节里分别介绍后掠翼转捩推迟与壁湍流流动控制的研究现状。


\section{介质阻挡放电等离子体激发器}

等离子体是除了液态、固态以及气态之外的物质第四态\cite{zhangguling2008,niewanshang2012}。等离子体可由高温或者强电场产生。由强电场电离产生的等离子体通常由大量的电子和相应成对出现的离子构成。在电场的作用下,等离子体可以表现出明显的集体行为
\cite{wuhong2015}。在本文中,主要是用的介质阻挡放电等离子体激发器。这种激发器由两片电电极和一层绝缘层构成(如图~\ref{fig:SchematicPlasma}所示)。当在两片电极上加上高电压时,两片电击之间的空气就会被电离,在介质表面形生成等离子体。等离子体受到电场的驱动,就会形成平行于壁面的“电风”(electric wind)。这一控制装置相比传统的控制装置有诸多优点\cite{Corke2010}。首先,该激励器的主要部件就是两个电极片和一张绝缘层,非常轻便。其次,其该激励器反应速度很快,激励频率可调且可调范围广,在流动控制中可以很方便的避开流动的特征频率。最后,在不需要控制的时候激励器对流动完全没有影响。

DBD等离子体激励器对流动的影响可以分为三类,分别是物性变化效应,动量加速效应以及温升热冲击效应\cite{yu2017}。对于不可压缩流动,动量加速效应占主导,另外两种效应可以忽略。在本文的研究也主要依赖DBD激发器的动量加速效应来改变流动。简单而言,动量加速效应的物理过程就是DBD激发器将电场力作用咋其附近的流体上,然后这种电场力会驱动流体加速。因此,确定激发器施加在流体上的作用力强度对流动控制的研究至关重要\cite{Enloe2004,Debien2012,Kriegseis2011,Hoskinson2010,Durscher2011}。实验研究发现,DBD激发器产生的总的力基本上随着其功率的上升而线性增加,效用系数大约为0.25mN/W\cite{Kriegseis2011}。在激励频率固定的情况下,激励器产生的总的力与电压的平方成正比\cite{Debien2012},即$F\propto(V-V_0)^2$。目前试验中测量到的最强的体积力为125mN/m\cite{Thomas2009},采用了40kV的电压和1kHz的频率。电介质采用的是6.35mm厚的石英。这最强的激发器的能耗为4W/cm。由DBD激发器产生的“电风”的最大速度会随着能耗的增加而增加,但是当能耗达到2W/cm之后,会达到饱和状态\cite{Jolibois2009}。除了测量最大风速,学者们还采用PIV(particle image velocimetry,粒子成像测速)技术对“电风”在空间的分布进行了测量\cite{Kotsonis2011,kriegseis2013velocity},其流速矢量场如\ref{f:Kotsonis2011}。由于本文的重点是探讨如何采用DBD激发器进行流动控制,其物理机理在这里就不做过多探讨,读者可以参考文献\cite{Moreau2007,Benard2014}。

\begin{figure}
  \centering
  \includegraphics[width=\textwidth]{plasma}
  \caption{等离子体激发器示意图\cite{Whalley2012}}\label{fig:SchematicPlasma}
\end{figure}
\begin{figure}
  \centering
  \includegraphics[width=0.6\textwidth]{ch1/Kotsonis2011.JPG}
  \caption{Kotsonis和Ghaemi\cite{Kotsonis2011}采用PIV测量得到的DBD激发器诱发出的流场矢量图}\label{f:Kotsonis2011}
\end{figure}
本文主要是采用数值模拟的方法研究这种激发器在流动控制中的应用,所以本文将在引言的第一部分重点介绍DBD等离子体激发器的数值模拟方法与其在流动控制领域的应用现状。


\subsection{介质阻挡放电等离子体激发器数值模拟方法}
传统的等离子体数值模型可以分为两类,一种是从第一原理出发的机理模型,还有一种是较为简单的唯像简化模型。从机理模型会对从空气电离到动量传导的每一个物理过程进行建模求解。这一类模型往往需要求解多组分的输运方程,电场的泊松方程以及纳维-斯托克斯(N-S)方程。Boeuf等人\cite{Boeuf2007}采用电子-离子两组分的模型模拟了DBD放电的全过程,并通过计算作用在带电离子上的电场力,得到了DBD产生的总的作用力。该课题组随后又将不带电的中性离子也纳入考虑范畴,解析了带电粒子与中性粒子之间的动量传输过程。Boeuf等人\cite{Boeuf2009}在2009年进一步研究发现,在激励的正电压周期(电介质上的电极是阳极),中性粒子获得的动量主要来自正离子,相反,在另外半个周期动量主要来自负离子。Nishida等人\cite{Nishida2016,Nishida2011}求解了展向均匀和展向非均匀(电极上有小的凸起)两种不同构型激发器的电离及气动激励过程。他们对比发现忽略三维效应会使得预测的体积力在流向分布范围变小,体积力强度也会变弱。基于第一原理的模型虽然能够给出体积力产生的详细物理过程,但是其计算量巨大,通常会比简化模型的计算量大一个量级。这些模型在研究等离子体产生机理以及优化激发器形状的时候非常有用,但是仅仅是进行流动控制的研究,这些模型有些过于浪费计算资源。

唯像简化模型并不求解组分的输运方程,只是先验的给出一个等离子体或电场的分布。其中最经典的莫过于史维等人提出来的三角模型\cite{shyy2002}。该模型假设电场线性的分布在两个磁极之间的三角形区域,且忽略掉了体积力随时间的变化。由于这个模型极大的简化了DBD产生等离子体并与流体相互作用的过程,因此深得流动控制领域研究者的青睐。先后有学者将其应用于槽道湍流减阻\cite{LiZX2015}和翼型上表面层流分离抑制\cite{Rizzetta2011}。Orlov等人\cite{Orlov2006,Corke2008,Mertz2011}则是将DBD放电的电流环路简化成一个电路系统。这一电路系统一般包含若干形式相同的支路系统,每一个支路系统对应两片电极之间一处特定的空间。这样在抑制两片电极上电压随时间变化的规律,就能够进一步求解出电流、空间电场强度以及体积力的分布。Suzen等人\cite{suzen2005,suzen2007}基于实验\cite{Enloe2004,enloe2006}的观察,假设覆盖在电极上的电介质上表面的电荷密度分布为半个高斯函数的形式,即:
\begin{equation}\label{e:suzen}
  \rho_{c,w}(x,t)=\rho_cf(t)e^{-\frac{(x-\mu)^2}{2\sigma^2}}
\end{equation}
基于这一假设,只需要再求解关于空间电荷电位分布的泊松方程,即可得到激发器作用于流体上的体积力分布。这种做法巧妙的回避了各种组分输运方程的求解,但同时还原了部分的关键物理过程。Ibrahim等人\cite{Ibrahim2014}采用这一模型模拟了由DBD激发器驱动的槽道流动,并与实验\cite{Debiasi2011}对比发现,数值模拟可以很好的预测槽道中心的最大流速,但是速度剖面的吻合欠佳。Eltaweel等人\cite{Eltaweel2014}应用该模型,研究了如何使用DBD激发器进行串联圆柱降噪。总的而言,这类模型求解方便,计算量小,但是在精度方面却有些逊色。

近些年来,很多学者提出了一种基于实验的等离子体模型\cite{albrecht2011method,Kotsonis2011,kriegseis2013velocity,Benard2014,wilke2009aerodynamische}。由于一般DBD的激励频率远高于流动的固有频率,DBD的激励周期远小于流动的特征尺度,所以可以近似认为流动对等离子体产生的体积力分布没有影响。所以只需要测定在某一特殊工况下的体积力分布,这一分布就可以拿来应用到其他控制工况的模拟中。在实验中,针对准二维流动,可以通过PIV测量得到两个方向的速度分量。然而,空间中的压力分布不知道,这就导致体积力的两个分量并不能够直接算出。为了解决这个问题,Wilke\cite{wilke2009aerodynamische}认为在DBD诱导出的流动中,体积力的强度远大于压力梯度,所以可以将压力梯度忽略掉。Albrecht\cite{albrecht2011method}用两个动量方程推导出涡量的输运方程。这一变化刚好约掉了压力项。另外,他认为,平行于壁面的体积力远大于垂直于壁面的体积力,并将后者忽略掉,从而得到了平行于壁面的体积力分布。D\"orr和Kloker\cite{dorr2015numerical}数值对比了这两种做法。他们先假设一个体积力分布,然后用直接数值模拟(DNS)计算得到这种预先设定的体积力会产生怎样的流场。之后在用上面提到的两种方法从流场反推体积力,来对比这两种方法的效果。他们发现,如果激发器是放置在边界层流动中,则压力项不能忽略,而若是在原本静止的流场中,则两种方法差别不大。Maden等人\cite{Maden2013}将基于实验的等离子体模型和之前介绍的两种唯像简化模型(史维模型\cite{shyy2002}和Suzen模型\cite{suzen2007})进行了对比,他们发现基于实验的模型要好于唯像简化模型。在他们的工作中,还提出了一种用复杂解析表达式近似拟合体积力分布的模型。在他们的结果中,这一模型的表现也要好于唯像的简化模型。在后文中,将这模型简记为Maden模型。 由于基于实验的模型计算量小,精度高,本文在研究中主要应用这一类模型,实验数据来自Kriegseis\cite{kriegseis2013velocity}的文章。另外,Maden模型也有用到。

\subsection{介质阻挡放电等离子体激发器在流动控制方面的应用}
DBD等离子体激发器已经被应用于诸多流动控制领域,如抑制流动分离\cite{Benard2016a,Sujar-Garrido2015,Benard2011,Benard2011a,Benard2016b,Little2010,Schneck2014,Jukes2009}、影响层流-湍流转捩\cite{Duchmann2014,Grundmann2007a,Grundmann2008,Kurz2014,Kotsonis2013,Hanson2010,Hanson2014}、湍流边界层减阻\cite{Jukes2006,Jukes2009,Mahfoze2017}、缓解激波/附面层干扰\cite{Im2012,Peschke2013}、提高压气机与涡轮性能\cite{huang2006,Ness2012}、控制管道流动\cite{Benard2008}和辅助飞行姿态控制\cite{He2009,Wei2013bang}等。这里重点介绍和本文有关的,也就是推迟转捩和湍流边界层减阻方面的应用。其他方面的应用可以参考综述文章\cite{Corke2010,wang2013Recent,wuyun2015}

在转捩推迟方面,DBD激发器最直接的应用就是依赖其产生的动量效应,对边界层流动进行加速,提高边界层流动的顺压梯度,从而抑制T-S波的发展。Riherd和Roy\cite{Riherd2013}采用数值模拟的方法验证了这一控制方案,Duchmann等人\cite{Duchmann2014,Grundmann2007a}和Grundmann等人\cite{Grundmann2007a}通过实验给出了证实。Riherd和Roy\cite{Riherd2014}还发现这一想法不仅对T-S波诱发的转捩有效,对流向条带诱发的转捩也能起到推迟的效果。除了增加边界层的顺压梯度之外,DBD还被应用于边界层内扰动的反向控制\cite{Grundmann2008,Kurz2014,Kotsonis2013,Simon2018}。这一应用主要得益于DBD激发器的快速响应能力。这种控方案通常需要在上游放置一个探测器。这样,DBD激发器所在位置的扰动相位就可以由上游探测器的信号准确的推算出来。然后用DBD激发器产生一个相当幅值但是反相位的扰动,就可以将边界层内原有的扰动波抵消掉。Hanson等人\cite{Hanson2010,Hanson2014}将DBD应用于控制边界层转捩过程中的瞬态增长。在他们研究的算例中,平板前缘有一排圆柱形的粗糙单元。这些粗糙单元会产生流向涡,并进一步导致扰动的瞬态增长。Hanson等人在下游放置DBD激发器来削弱这些流向涡,从而推迟了转捩。除了推迟二维边界转捩,DBD激发器还被应用于推迟三维边界层转捩。Schuele等人\cite{schuele2013control}将DBD激发器安装于一个带攻角的尖锥前端。这一流动的转捩本身是由横流失稳触发。实验中,DBD激发器起到了类似壁面粗糙单元的作用,激发出了次谐波模态。这一次谐波模态有效的推迟了转捩过程。D\"orr和Kloker\cite{dorr2016}用DBD激发器产生的“电风”削弱三维边界层中的二次流(横流)速度,从而降低了横流失稳模态的增长。本文也研究了类似的控制方法,并将其从后掠平板流动推广到实际的后掠翼流动中来。

在湍流减阻方面,Jukes和 Choi\cite{Jukes2006}在实验中采用等离子体激发器对$Re_\tau=380$的湍流边界层进行了控制。他们用激发器阵列产生近壁周期震荡的体积力,这些体积力会产生同方向旋转的流向涡。他们发现,当激励的周期在$T^+=16$的时候,减阻效果最好。其最大减阻率在激励器下游$x^+=75$的位置甚至可以达到45\%。随后,该课题组\cite{Choi2011,Whalley2014}通过改变不同激发器之间的相位差,在近壁产生了展向行波形式的激励,也成功降低了湍流边界层的阻力。Mahfoze和Laizet\cite{Mahfoze2017}试图将Jukes和Choi的减阻方法应用到湍流槽道中,但是很不幸的是他们发现周期振荡并不能降低阻力,而只向一个方向驱动流体反而可以将阻力降低33\%。本文也研究了采用DBD激发器进行展向周期激励的减阻方案,并通过相条件平均技术,进一步解释了控制机理。除了展向激励,流向激励也可以一定程度上降低摩阻。Li等人\cite{LiZX2015}在槽道的上下壁面均添加了一条激励带,他们可以产生沿着流向的体积力。通过直接数值模拟,Li等人发现虽然在激励的区域湍流的摩阻会上升,但是在其下游阻力会显著降低。

\section{通过推迟层流/湍流转捩减阻}
\label{sec:first}
众所周知,层流的摩擦阻力要比湍流的摩擦阻力小很多,所以流动减阻的一个重要方向就是扩大飞行器表面的层流范围。对于大型客机而言,由于机身长度过长,转捩总会无可避免的发生。相比之下,在机翼上发展和应用层流技术则有很大的前景。目前大多数客机使用的机翼还都是湍流机翼。湍流机翼发生从层流向湍流的转捩点一般在翼型弦长的10\%以前,而如果使用推迟转捩的层流技术,可以将转捩点推迟到20\%甚至70\%弦长之后。在这一小节,我们先简要介绍二维和三维边界层的失稳与转捩研究现状,最后再总结目前已经提出的转捩推迟手段。

\subsection{二维边界层失稳与转捩}
边界层转捩过程强烈依赖于来流条件和壁面条件,受到来流湍流度、来流马赫数、外流压力梯度、壁面温度、壁面粗糙度、壁面抽吸量及外部扰动特征参数等诸多因素的影响\cite{morkovin1969many},因此存在着多种物理机制。
在二维不可压缩边界层中,转捩过程可分为以下三种类型:当来流湍流度较低(小于0.1\%)时发生的是自然转捩(natural transition)\cite{papanastasiou1999viscous};而来流湍流度较高(大于1\%)时,转捩过程中小扰动的指数增长阶段将被跳过,这被称为跨越转捩(bypass transition)\cite{jacobs2001simulations};逆压梯度会导致层流边界层与壁面分离,从而引发分离流转捩(separation-induced transition)\cite{malkiel1995transition};反过来,顺压梯度会使湍流边界层再层流化\cite{walker1992role}。具体地,自然转捩过程分为四个阶段\cite{bradshaw1994turbulence}:第一阶段是所谓的边界层感受性过程(Receptivity)\cite{reshotko1984environment},指的是背景扰动如何进入边界层并产生不稳定波的机制;第二阶段是不稳定波的线性增长过程;第三阶段是不稳定波发展的非线性阶段,不稳定波发展到一定的幅值后,会出现波的相互作用和高阶不稳定性,从而导致以湍斑为特征的湍流结构的产生;最后一个阶段是从湍斑到完全湍流的发展过程。在自然转捩的第二阶段,扰动幅值相比于基本流非常小,一般采用线性稳定性理论进行描述。该理论假设扰动具有行波的形式,且不同频率,不同波长的扰动波之间不会互相干扰。基于这一假设,我们可以得到线性稳定性方程(Orr-Sommerfeld方程),并且得到不同频率和波长的扰动波的衰减或增长的情况,如图\ref{f:neutral_curve}所示。在此 图上,扰动的衰减(稳定)区和放大(不稳定)区可通过的扰动增长率等于零的线区分出来,这条线被称为中性稳定性曲线。令人特别关注的是曲线上 取最小值的点:小于此值的区域内,所有的扰动均会趋于稳定。这个最小的雷诺数被称为临界雷诺数。可见,速度剖面有拐点的边界层比没有的更不稳定,而且后者在 时仍存在不稳定频带,因此也被称为具有“无粘不稳定性”的剖面。实际上,上述频带可通过求解Rayleigh方程\cite{Schlichting1979}得到,此方程是Orr-Sommerfeld方程在 时的简化形式,基于此方程的理论被称为“无粘稳定性理论”。无粘稳定性理论中的拐点定理指出拐点的存在是流动失稳的充分必要条件。
\begin{figure}
  \centering
  \includegraphics[width=0.5\textwidth]{neutral_curve}
  \caption{二维边界层中二维扰动的中性稳定性曲线\cite{Schlichting1979}
  。图中,a曲线对应的是具有拐点PI的速度剖面,而b曲线对应的是无拐点的速度剖面。}\label{f:neutral_curve}
\end{figure}

\subsection{三维边界层失稳与转捩}
三维边界层转捩的研究起始于后掠层流机翼设计项目\cite{Gray1952},其目标是大幅降低机翼阻力。几十年来航空界一直致力于这一项目,然而由于三维边界层的稳定性涉及到边界层对自由流中的扰动与机翼表面粗糙度的感受性、基频扰动及其谐波与驻涡(crossflow vortices)等多种模态之间的相互作用等诸多问题,目前的研究与实际应用还有一定的距离。

\begin{figure}
  \centering
  \includegraphics[width=0.5\textwidth]{ch1/sweptwing.png}
  \caption{后掠机翼层流边界层示意图\cite{Aranl2000},其中u和w分别代表主流和横流,$\beta_o0$为边界层外缘流线与机翼弦向的夹角。$X_M$为边界层外缘流线的拐点。}\label{f:sweptwing}
\end{figure}


三维不可压缩边界层具有多种失稳机制\cite{Saric2003},其中横流不稳定性起主导作用。图\ref{f:sweptwing}显示了后掠机翼上的层流边界层流动。可见,由于沿机翼弦向压力梯度的存在,边界层外缘流线将发生扭曲,或者认为此处流体微团曲线运动产生的离心力与压力平衡。而在边界层内,流体微团的速度沿壁面法向逐渐减小,因此其产生的离心力减小,而压力却保持不变,这种不平衡性导致了垂直于主流方向的横流(crossflow velocity)的出现。横流速度剖面在壁面和边界层外缘必须为零,这也就意味着横流速度剖面必须有一个拐点。这个拐点就是横流失稳的根源所在。横流失稳会产生驻涡模态和行波模态,其中驻涡模态是由壁面粗糙度激发出来的,而行波模态是由来流湍流度激发出来的\cite{Schrader2008}。当来流湍流度很低的时候,转捩由驻涡模态主导,当来流湍流度较高时,转捩会由行波模态主导\cite{Bipps1999}。一般而言,在高空中,来流都很干净,湍流度较低,后掠翼上的转捩均由驻涡模态主导。所以本文主要研究驻涡模态主导的转捩。驻涡模态之所以叫“驻涡”是因为失稳之后其在边界层内会形成同方向旋转的涡。这些涡的涡轴方向大致与势流线平行(有3$^\circ$-4$^\circ$的夹角)。当这些横流涡模态的幅值达到一定的强度,法向和展向对扰动速度会产生很强的对流效应,在流向速度云图上回体现出一个翻转结构。这时候横流涡就进入了饱和阶段\cite{Malik1994,Haynes2000}。在饱和阶段,横流涡的幅值不再增长,但是由于其极大程度地扭曲了边界层,高频的二次失稳扰动迅速发展起来\cite{White2005}。Malik等人\cite{Malik1999}根据二次失稳扰动能量的来源不同,将其分为两类,分别是能量来源于纵向剪切的Y模态和能量来源于横向剪切的Z模态。图\ref{f:yzmodes}为Y模态与Z模态的扰动幅值等值线。
\begin{figure}
  \centering
  \includegraphics[width=0.6\textwidth]{ch1/zymodes.png}
  \caption{Z模态(上图)与Y模态(下图)扰动分布\cite{Malik1999}}\label{f:yzmodes}
\end{figure}
与此同时,他们还提出了基于二次失稳理论N转捩因子的预测方法。不仅驻涡会诱发二次失稳,行波模态也会诱发二次失稳。Li\cite{Li2014}等人对横流行涡的二次失稳也做了详细的研究,他们发现行涡的增长率相比于驻涡更大,并且在更低的幅值饱和,而二次失稳的幅值却不亚于驻涡,所以只有在行涡的初始幅值很低的时候才是驻涡主导转捩。定性而言,较低的首次失稳幅值会产生较慢的二次失稳增长率\cite{Li2015a}。因此,如果能够降低首次失稳模态的幅值,就能够推迟转捩。本文的提出的控制方案均针对于降低横流首次失稳主模态幅值。

对于横流失稳的分析,直接数值模拟\cite{Wassermann2005,Bonfigli2007,Duan2013,Hosseini2013}肯定是最强大且直接的手段。但是其计算量巨大,费时费力。较为传统的做法是线性稳定性分析(linear stability theory,LST)。这种做法假设扰动形式为行波解,忽略掉了非线性项,并最终将偏微分方程转化为一个特征值问题。Mack\cite{Mack1984}给出了这种方法的详细细节。在把线性稳定性理论应用于边界层的时候还需要做基本流平行性假设,也就是假设边界层厚度不增长。但是这种做法会高估失稳模态的增长率。为了克服这一缺点,Herbert\cite{Herbert1987,Herbert1997,Herbert1993}提出了抛物化扰动方程(parabolized stability equation,PSE)。这一方程可以求解线性非线性失稳阶段扰动在边界层内的发展过程。Bertolotti等人\cite{Bertolotti1991,Bertolotti1992}将这一方法应用于求解 Tollmien--Schlichting(TS)波在Blasius边界层内的演化过程。PSE方法在考虑了非线性项之后强大到甚至可以解析三维边界层的二次失稳和涡破碎过程\cite{LiFei2011,Li2015a}。这一方法的理论公式和数学推导在之后的章节中给出。

\subsection{转捩推迟方案研究进展}
在横流失稳的控制方面,最著名的结果无外乎Saric\cite{Saric1998}等人提出的通过人为引入亚谐波模态,减缓最不稳定模态增长的方法。在他们的实验中,亚谐波模态是通过翼型前缘一排间距略小于最不稳定模态展向波长的粗糙单元(discrete roughness element,DRE)引入的。在他们提出这一方法之后,诸多学者前赴后继的对这一方法进行实验和数值研究\cite{Malik1999,Haynes2000,Wassermann2002,Carpenter2008,LiFei2011,Hosseini2013,Li2015a}。
Malik\cite{Malik1999}等人采用NPSE进行计算,得到了有控制模态和没有控制模态时扰动的发展变化情况如图\ref{f:DRE1}。
\begin{figure}
  \centering
  \includegraphics[width=0.6\textwidth]{ch1/DRE1.png}
  \caption{亚谐波模态对扰动发展的影响\cite{Malik1999}}\label{f:DRE1}
\end{figure}
其中实线是控制模态和自然模态同时存在时它们的幅值沿着流向发展的情况,虚线是只有自然模态时的情况。可以看到在有控制模态的时候自然扰动模态的发展受到了抑制。Carpenter\cite{Carpenter2008}等人在2008年做了利用粗糙单元推迟转捩的飞行试验,实验发现在翼型前缘表面没有打磨得很光滑的时候该方法是有效的。随后,Li等人\cite{Li2009roughness}对该控制方法做了从线性失稳到二次失稳,涡破碎的全面数值模拟,进一步证实了其可行性。2013年,Hosseini等人\cite{Hosseini2013}用DNS更加精细的解析出了亚谐波控制下的横流涡发展过程。他们的计算将扰动传入边界层的感受性过程也包含了进来。2014年Lovig等人\cite{Lovig2014}在湍流度更低的风洞(来流湍流度0.04\%)中印证了这一方法,从而更加肯定了这种方法在实际飞行中的可行性。上面提到的做法都是试图在横流失稳的首次失稳阶段进行控制,Friederich和 Kloker\cite{Friederich2011,Friederich2012}提出了一种通过吸气控制二次失稳的方法。横流涡饱和之后在其上会发展出二次失稳模态。二次失稳模态的能量来源这正是被横流涡扭曲的边界层。Friederich和 Kloker在横流涡卷着流体上扬的位置放置吸气的小孔。这些小孔几乎直接将横流涡吸没了,使边界层重新回到了规则安静的状态。除了这些传统做法之外,采用等离子体削弱横流也是有效提高三维边界层稳定性的做法\cite{dorr2016},这已经在上一节介绍过,这里不再复述。


\section{通过控制壁湍流相干结构减阻}
在上世纪60年代,人们发现湍流中的脉动并不是完全随机的,而是包含一些可辨认的有序大尺度结构。这些结构之后便被称作湍流相干结构(coherent structure)。现代湍流控制的主要途径就是去干扰这种大尺度结构。这一小节先简要的介绍目前人们对湍流相关结构的认识,之后再简单综述近些年来在减阻方面的研究成果。
\subsection{湍流相干结构研究进展}
Robinson\cite{Robinson1991}将壁湍流相干结构分为八类,分别是低速条带、上抛、下扫、涡结构、强剪切层、近壁团状结构、背面结构和外层大尺度结构。其中条带主要出现在$y^+<30$的近壁区域\footnote{上标“+”表示用粘性尺度无量纲化的物理量,其特征速度为$u_\tau=\sqrt{\tau_w/\rho}$,特征长度为$\delta_\nu=\nu/u_\tau$}。
当在这一高度范围截取与壁面平行的流向脉动速度云图时,就会发现低速流体聚集在准周期分布的带状结构中。这就是所谓的低速条带。Kim等人\cite{Kim1987}指出,这些条带在流向可以长达1000$\delta_\nu$,其展向平均间距几乎不遂雷诺数变化。近壁面的涡结构主要是准流向涡。这些涡并不完全沿着流向,而是向左或向右的有4$^\circ$的偏角\cite{Jeong1997}。相比与条带,这些流向涡的长度要小很多,平均约为200$\delta_\nu$。这些流向涡的涡核位于$y^+=20$高度处,平均半径为15$\delta_\nu$。

Hamilton等人\cite{Hamilton1995}1995年首次提出了由流向涡和条带主导的近壁湍流自维持机制。他们认为近壁相干结构的再生可以分为三个阶段,首先是由流向涡导致条带的生成,其次是条带的失稳破碎,最后在通过非线性机机制生成流向涡。Jim\'enez\cite{Jimenez2005}通过对最小槽道的直接数值模拟,给出了自维持机制的周期约为$T^+=Tu^2_\tau/\nu\approx400$。同时他们指出,这一循环过程包含相当长的平静时期,猝发过程所占的时间只有大约整个循环过程的1/3。目前,学者们对条带的产生机制的认识比较统一。Jim\'enez等人\cite{Jimenez1999}认为条带是由流向涡的线性输运过程产生的。Del \'Alamo等人\cite{del2006}和Cossu等人\cite{Cossu2009}更进一步采用稳定性分析中的瞬态增长理论给出了解释。由于线化的N-S方程不是自伴随的,或者也可以说其微分算符不是对称的,这导致其特征向量并非正交的。特征向量对应的模态增长率指的是时间趋于无穷时的渐进结果,但是在特定的一段时间内,这些非正交的衰减模态可以组合出“暂时”增长的扰动形式,这就是瞬态增长。当采用湍流速度剖面最为基本流进行稳定性分析时,就会发现最优扰动对应的初始扰动分布即为一对向相反方向旋转的流向涡。这对流向涡随着扰动的增长就会形成条带结构,且条带间距与湍流近壁条带间距一致。与条带的生成不同,流向涡的生成则普遍被认为是一种非线性机制。有一类观点认为新的流向涡的生成必须依赖已有的涡结构\cite{Heist2000}。另一类笔者认为更可信的观点是流向涡是条带失稳破碎产生的。Jim\'enez\cite{Jimenez1999}在数值模拟中人为地减弱了条带结构,并发现这样做抑制了流向涡的再生。Schoppa和Hussain\cite{Schoppa2002}提出了基于条带瞬态增长(streak
transient growth,STG)的流向涡生成机制。他们将流向涡的生成分成三个阶段:首先,流向涡量层因为条带的瞬态增长机制而变形;之后,条带上的奇(sinuous)模态扰动快速增长并触发非线性机制;最后,流向涡量层因为$\p u/\p x$过大而破碎,并最终变为流向涡。


\subsection{湍流减阻技术研究进展}
\label{chap1:sample:table}
湍流减阻技术大致可以分为两类,一类是主动控制,一类是被动控制。被动控制比较成功的有在流体中添加高分子聚合物\cite{White2008Mechanics}和在壁面放置小肋\cite{Garcia2011}等。本文主要研究的等离子体控制属于主动控制范畴,这里主要介绍近些年主动控制的研究成果。

目前学者们研究比较多的是反馈控制方法,这种方法由Choi\cite{Choi1994}首先提出来。这种做法主要是针对流向涡设计的。其在近壁的一个平面上探测法向速度,然后再在壁面上施加相反方向的速度,从而抵消流向涡引发的上抛下扫运动。Hammond等人\cite{Hammond1998}发现反向控制的探测平面在$y^+_d<20$的时候才能起到效果,反之则会增加阻力。Deng和Xu\cite{Deng2012}基于条带瞬态增长产生流向涡的机理更加深入的研究了反向控制对于流向涡的影响。她们指出,探测平面在$y^+_d>20$处反向控制失效是因为法向脉动速度$v'$在$y^+=20$位置发生了反号。反馈控制不仅可以起到减阻的效果,还能从机理上说明阻力的来源,但是实际应用性并不强,因为实际中很难去探测空间一个平面的速度分布。

\begin{figure}[htb]
  \centering
  \includegraphics[width=0.8\textwidth]{ch1/POLIMIkzw.JPG}
  \caption{展向行波激励减阻率图谱(Xie博士论文\cite{Xie2014}中Fig 2.2)}\label{f:POLIMIkzw}
\end{figure}
Jung等人\cite{Jung1992}首先提出了采用壁面振荡的方法抑制湍流阻力。他们在$Re_\tau=200$的槽道施加周期为$T^+=100$的展向振动,获得了高达40\%的减阻率。Choi等人\cite{Choi2002}采用条件平均技术分析了壁面周期震荡的圆管湍流,并指出壁面振动使得条带和流向涡的位置关系发生扭曲,从而降低了上抛下扫对雷诺应力的贡献。Yakeno\cite{Yakeno2014}之后对湍流槽道做了类似的分析。由于这一减阻方法简单易实施,有大量学者对其进行了研究。其中Quadrio及其同事是这一领域的先驱,他们进行了大量的参数研究\cite{Quadrio2009,Quadrio2011,Gatti2013,Gatti2016}。他们不仅尝试了采用壁面周期振荡减阻的情况,还应用了采用流向行波和展向行波的振荡形式进行减阻。就流向行波形式的展向振动而言,其壁面边界条件可以写为:
\begin{equation}
   w_w(x,t)=A{\rm sin}(k_xx-\omega t)
\end{equation}
他们发现,当用向上游传播的流向行波形式的展向振动时(即$c=\omega/k_xx<0$),总能起到减阻的效果。但是若令波的传播方向向下游,则在一定相速度的时候会出现增阻的现象。在随着行波一起运动的坐标系中,最佳激励周期与壁面均匀振动的最佳周期相同,这也可以说明他们减阻的机制是相同的。Quadrio的博士生Xie也做了展向行波减阻方面的研究\cite{Xie2014},他发现在展向波数$k_z=0$的时候,也就是退化为展向均匀振动的时候减阻率最大。他们得到的不同参数激励下的减阻率图谱如图\ref{f:POLIMIkzw}。采用等离子体激发器进行湍流减阻一般也是采取展向激励的形式,相关工作已在之前的的小结提到,这里不再复述。更多关于壁湍流相干结构和控制方法的介绍,读者可以参考许春晓的综述文章\cite{Xu2015bi}



 
